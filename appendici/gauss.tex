\cleardoublepage
\chapter{Il teorema di Gauss nello spazio di Minkowski}
\label{cha:teorema-gauss}

Il teorema di Gauss in $\M$ mette in relazione l'integrale di volume
(spaziotemporale) della quadridivergenza di un quadrivettore $V^{\alpha}(x)$ con
un integrale di ipersuperficie di $V^{\alpha}(x)$
\begin{equation}
  \int\limits_{\Omega} \partial_{\alpha}V^{\alpha}(x) \dd^{4} x =
  \int\limits_{\partial\Omega} V^{\alpha}(x) \dd\Sigma_{\alpha},
\end{equation}
in cui $\partial\Omega$ è l'ipersuperficie di contorno del volume
quadrimensionale $\Omega$. Il quadrivettore $\dd\Sigma_{\alpha}$ è l'elemento di
ipersuperficie
\begin{equation}
  \dd\Sigma_{\alpha} = (\dd x^{1}\dd x^{2}\dd x^{3}, \dd x^{0}\dd x^{2}\dd
  x^{3}, \dd x^{0}\dd x^{1}\dd x^{3}, \dd x^{0}\dd x^{1}\dd x^{2})
\end{equation}
che ha come direzione la normale all'ipersuperficie e come modulo la sua area
$\dd\Sigma$.

% TODO: fare un semplice disegno come quello che ha fatto Ingrosso a lezione
% (11/04/2012) per aiutare a capire la geometria del sistema.
Questo teorema ha un'utile applicazione per quanto riguarda i quadrivettori
$V^{\alpha}(x)$ con quadridivergenza nulla
\begin{equation}
  \partial_{\alpha}V^{\alpha}(x) = 0.
\end{equation}
Integriamo $\partial_{\alpha}V^{\alpha}(x)$ sul volume $\Omega$ delimitato da
due ipersuperfici a tempo costante $\Sigma_{1}$ e $\Sigma_{2}$ (cioè individuate
rispettivamente dalle equazioni $t = t_{1}$ e $t = t_{2}$, con $t_{1}$ e $t_{2}$
costanti) e da un'ipersuperficie $\Sigma_{\infty}$ posta all'infinito spaziale.
Per il teorema di Gauss risulta
\begin{equation}
  0 = \int\limits_{\Omega} \partial_{\alpha}V^{\alpha}(x) \dd^{4}x =
  \int\limits_{\Sigma_{2}} V^{\alpha}(x) \dd\Sigma_{\alpha} -
  \int\limits_{\Sigma_{1}} V^{\alpha}(x) \dd\Sigma_{\alpha} +
  \int\limits_{\Sigma_{\infty}} V^{\alpha}(x) \dd\Sigma_{\alpha}.
\end{equation}
Il segno meno davanti all'integrale su $\Sigma_{1}$ è dovuto al fatto che
abbiamo orientato entrambe le normali a $\Sigma_{1}$ e $\Sigma_{2}$ nel verso
crescente del tempo.  Se $V^{\alpha}(x)$ tende abbastanza rapidamente a $0$
all'infinito spaziale, il terzo integrale nell'equazione precedente è nullo e
quindi
\begin{equation}
  \int\limits_{\Sigma_{2}} V^{\alpha}(x) \dd\Sigma_{\alpha} =
  \int\limits_{\Sigma_{1}} V^{\alpha}(x) \dd\Sigma_{\alpha}.
\end{equation}
Per l'arbitrarietà della scelta delle ipersuperfici $\Sigma_{1}$ e $\Sigma_{2}$,
ciascuna con la coordinata temporale fissata, dobbiamo concludere che
l'integrale
\begin{equation}
  I(\Sigma) = \int\limits_{\Sigma} V^{\alpha}(x) \dd\Sigma_{\alpha}
\end{equation}
non dipende dall'ipersuperficie $\Sigma$, cioè è una costante rispetto al tempo.
In particolare, se scegliamo un sistema di riferimento nel quale
$\dd\Sigma_{\alpha} = (\dd^{3} x, 0, 0, 0)$ allora abbiamo mostrato che
\begin{equation}
  \partial_{\alpha}V^{\alpha}(x) = 0 \implies \int V^{0}(x) \dd^{3} x =
  \text{costante in $t$}
\end{equation}
se l'integrale è esteso a una regione spaziale sufficientemente grande in modo
che $V^{\alpha}(x)$ si annulla sul suo bordo.

%%% Local Variables: 
%%% mode: latex
%%% TeX-master: "../astrofisica-teorica"
%%% fill-column: 80
%%% End: 
