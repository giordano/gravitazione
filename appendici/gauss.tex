\cleardoublepage
\chapter{Teorema di Gauss nello spazio di Minkowski}
\label{cha:teorema-gauss}

Il \index{teorema!di Gauss}teorema di Gauss in $\M$ mette in relazione
l'integrale di volume (spaziotemporale) della quadridivergenza di un
quadrivettore $V^{\alpha}(x)$ con un integrale di ipersuperficie di
$V^{\alpha}(x)$
\begin{equation}
  \int\limits_{\Omega} \partial_{\alpha}V^{\alpha}(x) \dd^{4} x =
  \int\limits_{\partial\Omega} V^{\alpha}(x) \dd\Sigma_{\alpha},
\end{equation}
in cui $\partial\Omega$ è l'ipersuperficie di contorno del volume
quadrimensionale $\Omega$. Il quadrivettore $\dd\Sigma_{\alpha}$ è l'elemento di
ipersuperficie
\begin{equation}
  \dd\Sigma_{\alpha} = (\dd x^{1}\dd x^{2}\dd x^{3}, \dd x^{0}\dd x^{2}\dd
  x^{3}, \dd x^{0}\dd x^{1}\dd x^{3}, \dd x^{0}\dd x^{1}\dd x^{2})
\end{equation}
che ha come direzione la normale all'ipersuperficie e come modulo la sua area
$\dd\Sigma$.

\begin{figure}
  \centering
  \tdplotsetmaincoords{70}{100}
  \begin{tikzpicture}[tdplot_main_coords,font=\footnotesize,scale=5,
    piano/.style={gray!20!white,opacity=0.85}]
    \node[shape=coordinate] (O1) at (0,0,0) {};
    \draw[->] (O1) -- ++(1,0,0) node[anchor=north east] {$x$}; % asse x
    \draw[->] (O1) -- ++(0,1,0) node[anchor=north west] {$y$}; % asse y
    \draw[->] (O1) -- ++(0,0,1) node[anchor=south] {$t$}; % asse temporale
    \filldraw[piano] (0,0,0.1) node[black,left] {$t_{1}$} -- ++(0,1,0) --
    ++(1,0,0) node[black,above left] {$\Sigma_{1}$} -- ++(0,-1,0) -- ++(-1,0,0);
    \filldraw[piano] (0,0,0.9) node[black,left] {$t_{2}$} -- ++(0,1,0) --
    ++(1,0,0) node[black,above left] {$\Sigma_{2}$} -- ++(0,-1,0) -- ++(-1,0,0);
    \draw[->] (0.5,0.5,0.1) -- ++(0,0,0.3) node[above] {$\versor{n}_{1}$};
    \draw[->] (0.5,0.5,0.9) -- ++(0,0,0.3) node[above] {$\versor{n}_{2}$};
    \node at (0.3,0.3,0.5) {$\Omega$};
  \end{tikzpicture}
  \caption[Volume quadrimensionale delimitato da iperpiani a tempo
  costante]{Volume quadrimensionale $\Omega$ delimitato dagli iperpiani a tempo
    $t$ costante $\Sigma_{1}$ e $\Sigma_{2}$.  Sono riportate solo due
    dimensioni spaziali, $x$ e $y$.  Gli iperpiani $\Sigma_{1}$ e $\Sigma_{2}$
    si estendono verso l'infinito spaziale e sono chiusi dall'iperpiano
    $\Sigma_{\infty}$, non riportato nella figura}
  \label{fig:teorema-gauss}
\end{figure}
Questo teorema ha un'utile applicazione per quanto riguarda i quadrivettori
$V^{\alpha}(x)$ con quadridivergenza nulla
\begin{equation}
  \partial_{\alpha}V^{\alpha}(x) = 0.
\end{equation}
Con riferimento alla figura~\ref{fig:teorema-gauss}, integriamo
$\partial_{\alpha}V^{\alpha}(x)$ sul volume $\Omega$ delimitato da due
ipersuperfici a tempo costante $\Sigma_{1}$ e $\Sigma_{2}$ (cioè parametrizzate
rispettivamente dalle equazioni $t = t_{1}$ e $t = t_{2}$, con $t_{1}$ e $t_{2}$
fissati) e da un'ipersuperficie $\Sigma_{\infty}$ posta all'infinito spaziale.
Per il teorema di Gauss risulta
\begin{equation}
  0 = \int\limits_{\Omega} \partial_{\alpha}V^{\alpha}(x) \dd^{4}x =
  \int\limits_{\Sigma_{2}} V^{\alpha}(x) \dd\Sigma_{\alpha} -
  \int\limits_{\Sigma_{1}} V^{\alpha}(x) \dd\Sigma_{\alpha} +
  \int\limits_{\Sigma_{\infty}} V^{\alpha}(x) \dd\Sigma_{\alpha}.
\end{equation}
Il segno meno davanti all'integrale su $\Sigma_{1}$ è dovuto al fatto che
abbiamo orientato le normali $\versor{n}_{1}$ e $\versor{n}_{2}$ degli
iperpiani, rispettivamente, $\Sigma_{1}$ e $\Sigma_{2}$ nel verso crescente del
tempo.  Se $V^{\alpha}(x)$ tende abbastanza rapidamente a $0$ all'infinito
spaziale, il terzo integrale nell'equazione precedente è nullo e quindi
\begin{equation}
  \int\limits_{\Sigma_{2}} V^{\alpha}(x) \dd\Sigma_{\alpha} =
  \int\limits_{\Sigma_{1}} V^{\alpha}(x) \dd\Sigma_{\alpha}.
\end{equation}
Per l'arbitrarietà della scelta delle ipersuperfici $\Sigma_{1}$ e $\Sigma_{2}$,
ciascuna con la coordinata temporale fissata, dobbiamo concludere che
l'integrale
\begin{equation}
  I(\Sigma) = \int\limits_{\Sigma} V^{\alpha}(x) \dd\Sigma_{\alpha}
\end{equation}
non dipende dall'ipersuperficie $\Sigma$, cioè è una costante rispetto al tempo.
In particolare, se scegliamo un sistema di riferimento nel quale
$\dd\Sigma_{\alpha} = (\dd^{3} \bm{x}, 0, 0, 0)$ allora abbiamo mostrato che
\begin{equation}
  \partial_{\alpha}V^{\alpha}(x) = 0 \implies \int V^{0}(x) \dd^{3} \bm{x} =
  \text{costante in $t$}
\end{equation}
se l'integrale è esteso a una regione spaziale sufficientemente grande in modo
che $V^{\alpha}(x)$ si annulla sul suo bordo.\index{teorema!di Gauss}

In maniera analoga si può far vedere che se per un tensore
$\tensor{T}{^{\alpha\beta}}$ si ha
\begin{equation}
  \partial_{\beta}T^{\alpha\beta} = 0,
\end{equation}
allora il quadrivettore
\begin{equation}
  Q^{\alpha}(\Sigma) = \int\limits_{\Sigma} \tensor{T}{^{\alpha\beta}}\dd
  \Sigma_{\beta},
\end{equation}
in cui $\Sigma$ è una ipersuperficie a tempo costante, non dipende dalla
superficie $\Sigma$ su cui si calcola l'integrale, cioè è indipendente dal
tempo, se l'integrale è esteso a tutto lo spazio tridimensionale.  In
particolare, scelto un sistema di riferimento nel quale
$\dd\Sigma_{\beta} = (\dd^{3} \bm{x}, 0, 0, 0)$, risulta
\begin{equation}
  \partial_{\beta}T^{\alpha\beta} = 0 \implies Q^{\alpha} = \int\limits_{\Sigma}
  T^{\alpha0} \dd^{3} \bm{x} = \text{costante in $t$}.
\end{equation}

Più in generale il \index{teorema!di Gauss}teorema di Gauss per un tensore
$T^{\alpha_{1}\alpha_{2}\dots\alpha_{k}}(x)$ di rango $k$ è
\begin{equation}
  \int\limits_{\Omega} \partial_{\alpha_{1}}
  T^{\alpha_{1}\alpha_{2}\dots\alpha_{k}}(x) \dd^{4} x = \int
  T^{\alpha_{1}\alpha_{2}\dots\alpha_{k}}(x) \dd\Sigma_{\alpha_{1}}.
\end{equation}
Se risulta
\begin{equation}
  \partial_{\alpha_{1}} T^{\alpha_{1}\alpha_{2}\dots\alpha_{k}}(x) = 0
\end{equation}
allora l'integrale su un'ipersuperficie $\Sigma$ a tempo fissato
\begin{equation}
  Q^{\alpha_{2}\dots\alpha_{k}}(\Sigma) = \int\limits_{\Sigma}
  T^{\alpha_{1}\alpha_{2}\dots\alpha_{k}}(x) \dd\Sigma_{\alpha_{1}}
\end{equation}
non dipende da $\Sigma$, quindi è costante rispetto al tempo, ed è un tensore di
rango $k-1$.

%%% Local Variables:
%%% mode: latex
%%% TeX-master: "../astrofisica-teorica"
%%% fill-column: 80
%%% End:
