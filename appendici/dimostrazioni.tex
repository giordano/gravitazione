\cleardoublepage
\chapter{Dimostrazioni di alcune relazioni}
\label{cha:dimostrazioni}

\section{Dimostrazione della
  relazione~\texorpdfstring{\eqref{eq:correzione-metrica-controvariante}}
  {(2.33)}}
\label{sec:dimostr-correz-metrica-controvariante}

Nel caso di campo gravitazionale debole, la correzione al tensore metrico
covariante al primo ordine negli infinitesimi è
\begin{equation}
  \label{eq:correzione-metrica-covariante}
  g_{\mu\nu} = \eta_{\mu\nu} + h_{\mu\nu},
\end{equation}
con $\abs{h_{\mu\nu}} \ll 1$.  Anche $g^{\mu\nu}$ dovrà differire di poco dal
tensore di Minkowski controvariante
\begin{equation}
    \label{eq:correzione-metrica-controvariante2}
  g^{\mu\nu} = \eta^{\mu\nu} + c^{\mu\nu}
\end{equation}
ma le correzioni $c^{\mu\nu}$ alle componenti del tensore metrico controvariante
non sono semplicemente
$c^{\mu\nu} = h^{\mu\nu} = \eta^{\mu\lambda}\eta^{\nu\sigma} h_{\lambda\sigma}$
perché $g^{\mu\nu}$ e $g_{\mu\nu}$ sono legati dalla relazione
\begin{equation}
  g_{\mu\nu}g^{\nu\lambda} = \tensor{\delta}{_{\mu}^{\lambda}}.
\end{equation}
Infatti sostituendo le espressioni~\eqref{eq:correzione-metrica-covariante} e
\eqref{eq:correzione-metrica-controvariante2} dei tensori metrici nell'equazione
precedente abbiamo
\begin{equation}
  \tensor{\delta}{_{\mu}^{\lambda}} = (\eta_{\mu\nu} +
  h_{\mu\nu})(\eta^{\nu\lambda} + c^{\nu\lambda}) =
  \tensor{\delta}{_{\mu}^{\lambda}} + \tensor{c}{_{\mu}^{\lambda}} +
  \tensor{h}{_{\mu}^{\lambda}} + h_{\mu\nu}c^{\nu\lambda},
\end{equation}
quindi deve risultare
\begin{equation}
  0 = \tensor{c}{_{\mu}^{\lambda}} + \tensor{h}{_{\mu}^{\lambda}} +
  h_{\mu\nu}c^{\nu\lambda} \implies \tensor{c}{_{\mu}^{\lambda}} =
  -\tensor{h}{_{\mu}^{\lambda}} - h_{\mu\nu}c^{\nu\lambda}.
\end{equation}
Innalzando in ambo i membri l'indice libero $\mu$ ricaviamo
\begin{equation}
  c^{\mu\lambda} = -h^{\mu\lambda} - \tensor{h}{^{\mu}_{\nu}}c^{\nu\lambda} =
  -h^{\mu\lambda} -\tensor{h}{^{\mu}_{\nu}}(-h^{\nu\lambda} -
  \tensor{h}{^{\nu}_{\rho}} c^{\rho\lambda}) = -h^{\mu\lambda} +
  \tensor{h}{^{\mu}_{\nu}}h^{\nu\lambda} + \mathcal{O}(h^{3}).
\end{equation}
Quindi trascurando termini infinitesimi superiori al terzo ordine, il tensore
metrico controvariante nell'approssimazione di campo debole è dato da
\begin{equation}
  g^{\mu\nu} = \eta^{\mu\nu} + c^{\mu\nu} \approx \eta^{\mu\nu} - h^{\mu\lambda}
  + \tensor{h}{^{\mu}_{\nu}}h^{\nu\lambda}.
\end{equation}
Fermandosi al primo ordine della perturbazione si ha
$g^{\mu\nu} \approx \eta^{\mu\nu} - h^{\mu\nu}$.

\section{Dimostrazione delle
  relazioni~\texorpdfstring{\eqref{eq:g_mu_rho-determinante}}{(3.51)}}
\label{sec:dimostr-determinante}

Sia $A$ una matrice quadrata reale di ordine $n$
\begin{equation}
  A =
  \begin{pmatrix}
    a_{11} & a_{12} & \cdots & a_{1n} \\
    a_{21} & a_{22} & \cdots & a_{2n} \\
    \vdots & \vdots & \ddots & \vdots \\
    a_{n1} & a_{n2} & \cdots & a_{nn}
  \end{pmatrix},
\end{equation}
il suo determinante $D$ si può calcolare, avendo fissato una riga $i$, con lo
\emph{sviluppo di Laplace}
\begin{equation}
  D = \det A = \sum_{j=1}^{n} a_{ij}A_{ij}
\end{equation}
in cui $A_{ij}$ è il \emph{complemento algebrico} dell'elemento $a_{ij}$, cioè
il determinante della matrice che si ottiene da $A$ sopprimendo l'$i$-esima riga
e la $j$-esima colonna moltiplicato per $+1$ se $i+j$ è pari, per $-1$ se $i+j$
è dispari.  Lo sviluppo di Laplace può anche essere svolto fissando la colonna
$j$ e sommando sull'indice di riga $i$.  Dunque, la derivata di $D$ fatta
rispetto all'elemento $a_{ij}$ è il suo coefficiente nello sviluppo di Laplace,
cioè il complemento algebrico corrispondente
\begin{equation}
  \label{eq:deriv-determinante}
  \parder{D}{a_{ij}} = A_{ij}.
\end{equation}

La matrice $B = A^{-1}$ inversa di $A$ con determinante non nullo è quella
matrice di ordine $n$ tale che
\begin{equation}
  A B = B A = I_{n},
\end{equation}
dove $I_{n} = (\delta_{ij})_{i,j=1}^{n}$ è la matrice identità di ordine
$n$. Per la regola di Cramer gli elementi $b_{ij}$ della matrice inversa sono
dati dal rapporto $A_{ji}/D$, cioè la matrice $B$ può essere così scritta
\begin{equation}
  B = A^{-1} =
  \begin{pmatrix}
    b_{11} & b_{12} & \cdots & b_{1n} \\
    b_{21} & b_{22} & \cdots & b_{2n} \\
    \vdots & \vdots & \ddots & \vdots \\
    b_{n1} & b_{n2} & \cdots & b_{nn}
  \end{pmatrix} =
  \begin{pmatrix}
    \dfrac{A_{11}}{D} & \dfrac{A_{21}}{D} & \cdots & \dfrac{A_{n1}}{D} \\[2.0ex]
    \dfrac{A_{11}}{D} & \dfrac{A_{22}}{D} & \cdots & \dfrac{A_{n2}}{D} \\[2.0ex]
    \vdots            & \vdots            & \ddots & \vdots            \\[2.0ex]
    \dfrac{A_{1n}}{D} & \dfrac{A_{2n}}{D} & \cdots & \dfrac{A_{nn}}{D}
  \end{pmatrix}.
\end{equation}
Per la regola di Binet risulta $\det B = 1/\det A = 1/D$.  Inoltre, se la
matrice $A$ è diagonale, si ricava subito che anche $B$ è diagonale con elementi
$b_{ii}=1/a_{ii}$.

Il tensore metrico covariante $g_{\mu\nu}$ è un insieme di $4^{2}$ quantità
reali che possono essere raccolte in una matrice $4 \times 4$
\begin{equation}
  g_{\mu\nu} =
  \begin{pmatrix}
    g_{00} & g_{01} & g_{02} & g_{03} \\
    g_{10} & g_{11} & g_{12} & g_{13} \\
    g_{20} & g_{21} & g_{22} & g_{23} \\
    g_{30} & g_{31} & g_{32} & g_{33}
  \end{pmatrix}.
\end{equation}
Dall'equazione~\eqref{eq:deriv-determinante} abbiamo che la derivata di
$g = -\det(g_{\mu\nu})$ rispetto all'elemento $g_{\mu\nu}$ è
\begin{equation}
  \parder{g}{g_{\mu\nu}} = -\Delta_{\mu\nu},
\end{equation}
in cui $\Delta_{\mu\nu}$ è il complemento algebrico dell'elemento $g_{\mu\nu}$.
Il tensore metrico controvariante $g^{\mu\nu}$ è la matrice inversa di
$g_{\mu\nu}$, quindi, sfruttando la simmetria del tensore metrico, abbiamo
\begin{equation}
  g^{\mu\nu} = g^{\nu\mu} = \frac{\Delta_{\mu\nu}}{-g} =
  \frac{1}{g} \parder{g}{g_{\mu\nu}}.
\end{equation}
Applicando lo stesso ragionamento a $g^{\mu\nu}$ troviamo
($G = -\det(g^{\mu\nu}) = 1/\det(g_{\mu\nu}) = 1/g$)
\begin{equation}
  g_{\mu\nu} = \frac{1}{G} \parder{G}{g^{\mu\nu}} =
  \frac{1}{G} \parder{G}{g} \parder{g}{g^{\mu\nu}} = g \bigg(- \frac{1}{g^{2}}
  \bigg) \parder{g}{g^{\mu\nu}} = - \frac{1}{g} \parder{g}{g^{\mu\nu}}.
\end{equation}

\section{Dimostrazione dell'identità di Bianchi}
\label{sec:dimostr-identita-bianchi}

Per verificare \index{identità!di Bianchi}l'identità di
Bianchi~\eqref{eq:bianchi} conviene verificarla in un sistema di coordinate
localmente inerziali, quindi sarà valida in tutto lo spazio dato il carattere
tensoriale di $R_{\lambda\mu\nu\kappa}$.  In questo sistema di coordinate la
connessione affine si annulla e la derivata covariante coincide con la derivata
ordinaria, dunque dalla~\eqref{eq:riemann-covariante} abbiamo
\begin{subequations}
  \begin{align}
    R_{\lambda\mu\nu\kappa;\eta} &=
    \frac{1}{2} \partial_{\eta}(\partial_{\kappa} \partial_{\mu} g_{\lambda\nu}
    - \partial_{\kappa} \partial_{\lambda} g_{\mu\nu}
    - \partial_{\nu} \partial_{\mu} g_{\lambda\kappa}
    + \partial_{\nu} \partial_{\lambda} g_{\mu\kappa}), \\
    R_{\lambda\mu\kappa\eta;\nu} &=
    \frac{1}{2} \partial_{\nu}(\partial_{\eta} \partial_{\mu} g_{\lambda\kappa}
    - \partial_{\nu} \partial_{\lambda} g_{\mu\kappa}
    - \partial_{\kappa} \partial_{\mu} g_{\lambda\eta}
    + \partial_{\kappa} \partial_{\lambda} g_{\mu\eta}), \\
    R_{\lambda\mu\eta\nu;\kappa} &= \frac{1}{2} \partial_{\kappa}(\partial_{\nu}
    \partial_{\mu} g_{\lambda\eta} - \partial_{\nu} \partial_{\lambda}
    g_{\mu\eta} - \partial_{\eta} \partial_{\mu} g_{\lambda\nu}
    + \partial_{\eta} \partial_{\lambda} g_{\mu\nu}).
  \end{align}
\end{subequations}
Sommando membro a membro queste tre relazioni si ottiene l'identità di Bianchi
\begin{equation}
  R_{\lambda\mu\nu\kappa;\eta} + R_{\lambda\mu\kappa\eta;\nu} +
  R_{\lambda\mu\eta\nu;\kappa} = 0.
\end{equation}

\section{Dimostrazione dell'identità di Palatini}
\label{sec:dimostr-palatini}

% vedi Weinberg, pagina 290
Se si fa variare la metrica $g_{\mu\nu} \to g_{\mu\nu} + \delta g_{\mu\nu}$, con
$\delta g_{\mu\nu}$ piccolo, la variazione del tensore di
Ricci~\eqref{eq:tens-ricci} al primo ordine in $\delta g_{\mu\nu}$ è data da
\begin{equation}
  \delta R_{\mu\nu}
  = \parder{\delta\tensor{\Gamma}{^{\lambda}_{\mu\lambda}}}{x^{\nu}}
  - \parder{\delta\tensor{\Gamma}{^{\lambda}_{\mu\nu}}}{x^{\lambda}} +
  \delta\tensor{\Gamma}{^{\eta}_{\mu\lambda}}
  \tensor{\Gamma}{^{\lambda}_{\eta\nu}} +
  \delta\tensor{\Gamma}{^{\lambda}_{\eta\nu}}
  \tensor{\Gamma}{^{\eta}_{\mu\lambda}} -
  \delta\tensor{\Gamma}{^{\eta}_{\mu\nu}}
  \tensor{\Gamma}{^{\lambda}_{\eta\lambda}} -
  \delta\tensor{\Gamma}{^{\lambda}_{\eta\lambda}}
  \tensor{\Gamma}{^{\eta}_{\mu\nu}},
\end{equation}
in cui $\delta\tensor{\Gamma}{^{\lambda}_{\mu\nu}}$ è la variazione della
connessione affine
\begin{equation}
  \delta\tensor{\Gamma}{^{\lambda}_{\mu\nu}} = -g^{\lambda\rho}\delta
  g_{\rho\sigma} \tensor{\Gamma}{^{\sigma}_{\mu\nu}} +
  \frac{1}{2}g^{\lambda\rho} \bigg(\parder{\delta g_{\rho\mu}}{x^{n}}
  + \parder{\delta g_{\rho\nu}}{x^{\mu}} - \parder{\delta
    g_{\mu\nu}}{x^{\rho}}\bigg).
\end{equation}
Le quantità $\delta \tensor{\Gamma}{^{\lambda}_{\mu\nu}}$ sono dei tensori,
essendo differenze fra due connessioni affini, quindi possiamo calcolare le loro
derivate covarianti (l'operazione di derivazione covarianti è calcolata usando
la connessione affine imperturbata $\tensor{\Gamma}{^{\lambda}_{\mu\nu}}$).  In
particolare risulta
\begin{subequations}
  \begin{align}
    (\delta \tensor{\Gamma}{^{\lambda}_{\mu\lambda}})_{;\nu} &= (\delta
    \tensor{\Gamma}{^{\lambda}_{\mu\lambda}})_{,\nu} +
    \delta\tensor{\Gamma}{^{\eta}_{\mu\lambda}}
    \tensor{\Gamma}{^{\lambda}_{\eta\nu}} -
    \delta\tensor{\Gamma}{^{\lambda}_{\eta\lambda}}
    \tensor{\Gamma}{^{\eta}_{\mu\nu}} -
    \delta\tensor{\Gamma}{^{\lambda}_{\mu\eta}}
    \tensor{\Gamma}{}^{\eta}_{\nu\lambda},  \\
    (\delta \tensor{\Gamma}{^{\lambda}_{\mu\nu}})_{;\lambda} &= (\delta
    \tensor{\Gamma}{^{\lambda}_{\mu\nu}})_{,\lambda} +
    \delta\tensor{\Gamma}{^{\eta}_{\mu\nu}}
    \tensor{\Gamma}{^{\lambda}_{\eta\lambda}} -
    \delta\tensor{\Gamma}{^{\lambda}_{\eta\nu}}
    \tensor{\Gamma}{^{\eta}_{\mu\lambda}} -
    \delta\tensor{\Gamma}{^{\lambda}_{\mu\eta}}
    \tensor{\Gamma}{}^{\eta}_{\nu\lambda}.
  \end{align}
\end{subequations}
È immediato verificare che la variazione del tensore di Ricci
$\delta R_{\mu\nu}$ è data dalla differenza di queste due quantità
\begin{equation}
  \delta R_{\mu\nu} = (\delta \tensor{\Gamma}{^{\lambda}_{\mu\lambda}})_{;\nu}
  - (\delta \tensor{\Gamma}{^{\lambda}_{\mu\nu}})_{;\lambda}.
\end{equation}
Questa è l'\index{identità!di Palatini}\emph{identità di Palatini}.  È stata
ottenuta con la solita convenzione dei segni $(+--)$
di~\textcite{weinberg:gravitation}.  Se si invece adotta la convenzione $(+--)$,
come \textcites{barone:relativita,landau:campi}, si ottiene
\begin{equation}
  \delta R_{\mu\nu} = (\delta \tensor{\Gamma}{^{\lambda}_{\mu\nu}})_{;\lambda} -
  (\delta \tensor{\Gamma}{^{\lambda}_{\mu\lambda}})_{;\nu}.
\end{equation}

%%% Local Variables:
%%% mode: latex
%%% TeX-master: "../gravitazione"
%%% fill-column: 80
%%% TeX-PDF-mode: t
%%% End:
