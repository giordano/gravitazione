% Non l'abbiamo fatto a lezione, però introdurre brevemente lo spin serve per la
% precessione di De Sitter e mi sembra che Weinberg lo spieghi abbastanza bene.
\chapter{Spin}
\label{cha:spin}

Il momento angolare totale di un sistema è dato dalla somma del
\emph{momento angolare orbitale}, dovuto al suo moto attorno a un asse fisso, e
un momento angolare intrinseco, detto \emph{spin}.  Naturalmente non tutti i
sistemi hanno necessariamente sia un momento angolare orbitale sia uno spin.
Nel sistema di riferimento del centro di massa, il sistema è a riposo e il
momento angolare orbitale è nullo ma è presente sempre lo stesso spin, da qui
deriva l'aggettivo ``intrinseco''.  Per esempio, la Terra possiede un momento
angolare orbitale dovuto al moto di rivoluzione del pianeta intorno al Sole, e
uno spin, dovuto al moto di rotazione attorno a un proprio asse.  Un altro
esempio di corpo dotato di spin è un giroscopio che orbita intorno alla Terra.
Usando il tensore energia-impulso $T^{\alpha\beta}$, introdotto nel
paragrafo~\ref{sec:tensore-energia-impulso}, possiamo definire in maniera
precisa queste quantità nella relatività speciale.

Definiamo il quadritensore
\begin{equation}
  M^{\gamma\alpha\beta} = x^{\alpha}T^{\beta\gamma} - x^{\beta}T^{\alpha\gamma}.
\end{equation}
Questo tensore è antisimmetrico per lo scambio degli ultimi due indici:
$M^{\gamma\alpha\beta} = - M^{\gamma\beta\alpha}$.  Il tensore energia-impulso è
simmetrico e inoltre soddisfa la relazione di conservazione
$\lparder{T^{\alpha\beta}}{x^{\beta}} = 0$, per questi motivi anche
$M^{\gamma\alpha\beta}$ si conserva
\begin{equation}
  \parder{M^{\gamma\alpha\beta}}{x^{\gamma}} = \parder{x^{\alpha}}{x^{\gamma}}
  T^{\beta\gamma} - \parder{x^{\beta}}{x^{\gamma}} T^{\alpha\gamma} =
  T^{\beta\alpha} - T^{\alpha\beta} = 0.
\end{equation}
Allora per il teorema di Gauss (vedi l'appendice~\ref{cha:teorema-gauss})
possiamo definire il seguente quadritensore $J^{\alpha\beta}$ di rango $2$ e
costante nel tempo
\begin{equation}
  J^{\alpha\beta} = \int M^{0\alpha\beta} \dd^{3} \bm{x} = -\int
  M^{0\beta\alpha} \dd^{3} \bm{x} = -J^{\beta\alpha}.
\end{equation}
Questo tensore prende il nome di \index{momento!angolare totale}
\emph{momento angolare totale}.  Possiamo giustificare questo nome scrivendo
esplicitamente le componenti spaziali
\begin{equation}
  J^{ij} = \int (x^{i}T^{j0} - x^{j}T^{i0})\dd^{3} \bm{x}.
\end{equation}
Abbiamo interpretato le componenti $T^{i0}$ del tensore energia-impulso come la
densità della componente $i$ della quantità di moto, dunque riconosciamo che
$\bm{L} = (J^{23}, J^{31}, J^{12}) = \bm{r} \times \bm{p}$ è il vettore
tridimensionale momento angolare.  Per le altre componenti non nulle
\begin{equation}
  J^{0i} = tp^{i} - \int x^{i}T^{00} \dd^{3} \bm{x}
\end{equation}
non esiste un'interpretazione fisica in termini di variabili dinamiche.  Queste
componenti possono peraltro essere annullate ponendo
$\int x^{i}T^{00} \dd^{3} \bm{x} = 0$ al tempo $t = 0$.

Il tensore momento angolare totale $J^{\alpha\beta}$ si comporta come un normale
tensore di rango $2$ rispetto alle trasformazioni di Lorentz omogenee
$x^{\alpha} \to x'^{\alpha} = \tensor{\Lambda}{^{\alpha}_{\beta}} x^{\beta}$, ma
per una traslazione $x^{\alpha} \to x'^{\alpha} = x^{\alpha} + a^{\alpha}$
abbiamo
\begin{equation}
  J^{\alpha\beta} \to J'^{\alpha\beta} = J^{\alpha\beta} + a^{\alpha}p^{\beta} -
  a^{\beta}p^{\alpha}.
\end{equation}
Ciò è dovuto al fatto che $J^{\alpha\beta}$ comprende anche il momento angolare
orbitale, definito in relazione a un asse di rotazione.  Per individuare la
parte ``intrinseca'' del momento angolare totale $J^{\alpha\beta}$, cioè una
parte che non cambi per effetto di una traslazione, definiamo il quadrivettore
di \index{spin}\index{momento!angolare intrinseco|see{spin}}\emph{spin}
\begin{equation}
  S_{\alpha} = \frac{1}{2}\epsilon_{\alpha\beta\gamma\delta} J^{\beta\gamma} U^{\delta},
\end{equation}
in cui $U^{\alpha}$ è il quadrivettore velocità del sistema.  Grazie a questa
definizione, il quadrivettore di spin non cambia per effetto di una traslazione.
È intuitivo capire che per una particella libera il quadrivettore di spin è
costante
\begin{equation}
  \toder{S_{\alpha}}{t} = 0.
\end{equation}
Nel sistema di riferimento del centro di massa del sistema
$U'^{\alpha} = (1, \bm{0})$, quindi
\begin{subequations}
  \begin{align}
    S'_{0} &= \epsilon_{0\beta\gamma\delta} J^{\alpha\beta} U'^{\gamma} =
    \epsilon_{0\beta\gamma 0} J^{\alpha\beta} = 0, \\
    S'_{i} &= \frac{1}{2}\epsilon_{ijk} J^{jk} = \int \epsilon_{ijk} x^{j}T^{0k}
    \dd^{3} \bm{x},
  \end{align}
\end{subequations}
cioè $S'_{\alpha} = (0, J^{23}, J^{31}, J^{12})$.  Nel sistema di riferimento
del centro di massa il momento angolare orbitale è nullo, pertanto è corretta
l'identificazione del quadrivettore di spin con il momento angolare intrinseco.
Anche quando la velocità del sistema non è nulla, $(U^{i}) \neq \bm{0}$, il
quadrivettore di spin $S_{\alpha}$ ha solo tre componenti indipendenti perché
il quadrivettore di spin e la quadrivelocità sono sempre ortogonali
\begin{equation}
  S_{\alpha}U^{\alpha} = \frac{1}{2}
  \epsilon_{\alpha\beta\gamma\delta}J^{\beta\gamma}U^{\delta}U^{\alpha} =
  -\frac{1}{2}
  \epsilon_{\delta\beta\gamma\alpha}J^{\beta\gamma}U^{\delta}U^{\alpha} = 0.
\end{equation}


%%% Local Variables:
%%% mode: latex
%%% TeX-master: "../gravitazione"
%%% End:
