\documentclass[a4paper,twoside,openright,fleqn,11pt]{book}
\usepackage[T1]{fontenc}
\usepackage[light,slantedGreeks]{kpfonts}
\usepackage[utf8]{inputenc}
\usepackage[italian]{babel}

%%%%% Pacchetti caricati
\usepackage{layaureo}
\usepackage[autostyle=true]{csquotes}
\usepackage[style=numeric,hyperref,abbreviate=false,backend=biber]{biblatex}
\usepackage{fancyhdr,emptypage}
\usepackage{tensor} % per rappresentare i tensori
\usepackage[font=small,format=hang,labelfont=bf,textformat=period]{caption}
% per tabelle e figure
\usepackage{booktabs,longtable,graphicx,subfig}
\usepackage{tikz,tikz-3dplot}
% `mathtools' serve per definire la norma e il valore assoluto
\usepackage{mathtools}
% pacchetti matematici dell'American Mathematical Society
\usepackage{amsthm}
% `bm' serve per scrivere i vettori in corsivo con il comando
% `\bm{vettore}'. Deve sostituire il comando `\mathbf{vettore}' perché questo
% restituisce erroneamente lettere in tondo, non in corsivo e non funziona con
% le lettere greche.
\usepackage{bm}
\usepackage{makeidx}
\usepackage{siunitx}
\usepackage{hyperref}
%%%%% Fine pacchetti

%%%%% Impostazioni
%% testatine.
\pagestyle{fancy}
\fancyhf{} % svuota tutti i campi delle testatine e dei piedi
\setlength{\headheight}{14pt}
% Nella testatina (`head') a destra (`R') nelle pagine pari (`E') mettiamo il
% nome del capitolo (`\leftmark'), non tutto maiuscolo (`\nouppercase'). Sulla
% sinistra (`L') delle pagine dispari (`O') mettiamo il nome del paragrafo
% (`\rightmark').
\fancyhead[RE]{\slshape\footnotesize\nouppercase{\leftmark}}
\fancyhead[LO]{\slshape\footnotesize\nouppercase{\rightmark}}
% All'esterno di ogni pagina (cioè a sinistra (`L') nelle pagine pari (`E') e a
% destra (`R') nelle pagine dispari (`O')), nella testatina (`head') mettiamo il
% numero della pagina (`\thepage').
\fancyhead[LE,RO]{\thepage}

% impostazioni per il pacchetto `hyperref'. Per l'elenco di tutte le opzioni del
% documento consulta il manuale: `texdoc hyperref'
\hypersetup{
  pdftitle={Gravitazione e cosmologia},
  pdfauthor={Mosè Giordano},
  breaklinks=true,% permette di spezzare i link su più righe
  bookmarksnumbered,% inserisce i numeri delle sezioni nei segnalibri
  hidelinks % link neri e senza bordi colorati, adatto per la stampa
}

\sisetup{per-mode=symbol,
  inter-unit-product=\ensuremath{{}\cdot{}},
  exponent-product=\cdot,
  output-product=\cdot,
  separate-uncertainty=true,
  math-micro=\muup,
  text-micro=\ensuremath{\muup},
}

% Posiziona le didascalie sopra le tabelle
\captionsetup[table]{position=top}

\addbibresource{bibliografia.bib} % nome del file contenente la bibliografia
\defbibheading{bibliography}[\refname]{\chapter*{#1}\markboth{#1}{#1}}

\theoremstyle{plain}
\newtheorem*{teorema}{Teorema}

% librerie ti TikZ
\usetikzlibrary{calc}
%%%%% Fine impostazioni

%%%%% Comandi personalizzati
% ridefinisco i comandi per alcune lettere greche in modo che si usino le
% varianti
\renewcommand{\phi}{\varphi}
\renewcommand{\epsilon}{\varepsilon}

% Comando per indicare la variazione di una quantità.  Deve produrre una delta
% maiuscola diritta, visto che carico il pacchetto `kpfonts' con l'opzione
% `slantedGreeks' uso la macro `\Deltaup'.
\newcommand{\var}{\Deltaup}

% comando per evidenziare le sezioni da completare o espandere.  NOTA: è un
% comando provvisorio, serve solo durante la scrittura del testo, ricordarsi di
% eliminarlo quando non serve più.
\newcommand{\completare}[1]{\textcolor{red}{#1}}

% Operatori
\newcommand*{\dd}{\mathop{}\!\mathrm{d}} % Operatore differenziale \dd
\newcommand*{\cdd}{\mathop{}\!\mathrm{D}} % Differenziale covariante \cdd
\DeclareMathOperator{\e}{\mathrm{e}} % Numero di Eulero
\DeclareMathOperator{\diag}{diag}
\DeclareMathOperator{\uimm}{\mathrm{i}} % unità immaginaria
% Operatore valore assoluto \abs{x}. Usa \abs*{} per le frazioni
\DeclarePairedDelimiter{\abs}{\lvert}{\rvert}
% Operatore norma \norm{x}. Usa \norm*{} per le frazioni
\DeclarePairedDelimiter{\norm}{\lVert}{\rVert}

%% Derivate
% Derivata totale: \toder[ordine]{funzione}{variabile}
\newcommand*{\toder}[3][]{\frac{{\dd^{#1}}#2}{\dd {#3}^{#1}}}
% Derivata covariante: \covder[ordine]{tensore}{variabile}
\newcommand*{\covder}[3][]{\frac{{\cdd^{#1}}#2}{\dd {#3}^{#1}}}
% Derivata lungo una curva: \covder[ordine]{tensore}
\newcommand*{\curder}[2][]{\frac{{\cdd^{#1}}#2}{\cdd \tau^{#1}}}
% operatore d'alambertiano.  Definisco un comando specifico nel caso in cui si
% decidesse di cambiare la notazione (per esempio \Box invece di \Box^{2}).
\newcommand{\dalamb}{\Box^{2}}


% Derivata parziale \parder[ordine]{funzione}{variabile}
% Per la definizione del comando `parder' (per inserire le derivate parziali)
% vedi
% http://www.guitex.org/home/index.php?option=com_kunena&func=view&catid=5&id=42178&Itemid=60#42199
\makeatletter
\newcommand{\parder}[2]{\begingroup
  \@tempswafalse\toks@={}\count@=\z@
  \@for\next:=#2\do
    {\expandafter\check@var\next\@nil
     \advance\count@\parder@exp
     \if@tempswa
       \toks@=\expandafter{\the\toks@\,}%
     \else
       \@tempswatrue
     \fi
     \toks@=\expandafter{\the\expandafter\toks@\expandafter\partial\parder@var}}%
  \frac{\partial\ifnum\count@=\@ne\else^{\number\count@}\fi#1}{\the\toks@}%
  \endgroup}
\def\check@var{\@ifstar{\mult@var}{\one@var}}
\def\mult@var#1#2\@nil{\def\parder@var{#2^{#1}}\def\parder@exp{#1}}
\def\one@var#1\@nil{\def\parder@var{#1}\chardef\parder@exp\@ne}
\makeatother

% Derivate per le formule in linea (usare \frac in linea è eccessivo). La `l'
% iniziale nel nome distingue questi comandi da quelli per le formule fuori
% corpo. Non uso `\dd' ma `\mathrm{d}' perché nelle formule in linea `\dd'
% aggiunge una spaziatura non adatta. Non sono dei comandi bellissimi, ma
% permettono di passare facilmente da formula in linea a fuori corpo e viceversa
% cambiando una lettera.
% Derivata totale: \ltoder[ordine]{funzione}{variabile}
\newcommand*{\ltoder}[3][]{\mathrm{d}^{#1}#2 / \mathrm{d} {#3}^{#1}}
% Derivata parziale: \lparder[ordine]{funzione}{variabile}
\newcommand*{\lparder}[3][]{\partial^{#1} #2 / \partial {#3}^{#1}}
% NOTA: `\parder' e `\lparder' non sono completamente interscambiabili, il primo
% comando è molto più complesso e permette di inserire le derivate miste, a
% differenza del secondo.

% Versore. Esempi: versore x: `\versore{x}', versore i: \versor{\imath}, versore
% j: \versor{\jmath} (solo `i' e `j' richiedono `\imath' e `\jmath', altrimenti
% il puntino litiga con `\hat')
\newcommand*{\versor}[1]{\hat{\bm{#1}}}

% Ambiente per scrivere sistemi di equazioni.
% Vedi `L'arte di scrivere con LaTeX' di Pantieri.
% Esempio d'uso (in ambiente matematico):
%	\begin{sistema}
%         x+y+z=0 \\
%         2x-y=1 \\
%         y-4z=-3
%       \end{sistema}
\newenvironment{sistema}%
{\left\lbrace\begin{array}{@{}l@{}}}%
    {\end{array}\right.}

%% Campi notevoli
\let\numberset\mathbb
\newcommand{\N}{\numberset{N}}	% Naturali \N
\newcommand{\Z}{\numberset{Z}}	% Interi \Z
\newcommand{\R}{\numberset{R}}	% Reali \R
\newcommand{\C}{\numberset{C}}	% Complessi \C
\newcommand{\M}{\numberset{M}}	% spazio-tempo di Minkowski

% unità di misura aggiuntive (non-SI)
\DeclareSIUnit\dyne{dyn} % dyne
\DeclareSIUnit\erg{erg} % erg
\DeclareSIUnit\tonne{t} % tonnellata ("tonne" in inglese)
\DeclareSIUnit\parsec{pc} % parsec
\DeclareSIUnit\lightyear{ly} % anno luce
\DeclareSIUnit\radian{rad} % radiante
\DeclareSIUnit\solarmass{\ensuremath{M_{\odot}}} % massa solare
%%%%% Fine comandi personalizzati

\makeindex{}

% titolo provvisorio per il frontespizio provvisorio
\title{Gravitazione e cosmologia}
\date{}
\begin{document}
\frontmatter{}
\maketitle

\cleardoublepage{}
\tableofcontents{} % indice generale
\cleardoublepage{}
\listoffigures{} % elenco delle figure

\cleardoublepage
\chapter{Convenzioni e notazioni usate}
\label{cha:notazioni}

Nella teoria della relatività generale sono diffuse diverse notazioni e
convenzioni sui segni e i fattori numerici che compaiono in alcune formule.
Nella tabella~\ref{tab:notazione} è riportata la notazione adottata negli
appunti.  Gran parte dei presenti appunti si basano
su~\textcite{weinberg:gravitation}, quindi, se non diversamente specificato,
adotteremo le stesse convenzioni di questo testo.  Riportiamo di seguito in
dettagli le convenzioni scelte.

Con le lettere greche ($\alpha,\beta,\dots,\mu,\nu,\dots$) rappresenteremo
indici che possono assumere i valori $0$, $1$, $2$ e $3$, mentre con le lettere
latine ($a,b,\dots,i,j,\dots$) indici che assumono i valori $1$, $2$, $3$.  Si
noti che \textcite{landau:campo} seguono la convenzione opposta: gli indici
latini assumono i valori $0$, $1$, $2$ e $3$, gli indici greci $1$, $2$ e $3$.

Dove non chiaramente dichiarato il contrario, adottiamo, come è di consuetudine
all'interno della relatività, la \index{convenzione!di Einstein per la somma}
\emph{convenzione di Einstein per la somma}: se in un termine un indice compare
due volte, di cui una volta controvariante (sopra) e una volta covariante
(sotto), esso è chiamato \index{indice!muto}\emph{muto} (o \emph{contratto}), si
sottintende una somma sui possibili valori che l'indice può assumere e si omette
il simbolo di sommatoria.  Gli indici non muti sono chiamati
\index{indice!libero}\emph{liberi}.  In un'identità fra tensori gli indici
liberi devono comparire in ugual numero e nelle stesse posizioni (superiormente
o inferiormente) in ambo i membri.  Sebbene possa succedere che, casualmente, in
un certo sistema di riferimento le componenti controvarianti di un tensore $A$
siano uguali alle componenti covarianti di un altro tensore $B$
\begin{equation}
  \tensor{A}{^{\alpha}_{\beta}^{\gamma}} = B^{\alpha\beta\gamma},
\end{equation}
questo non è valido in un altro sistema di riferimento e l'equazione precedente
non è, quindi, legittima.  Si noti inoltre che questa non è corretta dal punto
di vista tensoriale poiché l'indice libero $\beta$ è in un membro controvariante
e nell'altro covariante.  L'espressione
$A^{\mu\sigma}B_{\mu} = A^{0\sigma}B_{0} + A^{i\sigma}B_{i}$, scritta
utilizzando la convenzione di Einstein, equivale, esplicitando la sommatoria, a
\begin{equation}
  \sum_{\mu=0}^{3} A^{\mu\sigma}B_{\mu} = A^{0\sigma}B_{0} + \sum_{i=1}^{3}
  A^{i\sigma}B_{i} = A^{0\sigma}B_{0} + A^{1\sigma}B_{1} + A^{2\sigma}B_{2} +
  A^{3\sigma}B_{3},
\end{equation}
in cui gli indici $\mu$ e $i$ sono muti e $\sigma$ è libero.  Gli indici muti
sono così denominati poiché non compaiono nell'espressione esplicita.  Gli
indici muti possono essere rinominati (e spesso eseguiremo questa operazione
senza dichiararla esplicitamente) a patto di non utilizzare indici muti già
utilizzati in quel termine oppure indici liberi.  Quindi l'espressione
$\tensor{T}{^{\alpha\beta\gamma}_{\alpha\gamma}}$ può essere anche scritta come
$\tensor{T}{^{\mu\beta\nu}_{\mu\nu}}$ ma \emph{non} come
$\tensor{T}{^{\beta\beta\gamma}_{\beta\gamma}}$ oppure
$\tensor{T}{^{\alpha\beta\alpha}_{\alpha\alpha}}$.

La velocità della luce nel vuoto $c = \SI{299792458}{\metre\per\second}$ è posta
uguale a $1$.  Di conseguenza tutte le velocità sono adimensionali (in
particolare sono espresse in unità di $c$), lunghezze e tempo hanno la stessa
dimensione.  Per ristabilire la coerenza dimensionale delle equazioni bisogna
utilizzare l'adeguato numero di potenze di $c$.

Il tensore metrico di Minkowski $\eta_{\alpha\beta}$ ha, per noi, segnatura
$-+++$.  I testi \textcites{barone:relativita,landau:campo,ohanian:gravitazione}
adottano la segnatura opposta $+---$.

Il tensore di Riemann è definito come
\begin{equation}
    \tensor{R}{^{\lambda}_{\mu\nu\kappa}}
  = \parder{\tensor{\Gamma}{^{\lambda}_{\mu\nu}}}{x^{\kappa}}
  - \parder{\tensor{\Gamma}{^{\lambda}_{\mu\kappa}}}{x^{\nu}} +
  \tensor{\Gamma}{^{\eta}_{\mu\nu}} \tensor{\Gamma}{^{\lambda}_{\eta\kappa}} -
  \tensor{\Gamma}{^{\eta}_{\mu\kappa}} \tensor{\Gamma}{^{\lambda}_{\eta\nu}}.
\end{equation}
I testi \textcites{barone:relativita,landau:campo,ohanian:gravitazione}
definiscono il tensore di Riemann con il segno opposto rispetto a quello qui
stabilito.

Il tensore di Ricci $R_{\mu\kappa}$ è ottenuto contraendo il primo e il terzo
indice del tensore di Riemann $\tensor{R}{^{\lambda}_{\mu\nu\kappa}}$, o
equivalentemente il secondo e il quarto indice.  Al contrario,
\textcite{ohanian:gravitazione} definiscono il tensore di Ricci contraendo primo
e quarto indice $R_{\mu\kappa} = \tensor{R}{^{\lambda}_{\mu\kappa\lambda}}$ e si
ottiene così un tensore opposto rispetto a quello da noi definito.

\begin{longtable}{c
    p{.37\textwidth}
    p{.28\textwidth}}
  % intestazione iniziale
  \caption{Notazione usata in questi appunti}
  \label{tab:notazione} \\
  \toprule
  Simbolo & Significato & Riferimento \\
  \midrule
  \endfirsthead
  % intestazione normale
  \multicolumn{3}{l}{\footnotesize\itshape
    Continua dalla pagina precedente} \\
  \toprule
  Simbolo & Significato & Riferimento \\
  \midrule
  \endhead
  % piede normale
  \midrule \multicolumn{3}{r}{\footnotesize\itshape
    Continua nella prossima pagina} \\
  \endfoot
  % piede finale
  \bottomrule \multicolumn{3}{r}{\footnotesize\itshape
    Si conclude dalla pagina precedente} \\
  \endlastfoot
  % inizio corpo della tabella
  $\bm{A} = (A^{1}, A^{2}, A^{3})$ & Vettore tridimensionale di componenti
  $A^{1}$, $A^2$ e $A^{3}$. & \\
  $A^{\mu} = (A^{0}, \bm{A})$ & Quadrivettore controvariante di componente
  temporale $A^{0}$ e componenti spaziali $\bm{A}$. & Vedi
  paragrafo~\ref{sec:calcolo-tensoriale-minkowski} e
  capitolo~\ref{cha:calcolo-tensoriale}. \\
  $A_{\mu} = (A_{0}, \bm{A})$ & Quadrivettore covariante con componente
  temporale $A_{0}$ e componente spaziale $\bm{A}$. & Vedi
  paragrafo~\ref{sec:calcolo-tensoriale-minkowski} e
  capitolo~\ref{cha:calcolo-tensoriale}. \\
  $\tensor*{T}{^{\alpha_{1}\dots\alpha_{p}}_{\beta_{1}\dots\beta_{q}}}$ &
  Quadritensore. & Vedi paragrafo~\ref{sec:calcolo-tensoriale-minkowski} e
  capitolo~\ref{cha:calcolo-tensoriale}. \\
  $\dd \tau^{2} = \dd t^{2} - \dd \bm{x}^{2}$ & Metrica nello spazio di
  Minkowski. & Vedi paragrafo~\ref{sec:trasformazioni-lorentz}. \\
  $\eta_{\alpha\beta}$ & Tensore metrico di Minkowski. & Vedi
  paragrafo~\ref{sec:trasformazioni-lorentz}. \\
  $\tensor{\delta}{^{\alpha}_{\beta}}$ & Delta di Kronecker quadrimensionale. &
  Vedi paragrafo~\ref{sec:trasformazioni-lorentz}. \\
  $\tensor{\Lambda}{^{\alpha}_{\beta}}$ & Componente della matrice di Lorentz. &
  Vedi paragrafo~\ref{sec:trasformazioni-lorentz}. \\
  $\gamma(v) = \dfrac{1}{\sqrt{1-v^{2}}}$ & Fattore di Lorentz associato al
  boost $v$. & Vedi paragrafo~\ref{sec:trasformazioni-lorentz}. \\
  $\partial_{\mu} A_{\nu} = A_{\nu,\mu} =
  \displaystyle\parder{A_{\nu}}{x^{\mu}}$
  & Derivata ordinaria di $A_{\nu}$ rispetto alla coordinata $x^{\mu}$. & Vedi
  paragrafo~\ref{sec:derivazione-minkowski}. \\
  $\partial^{\mu} A_{\nu} = \tensor{A}{_{\nu}^{,\mu}} =
  \displaystyle\parder{A_{\nu}}{x_{\mu}}$
  & Derivata ordinaria di $A_{\nu}$ rispetto alla coordinata $x_{\mu}$. & Vedi
  paragrafo~\ref{sec:derivazione-minkowski}. \\
  $ \Box^{2} = \partial^{\alpha}\partial_{\alpha}$ & Operatore d'Alambertiano. &
  Vedi paragrafo~\ref{sec:derivazione-minkowski}. \\
  $\epsilon^{\alpha\beta\gamma\delta}$ & Tensore di Levi-Civita & Vedi
  paragrafo~\ref{sec:tensore-levi-civita}. \\
  $\dd^{4} x = \dd x^{0}\dd x^{1}\dd x^{2}\dd x^{3}$ & Elemento infinitesimo di
  volume quadrimensionale. & \\
  $\dd \Sigma_{\mu}$ & Elemento infinitesimo di ipersuperficie
  tridimensionale. & Vedi appendice~\ref{cha:teorema-gauss}. \\
  $\dd^{3} \bm{x} = \dd V$ & Elemento infinitesimo di volume tridimensionale. &
  \\
  $F^{\alpha\beta}$ & Tensore del campo elettromagnetico. & Vedi
  paragrafo~\ref{sec:equazioni-maxwell}. \\
  $S = \displaystyle\int \Lambda \dd^{4} x$ & Azione associata alla densità di
  lagrangiana $\Lambda$. & Vedi paragrafo~\ref{sec:tensore-energia-impulso}. \\
  $\tensor{T}{^{\beta}_{\alpha}}$ & Tensore energia-impulso. & Vedi
  paragrafo~\ref{sec:tensore-energia-impulso}. \\[1.6ex]
  $\tensor{\Gamma}{^{\lambda}_{\mu\nu}}$ & Connessione affine. & Vedi
  paragrafi~\ref{sec:equazione-moto}, \ref{sec:relazione-g-Gamma} e
  \ref{sec:derivazione-covariante}. \\
  $g_{\mu\nu}$ & Tensore metrico nella relatività generale. & Vedi
  paragrafi~\ref{sec:equazione-moto}, \ref{sec:relazione-g-Gamma} e
  \ref{sec:tensore-metrico}. \\
  $r_{\textup{G}} = r_{\textup{S}}$ & Raggio di Schwarzschild. & Vedi
  paragrafo~\ref{sec:limite-newtoniano}. \\[1.6ex]
  $\sqrt{g} = \sqrt{-\det(g_{\mu\nu})}$ & Radice quadrata dell'opposto del
  determinante del tensore metrico. & Vedi
  paragrafo~\ref{sec:tensore-metrico}. \\
  $V_{\mu;\nu} = \displaystyle\covder{V_{\mu}}{x^{\nu}}$ & Derivata covariante
  di $V_{\mu}$ rispetto alla coordinata $x^{\nu}$. & Vedi
  paragrafo~\ref{sec:derivazione-covariante}. \\
  $\cdd V_{\mu}$ & Differenziale covariante di $V_{\mu}$. & Vedi
  paragrafo~\ref{sec:differenziale-covariante}. \\[1.6ex]
  $\displaystyle\curder{V_{\mu}}$ & Derivata di $V_{\mu}$ lungo una curva
  parametrizzata da $\tau$. & Vedi
  paragrafo~\ref{sec:differenziale-covariante}. \\
  $\tensor{R}{^{\lambda}_{\mu\nu\kappa}}$ & Tensore di curvatura di
  Riemann-Christoffel. & Vedi paragrafo~\ref{sec:tensore-riemann}. \\
  $R_{\mu\kappa} = \tensor{R}{^{\lambda}_{\mu\lambda\kappa}}$ & Tensore di
  Ricci. & Vedi paragrafo~\ref{sec:tensore-riemann}. \\[1.6ex]
  $R = \tensor{R}{^{\mu}_{\mu}}$ & Curvatura scalare & Vedi
  paragrafo~\ref{sec:tensore-riemann}. \\[1.6ex]
  $G^{\mu\nu} = R^{\mu\nu} - \dfrac{1}{2} g^{\mu\nu}R$ & Tensore di Einstein &
  Vedi paragrafo~\ref{sec:identita-bianchi}. \\
\end{longtable}


%%% Local Variables:
%%% mode: latex
%%% TeX-master: "../astrofisica-teorica"
%%% fill-column: 80
%%% End:


\cleardoublepage{}
\mainmatter{}
\chapter{Richiami di relatività speciale}
\label{cha:relativita-speciale}

\section{Trasformazioni di Lorentz}
\label{sec:trasformazioni-lorentz}

\index{coordinata controvariante}Coordinata controvariante
\begin{equation}
  \label{eq:coord-controvariante}
  x^{\alpha} = (x^{0},x^{i}) = (t, \bm{x}),
\end{equation}
con $\alpha$ che varia fra $0$, $1$, $2$ e $3$ e $i$ fra $1$, $2$ e $3$.

Nello spazio ordinario si defisce
\begin{equation}
  \dd l^{2}   = \dd x^{2} + \dd y^{2} + \dd z^{2} = \sum_{i} (\dd x^{i})^{2} =
  \delta_{ij}\dd x^{i} \dd x^{j},
\end{equation}
con
\begin{equation}
  \delta_{ij} =
  \begin{pmatrix}
    1 & 0 & 0 \\
    0 & 1 & 0 \\
    0 & 0 & 1
  \end{pmatrix}.
\end{equation}
Il trasformato della coordinata controvariante~\eqref{eq:coord-controvariante} è
\begin{equation}
  x'^{i} = \tensor{R}{^{i}_{j}}x^{j} + a^{i}
\end{equation}
e la matrice $R$ è univocamente definita imponendo la condizione che $\dd l^{2}$
sia invariante cambiando sistema di riferimento.  Si scopre che $R$ è la matrice
di rotazione.

Nello spazio di Minkowski definiamo
\begin{equation}
  \dd \tau^{2} = \dd t^{2} - \dd \bm{x}^{2} = -\tensor{\eta}{_{\alpha\beta}}\dd
  x^{\alpha} \dd x^{\beta},
\end{equation}
in cui $\tensor{\eta}{_{\alpha\beta}}$ è il
\index{tensore!metrico covariante di
  Minkowski}\emph{tensore metrico covariante di Minkowski}
\begin{equation}
  \label{eq:tensore-metrico-controvariante}
  \tensor{\eta}{_{\alpha\beta}} =
  \begin{pmatrix}
   \eta_{00} & \eta_{01} & \eta_{02} & \eta_{03} \\
   \eta_{10} & \eta_{11} & \eta_{12} & \eta_{13} \\
   \eta_{20} & \eta_{21} & \eta_{22} & \eta_{23} \\
   \eta_{30} & \eta_{31} & \eta_{32} & \eta_{33}
  \end{pmatrix}
  =
  \begin{pmatrix}
    -1 & 0 & 0 & 0 \\
    0 & 1 & 0 & 0 \\
    0 & 0 & 1 & 0 \\
    0 & 0 & 0 & 1
  \end{pmatrix}.
\end{equation}
Questo tensore definisce una metrica pseudo euclidea con segnatura $(-+++)$.
Imponendo che $\dd \tau$ sia invariante per trasformazioni di Lorentz possiamo
passare da un sistema di riferimento inerziale $O$ a un altro sistema di
riferimento inerziale $O'$ in moto rispetto a $O$.  Consideriamo due eventi
$E_{1}$ ed $E_{2}$ relativi a un fotone
\begin{gather}
  E_{1}: \quad (0,\bm{0}), \\
  E_{2}: \quad (\dd t, \dd \bm{x})
\end{gather}
\begin{equation}
  \dd \tau^{2} = \dd t^{2}\Big( 1 - \toder{\bm{x}^{2}}{t^{2}}\Big) = \dd t^{2}
  (1 - 1^{2}) = 0
\end{equation}
perché per un fotone si ha
\begin{equation}
  \bm{v}^{2} = \Big(\toder{\bm{x}}{t}\Big)^{2} = c^{2} = 1.
\end{equation}


\begin{equation}
  \dd \tau'^{2} = -\tensor{\eta}{_{\alpha\beta}}\dd x'^{\alpha}\dd x'^{\beta}
\end{equation}
Richiediamo che la trasformazione $x^{\alpha} \to x'^{\alpha}$ sia lineare
per rispettare la condizione di omogeneità dello spazio (la lunghezza di una
sbarra non deve dipendere dalla posizione degli estremi), del tempo (la
lunghezza della sbarra non deve dipendere dall'istante di tempo in cui la
misuriamo) e di isotropia dello spazio (la lunghezza della sbarra non deve
dipendere neanche dalla direzione in cui la guardiamo).  La più generica
trasformazione lineare è
\begin{equation}
  \label{eq:trasformazione-lineare}
  x'^{\alpha} = \tensor{\Lambda}{^{\alpha}_{\beta}}x^{\beta} + a^{\alpha}.
\end{equation}
Imponiamo che l'intervallo $\dd \tau^{2}$ non vari passando da un sistema di
riferimento all'altro, cioè che $\dd \tau^{2} = \dd \tau'^{2}$, così abbiamo
\begin{equation}
  \dd \tau'^{2} = -\tensor{\eta}{_{\alpha\beta}}
  \tensor{\Lambda}{^{\alpha}_{\gamma}} \dd x^{\gamma}
  \tensor{\Lambda}{^{\beta}_{\delta}} \dd x^{\delta} = \dd \tau^{2} =
  -\tensor{\eta}{_{\alpha\beta}} \dd x^{\alpha} \dd x^{\beta},
\end{equation}
da cui, semplificando negli ultimi due membri $\dd x^{\gamma}\dd x^{\delta}$,
otteniamo la relazione fondamentale del tensore metrico
\begin{equation}
  \label{eq:rel-fond-tensore-metrico}
  \tensor{\eta}{_{\alpha\beta}} \tensor{\Lambda}{^{\alpha}_{\gamma}}
  \tensor{\Lambda}{^{\beta}_{\delta}} = \tensor{\eta}{_{\gamma\delta}}.
\end{equation}
Questa è la condizione che deve essere soddisfatta da una
trasformazione~\eqref{eq:trasformazione-lineare} affinché sia una trasformazione
di Lorentz.  L'equazione~\eqref{eq:rel-fond-tensore-metrico} impone $10$
condizioni indipendenti in quanto il tensore metrico
$\tensor{\eta}{_{\gamma\delta}}$ è simmetrico quindi solo $10$ delle sue $16$
componenti sono indipendenti.

L'equazione matriciale equivalente alla~\eqref{eq:rel-fond-tensore-metrico} è
\begin{equation}
  \Lambda^{\textup{T}}\eta\Lambda = \eta
\end{equation}
dalla quale abbiamo
\begin{equation}
  \det\eta = \det(\Lambda^{\textup{T}}\eta\Lambda) = \det\eta
  \det(\Lambda^{\textup{T}}\Lambda) = \det\eta (\det\Lambda)^{2}
\end{equation}
e quindi deve valere la condizione
\begin{equation}
  \det\Lambda = \pm 1.
\end{equation}
Le trasformazioni di Lorentz con $\det\Lambda = +1$ sono chiamate
\emph{proprie}, quelle con $\det\Lambda = -1$ sono chiamate \emph{improprie}.

Consideriamo una particella a riposo nell'origine del sistema di riferimento $O$
(quindi $\dd \bm{x} = \bm{0}$) che si muove con velocità
$\bm{v} = (v^{1},v^{2},v^{3})$ rispetto a un altro sistema di riferimento
inerziale $O'$
% TODO: probabilmente questi calcoli andranno spezzati in più ambienti
% `equation' spiegando che passaggi vengono fatti in mezzo
\begin{gather}
  \dd x'^{\alpha} = \tensor{\Lambda}{^{\alpha}_{\beta}} \dd x^{\beta} =
  \tensor{\Lambda}{^{\alpha}_{0}} \dd x^{0} + \tensor{\Lambda}{^{\alpha}_{i}}\dd
  x^{i} \\
  \dd x'^{i} = \tensor{\Lambda}{^{i}_{0}}\dd t \\
  \dd x'^{0} = \tensor{\Lambda}{^{0}_{0}}\dd t.
\end{gather}
Dividendo membro a membro abbiamo
\begin{equation}
  \toder{x'^{i}}{x'^{0}} = \toder{x'^{i}}{t'} = v_{i} =
  \frac{\tensor{\Lambda}{^{i}_{0}}}{\tensor{\Lambda}{^{0}_{0}}},
\end{equation}
da cui
\begin{equation}
  \tensor{\Lambda}{^{i}_{0}} = v_{i}\tensor{\Lambda}{^{0}_{0}}.
\end{equation}
Ponendo $\gamma = \delta = 0$ nella~\eqref{eq:rel-fond-tensore-metrico} risulta
\begin{equation}
  \begin{split}
    \tensor{\eta}{_{00}} &= -1 = \tensor{\eta}{_{\alpha\beta}}
    \tensor{\Lambda}{^{\alpha}_{0}} \tensor{\Lambda}{^{\beta}_{0}} =
    \tensor{\eta}{_{\alpha\beta}}(\tensor{\Lambda}{^{0}_{0}})^{2} +
    \tensor{\eta}{_{ij}} \tensor{\Lambda}{^{i}_{0}} \tensor{\Lambda}{^{j}_{0}}
    \\
    &= -(\tensor{\Lambda}{^{0}_{0}})^{2} + \delta_{ij}
    \tensor{\Lambda}{^{i}_{0}} \tensor{\Lambda}{^{j}_{0}} =
    -(\tensor{\Lambda}{^{0}_{0}})^{2} + \sum_{i}(\tensor{\Lambda}{^{i}_{0}})^{2},
  \end{split}
\end{equation}
quindi
\begin{equation}
  (\tensor{\Lambda}{^{0}_{0}})^{2} = 1 +
  \sum_{i}(\tensor{\Lambda}{^{i}_{0}})^{2} = 1 +
  \sum_{i}(v_{i}\tensor{\Lambda}{^{0}_{0}})^{2} \geq 1,
\end{equation}
cioè deve risultare
\begin{equation}
  \tensor{\Lambda}{^{0}_{0}} \geq 1 \text{ oppure } \tensor{\Lambda}{^{0}_{0}}
  \leq -1.
\end{equation}
Le trasformazioni con $\tensor{\Lambda}{^{0}_{0}} \geq +1$ sono chiamate
\emph{ortocrone}, quelle con $\tensor{\Lambda}{^{0}_{0}} \leq -1$
\emph{anticrone}.  Le trasformazioni di Lorentz proprie e ortocrone sono dette
\emph{trasformazioni di Lorentz ristrette} o \emph{speciali}.  Considerando una
trasformazione ristretta inoltre abbiamo
\begin{equation}
  \tensor{\Lambda}{^{0}_{0}} = \frac{1}{\sqrt{1-\sum_{i}(v_{i})^{2}}} =
  \frac{1}{\sqrt{1-v^{2}}} = \gamma.
\end{equation}
Quindi risulta anche
\begin{equation}
  \tensor{\Lambda}{^{i}_{0}} = v_{i}\tensor{\Lambda}{^{0}_{0}} = v_{i}\gamma.
\end{equation}
Le altre componenti $\tensor{\Lambda}{^{\alpha}_{\beta}}$ non sono univocamente
determinate.  Una scelta conveniente che permette di soddisfare
la~\eqref{eq:trasformazione-lineare} è
\begin{subequations}
  \begin{align}
    \tensor{\Lambda}{^{0}_{i}} &= \tensor{\Lambda}{^{i}_{0}} = v_{i}\gamma, \\
    \tensor{\Lambda}{^{i}_{j}} &= \delta_{ij} +
    v_{i}v_{j}\frac{\gamma-1}{v^{2}}.
  \end{align}
\end{subequations}

% TODO: scrivere qualcosa!
\section{Conseguenze delle trasformazioni di Lorentz}
\label{sec:conseguenze-lorentz}

\subsection{Dilatazione degli intervalli temporali}
\label{sec:dilatazione-tempo}

\subsection{Effetto Doppler}
\label{sec:effetto-doppler}

\section{Calcolo tensoriale nello spazio di Minkowski}
\label{sec:calcolo-tensoriale}

Un \index{quadrivettore!controvariante}\emph{quadrivettore controvariante}
$V^{\alpha} = (V^{0}, V^{1}, V^{2}, V^{3})$ è un insieme di quattro quantità
che, mediante una trasformazione di Lorentz ristretta, si trasformano come
\begin{equation}
  V^{\alpha} \to V'^{\alpha} = \tensor{\Lambda}{^{\alpha}_{\beta}}V^{\beta}.
\end{equation}
Un quadrivettore controvariante può anche essere rappresentato come
\begin{equation}
  V^{\alpha} = (V^{0}, V^{i}) = (V^{0}, \bm{V}),
\end{equation}
dove $\bm{V} = (V^{1}, V^{2}, V^{3})$.

Un \index{quadrivettore!covariante}\emph{quadrivettore covariante}
$U_{\alpha} = (U_{0}, U_{1}, U_{2}, U_{3})$ è un insieme di quattro quantità
che, sotto una trasformazione di Lorentz ristretta, si trasformano come
\begin{equation}
  U_{\alpha} \to U'_{\alpha} = \tensor{\Lambda}{_{\alpha}^{\beta}}U_{\beta}
\end{equation}
dove
\begin{equation}
  \tensor{\Lambda}{_{\alpha}^{\beta}} = \tensor{\eta}{_{\alpha\gamma}}
  \tensor{\eta}{^{\beta\delta}} \tensor{\Lambda}{^{\gamma}_{\delta}}.
\end{equation}
La matrice $\tensor{\eta}{^{\alpha\beta}}$ è il
\index{tensore!metrico controvariante di
  Minkowski}\emph{tensore
  metrico controvariante di Minkowski}
ed è numericamente uguale a $\tensor{\eta}{_{\alpha\beta}}$, vale a dire
\begin{equation}
  \tensor{\eta}{^{\alpha\beta}} = \tensor{\eta}{_{\alpha\beta}}.
\end{equation}
Osserviamo che
\begin{equation}
  \tensor{\eta}{^{\alpha\gamma}} \tensor{\eta}{_{\beta\gamma}} =
  \tensor{\delta}{^{\alpha}_{\beta}} = \tensor{\delta}{_{\alpha}^{\beta}} =
  \begin{cases}
    1 & \text{se } \alpha = \beta, \\
    0 & \text{se } \alpha \neq \beta.
  \end{cases}
\end{equation}
in cui $\tensor{\delta}{^{\alpha}_{\beta}} = \tensor{\delta}{_{\alpha}^{\beta}}$
è la
\index{delta!di Kronecker quadrimensionale}delta di Kronecker quadrimensionale.
Usando la relazione fondamentale delle trasformazioni di
Lorentz~\eqref{eq:rel-fond-tensore-metrico} abbiamo
\begin{equation}
  \tensor{\Lambda}{_{\alpha}^{\gamma}} \tensor{\Lambda}{^{\alpha}_{\beta}} =
  \tensor{\eta}{_{\alpha\delta}} \tensor{\eta}{^{\gamma\epsilon}}
  \tensor{\Lambda}{^{\delta}_{\epsilon}} \tensor{\Lambda}{^{\alpha}_{\beta}} =
  \tensor{\eta}{_{\epsilon\beta}} \tensor{\eta}{^{\gamma\epsilon}} =
  \tensor{\delta}{^{\gamma}_{\beta}}.
\end{equation}
Dunque $\tensor{\Lambda}{_{\alpha}^{\beta}}$ è la matrice inversa di
$\tensor{\Lambda}{^{\alpha}_{\beta}}$.

Vogliamo ora determinare la trasformazione inversa $x'^{\alpha} \to x^{\alpha}$.
Dalla trasformazione~\eqref{eq:trasformazione-lineare} risulta che
\begin{equation}
  x'^{\alpha} \tensor{\Lambda}{_{\alpha}^{\gamma}} =
  \tensor{\Lambda}{^{\alpha}_{\beta}} \tensor{\Lambda}{_{\alpha}^{\gamma}}
  x^{\beta} + \tensor{\Lambda}{_{\alpha}^{\gamma}} a^{\alpha} =
  \tensor{\delta}{^{\gamma}_{\beta}} x^{\beta} - a'^{\gamma} = x^{\gamma} -a'^{\gamma}
\end{equation}
da cui ricaviamo la trasformazione, rinominando gli indici,
\begin{equation}
  x'^{\alpha} \to x^{\alpha} = \tensor{\Lambda}{_{\beta}^{\alpha}}x'^{\beta} +
  a'^{\alpha}.
\end{equation}

Il prodotto scalare $V^{\alpha}U_{\alpha}$ fra un quadrivettore covariante e uno
controvariante è una quantità che non varia cambiando sistema di riferimento
inerziale, infatti
\begin{equation}
  V^{\alpha}U_{\alpha} \to V'^{\alpha}U'_{\alpha} =
  \tensor{\Lambda}{^{\alpha}_{\beta}} V^{\beta}
  \tensor{\Lambda}{_{\alpha}^{\gamma}} U_{\gamma} =
  \tensor{\delta}{_{\beta}^{\gamma}} V^{\beta} U_{\gamma} = V^{\gamma}U_{\gamma}
  = V^{\alpha} U_{\alpha}.
\end{equation}

Possiamo passare da un quadrivettore controvariante
$V^{\alpha} = (V^{0}, V^{i})$ al corrispondente quadrivettore covariante
$V_{\alpha} = (V_{0}, V_{i})$ moltiplicandolo per il tensore metrico covariante
\begin{equation}
  \label{eq:abbassamento-indice}
  \begin{split}
    V^{\alpha} \to V_{\alpha} &= \tensor{\eta}{_{\alpha\beta}}V^{\beta} =
    \tensor{\eta}{_{\alpha\beta}} (V^{0}, V^{i}) =
    (\tensor{\eta}{_{0\beta}}V^{0}, \tensor{\eta}{_{i\beta}}V^{i}) \\
    &= (\tensor{\eta}{_{00}}V^{0}, \delta_{i\beta}V^{\beta}) = (-V^{0}, V^{i}).
  \end{split}
\end{equation}
Pertanto, mediante questa operazione, chiamata a volte di
\emph{abbassamento degli indici}, la componente temporale di un quadrivettore
cambia segno mentre quelle spaziali rimangono immutate.  Analogamente si può
effettuare l'\emph{innalzamento degli indici} di un quadrivettore covariante
moltiplicandolo per il tensore metrico controvariante
\begin{equation}
  \label{eq:innalzamento-indice}
  U_{\alpha} \to U^{\alpha} = \tensor{\eta}{^{\alpha\beta}}U_{\beta}.
\end{equation}
Verifichiamo che il quadrivettore $V_{\alpha}$ definito
nella~\eqref{eq:abbassamento-indice} si trasforma effettivamente come un
quadrivettore covariante
\begin{equation}
  V_{\alpha} \to V'_{\alpha} = \tensor{\eta}{_{\alpha\beta}}V'^{\beta} =
  \tensor{\eta}{_{\alpha\beta}} \tensor{\Lambda}{^{\beta}_{\gamma}} V^{\gamma} =
  \tensor{\eta}{_{\alpha\beta}} \tensor{\Lambda}{^{\beta}_{\gamma}}
  \tensor{\eta}{^{\gamma\delta}}V_{\delta} =
  \tensor{\Lambda}{_{\alpha}^{\delta}} V_{\delta}.
\end{equation}
Allo stesso modo si può far vedere che il quadrivettore $U^{\alpha}$ definito
nella~\eqref{eq:innalzamento-indice} si trasforma come un quadrivettore
controvariante.

Possiamo generalizzare le definizioni date finora introducendo il concetto di
quadritensore.  Un \index{quadritensore}\emph{quadritensore}
$\tensor*{A}{^{\alpha_{1}\dots\alpha_{p}}_{\beta_{1}\dots\beta_{q}}}$ (con
$p + q = k$) di rango $k$ è un insieme di $4^{k}$ quantità che sotto una
trasformazione di Lorentz ristretta
$x^{\alpha} \to x'^{\alpha} = \tensor{\Lambda}{^{\alpha}_{\beta}}x^{\beta}$ si
trasformano come
\begin{equation}
  \tensor*{A}{^{\alpha_{1}\dots\alpha_{p}}_{\beta_{1}\dots\beta_{q}}} \to
  \tensor*{{A'}}{^{\alpha_{1}\dots\alpha_{p}}_{\beta_{1}\dots\beta_{q}}} =
  \tensor{\Lambda}{^{\alpha_{1}}_{\mu_{1}}} \cdots
  \tensor{\Lambda}{^{\alpha_{p}}_{\mu_{p}}}
  \tensor{\Lambda}{_{\beta_{1}}^{\nu_{1}}} \cdots
  \tensor{\Lambda}{_{\beta_{q}}^{\nu_{q}}} \tensor*{A}{^{\mu_{1}\dots\mu_{p}}_{\nu_{1}\dots\nu_{q}}}
\end{equation}
Il quadritensore
$\tensor*{A}{^{\alpha_{1}\dots\alpha_{p}}_{\beta_{1}\dots\beta_{q}}}$ è un
quadritensore misto $p$ volte controvariante e $q$ volte covariante, oppure di
tipo $(p,q)$.  Un quadritensore di rango $k$ completamente controvariante è del
tipo $\tensor{A}{^{\alpha_{1}\dots\alpha_{k}}}$ e si trasforma come
\begin{equation}
  \tensor{A}{^{\alpha_{1}\dots\alpha_{k}}} \to
  \tensor{{A'}}{^{\alpha_{1}\dots\alpha_{k}}} =
  \tensor{\Lambda}{^{\alpha_{1}}_{\mu_{1}}} \cdots
  \tensor{\Lambda}{^{\alpha_{k}}_{\mu_{k}}} \tensor{A}{^{\mu_{1}\dots\mu_{k}}},
\end{equation}
similmente un quadritensore di rango $k$ completamente covariante è del tipo
$\tensor{A}{_{\alpha_{1}\dots\alpha_{k}}}$ e si trasforma come
\begin{equation}
  \tensor{A}{_{\alpha_{1}\dots\alpha_{k}}} \to
  \tensor{{A'}}{_{\alpha_{1}\dots\alpha_{k}}} =
  \tensor{\Lambda}{_{\alpha_{1}}^{\mu_{1}}} \cdots
  \tensor{\Lambda}{_{\alpha_{k}}^{\mu_{k}}} \tensor{A}{_{\mu_{1}\dots\mu_{k}}}.
\end{equation}
I quadritensori di rango $1$ sono i \index{quadrivettore}\emph{quadrivettori}, i
quadritensori di rango $0$, che quindi non hanno indici, sono chiamati
\index{scalare!di Lorentz}\emph{scalari di Lorentz} (o brevemente
\emph{scalari}) e sono invarianti sotto le trasformazioni di Lorentz.

\subsection{Operazioni con i quadritensori}
\label{sec:operazioni-tensori}

\subsubsection{Combinazione lineare}
\label{sec:combinazione-lineare}

La combinazione lineare
\begin{equation}
  \tensor*{T}{^{\alpha_{1}\dots\alpha_{p}}_{\beta_{1}\dots\beta_{q}}} = a
  \tensor*{R}{^{\alpha_{1}\dots\alpha_{p}}_{\beta_{1}\dots\beta_{q}}} + b
  \tensor*{S}{^{\alpha_{1}\dots\alpha_{p}}_{\beta_{1}\dots\beta_{q}}},
\end{equation}
con $a$ e $b$ scalari, di due quadritensori dello stesso tipo è un quadritensore
ancora dello stesso tipo.  Per esempio, verifichiamo che
\begin{equation}
  \tensor{T}{^{\alpha}_{\beta}} = a \tensor{R}{^{\alpha}_{\beta}} + b
  \tensor{S}{^{\alpha}_{\beta}}
\end{equation}
è un quadritensore di rango $2$ una volta controvariante e una volta covariante
come lo sono $\tensor{R}{^{\alpha}_{\beta}}$ e
$\tensor{S}{^{\alpha}_{\beta}}$. Abbiamo
\begin{equation}
  \begin{split}
    \tensor{T}{^{\alpha}_{\beta}} \to \tensor{{T'}}{^{\alpha}_{\beta}} &= a
    \tensor{{R'}}{^{\alpha}_{\beta}} + b \tensor{{S'}}{^{\alpha}_{\beta}} = a
    \tensor{\Lambda}{^{\alpha}_{\gamma}} \tensor{\Lambda}{_{\beta}^{\delta}}
    \tensor{R}{^{\gamma}_{\delta}} + b \tensor{\Lambda}{^{\alpha}_{\gamma}}
    \tensor{\Lambda}{_{\beta}^{\delta}} \tensor{S}{^{\gamma}_{\delta}} \\
    &= \tensor{\Lambda}{^{\alpha}_{\gamma}}
    \tensor{\Lambda}{_{\beta}^{\delta}}(a \tensor{R}{^{\gamma}_{\delta}} + b
    \tensor{S}{^{\gamma}_{\delta}}) = \tensor{\Lambda}{^{\alpha}_{\gamma}}
    \tensor{\Lambda}{_{\beta}^{\delta}} \tensor{T}{^{\gamma}_{\delta}}.
  \end{split}
\end{equation}

\subsubsection{Prodotto diretto}
\label{sec:prodotto-diretto}

Il prodotto
$\tensor*{A}{^{\alpha_{1}\dots\alpha_{p}}_{\beta_{1}\dots\beta_{q}}}
\tensor*{B}{^{\alpha_{1}\dots\alpha_{r}}_{\beta_{1}\dots\beta_{s}}}$
di un quadritensore di tipo $(p,q)$ e di uno di tipo $(r,s)$ è un quadritensore
$\tensor*{T}{^{\alpha_{1}\dots\alpha_{p+r}}_{\beta_{1}\dots\beta_{q+s}}}$ di
tipo $(p+r,q+s)$.  Consideriamo per esempio
\begin{equation}
  \tensor{T}{^{\alpha}_{\beta}^{\delta}} = \tensor{A}{^{\alpha}_{\beta}}
  \tensor{B}{^{\gamma}}.
\end{equation}
Abbiamo
\begin{equation}
  \tensor{T}{^{\alpha}_{\beta}^{\delta}} \to
  \tensor{{T'}}{^{\alpha}_{\beta}^{\delta}} = \tensor{{A'}}{^{\alpha}_{\beta}}
  \tensor{{B'}}{^{\gamma}} = \tensor{\Lambda}{^{\alpha}_{\delta}}
  \tensor{\Lambda}{_{\beta}^{\epsilon}} \tensor{A}{^{\delta}_{\epsilon}}
  \tensor{\Lambda}{^{\gamma}_{\zeta}} \tensor{B}{^{\zeta}} =
  \tensor{\Lambda}{^{\alpha}_{\delta}} \tensor{\Lambda}{_{\beta}^{\epsilon}}
  \tensor{\Lambda}{^{\gamma}_{\zeta}} \tensor{T}{^{\delta}_{\epsilon}^{\zeta}}.
\end{equation}


\subsubsection{Contrazione}
\label{sec:contrazione}

\subsubsection{Derivazione}
\label{sec:derivazione}

La derivazione di uno scalare $\phi$ rispetto a una coordinata controvariante $x^{\alpha}$
\begin{equation}
  \toder{\phi}{x^{\alpha}} = U_{\alpha}
\end{equation}
produce un quadrivettore covariante.  Infatti, dalla trasformazione

\subsection{Quadritensori invarianti}
\label{sec:tensori-invarianti}


\section{Dinamica particellare}
\label{sec:dinamica-particellare}

Data una particella il cui moto è descritto dalla coordinata controvariante
$x^{\alpha}$, definiamo la sua \index{quadrivelocità}\emph{quadrivelocità} come
la derivata di $x^{\alpha}$ rispetto al tempo proprio $\tau$:
\begin{equation}
  v^{\alpha} = (v^{0}, v^{i}) =\toder{x^{\alpha}}{\tau}
\end{equation}
e poiché risulta
\begin{equation}
  \dd \tau = \sqrt{\dd t^{2} - \dd \bm{x}^{2}} = \dd t \sqrt{1 -
    \Big(\toder{\bm{x}}{t}\Big)^{2}} = \dd t\sqrt{1 - v^{2}} = \toder{\dd
    t}{\gamma}
\end{equation}
abbiamo che la quadrivelocità è uguale a
\begin{equation}
  v^{\alpha} = \toder{x^{\alpha}}{t}\toder{t}{\tau} = \toder{x^{\alpha}}{t}
  \gamma = \gamma \Big(\toder{x^{0}}{t}, \toder{\bm{x}}{t}\Big) =
  \gamma(1,\bm{v}) = (\gamma, \gamma\bm{v}).
\end{equation}
Il \index{quadrimomento}\emph{quadrimomento} della particella è definito come il
prodotto fra la massa a riposo $m$ della particella e la quadrivelocità
\begin{equation}
  p^{\alpha} = (p^{0},p^{i}) = m v^{\alpha} = m (v^{0}, v^{i}) = (m\gamma,
  m\gamma\bm{v}).
\end{equation}
Ricordando che l'energia totale $E$ di una particella è definita come $m\gamma$
troviamo che la prima componente del quadrivettore momento è proprio l'energia
totale, mentre le componenti $p^{i}$ sono le componenti del momento
$\bm{p} = m\gamma\bm{v}$
\begin{equation}
  p^{\alpha} = (E, \bm{p}).
\end{equation}
La \index{quadriforza}\emph{quadriforza} agente sulla particella è definita come
la derivata rispetto al tempo proprio del quadrimomento
\begin{equation}
  f^{\alpha} = (f^{0},f^{i}) =\toder{p^{\alpha}}{\tau} =
  m\toder{v^{\alpha}}{\tau} = mw^{\alpha}
\end{equation}
in cui
\begin{equation}
  w^{\alpha} = (w^{0},w^{i}) =\toder{v^{\alpha}}{\tau}
\end{equation}
è la \index{quadriaccelerazione}\emph{quadriaccelerazione}.  In un sistema di
riferimento $O$ in cui la particella è a riposo $\dd \bm{x} = \bm{0}$ quindi
$\dd \tau = \dd t$ e $f^{\alpha} = F^{\alpha}$, in cui $F^{i}$ sono le
componenti cartesiane della forza non relativistica $\bm{F}$ e $F^{0} = 0$:
\begin{equation}
  f^{\alpha} = (0, \bm{F}).
\end{equation}
Per calcolare la forza a cui è soggetta la particella in un altro sistema di
riferimento inerziale $O'$ in moto rispetto a $O$ con velocità $\bm{v}$ dobbiamo
effettuare la trasformazione di Lorentz
\begin{equation}
  f'^{\alpha} = \tensor{\Lambda}{^{\alpha}_{\beta}}f^{\beta}.
\end{equation}
Sviluppando i calcoli si ottiene
\begin{subequations}
  \begin{align}
    f^{0} &= \gamma\bm{v}\cdot\bm{F} = \bm{v}\cdot\bm{f}, \\
    \bm{f} &= \bm{F} + (\gamma - 1)\bm{v}\frac{\bm{v}\cdot\bm{F}}{v^{2}}.
  \end{align}
\end{subequations}
% TODO: mettere la relazione di mass shell se non è stata messa prima


%%% Local Variables:
%%% mode: latex
%%% TeX-master: "../astrofisica-teorica"
%%% fill-column: 80
%%% End:

\cleardoublepage
\chapter{Il principio di equivalenza}
\label{cha:principio-equivalenza}

\completare{Esperimento ideale di Einstein, uguaglianza massa inerziale e
  gravitazionale.}

\index{principio!di equivalenza}\emph{principio di equivalenza}:
\emph{per ogni evento dello spazio-tempo in un arbitrario campo gravitazionale è
  possibile scegliere un sistema di riferimento \emph{localmente inerziale} (in
  caduta libera nel campo gravitazionale), tale che in un intorno
  sufficientemente piccolo dell'evento gli effetti della gravità siano assenti e
  le leggi della natura assumano la stessa forma che hanno in un sistema di
  riferimento inerziale}.
Dunque in un sistema di riferimento localmente inerziale la fisica segue le
leggi della relatività speciale.

\section{Equazione del moto di una particella in un campo gravitazionale}
\label{sec:equazione-moto}

Consideriamo una particella in caduta libera all'interno di un campo
gravitazionale $\bm{g}$.  In un sistema di riferimento, di coordinate
$\xi^{\alpha}$, localmente inerziale in cui la particella si muove di moto
rettilineo uniforme nello spazio-tempo, l'equazione del moto è
\begin{equation}
  \label{eq:moto-caduta-libera}
  \toder[2]{\xi^{\alpha}}{\tau} = 0,
\end{equation}
in cui $\dd^{2} \tau = -\eta_{\alpha\beta}\dd\xi^{\alpha}\dd\xi^{\beta}$ è
l'intervallo di tempo proprio.  Le coordinate $\xi^{\alpha}$ possono essere
espresse come funzione di un altro qualsiasi sistema di coordinate $x^{\mu}$ e
viceversa
\begin{align}
  \xi^{\alpha} &= \xi^{\alpha}(x^{\mu}), \\
  x^{\mu} &= x^{\mu}(\xi^{\alpha}).
\end{align}
% Poiché la trasformazione è invertibile deve risultare
% \begin{equation}
%   \abs*{\parder{x^{\mu}}{\xi^{\alpha}}} \neq 0.
% \end{equation}
Inoltre le coordinate $x^{\mu}$ dipendono anche dal tempo proprio $\tau$, allora
dalla~\eqref{eq:moto-caduta-libera} abbiamo
\begin{equation}
  \label{eq:moto1}
  0 = \toder{}{\tau}\toder{\xi^{\alpha}}{\tau} = \toder{}{\tau}
  \Big( \parder{\xi^{\alpha}}{x^{\mu}}\toder{x^{\mu}}{\tau} \Big)
  = \parder{\xi^{\alpha}}{x^{\mu},x^{\nu}} \toder{x^{\nu}}{\tau}
  \toder{x^{\mu}}{\tau} + \parder{\xi^{\alpha}}{x^{\mu}}
  \toder[2]{x^{\mu}}{\tau}.
\end{equation}
Supponendo che ciascuna delle quattro coordinate $x^{\mu}$ sia indipendente
dalle altre possiamo scrivere le relazioni
\begin{equation}
  \parder{x^{\mu}}{x^{\nu}} = \tensor{\delta}{^{\mu}_{\nu}}
  \implies \parder{x^{\mu}}{\xi^{\alpha}} \parder{\xi^{\alpha}}{x^{\nu}} =
  \tensor{\delta}{^{\mu}_{\nu}},
\end{equation}
quindi moltiplicando la~\eqref{eq:moto1} per
$\lparder{x^{\lambda}}{\xi^{\alpha}}$ l'equazione del moto diventa
\begin{equation}
  \tensor{\delta}{^{\lambda}_{\mu}} \toder[2]{x^{\mu}}{\tau} +
  \Big( \parder{\xi^{\alpha}}{x^{\mu},x^{\nu}} \parder{x^{\lambda}}{\xi^{\alpha}}
  \Big) \toder{x^{\mu}}{\tau} \toder{x^{\nu}}{\tau} = 0.
\end{equation}
Definendo la \index{connessione!affine}\emph{connessione affine}
$\tensor{\Gamma}{^{\lambda}_{\mu\nu}}$ come
\begin{equation}
  \label{eq:connessione-affine}
  \tensor{\Gamma}{^{\lambda}_{\mu\nu}}
  = \parder{\xi^{\alpha}}{x^{\mu},x^{\nu}} \parder{x^{\lambda}}{\xi^{\alpha}}
\end{equation}
l'equazione del moto può essere riscritta nel seguente modo
\begin{equation}
  \label{eq:geodesica}
  \toder[2]{x^{\lambda}}{\tau} + \tensor{\Gamma}{^{\lambda}_{\mu\nu}}
  \toder{x^{\mu}}{\tau} \toder{x^{\nu}}{\tau} = 0,
\end{equation}
oppure, utilizzando la quadrivelocità $u^{\alpha} = \ltoder{x^{\alpha}}{\tau}$,
\begin{equation}
  \toder[2]{x^{\lambda}}{\tau} + \tensor{\Gamma}{^{\lambda}_{\mu\nu}}
  u^{\nu} u^{\nu} = 0.
\end{equation}
La~\eqref{eq:geodesica} è chiamata \emph{equazione della geodesica}.
$\tensor{\Gamma}{^{\lambda}_{\mu\nu}}$ è un insieme di $4^{3} = 64$ quantità che
tuttavia, come vedremo, % TODO: farlo vedere oppure non dire "come vedremo" :-P
non si trasformano, in generale, come un quadritensore di rango $3$.  Dalla
definizione di connessione affine abbiamo inoltre che questa è simmetrica
rispetto agli indici inferiori, cioè
$\tensor{\Gamma}{^{\lambda}_{\mu\nu}} = \tensor{\Gamma}{^{\lambda}_{\nu\mu}}$.

L'intervallo di tempo proprio
$\dd\tau^{2} = -\eta_{\alpha\beta} \dd\xi^{\alpha} \dd\xi^{\beta}$ può essere
espresso rispetto alle coordinate $x^{\mu}$ come
\begin{equation}
  \begin{split}
    \dd\tau^{2} &= -\eta_{\alpha\beta} \parder{\xi^{\alpha}}{x^{\mu}}\dd
    x^{\mu} \parder{\xi^{\beta}}{x^{\nu}}\dd x^{\nu} \\
    &= -g_{\mu\nu}\dd x^{\mu}\dd x^{\nu},
  \end{split}
\end{equation}
dove
\begin{equation}
  \label{eq:tensore-metrico}
  g_{\mu\nu} =
  \eta_{\alpha\beta} \parder{\xi^{\alpha}}{x^{\mu}} \parder{\xi^{\beta}}{x^{\nu}}
\end{equation}
è il \index{tensore!metrico}\emph{tensore metrico}.  Osserviamo che $g_{\mu\nu}$
è simmetrico rispetto allo scambio degli indici in quanto il tensore metrico di
Minkowski $\eta_{\alpha\beta}$ è simmetrico, infatti
\begin{equation}
  g_{\nu\mu} =
  \eta_{\alpha\beta} \parder{\xi^{\alpha}}{x^{\nu}}\parder{\xi^{\beta}}{x^{\mu}}
  =
  \eta_{\beta\alpha} \parder{\xi^{\beta}}{x^{\mu}} \parder{\xi^{\alpha}}{x^{\nu}}
  = g_{\mu\nu}.
\end{equation}

\section{Relazione fra tensore metrico e connessione affine}
\label{sec:relazione-g-Gamma}

Derivando rispetto a $x^{\lambda}$ l'espressione del tensore
metrico~\eqref{eq:tensore-metrico} abbiamo
\begin{equation}
  \partial_{\lambda}g_{\mu\nu}
  = \parder{\xi^{\alpha}}{x^{\lambda},x^{\mu}} \parder{\xi^{\beta}}{x^{\nu}}
  \eta_{\alpha\beta}
  + \parder{\xi^{\alpha}}{x^{\mu}} \parder{\xi^{\beta}}{x^{\lambda},x^{\nu}}
  \eta_{\alpha\beta}.
\end{equation}
Osserviamo che dalla definizione di connessione
affine~\eqref{eq:connessione-affine} risulta
\begin{equation}
  \parder{\xi^{\alpha}}{x^{\lambda}}\tensor{\Gamma}{^{\lambda}_{\mu\nu}}
  = \parder{\xi^{\alpha}}{x^{\mu},x^{\nu}},
\end{equation}
quindi
\begin{equation}
  \begin{split}
    \partial_{\lambda}g_{\mu\nu} &=
    \tensor{\Gamma}{^{\rho}_{\lambda\mu}} \parder{\xi^{\alpha}}{x^{\rho}}
    \parder{\xi^{\beta}}{x^{\nu}} \eta_{\alpha\beta} +
    \tensor{\Gamma}{^{\rho}_{\lambda\nu}} \parder{\xi^{\alpha}}{x^{\mu}}
    \parder{\xi^{\beta}}{x^{\rho}}
    \eta_{\alpha\beta} \\
    &= \tensor{\Gamma}{^{\rho}_{\lambda\mu}}g_{\rho\nu} +
    \tensor{\Gamma}{^{\rho}_{\lambda\nu}}g_{\rho\mu}.
  \end{split}
\end{equation}
Calcolando in maniera $\partial_{\mu}g_{\lambda\nu}$ e
$\partial_{\nu}g_{\mu\lambda}$ si giunge all'equazione
\begin{equation}
  \begin{split}
    \partial_{\lambda}g_{\mu\nu} + \partial_{\mu}g_{\lambda\nu}
    - \partial_{\nu}g_{\mu\lambda} &=
    \tensor{\Gamma}{^{\rho}_{\lambda\mu}}g_{\rho\nu} +
    \tensor{\Gamma}{^{\rho}_{\lambda\nu}}g_{\rho\mu} +
    \tensor{\Gamma}{^{\rho}_{\mu\lambda}}g_{\rho_{\nu}} +
    \tensor{\Gamma}{^{\rho}_{\mu\nu}}g_{\rho\lambda} \\
    &- \tensor{\Gamma}{^{\rho}_{\nu\mu}}g_{\rho\lambda} -
    \tensor{\Gamma}{^{\rho}_{\nu\lambda}}g_{\rho\mu} \\
    &= 2g_{\rho\nu} \tensor{\Gamma}{^{\rho}_{\lambda\mu}},
  \end{split}
\end{equation}
ricordando che $g_{\mu\nu}$ e $\tensor{\Gamma}{^{\rho}_{\mu\nu}}$ sono
simmetrici per scambi degli indici $\mu$ e $\nu$.  Definiamo la matrice
$g^{\nu\sigma}$ come l'inversa di $g_{\nu\sigma}$, cioè
$g^{\nu\sigma}g_{\rho\nu} = \tensor{\delta}{^{\sigma}_{\rho}}$, allora,
moltiplicando l'equazione precedente per $g^{\nu\sigma}$ otteniamo la seguente
relazione fra il tensore metrico e la connessione affine
\begin{equation}
  \begin{aligned}
    &g^{\nu\sigma}(\partial_{\lambda}g_{\mu\nu} + \partial_{\mu}g_{\lambda\nu}
    - \partial_{\nu}g_{\mu\lambda}) = 2\tensor{\Gamma}{^{\sigma}_{\lambda\mu}}
    \iff \\
    &\tensor{\Gamma}{^{\sigma}_{\lambda\mu}} =
    \frac{1}{2} g^{\nu\sigma} (\partial_{\lambda}g_{\mu\nu}
    + \partial_{\mu}g_{\lambda\nu} - \partial_{\nu}g_{\lambda\mu}).
  \end{aligned}
\end{equation}

\section{Limite newtoniano dell'equazione del moto}
\label{sec:limite-newtoniano}

Vogliamo vedere come l'equazione del moto~\eqref{eq:geodesica} di una particella
all'interno di un campo gravitazionale si riduce considerando il limite
newtoniano, vale a dire nelle condizioni
\begin{itemize}
\item il campo gravitazionale è debole, cioè può essere considerato come una
  perturbazione dello spazio-tempo piatto;
\item il campo gravitazionale inoltre è stazionario;
\item la velocità della particella è molto più piccola di quella della luce.
\end{itemize}
La prima condizione può essere formalizzata dicendo che il tensore metrico
$g_{\mu\nu}$ deve differire di poco dal tensore di Minkowski dello spazio-tempo
piatto
\begin{equation}
  g_{\mu\nu} = \eta_{\mu\nu} + h_{\mu\nu},
\end{equation}
con $\abs{h_{\mu\nu}} \ll 1$.  Calcoliamo la connessione affine al primo ordine
della perturbazione $h_{\mu\nu}$
\begin{equation}
  \label{eq:christoffel-approx}
  \tensor{\Gamma}{^{\sigma}_{\mu\nu}} = \frac{1}{2}g^{\lambda\sigma}
  (\partial_{\mu}g_{\nu\lambda} + \partial_{\nu}g_{\mu\lambda}
  - \partial_{\lambda}g_{\mu\nu}) \approx \frac{1}{2}\eta^{\lambda\sigma}
  (\partial_{\mu}h_{\nu\lambda} + \partial_{\nu}h_{\mu\lambda}
  - \partial_{\lambda}h_{\mu\nu}).
\end{equation}
La terza condizione può essere formalizzata come
\begin{equation}
  \toder{x^{i}}{\tau} \ll \toder{t}{\tau},
\end{equation}
quindi l'equazione della geodesica si riduce a
\begin{equation}
  \label{eq:geodesica-approx}
  \begin{split}
    0 &= \toder[2]{x^{\sigma}}{\tau} + \tensor{\Gamma}{^{\sigma}_{\lambda\mu}}
    \toder{x^{\lambda}}{\tau} \toder{x^{\mu}}{\tau} \\
    &= \toder[2]{x^{\sigma}}{\tau} + \tensor{\Gamma}{^{\sigma}_{00}} \Big(
    \toder{t}{\tau} \Big)^{2} + 2\tensor{\Gamma}{^{\sigma}_{0i}} \toder{t}{\tau}
    \toder{x^{i}}{\tau} + \tensor{\Gamma}{^{\sigma}_{ij}} \toder{x^{i}}{\tau}
    \toder{x^{j}}{\tau} \\
    &\approx \toder[2]{x^{\sigma}}{\tau} + \tensor{\Gamma}{^{\sigma}_{00}} \Big(
    \toder{t}{\tau} \Big)^{2}.
  \end{split}
\end{equation}
Dalla~\eqref{eq:christoffel-approx} risulta
\begin{equation}
  \tensor{\Gamma}{^{\sigma}_{00}} \approx
  \frac{1}{2}\eta^{\lambda\sigma}(\partial_{0}h_{0\lambda}
  + \partial_{0}h_{0\lambda} - \partial_{\lambda}h_{00}) =
  -\frac{1}{2}\eta^{\lambda\sigma} \partial_{\lambda}h_{00},
\end{equation}
in quanto le derivate temporali di $h_{\mu\nu}$ sono nulle in base alla seconda
condizione che descrive il limite newtoniano.  Notiamo che per $\sigma=0$ si ha
$\tensor{\Gamma}{^{0}_{00}}=0$.  La~\eqref{eq:geodesica-approx} corrisponde, al
variare di $\sigma$, alle equazioni
\begin{subequations}
  \begin{align}
    \toder[2]{t}{\tau} &= 0 &\text{per $\sigma = 0$}, \\
    \toder[2]{x^{i}}{\tau} &= \frac{1}{2} \Big( \toder{t}{\tau} \Big)^{2}
    (\nabla h_{00})_{i} &\text{per $\sigma = i$}.
  \end{align}
\end{subequations}
La seconda equazione ci dice che $\ltoder{t}{\tau}$ è una costante, mentre dalla
prima, dividendo per $(\ltoder{t}{\tau})^{2}$, abbiamo
\begin{equation}
  \toder[2]{\bm{x}}{t} = \frac{1}{2}\nabla h_{00}.
\end{equation}
L'equazione newtoniana del moto è
\begin{equation}
  \bm{a}_{\textup{N}} = \toder[2]{\bm{x}}{t} = -\nabla\phi,
\end{equation}
in cui $\phi$ è il potenziale del campo gravitazionale che a distanza $r$ da un
corpo sferico di massa $M$ assume l'espressione $\phi = -GM/r$.  Confrontando le
due equazioni precedenti troviamo
\begin{equation}
  h_{00} = -2\phi + \text{costante}.
\end{equation}
A grandi distanze dal corpo, cioè nel limite $r \to \infty$, il potenziale
$\phi$ si annulla, il campo gravitazionale è assente e il tensore metrico
$g_{\mu\nu}$ deve tendere al tensore di Minkowski $\eta_{\mu\nu}$ dello
spazio-tempo piatto, cioè si devono annullare tutte le componenti di
$h_{\mu\nu}$.  Pertanto ricaviamo che la costante deve essere uguale a $0$ e
$h_{00} = -2\phi$ o, ripristinando la velocità $c$ della luce  nel vuoto,
\begin{equation}
  h_{00} = -\frac{2\phi}{c^{2}} = \frac{2GM}{rc^{2}} = \frac{r_{\textup{g}}}{r},
\end{equation}
in cui
\begin{equation}
  r_{\textup{g}} = \frac{2GM}{c^{2}}
\end{equation}
è il \index{raggio!di Schwarzschild}\emph{raggio di Schwarzschild} del corpo che
genera il campo gravitazionale.  Dunque
\begin{equation}
  g_{00} = \eta_{00} + h_{00} = - \Big(1 - \frac{2GM}{rc^{2}} \Big)
\end{equation}
e la quantità $h_{00} = r_{\textup{g}}/r$ è una misura delle deviazioni della
relatività generale rispetto alle previsioni della gravità newtoniana.

\begin{table}
  \centering
  \caption[Valori del raggio di Schwarzschild per diversi corpi]{Valori del
    raggio di Schwarzschild $r_{\textup{g}}$ per diversi oggetti e valore della
    correzione $r_{\textup{g}}/r$ sulla superficie dei corpi.  $r$ è il raggio
    degli oggetti}
  \label{tab:Schwarzschild}
  \begin{tabular}{lSSSS}
    \toprule
    corpo & {$r_{\textup{g}}$ (\si{\metre})} & {$r$ (\si{\metre})} &
    {$r_{\textup{g}}/r$} \\
    \midrule
    Terra              & 8.7e-3 & 6.4e6 & 1.4e-9 \\
    Sole               & 3e3    & 7e8   & 4.3e-6 \\
    nana bianca        & 3e3    & 1e6   & 3e-3   \\
    stella di neutroni & 3e3    & 1e4   & 0.3    \\
    buco nero stellare & 3e4    & 3e4   & 1      \\
    \bottomrule
  \end{tabular}
\end{table}
Nella tabella~\ref{tab:Schwarzschild} sono riportati i valori delle deviazioni
$r_{\textup{g}}/r$ dalla gravitazione newtoniana sulla superficie di alcuni
corpi.  Mentre per la Terra, il Sole e le nane bianche queste correzioni sono
molto piccole, non è possibile descrivere le vicinanze di corpi come le stelle
di neutroni o i buchi neri utilizzando la fisica newtoniana ma bisogna tenere
necessariamente in considerazione la relatività generale.



%%% Local Variables:
%%% mode: latex
%%% TeX-master: "../astrofisica-teorica"
%%% fill-column: 80
%%% End:

\cleardoublepage
\chapter{Calcolo tensoriale nella relatività generale}
\label{cha:calcolo-tensoriale}

\emph{Nota: da questo capitolo in poi ometteremo, per brevità, il prefisso
  \emph{quadri-} dove ciò non dovesse generare ambiguità.  Per esempio, useremo
  i termini \emph{tensore} e \emph{vettore} in luogo di, rispettivamente,
  \emph{quadritensore} e \emph{quadrivettore}.}

Nel paragrafo~\ref{sec:calcolo-tensoriale-minkowski} abbiamo introdotto, nel
contesto della relatività speciale, i tensori, specificando come si modificano
per effetto di una trasformazione di Lorentz, la quale è una trasformazione
lineare fra sistemi di riferimento inerziali.  Nella definizione dei tensori
intervengono le componenti $\tensor{\Lambda}{^{\alpha}_{\beta}}$ e
$\tensor{\Lambda}{_{\alpha}^{\beta}}$ delle matrici di Lorentz associate alle
specifiche trasformazioni considerate.

Anche nella relatività generale è utile introdurre i tensori, ma in questa
teoria si considerano trasformazioni arbitrarie, quindi non necessariamente
lineari, fra sistemi di riferimento anche non inerziali.  Abbiamo già notato
che, nella relatività speciale, gli elementi delle matrici delle trasformazioni
di Lorentz $x^{\alpha} \to x'^{\alpha}$ sono dati da
\begin{subequations}
  \begin{align}
    \tensor{\Lambda}{^{\alpha}_{\beta}} &= \parder{x'^{\alpha}}{x^{\beta}}, \\
    \tensor{\Lambda}{_{\beta}^{\alpha}} &= \parder{x^{\alpha}}{x'^{\beta}}.
  \end{align}
\end{subequations}
Definiremo i tensori, in relatività generale, in maniera analoga a quanto fatto
all'interno della relatività speciale, cioè prendendo come riferimento le
trasformazioni dei differenziali delle coordinate controvarianti che però adesso
non saranno le costanti $\tensor{\Lambda}{^{\alpha}_{\beta}}$ ma, in generale,
delle funzioni delle coordinate.  Rispetto al caso delle relatività speciale
bisognerà allora effettuare le sostituzioni
\begin{subequations}
  \label{eq:sostituzioni}
  \begin{align}
    \tensor{\Lambda}{^{\alpha}_{\beta}} &\to \parder{x'^{\alpha}}{x^{\beta}}, \\
    \tensor{\Lambda}{_{\beta}^{\alpha}} &\to \parder{x^{\alpha}}{x'^{\beta}}.
  \end{align}
\end{subequations}

Le uniche ipotesi che facciamo sulla trasformazione delle coordinate
$x^{\mu} \to x'^{\mu}$ è che sia rappresentata da una funzione biunivoca,
differenziabile (in modo che abbia senso calcolare la matrice jacobiana
$\lparder{x'^{\mu}}{x^{\mu}}$ della trasformazione), invertibile e con inverso a
sua volta differenziabile.  La trasformazione deve quindi avere determinante
jacobiano $\abs{\lparder{x'^{\mu}}{x^{\mu}}}$ non nullo, cioè non singolare, e
questo ci permetterà di definire anche nel presente contesto gli pseudotensori.

\section{Tensori}
\label{sec:tensori}

I differenziali delle coordinate controvarianti si trasformano, per effetto di
una trasformazione arbitraria delle coordinate $x^{\alpha} \to x'^{\alpha}$,
come
\begin{equation}
  \dd x'^{\alpha} = \parder{x'^{\alpha}}{x^{\mu}} \dd x^{\mu}.
\end{equation}
Un \emph{tensore}
$\tensor*{A}{^{\alpha_{1}\dots\alpha_{p}}_{\beta_{1}\dots\beta_{q}}}$ (con
$p + q = k$) di rango $k$ è un insieme di $4^{k}$ quantità che sotto una
trasformazione arbitraria delle coordinate $x^{\alpha} \to x'^{\alpha}$ si
trasformano come il differenziale delle coordinate controvarianti
\begin{equation}
  \tensor*{A}{^{\alpha_{1}\dots\alpha_{p}}_{\beta_{1}\dots\beta_{q}}} \to
  \tensor*{{A'}}{^{\alpha_{1}\dots\alpha_{p}}_{\beta_{1}\dots\beta_{q}}} =
  \parder{x'^{\alpha_{1}}}{x^{\mu_{1}}}
  \cdots \parder{x'^{\alpha_{p}}}{x^{\mu_{p}}} \parder{x^{\nu_{1}}}{x'^{\beta_{1}}}
  \cdots \parder{x^{\nu_{q}}}{x'^{\beta_{q}}}
  \tensor*{A}{^{\mu_{1}\dots\mu_{p}}_{\nu_{1}\dots\nu_{q}}}.
\end{equation}
Il tensore $\tensor*{A}{^{\alpha_{1}\dots\alpha_{p}}_{\beta_{1}\dots\beta_{q}}}$
è un tensore misto $p$ volte controvariante e $q$ volte covariante, oppure di
tipo $(p,q)$.  Un tensore di rango $k$ completamente controvariante è del tipo
$\tensor{A}{^{\alpha_{1}\dots\alpha_{k}}}$ e si trasforma come
\begin{equation}
  \tensor{A}{^{\alpha_{1}\dots\alpha_{k}}} \to
  \tensor{{A'}}{^{\alpha_{1}\dots\alpha_{k}}} =
  \parder{x'^{\alpha_{1}}}{x^{\mu_{1}}}
  \cdots \parder{x'^{\alpha_{k}}}{x^{\mu_{k}}}
  \tensor{A}{^{\mu_{1}\dots\mu_{k}}}.
\end{equation}
Un tensore di rango $k$ completamente covariante è del tipo
$\tensor{A}{_{\alpha_{1}\dots\alpha_{k}}}$ e si trasforma come
\begin{equation}
  \tensor{A}{_{\alpha_{1}\dots\alpha_{k}}} \to
  \tensor{{A'}}{_{\alpha_{1}\dots\alpha_{k}}} =
  \parder{x^{\mu_{1}}}{x'^{\alpha_{1}}}
  \cdots \parder{x^{\mu_{k}}}{x'^{\alpha_{k}}}
  \tensor{A}{_{\mu_{1}\dots\mu_{k}}}.
\end{equation}
In particolare, i tensori di rango $0$, che quindi non hanno indici, sono
chiamati \index{scalare}\emph{scalari} e sono invarianti sotto trasformazioni
arbitrarie di coordinate, i tensori di rango $1$ sono i \emph{vettori}.
Distinguiamo fra vettori controvarianti $V^{\alpha}$, i quali si trasformano,
per cambianti di coordinate $x^{\alpha} \to x'^{\alpha}$, come
\begin{equation}
  V^{\alpha} \to V'^{\alpha} = \parder{x'^{\alpha}}{x^{\mu}}V^{\mu},
\end{equation}
e vettori covarianti $U_{\alpha}$, che si trasformano come
\begin{equation}
  U_{\alpha} \to U'_{\alpha} = \parder{x^{\mu}}{x'^{\alpha}} U_{\mu}.
\end{equation}
Le trasformazioni inverse per i vettori controvarianti e covarianti sono
\begin{gather}
  V'^{\alpha} \to V^{\alpha} = \parder{x^{\alpha}}{x'^{\mu}} V'^{\mu}, \\
  U'_{\alpha} \to U_{\alpha} = \parder{x'^{\mu}}{x^{\alpha}} U'_{\mu}.
\end{gather}

Uno \index{pseudotensore}pseudotensore (o \emph{densità tensoriale})
$\tensor*{A}{^{\alpha_{1}\dots\alpha_{p}}_{\beta_{1}\dots\beta_{q}}}$ (con
$p+q=k$) di rango $k$ e peso $w$ è un insieme di $4^{k}$ quantità che per
effetto di un'arbitraria trasformazione delle coordinate
$x^{\alpha} \to x'^{\alpha}$ si trasformano come
\begin{equation}
  \tensor*{A}{^{\alpha_{1}\dots\alpha_{p}}_{\beta_{1}\dots\beta_{q}}} \to
  \tensor*{{A'}}{^{\alpha_{1}\dots\alpha_{p}}_{\beta_{1}\dots\beta_{q}}}
  = \abs*{\parder{x'}{x}}^{w} \parder{x'^{\alpha_{1}}}{x^{\mu_{1}}}
  \cdots \parder{x'^{\alpha_{p}}}{x^{\mu_{p}}} \parder{x^{\nu_{1}}}{x'^{\beta_{1}}}
  \cdots \parder{x^{\nu_{q}}}{x'^{\beta_{q}}}
  \tensor*{A}{^{\mu_{1}\dots\mu_{p}}_{\nu_{1}\dots\nu_{q}}},
\end{equation}
in cui $\abs{\lparder{x'}{x}}$ è lo jacobiano della trasformazione delle
coordinate.

La \index{delta!di Kronecker quadrimensionale}delta di Kronecker definita
nella~\eqref{eq:delta-kronecker} è l'unico tensore le cui componenti sono uguali
in qualsiasi sistema di riferimento, oltre agli \index{scalare}scalari e al
\index{tensore!nullo}tensore nullo.  Poiché anche nella relatività generale il
tensore nullo rimane invariato in ogni sistema di riferimento, è valido anche in
questo contesto il fatto che
\emph{se due tensori, dello stesso tipo, sono uguali in un sistema di
  riferimento, saranno uguali in qualsiasi sistema di riferimento ottenuto dal
  primo mediante una trasformazione arbitraria delle coordinate}.
Pertanto
\emph{una legge fisica espressa sotto forma di identità tensoriale è
  automaticamente generalmente covariante}.

Si noti che adesso le coordinate controvarianti non costituiscono più un
vettore.  Infatti, dopo una trasformazione generica, le nuove coordinate
$x'^{\alpha} = x'^{\alpha}(x)$ sono una funzione delle vecchie coordinate
$x^{\alpha} = x^{\alpha}(x')$.  Questa funzione è del tutto arbitraria in quanto
dipende dalla trasformazione, solo nel caso di trasformazioni lineari si ha
$x'^{\alpha} = \tensor{m}{^{\alpha}_{\beta}} x^{\beta}$, con
$\tensor{m}{^{\alpha}_{\beta}}$ matrice costante, e si può anche scrivere
\begin{equation}
  x'^{\alpha} = \parder{x'^{\alpha}}{x^{\beta}} x^{\beta}.
\end{equation}
Questo è proprio il caso delle trasformazioni di Lorentz.

\subsection{Tensore metrico}
\label{sec:tensore-metrico}

In un generico sistema di coordinate $x^{\mu}$, il
\index{tensore!metrico}tensore metrico è dato per definizione da
\begin{equation}
  g_{\mu\nu} =
  \eta_{\alpha\beta} \parder{\xi^{\alpha}}{x^{\mu}} \parder{\xi^{\beta}}{x^{\nu}},
\end{equation}
in cui $\xi^{\alpha}$ è un sistema di coordinate localmente inerziale.  In un
altro sistema di coordinate $x'^{\mu}$ risulta
\begin{equation}
  \label{eq:trasf-tensore-metrico}
  \begin{split}
    g'_{\mu\nu} &=
    \eta_{\alpha\beta} \parder{\xi^{\alpha}}{x'^{\mu}}
    \parder{\xi^{\beta}}{x'^{\nu}} =
    \eta_{\alpha\beta} \parder{\xi^{\alpha}}{x^{\rho}}
    \parder{x^{\rho}}{x'^{\mu}} \parder{\xi^{\beta}}{x^{\sigma}}
    \parder{x^{\sigma}}{x'^{\nu}} \\
    &= g_{\rho\sigma} \parder{x^{\rho}}{x'^{\mu}} \parder{x^{\sigma}}{x'^{\nu}}.
  \end{split}
\end{equation}
Dunque si tratta effettivamente di un tensore covariante di rango $2$.
Osserviamo che, a differenza del caso dello spazio di Minkowski piatto, nello
spazio-tempo curvo della relatività generale il \index{tensore!metrico}tensore
metrico $g_{\mu\nu}$ non è invariante per trasformazioni delle coordinate.  Nel
capitolo precedente abbiamo già introdotto il tensore metrico controvariante
$g^{\mu\nu}$ come quel tensore tale che
\begin{equation}
  g_{\mu\rho}g^{\mu\sigma} = \tensor{g}{^{\sigma}_{\rho}} =
  \tensor{\delta}{^{\sigma}_{\rho}}.
\end{equation}
Si dimostra che $g^{\mu\nu}$ è effettivamente un tensore controvariante di rango
$2$.

L'opposto del determinante della matrice che rappresenta il tensore metrico
\begin{equation}
  g = -\det(g_{\mu\nu})
\end{equation}
è uno pseudoscalare di peso $-2$.  Infatti, la
relazione~\eqref{eq:trasf-tensore-metrico} può essere scritta in forma
matriciale come
\begin{equation}
  g'_{\mu\nu} = \bigg( \parder{x}{x'}\bigg)^{\textup{T}} g_{\mu\nu}
  \bigg( \parder{x}{x'} \bigg),
\end{equation}
da cui, passando ai determinanti,
\begin{equation}
  \det(g'_{\mu\nu}) = \det(g_{\mu\nu}) \abs*{\parder{x}{x'}}^{2} \implies g' = g
  \abs*{\parder{x}{x'}}^{2} = g\abs*{\parder{x'}{x}}^{-2}.
\end{equation}
In particolare, considerando una trasformazione da un sistema di riferimento
inerziale, con $g_{\mu\nu} = \eta_{\mu\nu}$, a uno arbitrario abbiamo
\begin{equation}
  \det(g'_{\mu\nu}) = \det(\eta_{\mu\nu})\abs*{\parder{x}{x'}}^{2} = -
  \abs*{\parder{x}{x'}}^{2} < 0.
\end{equation}
Quindi si ha sempre $g > 0$ e ha senso considerare la sua radice quadrata
$\sqrt{g}$.  Poiché $g$ è uno pseudoscalare di peso $-2$, $\sqrt{g}$ è uno
pseudoscalare di peso $-1$
\begin{equation}
  \sqrt{g'} = \sqrt{g}\abs*{\parder{x'}{x}}^{-1}.
\end{equation}
Possiamo usare $\sqrt{g}$ per costruire un elemento infinitesimo di ipervolume
quadrimensionale invariante per arbitrarie trasformazione di coordinate.
Osserviamo che, come nella relatività
speciale,\footnote{Nella relatività speciale $\dd^{4} x$ può essere considerato
  uno scalare di Lorentz vero e proprio solo se ci si limita a considerare
  unicamente trasformazioni di Lorentz proprie, per le quali $\det \Lambda =
  1$.} $\dd^{4} x$ è uno pseudoscalare di peso $1$
\begin{equation}
  \dd^{4} x' = \dd^{4} x \abs*{\parder{x'}{x}}.
\end{equation}
Dunque l'elemento infinitesimo di ipersuperficie invariante è
$\sqrt{g}\dd^{4} x$
\begin{equation}
  \sqrt{g'}\dd^{4} x' = \sqrt{g}\abs*{\parder{x'}{x}}^{-1} \dd^{4} x
  \abs*{\parder{x'}{x}} = \sqrt{g}\dd^{4} x.
\end{equation}

Anche nella relatività generale il \index{tensore!metrico}tensore metrico può
essere usato per innalzare a abbassare gli indici dei tensori, per esempio
\begin{gather}
  V^{\alpha} \to V_{\alpha} = g_{\alpha\beta} V^{\beta}, \\
  U_{\alpha} \to U^{\alpha} = g^{\alpha\beta} U_{\beta}.
\end{gather}

\section{Operazioni con i tensori}
\label{sec:operazioni-tensori}

In questo paragrafo riporteremo le operazioni, già analizzate nel
paragrafo~\ref{sec:operazioni-tensori-minkowski}, che si possono effettuare con
i tensori anche all'interno della relatività generale.  Ometteremo le
dimostrazioni delle proprietà che sono state già svolte per lo spazio di
Minkowski poiché è sufficiente effettuare le
sostituzioni~\eqref{eq:sostituzioni} e usare il tensore metrico $g_{\mu\nu}$ al
posto del tensore di Minkowski $\eta_{\mu\nu}$.

\begin{description}
\item[Combinazione lineare] La combinazione lineare
  \begin{equation}
    a \tensor*{R}{^{\alpha_{1}\dots\alpha_{p}}_{\beta_{1}\dots\beta_{q}}} + b
    \tensor*{S}{^{\alpha_{1}\dots\alpha_{p}}_{\beta_{1}\dots\beta_{q}}},
  \end{equation}
  con $a$ e $b$ scalari, di due tensori dello stesso tipo è un tensore
  $\tensor*{T}{^{\alpha_{1}\dots\alpha_{p}}_{\beta_{1}\dots\beta_{q}}}$ ancora
  dello stesso tipo.
\item[Prodotto diretto] Il \index{prodotto!diretto}prodotto
  $\tensor*{A}{^{\alpha_{1}\dots\alpha_{p}}_{\beta_{1}\dots\beta_{q}}}
  \tensor*{B}{^{\alpha_{1}\dots\alpha_{r}}_{\beta_{1}\dots\beta_{s}}}$
  di un tensore di tipo $(p,q)$ e di uno di tipo $(r,s)$ è un tensore
  $\tensor*{T}{^{\alpha_{1}\dots\alpha_{p+r}}_{\beta_{1}\dots\beta_{q+s}}}$ di
  tipo $(p+r,q+s)$.
\item[Contrazione degli indici] \index{contrazione!degli indici}Se in un tensore
  di tipo $(p,q)$ si contraggono $n$ indici controvarianti con $n$ indici
  covarianti si ottiene un tensore di tipo $(p-n,q-n)$.  La contrazione di un
  tensore di rango superiore a $2$ non è univoca, si può scegliere di contrarre
  differenti coppie di indici ottenendo di volta in volta un risultato, in
  generale, diverso.  Ciascuna operazione di contrazione si effettua
  moltiplicando per il tensore metrico che ha per indici i due indici che si
  vogliono contrarre.  Due indici contratti possono essere scambiati di posto:
  l'indice covariante può diventare controvariante e, allo stesso tempo,
  l'indice controvariante diventare covariante, senza modificare il risultato.
  Non è invece possibile abbassare o alzare solo uno dei due indici muti.  Si
  può effettuare la contrazione degli indici anche nel prodotto di due tensori:
  il prodotto di un tensore di tipo $(p,q)$ e di uno di tipo $(r,s)$ è un
  tensore di tipo $(p+r,q+s)$, ma se si contraggono $n$ indici si ottiene un
  tensore di tipo $(p+r-n,q+s-n)$.
\item[Prodotto scalare] Il prodotto scalare fra due vettori $V$ e $U$ è definito
  come
  \begin{equation}
    V\cdot U = V^{\alpha}U_{\alpha} = g_{\alpha\beta}V^{\alpha}U^{\beta}.
  \end{equation}
  Il prodotto scalare di due vettori è un caso particolare di contrazione degli
  indici nel prodotto fra due tensori.  Dunque il loro prodotto è un tensore di
  rango $0$, cioè uno scalare e quindi una quantità che non varia per effetto di
  una trasformazione delle coordinate.
\end{description}

Non abbiamo ricordato fra le operazioni la derivazione poiché la derivazione
ordinaria nella relatività generale non ha le stesse importanti proprietà che
possiede nella relatività speciale.  Approfondiremo questo discorso nel
paragrafo~\ref{sec:derivazione-covariante}.

\section{Connessione affine}
\label{sec:connessione-affine}

Abbiamo definito la \index{connessione!affine}connessione affine come
\begin{equation}
  \tensor{\Gamma}{^{\lambda}_{\mu\nu}}
  = \parder{x^{\lambda}}{\xi^{\alpha}} \parder{\xi^{\alpha}}{x^{\mu},x^{\nu}},
\end{equation}
in cui $x^{\alpha}$ è un generico sistema di coordinate e
$\xi^{\alpha} = \xi^{\alpha}(x^{\alpha})$ un sistema localmente inerziale.
Considerando un altro sistema di coordinate $x'^{\mu} = x'^{\mu}(\xi^{\alpha})$
abbiamo
\begin{equation}
  \begin{split}
    \tensor{{\Gamma{}'}}{^{\lambda}_{\mu\nu}}
    &= \parder{x'^{\lambda}}{\xi^{\alpha}} \parder{}{x'^{\mu}}
    \bigg( \parder{\xi^{\alpha}(x)}{x'^{\nu}} \bigg)
    = \parder{x'^{\lambda}}{x^{\rho}} \parder{x^{\rho}}{\xi^{\alpha}}
    \parder{}{x'^{\mu}} \bigg( \parder{x^{\sigma}}{x'^{\nu}}
    \parder{\xi^{\alpha}}{x^{\sigma}} \bigg) \\
    &=  \parder{x'^{\lambda}}{x^{\rho}} \parder{x^{\rho}}{\xi^{\alpha}}
    \bigg( \parder{x^{\sigma}}{x'^{\nu}}
    \parder{\xi^{\alpha}}{x'^{\mu},x^{\sigma}} +
    \parder{x^{\sigma}}{x'^{\mu},x'^{\nu}} \parder{\xi^{\alpha}}{x^{\sigma}}
    \bigg) \\
    &= \parder{x'^{\lambda}}{x^{\rho}} \parder{x^{\rho}}{\xi^{\alpha}}
    \bigg( \parder{x^{\sigma}}{x'^{\nu}} \parder{x^{\tau}}{x'^{\mu}}
    \parder{\xi^{\alpha}}{x^{\tau},x^{\sigma}} +
    \parder{x^{\sigma}}{x'^{\mu},x'^{\nu}} \parder{\xi^{\alpha}}{x^{\sigma}}
    \bigg) \\
    &= \parder{x'^{\lambda}}{x^{\rho}} \parder{x^{\tau}}{x'^{\mu}}
    \parder{x^{\sigma}}{x'^{\nu}} \tensor{\Gamma}{^{\rho}_{\tau\sigma}}
    + \parder{x'^{\lambda}}{x^{\rho}} \parder{x^{\rho}}{x'^{\mu},x'^{\nu}}.
  \end{split}
\end{equation}
Questa relazione mostra che la connessione affine non è un tensore: se non ci
fosse il secondo termine nell'ultimo membro avremmo la formula di trasformazione
di un tensore di tipo $(1,2)$.

Riscriviamo il secondo termine dell'ultimo membro dell'equazione precedente in
una forma diversa che ci sarà utile in seguito.
% verrà usata nel paragrafo sulla derivazione covariante, *non* cancellare
% questo capoverso!
Derivando rispetto a $x'^{\mu}$
il primo e l'ultimo membro dell'identità
\begin{equation}
  \tensor{\delta}{^{\lambda}_{\nu}} = \parder{x'^{\lambda}}{x'^{\nu}}
  = \parder{x'^{\lambda}}{x^{\rho}} \parder{x^{\rho}}{x'^{\nu}}
\end{equation}
abbiamo
\begin{equation}
  0 = \parder{\tensor{\delta}{^{\lambda}_{\nu}}}{x'^{\mu}}
  = \parder{x'^{\lambda}}{x^{\rho}} \parder{x^{\rho}}{x'^{\mu}, x'^{\nu}}
  + \parder{x^{\rho}}{x'^{\nu}} \parder{x^{\sigma}}{x'^{\mu}}
  \parder{x'^{\lambda}}{x^{\rho},x^{\sigma}}.
\end{equation}
In questo modo la trasformazione della \index{connessione!affine}connessione
affine può essere riscritta come
\begin{equation}
  \tensor{{\Gamma{}'}}{^{\lambda}_{\mu\nu}}
  = \parder{x'^{\lambda}}{x^{\rho}} \parder{x^{\tau}}{x'^{\mu}}
  \parder{x^{\sigma}}{x'^{\nu}} \tensor{\Gamma}{^{\rho}_{\tau\sigma}}
  - \parder{x^{\rho}}{x'^{\nu}} \parder{x^{\sigma}}{x'^{\mu}}
  \parder{x'^{\lambda}}{x^{\rho},x^{\sigma}}.
\end{equation}

% La dimostrazione seguente è presa da Moshe Carmeli "Relativity: Modern
% Large-Scale Spacetime Structure of the Cosmos".  Vedi
% http://books.google.it/books?id=tZ2ilHyrkGUC&pg=PA110&lpg=PA110&dq=%22difference+between+two+affine+connections%22&source=bl&ots=Y1irJ0ao0z&sig=d2qZCZDSHPuD5-ykQM6C6hnXufQ&hl=it&sa=X&ei=Qn2mT87ALKzP4QTaluG_CQ&ved=0CE4Q6AEwAQ#v=onepage&q=%22difference%20between%20two%20affine%20connections%22&f=false
La connessione affine non è un tensore, ma la
\index{connessione!affine!differenza fra due connessioni affini}differenza fra
due connessioni affini è un tensore.  Infatti, siano
$\tensor{\Gamma}{^{\lambda}_{\mu\nu}}$ e
$\tensor{\tilde{\Gamma}}{^{\lambda}_{\mu\nu}}$ due distinte connessioni affini,
allora per un cambiamento di coordinate $x^{\mu} \to x'^{\mu}$ la loro
differenza
$\tensor{T}{^{\lambda}_{\mu\nu}} = \tensor{\Gamma}{^{\lambda}_{\mu\nu}} -
\tensor{\tilde{\Gamma}}{^{\lambda}_{\mu\nu}}$
si trasforma come un tensore di tipo $(1,2)$
\begin{equation}
  \begin{split}
    \tensor{T}{^{\lambda}_{\mu\nu}} \to \tensor{{T'}}{^{\lambda}_{\mu\nu}}
    &= \parder{x'^{\lambda}}{x^{\rho}} \parder{x^{\tau}}{x'^{\mu}}
    \parder{x^{\sigma}}{x'^{\nu}} \tensor{\Gamma}{^{\rho}_{\tau\sigma}}
    + \parder{x'^{\lambda}}{x^{\rho}} \parder{x^{\rho}}{x'^{\mu},x'^{\nu}} \\
    &- \parder{x'^{\lambda}}{x^{\rho}} \parder{x^{\tau}}{x'^{\mu}}
    \parder{x^{\sigma}}{x'^{\nu}} \tensor{\tilde{\Gamma}}{^{\rho}_{\tau\sigma}}
    - \parder{x'^{\lambda}}{x^{\rho}} \parder{x^{\rho}}{x'^{\mu},x'^{\nu}} \\
    &= \parder{x'^{\lambda}}{x^{\rho}} \parder{x^{\tau}}{x'^{\mu}}
    \parder{x^{\sigma}}{x'^{\nu}} \tensor{\Gamma}{^{\rho}_{\tau\sigma}}
    - \parder{x'^{\lambda}}{x^{\rho}} \parder{x^{\tau}}{x'^{\mu}}
    \parder{x^{\sigma}}{x'^{\nu}}\tensor{\tilde{\Gamma}}{^{\rho}_{\tau\sigma}}\\
    &= \parder{x'^{\lambda}}{x^{\rho}} \parder{x^{\tau}}{x'^{\mu}}
    \parder{x^{\sigma}}{x'^{\nu}} \tensor{T}{^{\rho}_{\tau\sigma}}.
  \end{split}
\end{equation}

\section{Derivazione covariante}
\label{sec:derivazione-covariante}

Nella relatività ristretta, la derivazione ordinaria di un tensore di rango $k$
rispetto a una coordinata $x^{\lambda}$ produce un tensore di rango $k+1$.
All'interno della teoria della relatività generale la situazione è differente:
la derivazione ordinaria di un tensore di rango $k$ per una coordinata
$x^{\lambda}$ produce un ente che non è, in generale, un tensore di rango $k+1$.
Per esempio, consideriamo un vettore controvariante $V^{\mu}$ il quale, per
definizione, si trasformerà, per effetto di un cambiamento di coordinate, come
\begin{equation}
  \label{eq:trasf-vettore}
  V^{\mu} \to V'^{\mu} = \parder{x'^{\mu}}{x^{\nu}} V^{\nu}.
\end{equation}
Derivando rispetto a $x'^{\lambda}$ abbiamo
\begin{equation}
  \begin{split}
    \parder{V'^{\mu}}{x'^{\lambda}}
    &= \parder{x'^{\mu}}{x^{\nu}} \parder{V^{\nu}}{x'^{\lambda}}
    + \parder{x'^{\mu}}{x'^{\lambda},x^{\nu}} V^{\nu} \\
    &= \parder{x'^{\mu}}{x^{\nu}} \parder{x^{\rho}}{x'^{\lambda}}
    \parder{V^{\nu}}{x^{\rho}}
    + \parder{x'^{\mu}}{x^{\nu},x^{\rho}} \parder{x^{\rho}}{x'^{\lambda}}
    V^{\nu}.
  \end{split}
\end{equation}
Se nell'ultimo membro ci fosse solo il primo termine avremmo che
$\lparder{V'^{\mu}}{x'^{\lambda}}$ sarebbe un tensore di rango $2$, tuttavia il
secondo termine è, in generale, diverso da zero.  Osserviamo che per
trasformazioni delle coordinate $x^{\mu} \to x'^{\mu}$ lineari, come le
trasformazioni di Lorentz, il secondo termine si annulla.

Vediamo come si trasforma per effetto di un cambiamento delle coordinate la
quantità $\tensor{\Gamma}{^{\mu}_{\nu\kappa}} V^{\kappa}$
\begin{equation}
  \begin{split}
    \tensor{{\Gamma{}'}}{^{\mu}_{\lambda\kappa}} V'^{\kappa} &=
    \bigg( \parder{x'^{\mu}}{x^{\nu}} \parder{x^{\sigma}}{x'^{\kappa}}
    \parder{x^{\rho}}{x'^{\lambda}} \tensor{\Gamma}{^{\nu}_{\rho\sigma}}
    -\parder{x^{\rho}}{x'^{\nu}} \parder{x^{\sigma}}{x'^{\kappa}}
    \parder{x'^{\mu}}{x^{\rho},x^{\sigma}} \bigg) \parder{x'^{\kappa}}{x^{\eta}}
    V^{\eta} \\
    &= \parder{x'^{\mu}}{x^{\nu}}
    \underbrace{\parder{x^{\sigma}}{x^{\eta}}}_{\tensor{\delta}{^{\sigma}_{\eta}}}
    \parder{x^{\rho}}{x'^{\lambda}} \tensor{\Gamma}{^{\nu}_{\rho\sigma}}
    V^{\eta} - \parder{x^{\rho}}{x'^{\nu}}
    \underbrace{\parder{x^{\sigma}}{x^{\eta}}}_{\tensor{\delta}{^{\sigma}_{\eta}}}
    \parder{x'^{\mu}}{x^{\rho},x^{\sigma}} V^{\eta} \\
    &= \parder{x'^{\mu}}{x^{\nu}} \parder{x^{\rho}}{x'^{\lambda}}
    \tensor{\Gamma}{^{\nu}_{\rho\sigma}} V^{\sigma}
    - \parder{x^{\rho}}{x'^{\nu}} \parder{x'^{\mu}}{x^{\rho},x^{\sigma}}
    V^{\sigma}.
  \end{split}
\end{equation}
Anche in questo caso, se nell'ultimo membro ci fosse solo il primo termine,
$\tensor{\Gamma}{^{\mu}_{\lambda\kappa}} V^{\kappa}$ si trasformerebbe come un
tensore misto di rango $2$, tuttavia il termine addizionale è opposto al termine
addizionale che compare nella trasformazione di
$\lparder{V^{\mu}}{x^{\lambda}}$, quindi la somma delle due quantità si
trasforma come un tensore misto di rango $2$, infatti
\begin{equation}
  \parder{V'^{\mu}}{x'^{\lambda}} + \tensor{{\Gamma{}'}}{^{\mu}_{\lambda\kappa}}
  V'^{\kappa} = \parder{x'^{\mu}}{x^{\nu}} \parder{x^{\rho}}{x'^{\lambda}}
  \bigg( \parder{V^{\nu}}{x^{\rho}} + \tensor{\Gamma}{^{\nu}_{\rho\sigma}}
  V'^{\sigma} \bigg).
\end{equation}
Definiamo allora la \index{derivata!covariante}\emph{derivata covariante}
$\tensor{V}{^{\mu}_{;\lambda}}$ di un vettore controvariante $V^{\mu}$ come
\begin{equation}
  \tensor{V}{^{\mu}_{;\lambda}} = \tensor{V}{^{\mu}_{,\lambda}} +
  \tensor{\Gamma}{^{\mu}_{\kappa\lambda}} V^{\kappa}
\end{equation}
e per un cambiamento delle coordinate $x^{\mu} \to x'^{\mu}$ si ha
\begin{equation}
  \tensor{V}{^{\mu}_{;\lambda}} \to \tensor{{V'}}{^{\mu}_{;\lambda}}
  = \parder{x'^{\mu}}{x^{\nu}} \parder{x^{\rho}}{x'^{\lambda}}
  \tensor{V}{^{\nu}_{;\rho}}
\end{equation}
cioè la derivata covariante di un vettore è un tensore di rango $2$, in analogia
alla derivazione ordinaria di un vettore nello spazio di Minkowski.

Ripetendo gli stessi calcoli per un vettore covariante si giunge a definire la
\index{derivata!covariante}derivata covariante $V_{\mu;\nu}$ di un vettore
covariante $V_{\mu}$ come
\begin{equation}
  V_{\mu;\nu} = V_{\mu,\nu} - \tensor{\Gamma}{^{\lambda}_{\mu\nu}} V_{\lambda}
\end{equation}
e per una trasformazione delle coordinate $x^{\mu} \to x'^{\mu}$ risulta
\begin{equation}
  V_{\mu;\nu} \to V'_{\mu;\nu}
  = \parder{x^{\lambda}}{x'^{\mu}} \parder{x^{\rho}}{x'^{\nu}} V_{\lambda;\rho}.
\end{equation}

Più in generale è possibile definire la \index{derivata!covariante}derivata
covariante di un tensore.  Vediamo il caso particolare di un tensore misto di
rango
tre\footnote{Presentiamo un metodo per ricordare la formula di derivazione
  covariante di un tensore di rango qualsiasi: il primo termine è la
  corrispondente derivazione ordinaria del tensore, i successivi sono del tipo
  $s\Gamma T$ cui $T$ ha, di volta in volta a rotazione, un solo indice muto e i
  restanti liberi, $\Gamma$ ha gli indici adeguati per saturare l'indice muto e
  compensare i due restanti indici liberi (di cui uno è sempre quello inferiore
  rispetto a cui si deriva, $\rho$ nell'esempio), $s$ vale $+1$ se l'indice muto
  di $T$ è controvariante, $-1$ se è covariante.}
\begin{subequations}
  \begin{align}
    \tensor{T}{^{\mu\sigma}_{\lambda;\rho}} &=
    \tensor{T}{^{\mu\sigma}_{\lambda,\rho}} + \tensor{\Gamma}{^{\mu}_{\rho\nu}}
    \tensor{T}{^{\nu\sigma}_{\lambda}} + \tensor{\Gamma}{^{\sigma}_{\rho\nu}}
    \tensor{T}{^{\mu\nu}_{\lambda}} - \tensor{\Gamma}{^{\nu}_{\rho\lambda}}
    \tensor{T}{^{\mu\sigma}_{\nu}}, \\
    \tensor{{T'}}{^{\mu\sigma}_{\lambda;\rho}}
    &= \parder{x'^{\mu}}{x^{\omega}} \parder{x'^{\sigma}}{x^{\tau}}
    \parder{x^{\alpha}}{x'^{\lambda}} \parder{x^{\beta}}{x'^{\rho}}
    \tensor{T}{^{\omega\tau}_{\alpha;\beta}}.
  \end{align}
\end{subequations}

Grazie alle proprietà algebriche dei tensori viste nel
paragrafo~\ref{sec:operazioni-tensori} si possono dimostrare le seguenti
proprietà della derivata covariante
\begin{enumerate}
\item la derivata covariante della combinazione lineare di tensori è uguale alla
  combinazione lineare delle derivate dei tensori
  \begin{equation}
    (\alpha \tensor*{A}{^{\mu_{1}\cdots\mu_{p}}_{\nu_{1}\cdots\nu_{q}}} + \beta
    \tensor*{B}{^{\mu_{1}\cdots\mu_{p}}_{\nu_{1}\cdots\nu_{q}}})_{;\lambda} =
    \alpha \tensor*{A}{^{\mu_{1}\cdots\mu_{p}}_{\nu_{1}\cdots\nu_{q};\lambda}} +
    \beta \tensor*{B}{^{\mu_{1}\cdots\mu_{p}}_{\nu_{1}\cdots\nu_{q};\lambda}},
  \end{equation}
  con $\alpha$ e $\beta$ scalari;
\item la derivata covariante del \index{prodotto!diretto}prodotto diretto di due
  tensori segue la \index{regola!di Leibniz}\emph{regola di Leibniz}
  \begin{equation}
    (\tensor*{A}{^{\mu_{1}\cdots\mu_{p}}_{\nu_{1}\cdots\nu_{q}}}
    \tensor*{B}{^{\mu_{1}\cdots\mu_{r}}_{\nu_{1}\cdots\nu_{s}}})_{;\lambda} =
    \tensor*{A}{^{\mu_{1}\cdots\mu_{p}}_{\nu_{1}\cdots\nu_{q};\lambda}}
    \tensor*{B}{^{\mu_{1}\cdots\mu_{r}}_{\nu_{1}\cdots\nu_{s}}} +
    \tensor*{A}{^{\mu_{1}\cdots\mu_{p}}_{\nu_{1}\cdots\nu_{q}}}
    \tensor*{B}{^{\mu_{1}\cdots\mu_{r}}_{\nu_{1}\cdots\nu_{s};\lambda}};
  \end{equation}
\item la derivata covariante della \index{contrazione!degli indici}contrazione
  di un tensore è uguale alla contrazione della derivata covariante del
  tensore.  Per esempio
  \begin{equation}
    \tensor{T}{^{\mu\lambda}_{\lambda;\rho}} =
    \tensor{T}{^{\mu\lambda}_{\lambda,\rho}} +
    \tensor{\Gamma}{^{\mu}_{\sigma\rho}}\tensor{T}{^{\sigma\lambda}_{\lambda}}.
  \end{equation}
  Infatti
  \begin{equation}
    \begin{split}
      \tensor{T}{^{\mu\lambda}_{\lambda;\rho}} &=
      \tensor{T}{^{\mu\lambda}_{\lambda,\rho}} +
      \tensor{\Gamma}{^{\mu}_{\sigma\rho}}\tensor{T}{^{\sigma\lambda}_{\lambda}}
      + \tensor{\Gamma}{^{\lambda}_{\sigma\rho}}
      \tensor{T}{^{\mu\sigma}_{\lambda}} -
      \tensor{\Gamma}{^{\sigma}_{\lambda\rho}}
      \tensor{T}{^{\mu\lambda}_{\sigma}} \\
      &= \tensor{T}{^{\mu\lambda}_{\lambda,\rho}} +
      \tensor{\Gamma}{^{\mu}_{\sigma\rho}}\tensor{T}{^{\sigma\lambda}_{\lambda}}
      + \tensor{\Gamma}{^{\sigma}_{\lambda\rho}}
      \tensor{T}{^{\mu\lambda}_{\sigma}} -
      \tensor{\Gamma}{^{\sigma}_{\lambda\rho}}
      \tensor{T}{^{\mu\lambda}_{\sigma}} \\
      &= \tensor{T}{^{\mu\lambda}_{\lambda,\rho}} +
      \tensor{\Gamma}{^{\mu}_{\sigma\rho}}\tensor{T}{^{\sigma\lambda}_{\lambda}}
    \end{split}
  \end{equation}
  in cui nel penultimo passaggio abbiamo scambiato gli indici muti $\lambda$ e
  $\sigma$ del terzo termine.
\end{enumerate}

La derivata covariante del \index{tensore!metrico}tensore metrico $g_{\mu\nu}$ è
nulla poiché in un sistema di coordinate localmente inerziale il tensore metrico
è uguale al \index{tensore!metrico di Minkowski}tensore di Minkowski
$\eta_{\mu\nu}$ e la derivata covariante del tensore di Minkowski è uguale al
tensore nullo.  Poiché il tensore nullo non cambia per trasformazioni arbitrarie
delle coordinate, la derivata covariante del tensore metrico è nulla in
qualsiasi sistema di coordinate.  Formalmente questo può essere fatto vedere
calcolando direttamente la derivata covariante di $g_{\mu\nu}$
\begin{equation}
  g_{\mu\nu;\lambda} = g_{\mu\nu,\lambda} -
  \tensor{\Gamma}{^{\rho}_{\lambda\mu}}g_{\rho\nu} -
  \tensor{\Gamma}{^{\rho}_{\lambda\nu}}g_{\rho\mu}
\end{equation}
e dalla~\eqref{eq:foo} abbiamo proprio $g_{\mu\nu;\lambda} = 0$.  Questo
risultato significa che l'operazione di abbassamento degli indici (o
innalzamento nel caso di $g^{\mu\nu}$) e derivazione covariante commutano, cioè
\begin{subequations}
  \begin{gather}
    (g_{\mu\nu}V^{\nu})_{;\lambda} = g_{\mu\nu;\lambda}V^{\nu} +
    g_{\mu\nu}\tensor{V}{^{\nu}_{;\lambda}} =
    g_{\mu\nu}\tensor{V}{^{\nu}_{;\lambda}}, \\
    (g^{\mu\nu}V_{\nu})_{;\lambda} = \tensor{g}{^{\mu\nu}_{;\lambda}}V_{\nu} +
    g^{\mu\nu}V_{\nu;\lambda} = g^{\mu\nu}V_{\nu;\lambda}.
  \end{gather}
\end{subequations}

La derivata covariante di uno scalare è uguale alla sua derivata ordinaria.
Infatti, sia $S$ uno scalare e $\tensor{V}{^{\mu}_{\nu}}$ un tensore tale che la
sua contrazione $\tensor{V}{^{\mu}_{\mu}}$ sia uguale a $S$, allora
\begin{equation}
  S_{;\lambda} = \tensor{V}{^{\mu}_{\mu;\lambda}} =
  \tensor{V}{^{\mu}_{\mu,\lambda}} + \tensor{\Gamma}{^{\mu}_{\sigma\lambda}}
  \tensor{V}{^{\sigma}_{\mu}} - \tensor{\Gamma}{^{\mu}_{\sigma\lambda}}
  \tensor{V}{^{\sigma}_{\mu}} = \tensor{V}{^{\mu}_{\mu,\lambda}} = S_{,\lambda}.
\end{equation}

Poiché la connessione affine è simmetrica rispetto ai suoi due indici inferiori,
il rotore covariante è uguale al rotore ordinario, infatti
\begin{equation}
  V_{\mu;\nu} - V_{\nu;\mu} = (V_{\mu,\nu} -
  \tensor{\Gamma}{^{\sigma}_{\mu\nu}}V_{\sigma}) - (V_{\nu,\mu} -
  \tensor{\Gamma}{^{\sigma}_{\nu\mu}} V_{\sigma}) = V_{\mu,\nu} - V_{\nu,\mu}.
\end{equation}

Calcoliamo la divergenza covariante $\tensor{V}{^{\mu}_{;\mu}}$ di un vettore
controvariante $V^{\mu}$
\begin{equation}
  \tensor{V}{^{\mu}_{;\mu}} = \tensor{V}{^{\mu}_{,\mu}} +
  \tensor{\Gamma}{^{\mu}_{\mu\lambda}} V^{\lambda}.
\end{equation}
Osserviamo che
\begin{equation}
  \tensor{\Gamma}{^{\mu}_{\mu\lambda}} =
  \frac{1}{2}(g^{\mu\rho}g_{\mu\rho,\lambda} + g^{\mu\rho}g_{\lambda\rho,\mu}
  - g^{\mu\rho}g_{\mu\lambda,\rho}) = \frac{1}{2}g^{\mu\rho}g_{\mu\rho,\lambda}.
\end{equation}
Inoltre si dimostrano le seguenti relazioni (vedi
l'appendice~\ref{sec:dimostr-determinante})
\begin{subequations}
  \label{eq:g_mu_rho-determinante}
  \begin{align}
    g^{\mu\rho} &= \frac{1}{g} \parder{g}{g_{\mu\rho}}, \\
    g_{\mu\rho} &= -
    \frac{1}{g} \parder{g}{g^{\mu\rho}}, \label{eq:g_mu_rho-determinante2}
  \end{align}
\end{subequations}
in cui $g = -\det(g_{\mu\nu})$, allora, usando in particolare la prima, possiamo
scrivere
\begin{equation}
  \tensor{\Gamma}{^{\mu}_{\mu\lambda}} = \frac{1}{2}
  \frac{1}{g} \parder{g}{g_{\mu\rho}} \parder{g_{\mu\rho}}{x^{\lambda}} =
  \frac{1}{2} \frac{1}{g} \parder{g}{x^{\lambda}} = \frac{1}{2} \parder{\ln
    g}{x^{\lambda}} = \parder{\ln \sqrt{g}}{x^{\lambda}} =
  \frac{1}{\sqrt{g}} \parder{\sqrt{g}}{x^{\lambda}}
\end{equation}
e quindi la divergenza covariante è
\begin{equation}
  \tensor{V}{^{\mu}_{;\mu}} = \parder{V^{\mu}}{x^{\mu}} +
  \tensor{\Gamma}{^{\mu}_{\mu\lambda}} V^{\lambda} = \parder{V^{\mu}}{x^{\mu}} +
  \frac{1}{\sqrt{g}} \parder{\sqrt{g}}{x^{\lambda}}V^{\lambda} =
  \frac{1}{\sqrt{g}} \parder{}{x^{\mu}}(\sqrt{g}V^{\mu}).
\end{equation}
Questo risultato ci permette di scrivere il \index{teorema!di Gauss}teorema di
Gauss in forma covariante, ricordando che l'elemento invariante infinitesimo di
volume quadrimensionale è $\sqrt{g}\dd^{4} x$,
\begin{equation}
  \label{eq:gauss-covariante}
  \int\limits_{\Omega} \tensor{V}{^{\mu}_{;\mu}} \sqrt{g} \dd^{4} x =
  \int\limits_{\Omega} \frac{1}{\sqrt{g}} \parder{}{x^{\mu}}(\sqrt{g}V^{\mu})
  \sqrt{g} \dd^{4} x = \int\limits_{\partial\Omega} V^{\mu} \sqrt{g}
  \dd\Sigma_{\mu}.
\end{equation}

\subsection{Differenziale covariante}
\label{sec:differenziale-covariante}

Abbiamo visto che la derivata ordinaria $\lparder{V^{\mu}}{x^{\lambda}}$ di un
vettore non è un tensore di rango $2$, in quanto il differenziale ordinario
$\dd V^{\mu}$ di un vettore $V^{\mu}$ non è un vettore, come invece succede
nello spazio piatto di Minkowski.  Infatti dalla~\eqref{eq:trasf-vettore}
abbiamo
\begin{equation}
  \dd V^{\mu} \to \dd V'^{\mu} = \parder{x'^{\mu}}{x^{\nu}} \dd V^{\nu}
  + \parder{x'^{\mu}}{x^{\nu},x^{\lambda}} V^{\nu} \dd x^{\lambda}
\end{equation}
e solo nel caso di trasformazioni lineari delle coordinate il secondo termine si
annulla e il differenziale si trasforma come un vettore.  Il motivo di ciò
risiede nel fatto che il differenziale di un vettore è dato dalla differenza dei
vettori $V^{\mu}(x + \dd x)$ e $V^{\mu}(x)$ valutati in due punti diversi dello
spazio e le leggi di trasformazione dei tensori in uno spazio curvo, come quello
della relatività generale, dipendono dalla posizione
\begin{equation}
  \begin{split}
    \dd V'^{\mu} &= V'^{\mu}(x' + \dd x') - V'^{\mu}(x') \\
    &= \bigg( \parder{x'^{\mu}}{x^{\nu}} \bigg) \bigg|_{x' + \dd x'}V^{\nu}(x +
    \dd x) - \bigg( \parder{x'^{\mu}}{x^{\nu}} \bigg) \bigg|_{x'} V^{\nu}(x).
  \end{split}
\end{equation}

Nasce allora l'esigenza di definire un differenziale di un vettore (o, più in
generale, di un tensore) che abbia a sua volta natura vettoriale (o tensoriale).
Per ovviare al problema della diversa posizione in cui vengono valutati i
vettori $V^{\mu}(x + \dd x)$ e $V^{\mu}(x)$ possiamo pensare di ``trasportare''
il vettore $V^{\mu}(x)$ dal punto $x$ al punto $x + \dd x$, mantenendo lo stesso
modulo e la stessa direzione rispetto alla curva $x^{\lambda}$.  In questo modo
si può introdurre il
\index{differenziale!covariante}\emph{differenziale covariante} $\cdd V^{\mu}$
di un vettore $V^{\mu}$ definito come la differenza fra il vettore
$V^{\mu}(x + \dd x)$ valutato nel punto $x + \dd x$ e il vettore $V^{\mu}(x)$
\emph{trasportato parallelamente} nel punto $x + \dd x$ (indichiamo con
$V^{*\mu}(x + \dd x)$ il vettore trasportato parallelamente)
\begin{equation}
  \begin{split}
    \cdd V^{\mu} &= V^{\mu}(x + \dd x) - V^{*\mu}(x + \dd x) = V^{\mu}(x + \dd x)
    - (V^{\mu}(x) + \delta V^{\mu}) \\
    &= \dd V^{\mu} - \delta V^{\mu},
  \end{split}
\end{equation}
in cui $\delta V^{\mu}$ è la variazione del vettore dovuta al suo
\index{trasporto!parallelo}\emph{trasporto parallelo} dal punto $x^{\lambda}$ al
punto $x^{\lambda} + \dd x^{\lambda}$.  Questa variazione deve essere nulla in
uno spazio piatto.

Definiamo $\delta V^{\mu}$ in modo che $\cdd V^{\mu}$ sia un vettore
controvariante.  Per fare questo consideriamo un sistema di coordinate inerziali
$x'^{\mu}$ e imponiamo che $\cdd V^{\mu}$ si trasformi, per effetto di un
cambiamento di coordinate $x^{\mu} \to x'^{\mu}$, come un vettore, cioè
\begin{equation}
  \cdd V^{\mu} \to \cdd V'^{\mu} = \parder{x'^{\mu}}{x^{\nu}} \cdd V^{\nu}
  = \parder{x'^{\mu}}{x^{\nu}} \dd V^{\nu} - \parder{x'^{\mu}}{x^{\nu}} \delta
  V^{\nu}.
\end{equation}
D'altra parte nel sistema di coordinate $x'^{\mu}$ abbiamo
$\cdd V'^{\mu} = \dd V'^{\mu} - \delta V'^{\mu}$, ma poiché questo sistema di
coordinate è piatto la variazione $\delta V'^{\mu}$ deve essere nulla, cioè il
differenziale covariante si deve trasformare come il differenziale ordinario
\begin{equation}
  \cdd V'^{\mu} = \dd V'^{\mu} = \parder{x'^{\mu}}{x^{\nu}} \dd V^{\nu}
  + \parder{x'^{\mu}}{x^{\rho},x^{\lambda}} V^{\rho} \dd x^{\lambda}
\end{equation}
quindi, per confronto ricaviamo
\begin{equation}
  - \parder{x'^{\mu}}{x^{\nu}} \delta V^{\nu}
  = \parder{x'^{\mu}}{x^{\rho},x^{\lambda}} V^{\rho} \dd x^{\lambda}
\end{equation}
da cui
\begin{equation}
  \delta V^{\nu} =
  - \parder{x'^{\mu}}{x^{\rho},x^{\lambda}} \parder{x^{\nu}}{x'^{\mu}} V^{\rho}
  \dd x^{\lambda}.
\end{equation}
Riconosciamo che il fattore di fronte a $V^{\rho} \dd x^{\lambda}$ è la
connessione affine $\tensor{\Gamma}{^{\nu}_{\rho\lambda}}$, quindi possiamo
definire il \index{differenziale!covariante}\emph{differenziale covariante} di
un vettore controvariante come
\begin{equation}
  \cdd V^{\mu} = \dd V^{\mu} + \tensor{\Gamma}{^{\mu}_{\rho\lambda}} V^{\rho} \dd
  x^{\lambda}.
\end{equation}
Osserviamo inoltre che risulta
\begin{equation}
  \cdd V^{\mu} = \parder{V^{\mu}}{x^{\lambda}} \dd x^{\lambda} +
  \tensor{\Gamma}{^{\mu}_{\rho\lambda}} V^{\rho} \dd x^{\lambda} =
  \tensor{V}{^{\mu}_{;\lambda}} \dd x^{\lambda}.
\end{equation}

Consideriamo una curva $x^{\lambda}(\tau)$ parametrizzata dal tempo proprio
$\tau$.  Uno spostamento infinitesimo lungo tale curva è dato da
$\dd x^{\lambda} = (\lparder{x^{\lambda}}{\tau}) \dd \tau = u^{\lambda}\dd\tau$,
quindi il differenziale covariante di $V^{\mu}$ si può scrivere come
\begin{equation}
  \cdd V^{\mu} = \dd V^{\mu} + \tensor{\Gamma}{^{\mu}_{\rho\lambda}} V^{\rho}
  u^{\lambda }\dd \tau.
\end{equation}
Allora il limite del rapporto incrementale $\cdd V^{\mu}/\dd \tau$ definisce la
\index{derivata!lungo una curva}\emph{derivata lungo la curva}
$x^{\lambda}(\tau)$ di $V^{\mu}$ come
\begin{equation}
  \curder{V^{\mu}} = \tensor{V}{^{\mu}_{;\lambda}} \toder{x^{\lambda}}{\tau} =
  \toder{V^{\mu}}{\tau} + \tensor{\Gamma}{^{\mu}_{\rho\lambda}} V^{\rho}
  \toder{x^{\lambda}}{\tau}.
\end{equation}
La derivata lungo una curva è un vettore in quanto è la contrazione fra il
tensore $\tensor{V}{^{\mu}_{;\lambda}}$ e la quadrivelocità
$\ltoder{x^{\lambda}}{\tau} = u^{\lambda}$.

Poiché $\delta V^{\mu} = 0$ in un sistema inerziale, in un tale sistema si deve
anche annullare la \index{connessione!affine}connessione affine.  Questo fa
vedere di nuovo che $\tensor{\Gamma}{^{\mu}_{\rho\lambda}}$ non può essere un
tensore: un tensore nullo in un sistema di coordinate è nullo in ogni altro
sistema di coordinate.

% TODO: scrivere questa sezione.  Lezione del 27/04/2012
\subsubsection{\completare{Esempio}}
\label{sec:esempio}

% TODO: scrivere questa sezione.  Lezione del 27/04/2012.  Vedi anche
% Ohanian-Ruffini, pagina 351.  Ricordati di sottolineare la differenza fra
% Lense-Thirring e De Sitter: De Sitter è dovuto alla sola presenza di una
% massa, Lense-Thirring è causato dalla rotazione della massa.
\subsection{\completare{Effetto De Sitter}}
\label{sec:effetto-de-sitter}

\section{Applicazioni del principio di generale covarianza}
\label{sec:applicazioni-generale-covarianza}

Grazie al formalismo introdotto in questo capitolo possiamo adottare l'algoritmo
che ci apprestiamo a esporre per determinare quale sia l'effetto della gravità
sui sistemi fisici: si scrive l'equazione che descrive il sistema, in assenza di
gravità, con il formalismo della relatività speciale e
\begin{itemize}
\item si sostituisce il \index{tensore!metrico}tensore metrico $g_{\mu\nu}$ al
  \index{tensore!metrico di Minkowski}tensore metrico di Minkowski
  $\eta_{\mu\nu}$;
\item si sostituiscono le \index{derivata!covariante}derivate covarianti alle
  derivate ordinarie e i \index{differenziale!covariante}differenziali
  covarianti ai differenziali ordinari;
\item si utilizzano opportune potenze di $\sqrt{g}$ per saturare a $0$ i pesi
  degli pseudotensori.
\end{itemize}
L'equazione così ottenuta sarà generalmente covariante e valida in assenza di
gravità e quindi, per il \index{principio!di generale covarianza}principio di
generale covarianza, sarà valida anche in presenza di campi gravitazionali.  È
possibile fare ciò a patto che si lavori su scale sufficientemente piccole
rispetto al raggio d'azione del campo gravitazionale.

Questo algoritmo ci permette, per esempio, di ricavare in maniera rapida
l'equazione del moto di una particella soggetta solo a un campo gravitazionale.
In assenza di gravità la particella è libera, quindi la sua equazione del moto è
\begin{equation}
  \toder{u^{\mu}}{\tau} = 0,
\end{equation}
in cui $u^{\mu}$ è la \index{quadrivelocità}quadrivelocità della particella.  In
presenza del campo gravitazionale, secondo le prescrizioni precedenti,
l'equazione del moto si scrive come
\begin{equation}
  0 = \curder{u^{\mu}} = \toder{u^{\mu}}{\tau} +
  \tensor{\Gamma}{^{\mu}_{\lambda\nu}} u^{\lambda} \toder{x^{\nu}}{\tau} =
  \toder{u^{\mu}}{\tau} + \tensor{\Gamma}{^{\mu}_{\lambda\nu}} u^{\lambda}
  u^{\nu},
\end{equation}
che è l'\index{equazione!della geodetica}equazione della
geodetica~\eqref{eq:geodetica2}.

%%% Local Variables:
%%% mode: latex
%%% TeX-master: "../gravitazione"
%%% fill-column: 80
%%% TeX-PDF-mode: t
%%% End:

\cleardoublepage
\chapter{Equazioni del campo gravitazionale}
\label{cha:equazioni-campo-grav}

\section{Tensore di Riemann}
\label{sec:tensore-riemann}

\subsection{Definizione e proprietà algebriche}
\label{sec:definizione-proprietà-riemann}

Si può far vedere\footnote{Vedi~\textcite[133-135]{weinberg:gravitation}.} che
l'unico tensore ottenibile dal tensore metrico e dalle sue derivate prime e
seconde è il \emph{tensore di curvatura di Riemann-Christoffel} o,
semplicemente, \index{tensore!di Riemann}\emph{tensore di Riemann}
\begin{equation}
  \tensor{R}{^{\lambda}_{\mu\nu\kappa}}
  = \parder{\tensor{\Gamma}{^{\lambda}_{\mu\nu}}}{x^{\kappa}}
  - \parder{\tensor{\Gamma}{^{\lambda}_{\mu\kappa}}}{x^{\nu}} +
  \tensor{\Gamma}{^{\eta}_{\mu\nu}} \tensor{\Gamma}{^{\lambda}_{\eta\kappa}} -
  \tensor{\Gamma}{^{\eta}_{\mu\kappa}} \tensor{\Gamma}{^{\lambda}_{\eta\nu}}.
\end{equation}
Il tensore di Riemann è lineare nelle derivate seconde del tensore metrico.

Per analizzare le proprietà algebriche del tensore di Riemann conviene
considerare la sua forma completamente covariante
\begin{equation}
  \label{eq:riemann-covariante}
  \begin{split}
    R_{\lambda\mu\nu\kappa} &= g_{\lambda\sigma}
    \tensor{R}{^{\sigma}_{\mu\nu\kappa}} = g_{\lambda\sigma}
    \bigg( \parder{\tensor{\Gamma}{^{\lambda}_{\mu\nu}}}{x^{\kappa}}
    - \parder{\tensor{\Gamma}{^{\lambda}_{\mu\kappa}}}{x^{\nu}} +
    \tensor{\Gamma}{^{\eta}_{\mu\nu}} \tensor{\Gamma}{^{\lambda}_{\eta\kappa}} -
    \tensor{\Gamma}{^{\eta}_{\mu\kappa}}
    \tensor{\Gamma}{^{\lambda}_{\eta\nu}} \bigg) \\
    &= \frac{1}{2} \bigg( \parder{g_{\lambda\nu}}{x^{\kappa}, x^{\mu}}
    - \parder{g_{\mu\nu}}{x^{\kappa}, x^{\lambda}}
    - \parder{g_{\lambda\kappa}}{x^{\nu}, x^{\mu}}
    + \parder{g_{\mu\kappa}}{x^{\nu}, x^{\lambda}} \bigg) \\
    &+ g_{\eta\sigma} (\tensor{\Gamma}{^{\eta}_{\nu\lambda}}
    \tensor{\Gamma}{^{\sigma}_{\mu\kappa}} -
    \tensor{\Gamma}{^{\eta}_{\kappa\lambda}}
    \tensor{\Gamma}{^{\sigma}_{\mu\nu}}).
  \end{split}
\end{equation}
Questa relazione ci tornerà utile quando considereremo il limite newtoniano
delle equazioni di Einstein.  Da questa espressione si vede che
\begin{enumerate}
\item \label{item:antisimmetria-riemann} il tensore di Riemann è antisimmetrico
  rispetto allo scambio fra primo e secondo indice e allo scambio fra terzo e
  quarto indice
  \begin{equation}
    R_{\lambda\mu\nu\kappa} = -R_{\mu\lambda\nu\kappa} =
    -R_{\lambda\mu\kappa\nu} = R_{\mu\lambda\kappa\nu};
  \end{equation}
\item \label{item:simmetria-riemann} il tensore di Riemann
  $R_{\lambda\mu\nu\kappa}$ è simmetrico rispetto allo scambio fra la prima
  coppia indici $(\lambda,\mu)$ con la seconda coppia $(\nu,\kappa)$
  \begin{equation}
    R_{\lambda\mu\nu\kappa} = R_{\nu\kappa\lambda\mu};
  \end{equation}
\item \label{item:ciclicita-riemann} è valida la seguente relazione di ciclicità
  (si noti che il primo indice è tenuto fisso, si fanno ruotare gli altri tre)
  \begin{equation}
    \label{eq:ciclicita-riemann}
    R_{\lambda\mu\nu\kappa} + R_{\lambda\kappa\mu\nu} + R_{\lambda\nu\kappa\mu}
    = 0.
  \end{equation}
\end{enumerate}

A causa delle proprietà appena elencate, delle $4^{4} = 256$ componenti del
tensore di Riemann solo $20$ sono indipendenti.  Per far vedere ciò consideriamo
il caso generale di uno spazio $n$-dimensionale.  Il tensore di Riemann può
essere immaginato come una matrice quadrata simmetrica (per la proprietà di
simmetria~\ref{item:simmetria-riemann}) di ordine $d$ del tipo
$R_{(\lambda\mu)(\nu\kappa)}$, con indici $(\lambda\mu)$ e $(\nu\kappa)$.  Il
valore dell'ordine $d$ di questa matrice è dato dal numero di valori
indipendenti che assumono i due indici che, per la proprietà di
antisimmetria~\ref{item:antisimmetria-riemann}, sono uguali al numero di
elementi indipendenti di una matrice $n \times n$ antisimmetrica, cioè
$d = n(n-1)/2$.  Dunque gli elementi indipendenti di
$R_{(\lambda\mu)(\nu\kappa)}$ si riducono
\begin{equation}
  \frac{1}{2} d (d + 1) = \frac{1}{2}\bigg( \frac{1}{2}n(n-1) \bigg) \bigg(
  \frac{1}{2}n(n-1) +1 \bigg) = \frac{1}{8}n(n-1)(n^{2} - n +2).
\end{equation}
% TODO: provare a capire per bene il perché dell'affermazione successiva e a
% spiegarlo in maniera semplice.  Per esempio vedi
% http://www.maths.tcd.ie/~houghton/TEACHING/442/notes/22Oct2002.pdf
% http://sergeysk.wordpress.com/2011/01/21/number-of-independent-components-of-the-riemann-curvature-tensor/
% http://ned.ipac.caltech.edu/level5/March01/Carroll3/Carroll3.html (è il
% capitolo 3 del Carrol)
Infine, la relazione di ciclicità~\eqref{eq:ciclicita-riemann} corrisponde a
$\binom{n}{4}$ equazioni differenti, allora abbiamo altrettanti ulteriori
vincoli, così che gli elementi indipendenti del tensore di Riemann
$n$-dimensionale sono
\begin{equation}
  C_{n} = \frac{1}{8}n(n-1)(n^{2} - n +2) - \binom{n}{4} =
  \frac{1}{12}n^{2}(n^{2} - 1).
\end{equation}
Ponendo $n=4$ abbiamo che nello spazio-tempo quadrimensionale gli elementi
indipendenti del tensore di Riemann sono $C_{4} = 20$, come avevamo
preannunciato.

La contrazione fra primo e terzo indice del tensore di Riemann dà il
\index{tensore!di Ricci}\emph{tensore di Ricci}
\begin{equation}
  \label{eq:tens-ricci}
  R_{\mu\kappa} = g^{\lambda\nu} R_{\lambda\mu\nu\kappa} =
  \tensor{R}{^{\nu}_{\mu\nu\kappa}}
  = \parder{\tensor{\Gamma}{^{\nu}_{\mu\nu}}}{x^{\kappa}}
  - \parder{\tensor{\Gamma}{^{\nu}_{\mu\kappa}}}{x^{\nu}} +
  \tensor{\Gamma}{^{\eta}_{\mu\nu}} \tensor{\Gamma}{^{\nu}_{\eta\kappa}} -
  \tensor{\Gamma}{^{\eta}_{\mu\kappa}} \tensor{\Gamma}{^{\nu}_{\eta\nu}}.
\end{equation}
Il tensore di Ricci è simmetrico per la proprietà~\ref{item:simmetria-riemann}
di simmetria del tensore di Riemann, infatti
\begin{equation}
  \label{eq:simmetria-riemann}
  R_{\mu\kappa} = g^{\lambda\nu} R_{\lambda\mu\nu\kappa} = g^{\lambda\nu}
  R_{\nu\kappa\lambda\mu} = g^{\nu\lambda} R_{\nu\kappa\lambda\mu} =
  R_{\kappa\mu}.
\end{equation}
Dunque delle $4^{2} = 16$ componenti del tensore solo $10$ sono indipendenti.
Inoltre il tensore di Ricci è essenzialmente l'unico tensore non nullo che si
può ottenere dalla contrazione del tensore di Riemann.  Contraendo primo e
secondo indice, per la proprietà~\ref{item:antisimmetria-riemann} di
antisimmetria del tensore di Riemann si ha
\begin{equation}
  \tensor{R}{^{\mu}_{\mu\nu\kappa}} = g^{\lambda\mu}
  R_{\lambda\mu\nu\kappa} = -g^{\lambda\mu} R_{\mu\lambda\nu\kappa} =
  -g^{\mu\lambda}R_{\mu\lambda\nu\kappa} =
  -g^{\lambda\mu}R_{\lambda\mu\nu\kappa} = 0.
\end{equation}
Analogamente, contraendo terzo e quarto indice si ottiene
\begin{equation}
  \tensor{R}{_{\lambda\mu}^{\kappa}_{\kappa}} = g^{\nu\kappa}
  R_{\lambda\mu\nu\kappa} = 0.
\end{equation}
Contraendo secondo e quarto indice del tensore di Riemann si ottiene di nuovo il
tensore di Ricci, invece contraendo primo e quarto indice oppure secondo terzo
si ottiene ancora il tensore di Ricci così come da noi definito ma con il segno
opposto.  Infatti, moltiplicando tutti i membri
della~\eqref{eq:simmetria-riemann} per $g^{\lambda\nu}$ troviamo
\begin{equation}
  \tensor{R}{^{\nu}_{\mu\nu\kappa}} = -\tensor{R}{_{\mu}^{\nu}_{\nu\kappa}} =
  -\tensor{R}{^{\nu}_{\mu\kappa\nu}} = \tensor{R}{_{\mu}^{\nu}_{\kappa\nu}}.
\end{equation}

L'ulteriore contrazione del tensore di Riemann dà la
\index{curvatura scalare}\emph{curvatura scalare}
\begin{equation}
  R = g^{\mu\nu}R_{\mu\nu} = \tensor{R}{^{\mu}_{\mu}}.
\end{equation}

\subsection{Tensore di Riemann e curvatura dello spazio}
\label{sec:riemann-curvatura}

Anche in uno spazio piatto è possibile introdurre una metrica non costante su
tutto lo spazio che faccia apparire lo spazio curvo, come se ci fosse un campo
gravitazionale.  Questo è il caso, per esempio, delle coordinate sferiche o
cilindriche nello spazio cartesiano $\R^{3}$.  Siamo allora interessati a
cercare uno strumento che ci permetta di discriminare fra uno spazio-tempo
intrinsecamente curvo, perché permeato da un reale campo gravitazionale, e uno
spazio-tempo piatto, cioè lo spazio di Minkowski, in cui si fa uso di coordinate
curvilinee.  Se siamo nella seconda situazione, in questo spazio di Minkowski è
possibile effettuare una trasformazione delle coordinate
$x^{\alpha} \to \xi^{\alpha}(x)$ da un sistema di coordinate curvilinee
$x^{\alpha}$ con tensore metrico $\gamma_{\mu\nu}$ non costante a un sistema di
coordinate $\xi^{\alpha}(x)$ con tensore metrico uguale al tensore di Minkowski
$\eta_{\alpha\beta}$ e in tutto lo spazio si ha
\begin{equation}
  \eta_{\alpha\beta} =
  g_{\mu\nu} \parder{x^{\mu}}{\xi^{\alpha}} \parder{x^{\nu}}{\xi^{\beta}}.
\end{equation}
Per il \index{principio!di equivalenza}principio di equivalenza, in ogni punto
$X$ dello spazio è possibile individuare un sistema di coordinate localmente
inerziali $\xi_{X}$ che soddisfano la relazione precedente in un intorno
infinitesimo di $X$.  Noi invece vogliamo scoprire se esista un sistema di
coordinate $\xi^{\alpha}$ che soddisfi la relazione in tutto lo spazio.

Lo strumento a cui siamo interessati è il tensore di Riemann, infatti è
possibile dimostrare\footnote{Vedi~\textcite[138]{weinberg:gravitation}.} il
seguente
\begin{teorema}
  Condizioni necessarie e sufficienti affinché sia possibile effettuare una
  trasformazione di coordinate $x \to \xi$ tale che in tutto lo spazio sia
  valida la relazione
  \begin{equation}
    \label{eq:tensore-spazio-piatto}
    \eta_{\alpha\beta} =
    g_{\mu\nu} \parder{x^{\mu}}{\xi^{\alpha}} \parder{x^{\nu}}{\xi^{\beta}}
  \end{equation}
  sono
  \begin{enumerate}
  \item il \index{tensore!di Riemann}tensore di Riemann si annulla in tutto lo
    spazio
    \begin{equation}
      \tensor{R}{^{\lambda}_{\mu\nu\kappa}} = 0;
    \end{equation}
  \item in qualche punto $X$ la matrice $g_{\mu\nu}(X)$ ha tre autovalori
    positivi e uno negativo.
  \end{enumerate}
\end{teorema}
Il fatto che sia valida la relazione~\eqref{eq:tensore-spazio-piatto} significa
che la metrica è quella usuale dello spazio di Minkowski
$\dd \tau^{2} = \dd t^{2} - \dd\bm{x}^{2}$, l'annullarsi del tensore di Riemann
invece significa che la curvatura dello spazio è nulla, il teorema precedente
dichiara l'equivalenza fra queste due condizioni.  In questo modo il tensore di
Riemann ci permette di verificare se nello spazio sia presente un reale campo
gravitazionale.

Un altro modo che permette di mostrare che il tensore di Riemann caratterizza la
curvatura dello spazio, e quindi evidenzia la presenza di un campo
gravitazionale, è il seguente.  Dato un vettore covariante $V_{\mu}$ si dimostra
(vedi l'appendice~\ref{sec:dimostr-derivate-miste-vettore}) che le sue derivate
covarianti seconde miste non commutano
\begin{equation}
  \label{eq:derivate-miste-vettore}
  V_{\mu;\nu;\kappa} - V_{\mu;\kappa;\nu} = -V_{\lambda}
  \tensor{R}{^{\lambda}_{\mu\nu\kappa}}.
\end{equation}
Si trovano inoltre analoghe relazioni per i vettori controvarianti o, più in
generale, per i tensori
\begin{subequations}
  \begin{gather}
    \tensor{V}{^{\mu}_{;\nu;\kappa}} - \tensor{V}{^{\mu}_{;\kappa;\nu}} =
    V^{\sigma} \tensor{R}{^{\lambda}_{\sigma\nu\kappa}}, \\
    \tensor{T}{^{\lambda}_{\mu;\nu;\kappa}} -
    \tensor{T}{^{\lambda}_{\mu;\kappa;\nu}} = \tensor{T}{^{\sigma}_{\mu}}
    \tensor{R}{^{\lambda}_{\sigma\nu\kappa}} - \tensor{T}{^{\lambda}_{\sigma}}
    \tensor{R}{^{\sigma}_{\mu\nu\kappa}}.
  \end{gather}
\end{subequations}
Nello spazio di Minkowski le derivate covarianti coincidono con le derivate
ordinarie e le derivate seconde miste commutano
\begin{equation}
  V_{\mu,\nu,\kappa} = \parder{V_{\mu}}{x^{\nu}, x^{\kappa}}
  = \parder{V_{\mu}}{x^{\kappa}, x^{\nu}} = V_{\mu,\kappa,\nu}
\end{equation}
e questo è possibile solo se il tensore di Riemann è identicamente nullo in
tutto lo spazio.

\subsection{Identità di Bianchi}
\label{sec:identita-bianchi}

Si dimostra (vedi l'appendice~\ref{sec:dimostr-identita-bianchi}) la seguente
relazione
\begin{equation}
  \label{eq:bianchi}
  R_{\lambda\mu\nu\kappa;\eta} + R_{\lambda\mu\kappa\eta;\nu} +
  R_{\lambda\mu\eta\nu;\kappa} = 0
\end{equation}
chiamata \index{identità!di Bianchi}\emph{identità di Bianchi}.  Contraiamo gli
indici $\lambda$ e $\nu$ moltiplicando ambo i membri per $g^{\lambda\nu}$
\begin{equation}
  \begin{split}
    \tensor{R}{^{\nu}_{\mu\nu\kappa;\eta}} +
      \tensor{R}{^{\nu}_{\mu\kappa\eta;\nu}} +
    \tensor{R}{^{\nu}_{\mu\eta\nu;\kappa}} &= R_{\mu\kappa;\eta} +
    \tensor{R}{^{\nu}_{\mu\kappa\eta;\nu}} - R_{\mu\eta;\kappa} \\
    &= R_{\mu\kappa;\eta} - \tensor{R}{^{\nu}_{\mu\eta\kappa;\nu}} -
    R_{\mu\eta;\kappa} = 0.
  \end{split}
\end{equation}
Abbiamo sfruttato il fatto che innalzamento degli indici e derivazione
covariante commutano.  Contraendo ulteriormente gli indici $\mu$ e $\kappa$,
cioè moltiplicando gli ultimi due membri per $g^{\mu\kappa}$, otteniamo
\begin{equation}
  R_{;\eta} - \tensor{R}{^{\nu}_{\eta;\nu}} -
  \tensor{R}{^{\kappa}_{\eta;\kappa}} = R_{;\eta} - 2
  \tensor{R}{^{\mu}_{\eta;\mu}} = 0,
\end{equation}
poiché $\kappa$ e $\nu$ sono entrambi indici muti che abbiamo rinominato $\mu$.
Questa relazione può anche essere scritta come
\begin{equation}
  \bigg( \tensor{R}{^{\mu}_{\eta}} - \frac{1}{2} \tensor{\delta}{^{\mu}_{\eta}}
  R \bigg)_{;\mu} = 0
\end{equation}
e moltiplicando ancora per $g^{\eta\nu}$
\begin{equation}
  \bigg( R^{\mu\nu} - \frac{1}{2} g^{\mu\nu} R\bigg)_{;\mu} = 0.
\end{equation}
La quantità all'interno delle parentesi è il
\index{tensore!di Einstein}\emph{tensore di Einstein}
\begin{equation}
  G^{\mu\nu} = R^{\mu\nu} - \frac{1}{2} g^{\mu\nu}R
\end{equation}
e abbiamo dunque mostrato che ha quadridivergenza covariante nulla
\begin{equation}
  \label{eq:divergenza-tens-einstein}
  \tensor{G}{^{\mu\nu}_{;\mu}} = \bigg( R^{\mu\nu} - \frac{1}{2} g^{\mu\nu}
  R\bigg)_{;\mu} = 0.
\end{equation}

% TODO: scrivere qualcosa.  Vedi lezioni dei giorni 03-04/05/2012
\subsection{\completare{Equazione di deviazione geodetica}}
\label{sec:deviazione-geodetica}

\section{Equazioni di Einstein}
\label{sec:equazioni-einstein}

\emph{Nota: in questo paragrafo indicheremo esplicitamente la velocità della
  luce nel vuoto $c$.}

Vogliamo ora determinare le equazioni del campo gravitazionale $g_{\mu\nu}$ a
partire dal principio variazionale di Hamilton.  Assumiamo che l'azione $S$ sia
somma di una componente $S_{\textup{g}}$ che descrive solo il campo
gravitazionale e una componente $S_{\textup{m}}$ relativa alla materia che
interagisce con il campo.  L'azione deve essere uno scalare, in modo che risulti
invariante per trasformazioni arbitrarie delle coordinate, e dipenderà dal
tensore metrico, dalle sue derivate prime, dalle coordinate generalizzate $q$ e
dalle velocità generalizzate $\lparder{q}{x^{\lambda}}$,
$S = S(g_{\mu\nu}, \lparder{g_{\mu\nu}}{x^{\lambda}}; q,
\lparder{q}{x^{\lambda}})$.
Hilbert propose questa espressione per l'azione del campo
gravitazionale\footnote{Il fattore davanti all'integrale è stato scelto in
  maniera tale che nel limite newtoniano si ottenga l'equazione di Poisson
  $\nabla^{2} \phi = 4\pi G\rho$.}
\begin{equation}
  S_{\textup{g}} = -\frac{c^{3}}{16\pi G} \int R \sqrt{g} \dd^{4} x,
\end{equation}
chiamata \index{azione!di Hilbert-Einstein} \emph{azione di Hilbert-Einstein}.
$G = \SI{6.673 84(80)e-11}{\cubic\metre\per\kilo\metre\per\second\squared}$ è la
\index{costante!di gravitazione universale}
\emph{costante di gravitazione universale}.  L'azione della materia ha invece
espressione\footnote{In uno spazio-tempo piatto $g = 1$, quindi
  $S_{\textup{m}} = (1/c)\int \Lambda \dd^{4} x = \iint \Lambda \dd t\dd^{3}
  \bm{x}$, come nella~\eqref{eq:azione-campo}.}
\begin{equation}
  S_{\textup{m}} = \frac{1}{c} \int \Lambda \sqrt{g} \dd^{4} x,
\end{equation}
in cui $\Lambda = \Lambda(g^{\mu\nu}, \partial_{\lambda}g^{\mu\nu})$ è la densità
di lagrangiana associata al sistema che interagisce con il campo
gravitazionale.  L'interazione fra il campo e la materia è inclusa in
$S_{\textup{m}}$.

Per determinare le equazioni del campo gravitazionale dobbiamo richiedere che
l'azione sia stazionaria rispetto alle variazioni, a volume fissato,
$\delta g_{\mu\nu}$ del campo che devono soddisfare inoltre la condizione che si
annullino sul bordo dell'ipervolume di integrazione.  Dunque dobbiamo imporre
$\delta S = \delta(S_{\textup{g}} + S_{\textup{m}}) = 0$.  Calcoliamo il
contributo di $\delta S_{\textup{g}}$
\begin{equation}
  \label{eq:var-HE}
  \begin{split}
    \delta \int R \sqrt{g} \dd^{4} x &= \delta \int g^{\mu\nu} R_{\mu\nu}
    \sqrt{g} \dd^{4} x \\
    &= \int (\delta g^{\mu\nu} R_{\mu\nu} \sqrt{g} + g^{\mu\nu} R_{\mu\nu}
    \delta \sqrt{g} + g^{\mu\nu} \delta R_{\mu\nu} \sqrt{g}) \dd^{4} x.
  \end{split}
\end{equation}
Osserviamo che
\begin{equation}
  \delta\sqrt{g} = \parder{\sqrt{g}}{g} \parder{g}{g^{\mu\nu}} \delta g^{\mu\nu}
  = \frac{1}{2 \sqrt{g}} (-g g_{\mu\nu} ) \delta g^{\mu\nu} = - \frac{1}{2}
  g_{\mu\nu} \sqrt{g} \delta g^{\mu\nu},
\end{equation}
in cui abbiamo ricordato la~\eqref{eq:g_mu_rho-determinante2}.  Sostituendo
nella~\eqref{eq:var-HE} otteniamo
\begin{equation}
  \delta \int R \sqrt{g} \dd^{4} x = \int \bigg( R_{\mu\nu} - \frac{1}{2}
  g_{\mu\nu} R \bigg)\sqrt{g} \delta g^{\mu\nu} \dd^{4} x + \int g^{\mu\nu}
  \delta R_{\mu\nu} \sqrt{g} \dd^{4} x.
\end{equation}
Mostriamo che il secondo integrale è nullo.  Innanzitutto la
\index{tensore!di Ricci!variazione del}variazione del tensore di Ricci può
essere espressa in termini di variazioni della connessione affine mediante
l'\index{identità!di Palatini}% TODO: vedere di dimostrarla
\emph{identità di
  Palatini}\footnote{Si
  noti che la variazione della connessione affine è un tensore poiché è la
  differenza fra due connessioni affini, la quale è un tensore come visto nel
  paragrafo~\ref{sec:connessione-affine}.}
\begin{equation}
  \delta R_{\mu\nu} = (\delta \tensor{\Gamma}{^{\lambda}_{\mu\lambda}})_{;\nu} -
  (\delta \tensor{\Gamma}{^{\lambda}_{\mu\nu}})_{;\lambda}.
\end{equation}
Allora si può riscrivere l'integrale in esame come
\begin{equation}
  \int g^{\mu\nu} \delta R_{\mu\nu} \sqrt{g} \dd^{4} x = \int (g^{\mu\nu} \delta
  \tensor{\Gamma}{^{\lambda}_{\mu\lambda;\nu}} - g^{\mu\nu} \delta
  \tensor{\Gamma}{^{\lambda}_{\mu\nu;\lambda}}) \sqrt{g} \dd^{4} x
\end{equation}
e per il teorema di Gauss in forma covariante~\eqref{eq:gauss-covariante}
ricaviamo
\begin{equation}
  \begin{split}
    \int g^{\mu\nu} \delta R_{\mu\nu} \sqrt{g} \dd^{4} x &= \int (\delta
    \tensor{\Gamma}{^{\lambda\nu}_{\lambda;\nu}} - \delta
    \tensor{\Gamma}{^{\lambda\nu}_{\nu;\lambda}}) \sqrt{g} \dd^{4} x \\
    &= \int \delta \tensor{\Gamma}{^{\lambda\nu}_{\lambda}} \sqrt{g} \dd
    \Sigma_{\nu} - \int \delta \tensor{\Gamma}{^{\lambda\nu}_{\lambda}} \sqrt{g}
    \dd \Sigma_{\lambda} = 0
  \end{split}
\end{equation}
poiché gli integrali sulle ipersuperfici vanno calcolati sul bordo
dell'ipervolume costante su cui si calcola la variazione dell'azione e sul bordo
sono nulle le variazioni di $g_{\mu\nu}$ e quindi anche della connessione
affine, la quale dipende dal tensore metrico.  Dunque abbiamo trovato che la
variazione dell'azione di Hilbert-Einstein è
\begin{equation}
  \delta S_{\textup{g}} = -\frac{c^{3}}{16\pi G} \int \bigg( R_{\mu\nu} -
  \frac{1}{2} g_{\mu\nu} R \bigg)\sqrt{g} \delta g^{\mu\nu} \dd^{4} x.
\end{equation}

Per calcolare la variazione $\delta S_{\textup{m}}$ dell'azione associata alla
materia ragioniamo in maniera analoga a quanto fatto nel
paragrafo~\ref{sec:tensore-energia-impulso}
\begin{equation}
  \begin{split}
    \delta S_{\textup{m}} &= \frac{1}{c} \delta \int \Lambda(g^{\mu\nu},
    \tensor{g}{^{\mu\nu}_{,\lambda}}) \sqrt{g} \dd^{4} x = \frac{1}{c} \int
    \bigg( \parder{(\sqrt{g}\Lambda)}{g^{\mu\nu}} \delta g^{\mu\nu}
    + \parder{(\sqrt{g}\Lambda)}{\tensor{g}{^{\mu\nu}_{,\lambda}}}
    \delta \tensor{g}{^{\mu\nu}_{,\lambda}} \bigg) \dd^{4} x \\
    &= \frac{1}{c} \int \bigg( \parder{(\sqrt{g}\Lambda)}{g^{\mu\nu}} \delta
    g^{\mu\nu} + \parder{}{x^{\lambda}}
    \bigg( \parder{(\sqrt{g}\Lambda)}{\tensor{g}{^{\mu\nu}_{,\lambda}}} \delta
    g^{\mu\nu}\bigg) - \delta g^{\mu\nu} \parder{}{x^{\lambda}}
    \bigg( \parder{(\sqrt{g}\Lambda)}{\tensor{g}{^{\mu\nu}_{,\lambda}}} \bigg)
    \bigg) \dd^{4} x \\
    &= \frac{1}{c} \int \bigg( \parder{(\sqrt{g}\Lambda)}{g^{\mu\nu}}
    - \parder{}{x^{\lambda}}
    \bigg( \parder{(\sqrt{g}\Lambda)}{\tensor{g}{^{\mu\nu}_{,\lambda}}} \bigg)
    \bigg) \delta g^{\mu\nu} \dd^{4} x.
  \end{split}
\end{equation}
Nell'ultimo passaggio abbiamo sfruttato il fatto che l'integrale del secondo
termine si trasforma, per il teorema di Gauss, in un integrale sul bordo del
volume di integrazione sul quale le variazioni $\delta g^{\mu\nu}$ sono nulle.
Introduciamo il tensore $T_{\mu\nu}$ definito
da\footnote{Si noti che \textcites{barone:relativita,landau:campi} definisco
  questo tensore con il segno opposto a causa della diversa convenzione sui
  segni.  Vedi pagina~\pageref{eq:convenzione-segni}.}
\begin{equation}
  -\frac{1}{2} \sqrt{g} T_{\mu\nu} = \parder{(\sqrt{g}\Lambda)}{g^{\mu\nu}}
  - \parder{}{x^{\lambda}}
  \bigg( \parder{(\sqrt{g}\Lambda)}{\tensor{g}{^{\mu\nu}_{,\lambda}}} \bigg)
\end{equation}
% TODO: spiegare da qualche parte che questo è proprio il tensore
% energia-impulso (e mettere \index{tensore!energia-impulso})
così la variazione dell'azione della materia diventa
\begin{equation}
  \delta S_{\textup{m}} = -\frac{1}{2c} \int T_{\mu\nu} \delta g^{\mu\nu}
  \sqrt{g} \dd^{4} x
\end{equation}
e la condizione di stazionarietà dell'azione totale del sistema campo
gravitazionale + particelle interagenti è
\begin{equation}
  \delta S = \delta(S_{\textup{g}} + S_{\textup{m}}) = -\int \bigg(
  \frac{c^{3}}{16\pi G} \bigg( R_{\mu\nu} - \frac{1}{2} g_{\mu\nu} R \bigg) +
  \frac{1}{2c} T_{\mu\nu} \bigg) \sqrt{g} \delta g^{\mu\nu} \dd^{4} x = 0.
\end{equation}
Data l'arbitrarietà delle variazioni $\delta g^{\mu\nu}$ ricaviamo
\begin{equation}
  \label{eq:einstein}
  G_{\mu\nu} = R_{\mu\nu} - \frac{1}{2} g_{\mu\nu} R = -\frac{8\pi G}{c^{4}}
  T_{\mu\nu}
\end{equation}
che sono le \emph{equazioni del campo gravitazionale} in presenza di sorgenti,
chiamate anche \index{equazioni!di Einstein}\emph{equazioni di Einstein}.
Osserviamo che dalla~\eqref{eq:divergenza-tens-einstein} deriva che anche il
tensore energia-impulso della materia ha quadridivergenza covariante nulla
$\tensor{T}{^{\mu\nu}_{;\mu}} = 0$.

Moltiplicando gli ultimi due membri per $g^{\mu\nu}$, in modo da contrarre gli
indici $\mu$ e $\nu$, otteniamo
\begin{equation}
  R - \frac{1}{2}4R = -R = -\frac{8\pi G}{c^{4}}\tensor{T}{^{\lambda}_{\lambda}}
\end{equation}
e sostituendo nella~\eqref{eq:einstein} otteniamo un'espressione alternativa per
le \index{equazioni!di Einstein}equazioni di Einstein
\begin{equation}
  R_{\mu\nu} = -\frac{8\pi G}{c^{4}} \bigg( T_{\mu\nu} -
  \frac{1}{2}g_{\mu\nu}\tensor{T}{^{\lambda}_{\lambda}} \bigg).
\end{equation}
Da qui possiamo osservare che nel vuoto, cioè in assenza di materia, il tensore
energia-impulso è nullo quindi le \index{equazioni!di Einstein}equazioni di
Einstein si riducono a
\begin{equation}
  R_{\mu\nu} = 0.
\end{equation}
Si noti che questa equazione non significa che uno spazio-tempo vuoto è anche
piatto, per avere quest'ultima condizione deve verificarsi la condizione più
forte $\tensor{R}{^{\lambda}_{\mu\nu\kappa}} = 0$.

\subsection{Limite newtoniano delle equazioni di Einstein}
\label{sec:limite-newtoniano-einstein}

Vogliamo determinare il limite newtoniano (campi gravitazionali deboli e
statici, velocità non relativistiche) delle equazioni di Einstein e per fare
questo lavoriamo in maniera analoga a quanto fatto nel
paragrafo~\ref{sec:limite-newtoniano}, quindi poniamo
$g_{\mu\nu} = \eta_{\mu\nu} + h_{\mu\nu}$.  Ci aspettiamo che il tensore
energia-impulso sia in queste condizioni quello ricavato nel limite di basse
velocità nel paragrafo~\ref{sec:fluido-perfetto} per un corpo macroscopico
continuo (come un \index{fluido!perfetto}fluido perfetto)
\begin{subequations}
  \begin{align}
    T_{ij} &= \rho c^{2} u_{i} u_{j}, \\
    T_{00} &= \rho c^{2},
  \end{align}
\end{subequations}
in cui qui con $\rho$ indichiamo la densità di massa a riposo (e quindi
$\rho c^{2}$ è la densità di energia a riposo, per brevità omettiamo il pedice
$0$) e $u_{\mu}$ la quadrivelocità.  Per l'ipotesi di velocità non
relativistiche si devono trascurare le componenti spaziali della quadrivelocità
rispetto alla componente temporale, cioè $u_{i} = 0$, e quindi $T_{ij} = 0$ da
cui abbiamo inoltre
\begin{equation}
  0 = G_{ij} \approx R_{ij} - \frac{1}{2}\eta_{ij}R.
\end{equation}
Allora l'unica equazione di Einstein rilevante è quella con $G_{00}$.
Contraendo gli indici dell'equazione precedente otteniamo
\begin{equation}
  \eta^{ij}R_{ij} \approx \frac{1}{2} \eta_{ij}\eta^{ij} R = \frac{3}{2} R.
\end{equation}
Il tensore di Ricci può essere approssimato come
\begin{equation}
  R = g^{\mu\nu}R_{\mu\nu} \approx \eta^{\mu\nu} R_{\mu\nu} = \eta^{00}R_{00} +
  \eta^{ij}R_{ij} = -R_{00} + \frac{3}{2} R
\end{equation}
dunque $R \approx 2 R_{00}$.  Sostituendo nell'equazione di Einstein abbiamo
\begin{equation}
  G_{00} = R_{00} - \frac{1}{2} \eta_{00}R = R_{00} + \frac{1}{2} R = 2 R_{00} =
  -\frac{8\pi G}{c^{4}} T_{00} = -\frac{8\pi G}{c^{4}} \rho c^{2}
\end{equation}
da cui
\begin{equation}
  R_{00} = -\frac{4\pi G}{c^{2}} \rho.
\end{equation}
D'altra parte osserviamo che
\begin{equation}
  R_{00} = g^{\mu\nu}R_{\mu 0\nu 0} \approx \eta^{\mu\nu}R_{\mu 0\nu 0}
\end{equation}
Dalla~\eqref{eq:riemann-covariante}, arrestandoci ai termini del primo ordine in
$h_{\mu\nu}$, abbiamo
\begin{equation}
  \label{eq:riemann-approx}
  \begin{split}
    R_{\lambda\mu\nu\kappa} &\approx \frac{1}{2}
    \bigg( \parder{h_{\lambda\nu}}{x^{\kappa}, x^{\mu}}
    - \parder{h_{\mu\nu}}{x^{\kappa},
      x^{\lambda}}- \parder{h_{\lambda\kappa}}{x^{\nu}, x^{\mu}}
    + \parder{h_{\mu\kappa}}{x^{\nu}, x^{\lambda}} \bigg) \\
    &+ \eta_{\eta\sigma}
    (\underbrace{\tensor{\Gamma}{^{\eta}_{\nu\lambda}}
      \tensor{\Gamma}{^{\sigma}_{\mu\kappa}} -
      \tensor{\Gamma}{^{\eta}_{\kappa\lambda}}
      \tensor{\Gamma}{^{\sigma}_{\mu\nu}}}_{\mathcal{O}(h^{2})}) \\
    &= \frac{1}{2} \bigg( \parder{h_{\lambda\nu}}{x^{\kappa}, x^{\mu}}
    - \parder{h_{\mu\nu}}{x^{\kappa},
      x^{\lambda}}- \parder{h_{\lambda\kappa}}{x^{\nu}, x^{\mu}}
    + \parder{h_{\mu\kappa}}{x^{\nu}, x^{\lambda}} \bigg) + \mathcal{O}(h^{2})
  \end{split}
\end{equation}
e in particolare, sfruttando l'ipotesi di campi costanti,
\begin{subequations}
  \begin{align}
    R_{0000} &\approx 0, \\
    R_{i0j0} &\approx \frac{1}{2} \parder{h_{00}}{x^{i},x^{j}}
  \end{align}
\end{subequations}
allora
\begin{equation}
  R_{00} = \eta^{00}R_{0000} + \eta^{ij}R_{i0j0} = \delta^{ij}R_{i0j0} \approx
  \frac{1}{2}\nabla^{2} h_{00} = -\nabla^{2}\frac{\phi}{c^{2}}
\end{equation}
avendo ricordato la~\eqref{eq:h00}.  Dunque mettendo insieme le due espressioni
di $R_{00}$ ottenute troviamo l'equazione di Poisson per il campo gravitazionale
\begin{equation}
  \nabla^{2} \phi = 4\pi G\rho.
\end{equation}

\section{Campo gravitazionale statico e a simmetria sferica}
\label{sec:campo-statico-sferico}

Vogliamo determinare la metrica per un campo gravitazionale a simmetria sferica
e statico, cioè tale che $\dd\tau^{2} = -g_{\mu\nu}\dd x^{\mu}\dd x^{\nu}$ non
dipenda dalla coordinata temporale $x^{0}$ e dipenda dalle coordinate spaziali
solo attraverso delle quantità invarianti per rotazioni, cioè
$\bm{x} \cdot \dd\bm{x}$, $\bm{x} \cdot \bm{x}$, $\dd\bm{x} \cdot \dd\bm{x}$.
La forma più generale per una metrica con queste proprietà è
\begin{equation}
  \dd\tau^{2} = F(r) \dd t^{2} - 2E(r) \bm{x} \cdot \dd\bm{x} \dd t -
  D(r)(\bm{x} \cdot \dd\bm{x})^{2} - C(r)\dd\bm{x}^{2},
\end{equation}
in cui $F$, $E$, $D$ e $C$ sono funzioni della distanza dall'origine del campo
gravitazionale $r = \norm{\bm{x}}$.  Data la simmetria del sistema che vogliamo
analizzare conviene passare in coordinate sferiche
\begin{subequations}
  \begin{align}
    x^{1} &= r \sin\theta \cos\phi, \\
    x^{2} &= r \sin\theta \sin\phi, \\
    x^{3} &= r \cos\theta
  \end{align}
\end{subequations}
così che la metrica diventa
\begin{equation}
  \dd\tau^{2} = F(r) \dd t^{2} - 2E(r) r\dd r\dd t - r^{2}D(r)\dd r^{2} -
  C(r)(\dd r^{2} + r^{2}\dd\theta^{2} + r^{2}\sin^{2}\theta \dd\phi^{2}).
\end{equation}
La libertà che si ha, in relatività generale, nella scelta del sistema di
coordinate ci consente di semplificare questa espressione.  Consideriamo dunque
la trasformazione della coordinate temporale $t \to t' = t + \phi(r)$ con la
condizione
\begin{equation}
  \toder{\phi}{r} = - r\frac{E(r)}{F(r)}
\end{equation}
in modo da annullare il termine $g_{rt}$ del tensore metrico.  Così la metrica
diventa
\begin{equation}
  \dd \tau^{2} = F(r)\dd t'^{2} - G(r)\dd r^{2} -C(r)(\dd r^{2} +
  r^{2}\dd\theta^{2} + r^{2}\sin^{2}\theta \dd\phi^{2})
\end{equation}
con
\begin{equation}
  G(r) = r^{2} \bigg(D(r) + \frac{E^{2}(r)}{F(r)} \bigg).
\end{equation}
Effettuando inoltre la trasformazione $r \to r' = C(r)r^{2}$ otteniamo la
\index{metrica!forma standard della}\emph{forma standard} della metrica
\begin{equation}
  \label{eq:metrica-standard}
  \dd\tau^{2} = B(r')\dd t'^{2} - A(r')\dd r'^{2} - r'^{2}(\dd\theta^{2} +
  \sin^{2}\theta \dd\phi^{2}),
\end{equation}
in cui
\begin{subequations}
  \begin{align}
    B(r) &= F(r), \\
    A(r) &= \bigg(1 + \frac{G(r)}{C(r)} \bigg) \bigg( 1 + \frac{r}{2C(r)}
    \toder{C(r)}{r} \bigg)^{-2}.
  \end{align}
\end{subequations}
Rinominando le variabili $r'$ e $t'$ in $r$ e $t$, il tensore metrico è
rappresentato dalla seguente matrice diagonale
\begin{equation}
  g_{\mu\nu} =
  \begin{pmatrix}
    g_{tt}       & g_{tr}       & g_{t\theta}      & g_{t\phi}      \\
    g_{rt}       & g_{rr}       & g_{r\theta}      & g_{r\phi}      \\
    g_{\theta t} & g_{\theta r} & g_{\theta\theta} & g_{\theta\phi} \\
    g_{\phi t}   & g_{\phi r}   & g_{\phi\theta}   & g_{\phi\phi}
  \end{pmatrix}
  =
  \begin{pmatrix}
    -B(r) & 0    & 0     & 0 \\
    0     & A(r) & 0     & 0 \\
    0     & 0    & r^{2} & 0 \\
    0     & 0    & 0     & r^{2}\sin^{2}\theta
  \end{pmatrix}.
\end{equation}
Poiché $g_{\mu\nu}$ è diagonale, gli elementi del tensore metrico controvariante
sono semplicemente $g^{\mu\nu} = 1/g_{\mu\nu}$, quindi
$g^{\mu\nu} = \diag(-B^{-1}(r), A^{-1}(r), r^{-2}, r^{-2}\sin^{-2}\theta)$.
Osserviamo che $g = -\det(g_{\mu\nu}) = r^{4}A(r)B(r)\sin^{2}\theta$, quindi
l'elemento di volume invariante è
\begin{equation}
  \sqrt{-g} \dd^{4} x = r^{2}\sqrt{A(r) B(r)} \sin\theta \dd^{4} x.
\end{equation}

Le connessioni affini possono essere calcolate con la
formula~\eqref{eq:connessione-metrica}.  Si trova che le uniche componenti non
nulle sono
\begin{subequations}
  \begin{align}
    \tensor{\Gamma}{^{r}_{rr}} &= \frac{A'(r)}{2A(r)}, &
    \tensor{\Gamma}{^{r}_{\theta\theta}} &= -\frac{r}{A(r)}, &
    \tensor{\Gamma}{^{r}_{\phi\phi}} &= -\frac{r \sin^{2}\theta}{A(r)}, \\
    \tensor{\Gamma}{^{r}_{tt}} &= \frac{B'(r)}{2A(r)}, &
    \tensor{\Gamma}{^{\theta}_{r\theta}} &=
    \tensor{\Gamma}{^{\theta}_{\theta r}} = \frac{1}{r}, &
    \tensor{\Gamma}{^{\theta}_{\phi\phi}} &= -\sin\theta \cos\theta, \\
    \tensor{\Gamma}{^{\phi}_{\phi r}} &= \tensor{\Gamma}{^{\phi}_{r\phi}} =
    \frac{1}{r}, & \tensor{\Gamma}{^{\phi}_{\phi\theta}} &=
    \tensor{\Gamma}{^{\phi}_{\theta\phi}} = \cot\theta, &
    \tensor{\Gamma}{^{t}_{tr}} &= \tensor{\Gamma}{^{t}_{rt}} =
    \frac{B'(r)}{2B(r)}.
  \end{align}
\end{subequations}
L'apice indica la derivazione rispetto a $r$.

Dopo di ciò è possibile calcolare il tensore di Ricci usando la sua
definizione~\eqref{eq:tens-ricci}.  Grazie all'invarianza per rotazioni e
l'indipendenza dal tempo della metrica, anche il tensore di Ricci è diagonale
con componenti non nulle
\begin{subequations}
  \label{eq:ricci-sferico}
  \begin{align}
    R_{tt} &= -\frac{B''(r)}{2A(r)} +
    \frac{B'(r)}{A(r)}\frac{B'(r)}{B(r)}\bigg(\frac{A'(r)}{A(r)} +
    \frac{B'(r)}{B(r)}\bigg) -
    \frac{1}{r}\frac{B'(r)}{A(r)}, \\
    R_{rr} &= \frac{B''(r)}{2B(r)} - \frac{1}{4}
    \frac{B'(r)}{B(r)}\bigg(\frac{A'(r)}{A(r)} + \frac{B'(r)}{B(r)}\bigg) -
    \frac{1}{r}\frac{A'(r)}{A(r)}, \\
    R_{\theta\theta} &= -1 + \frac{r}{2A(r)}\bigg(-\frac{A'(r)}{A(r)} +
    \frac{B'(r)}{B(r)}\bigg) + \frac{1}{A(r)}, \\
    R_{\phi\phi} &= R_{\theta\theta} \sin^{2}\theta.
  \end{align}
\end{subequations}
Osserviamo che risulta
\begin{equation}
  \label{eq:bar}
  \frac{R_{tt}}{B(r)} + \frac{R_{rr}}{A(r)} =
  -\frac{1}{rA(r)}\bigg(\frac{A'(r)}{A(r)} + \frac{B'(r)}{B(r)}\bigg).
\end{equation}

Concludiamo dicendo che si può
dimostrare\footnote{Vedi~\textcite[390-393]{landau:campi},
  \textcite[843-844]{misner:gravitation},
  \textcite[335-337]{weinberg:gravitation}.}
il \index{teorema!di Birkhoff}\emph{teorema di Birkhoff} il quale afferma che
\emph{un campo gravitazionale a simmetria sferica in assenza di materia deve
  essere statico},
quindi l'ipotesi di staticità del campo è superflua in quanto conseguenza della
sua particolare simmetria.

\subsection{Metrica di Schwarzschild}
\label{sec:metrica-schwarzschild}

La soluzione di Schwarzschild delle equazioni di Einstein descrive il campo
gravitazionale all'esterno di un corpo sferico non rotante e in assenza di
materia, cioè con tensore energia-impulso nullo.  Applicheremo i risultati che
otterremo al campo gravitazionale generato dal Sole.

Abbiamo visto che in assenza di materia le equazioni di Einstein comportano che
$R_{\mu\nu} = 0$.  Di conseguenza, ponendo uguale a $0$ il secondo membro
della~\eqref{eq:bar} abbiamo
\begin{equation}
  \frac{A'(r)}{A(r)} = -\frac{B'(r)}{B(r)}
\end{equation}
da cui
\begin{equation}
  \ln A(r) = - \ln B(r) + \text{costante} \implies A(r)B(R) = \text{costante}.
\end{equation}
Per determinare il valore della costante notiamo che a distanza infinita dal
campo gravitazionale la metrica deve tendere a quella di Minkowski, che in
coordinate sferiche si esprime come
\begin{equation}
  \dd\tau^{2} = \dd t^{2} -\dd r^{2} - r^{2}\dd\theta^{2} - r^{2}\sin^{2}\theta
  \dd\phi^{2}.
\end{equation}
Confrontando con la metrica nella forma standard~\eqref{eq:metrica-standard}
troviamo che deve risultare $A(r), B(R) \xrightarrow{r \to +\infty} 1$ e dalla
costanza del prodotto $A(r)B(r)$ deduciamo che esso vale $1$ e quindi
$A(r) = 1/B(r)$.  Sostituendo questo risultato nelle
componenti~\eqref{eq:ricci-sferico} del tensore di Ricci abbiamo
\begin{subequations}
  \begin{align}
    R_{\theta\theta} &= -1 + rB'(r) + B(r), \\
    R_{rr} &= \frac{B''(r)}{2B(r)} + \frac{B'(r)}{rB(r)} =
    \frac{R'_{\theta\theta}(r)}{2rB(r)}.
  \end{align}
\end{subequations}
Imponendo che anche la componente $R_{\theta\theta}$ si annulli, da cui deriva
inoltre che anche $R_{rr}$ e $R_{\phi\phi}$ sono nulle, si ricava
\begin{equation}
  \toder{}{r}(rBr(r)) = 1 \implies rB(r) = r + \text{costante} \implies B(r) = 1
  + \frac{\text{costante}}{r}.
\end{equation}
Per determinare il valore di questa nuova costante poniamo la condizione che a
grandi distanze dal corpo che genera il campo, in cui il campo diventa debole,
sia valida l'approssimazione vista nel paragrafo~\ref{sec:limite-newtoniano}
\begin{equation}
  g_{00} = -1 + \frac{r_{\textup{S}}}{r},
\end{equation}
con $r_{\textup{S}}$ raggio di Schwarzschild del corpo.  Osservando allora che
$g_{00} = g_{tt} = -B(r)$ abbiamo che la costante che vogliamo determinare vale
proprio $r_{\textup{S}}$ e in definitiva la
\index{metrica!di Schwarzschild}\emph{metrica di Schwarzschild}, per campi a
simmetria sferica e indipendenti dal tempo, è data da
\begin{equation}
  \label{eq:metrica-schwarzschild}
  \dd\tau^{2} = \bigg(1 - \frac{r_{\textup{S}}}{r} \bigg) \dd t^{2} - \bigg(1 -
  \frac{r_{\textup{S}}}{r}\bigg)^{-1}\dd r^{2} - r^{2}\dd\theta^{2} -
  r^{2}\sin^{2}\theta \dd\phi^{2}.
\end{equation}
Osserviamo che questa metrica ha una singolarità per $r = r_{\textup{S}}$ e
un'altra per $r = 0$.  Nel caso specifico del Sole, e di numerosi altri corpi,
la prima singolarità non costituisce un problema poiché la metrica vale
all'esterno del corpo e il raggio di questa stella è molto più piccolo del suo
raggio di Schwarzschild.  Si
dimostra\footnote{Vedi~\textcite[511]{barone:relativita},
  \textcite[403-404]{landau:campi}, \textcite[207-208]{weinberg:gravitation}.}
inoltre che la singolarità per $r = r_{\textup{S}}$ è eliminabile adottando un
opportuno sistema di coordinate, quindi non è, in realtà, una singolarità
fisica.

\section{Verifiche sperimentali della relatività generale}
\label{sec:verififiche-relativita}

Adottando la metrica di Schwarzschild siamo in grado di fare delle previsioni
teoriche riguardo alcuni fenomeni osservabili nel sistema solare, le quali sono
state confermate sperimentalmente e hanno permesso il riconoscimento della
validità della teoria della relatività generale.

\subsection{Precessione del perielio}
\label{sec:precessione-perielio}

\begin{figure}
  \centering
  \includegraphics{figure/precessione_perielio}
  % figura presa da
  % http://commons.wikimedia.org/wiki/File:Perihelion_precession.svg
  \caption{Precessione del perielio di un pianeta}
  \label{fig:precessione-perielio}
\end{figure}
La prima legge di Keplero afferma che l'orbita descritta dai pianeti è
un'ellisse.  Questo risultato è rigorosamente vero nel caso in cui il pianeta
interagisce solo con la stella intorno a cui orbita.  La presenza di altri
eventuali pianeti perturba il sistema facendo in modo che l'orbita non sia
chiusa ma piuttosto una rosetta, si ha quindi una precessione del perielio, come
rappresentato nella figura~\ref{fig:precessione-perielio}.  Per esempio, dalle
osservazioni astronomiche si è scoperto che il perielio di Mercurio si sposta di
circa \ang{;;575} ogni cento anni.  Utilizzando la meccanica newtoniana e
tenendo conto della presenza di tutti i pianeti conosciuti si prevede una
precessione del perielio di Mercurio pari a circa \ang{;;532}.  Per spiegare
questa incongruenza sono state proposte diverse ipotesi, fra cui la presenza di
un corpo perturbante come un nuovo pianeta (Vulcano), più interno di Mercurio, o
un satellite di Mercurio.  Tuttavia questi corpi non sono mai stati trovati e
faremo vedere che la relatività generale è in grado di risolvere questo
problema.

% TODO: capire per bene i passaggi dell'Adler
Per determinare l'orbita di un pianeta abbiamo bisogno di quattro equazioni per
le quattro coordinate spazio-temporali sferiche $t$, $r$, $\theta$ e $\phi$.
Una prima equazione differenziale per la coordinata $r$ si ottiene dividendo
l'intervallo di tempo proprio di Schwarzschild~\eqref{eq:metrica-schwarzschild}
per $\dd\tau^{2}$
\begin{equation}
  \label{eq:mercurio-r}
  \bigg(\toder{\tau}{\tau}\bigg)^{2} = 1 = \bigg(1 -
  \frac{r_{\textup{S}}}{r}\bigg)\dot{t}^{2} - \bigg(1 -
  \frac{r_{\textup{S}}}{r}\bigg)^{-1}\dot{r}^{2} - r^{2}(\dot{\theta}^{2} +
  \sin^{2}\theta \dot{\phi}^{2}).
\end{equation}
Il punto sopra le variabili indica la derivazione rispetto al tempo proprio
$\tau$.  Per determinare le altre equazioni dobbiamo risolvere il problema
variazionale
\begin{equation}
  \delta \int \dd \tau = 0
\end{equation}
in cui $\dd\tau$ è la metrica di Schwarzschild.  Questo problema può essere
risolto più facilmente considerando la seguente formulazione equivalente
\begin{equation}
  \begin{split}
    \delta\int \dd\tau &= \delta \int \bigg(\toder{\tau}{\tau}\bigg)^{2} \dd
    \tau \\
    &= \delta \int \bigg(\bigg(1 - \frac{r_{\textup{S}}}{r}\bigg)\dot{t}^{2} -
    \bigg(1 - \frac{r_{\textup{S}}}{r}\bigg)^{-1}\dot{r}^{2} -
    r^{2}(\dot{\theta}^{2} + \sin^{2}\theta \dot{\phi}^{2}) \bigg) \dd \tau = 0.
  \end{split}
\end{equation}
Da qui si ottengono tre equazioni di Eulero-Lagrange per le coordinate,
rispettivamente, $\theta$, $\phi$ e $t$
\begin{subequations}
  \begin{gather}
    \toder{}{\tau}(r^{2}\dot{\theta}) = r^{2}\sin\theta\cos\theta
    \dot{\phi}^{2}, \label{eq:mercurio-mu} \\
    \toder{}{\tau}(r^{2} \sin^{2}\theta \dot{\phi}) = 0, \label{eq:mercurio-phi}
    \\
    \toder{}{\tau}\bigg(\bigg(1 - \frac{r_{\textup{S}}}{r}\bigg) \dot{t} \bigg)
    = 0. \label{eq:mercurio-t}
  \end{gather}
\end{subequations}
Non abbiamo scritto l'equazione di Eulero-Lagrange per la coordinata $r$ poiché
utilizzeremo la~\eqref{eq:mercurio-r}.

Notiamo che associando all'equazione differenziale~\eqref{eq:mercurio-mu} le
condizioni iniziali $\theta = \pi/2$ e $\dot{\theta} = 0$, la funzione costante
$\theta(\tau) = \pi/2$ costituisce l'unica soluzione di questo problema di
Cauchy.  Ciò significa che con un'opportuna scelta delle coordinate il moto del
pianeta si svolge in un piano, così come succede anche in meccanica classica.
Sostituendo $\theta = \pi/2$ nella~\eqref{eq:mercurio-phi} abbiamo
\begin{equation}
  \label{eq:mercurio-phi2}
  \toder{}{\tau}(r^{2}\dot{\phi}) = 0 \implies r^{2}\dot{\phi} = h =
  \text{costante}
\end{equation}
cioè il pianeta si muove con velocità areolare costante, come nel problema di
Keplero classico.  Invece dalla~\eqref{eq:mercurio-t} otteniamo
\begin{equation}
  \label{eq:mercurio-t2}
  \bigg(1 - \frac{r_{\textup{S}}}{r}\bigg)\dot{t} = l = \text{costante}.
\end{equation}
Sostituendo tutti questi risultati nella~\eqref{eq:mercurio-r} troviamo
\begin{equation}
  \label{eq:mercurio-r2}
  1 = \bigg(1 - \frac{r_{\textup{S}}}{r}\bigg)^{-1}l^{2} - \bigg(1 -
  \frac{r_{\textup{S}}}{r}\bigg)^{-1}\dot{r}^{2} - \frac{h^{2}}{r^{2}}.
\end{equation}

Per determinare l'equazione dell'orbita conviene pensare la coordinata $r$ come
funzione di $\phi$, quindi indicando con l'apice la derivazione rispetto a
$\phi$ abbiamo
\begin{equation}
  \dot{r} = \toder{r}{\tau} = \toder{r}{\phi} \toder{\phi}{\tau} = r' \dot{\phi}
  = r' \frac{h}{r^{2}}
\end{equation}
e ponendo $u = 1/r$, con
\begin{equation}
  \dot{r} = \frac{h}{r^{2}}r' = \frac{h}{r^{2}}\toder{r}{\phi} =
  hu^{2}\toder{}{\phi} \frac{1}{u} = -hu^{2}\frac{u'}{u^{2}} = -hu',
\end{equation}
la~\eqref{eq:mercurio-r2} diventa
\begin{equation}
  (1 - r_{\textup{S}}u) = l^{2} - h^{2}u'^{2} - h^{2}u^{2}(1 - r_{\textup{S}}u),
\end{equation}
da cui, riarrangiando,
\begin{equation}
  \label{eq:mercurio-u'}
  u'^{2} = \frac{l^{2} - 1}{h^{2}} + \frac{r_{\textup{S}}}{h^{2}}u - u^{2} +
  r_{\textup{S}}u^{3} \equiv f(u).
\end{equation}
Questa equazione può essere integrata nel seguente modo
\begin{equation}
  \toder{u}{\phi} = \sqrt{f(u)} \implies \dd\phi = f^{-1/2}(u) \dd u \implies
  \phi = \phi_{0} + \int_{u_{0}}^{u} f^{-1/2}(\tilde{u}) \dd \tilde{u}.
\end{equation}
Sebbene questa sia una soluzione esatta, non è utile a livello pratico perché
fornisce solo un'espressione implicita di $u$, e quindi $r$, rispetto a $\phi$.
L'idea, a questo punto, è quella di scrivere un'equazione differenziale del
secondo ordine.  Derivando rispetto a $\phi$ la~\eqref{eq:mercurio-u'} abbiamo
\begin{equation}
  \label{eq:mercurio1}
  2u'u'' = \frac{r_{\textup{S}}}{h^{2}}u' - 2uu' + 3r_{\textup{S}}u^{2}u'
  \implies u'\bigg(2u'' - \frac{r_{\textup{S}}}{h^{2}} + 2u -
  3r_{\textup{S}}u^{2}\bigg) = 0.
\end{equation}
Una soluzione di questa equazione è $u' = 0$, cioè
$u(\phi) = 1/r(\phi) = \text{costante}$.  Tuttavia questa situazione non ci
interessa poiché descrive un moto circolare, essendo la distanza dal centro del
campo costante, quindi non c'è un perielio e non è possibile osservare la sua
precessione.  L'altra possibile soluzione è
\begin{equation}
  \label{eq:binet-relativ}
  2u'' - \frac{r_{\textup{S}}}{h^{2}} + 2u - 3r_{\textup{S}}u^{2} = 0 \implies
  u'' + u = \frac{r_{\textup{S}}}{2h^{2}} + \frac{3}{2}r_{\textup{S}}u^{2} =
  \frac{GM}{r^{4}(\ltoder{\phi}{\tau})^{2}} + \frac{3}{2}r_{\textup{S}}u^{2},
\end{equation}
in cui $M$ è la massa del corpo che genera il campo gravitazionale.  Ricordiamo
che nel problema di Keplero classico, l'\index{equazione!di Binet}equazione di
Binet è
\begin{equation}
  \label{eq:binet}
  u'' + u = \frac{GM}{r^{4}(\ltoder{\phi}{t})^{2}}.
\end{equation}
Osserviamo che per piccole velocità
$\ltoder{\phi}{\tau} \approx \ltoder{\phi}{t}$.  Inoltre per i pianeti del
sistema solare risulta
\begin{equation}
  \frac{3r_{\textup{S}}u^{2}/2}{r_{\textup{S}}/(2h^{2})} = 3u^{2}h^{2} \ll 1,
\end{equation}
infatti
\begin{equation}
  3u^{2}h^{2} = \frac{3}{r^{2}}r^{4} \bigg(\toder{\phi}{\tau}\bigg)^{2} = 3r^{2}
  \bigg(\toder{\phi}{\tau}\bigg)^{2} = 3 v_{\textup{t}}^{2},
\end{equation}
in cui $v_{\textup{t}} = r\ltoder{\phi}{t}$ è la velocità tangenziale del corpo,
espressa in unità di $c$ poiché abbiamo posto $c = 1$.  Per Mercurio, il pianeta
più interno del sistema solare e quindi con la maggior velocità tangenziale,
$3 v_{\textup{t}}^{2} = \num{7.7e-8}$, quindi possiamo considerare il termine
$3r_{\textup{S}}u^{2}/2$ nella~\eqref{eq:binet-relativ} come una debole
perturbazione.  In dettaglio, vogliamo risolvere l'equazione
\begin{equation}
  u'' + u = A + \frac{\epsilon u^{2}}{A},
\end{equation}
con $A = r_{\textup{S}}/(2h^{2})$ e $\epsilon = 3Ar_{\textup{S}}/2  \ll 1$.
Cerchiamo soluzioni del tipo
\begin{equation}
  u(\phi) = u_{0}(\phi) + \epsilon v(\phi),
\end{equation}
con $u_{0}(\phi)$ soluzione dell'equazione di Binet classica
\begin{equation}
  \label{eq:binet-2}
  u_{0}'' + u_{0} = A
\end{equation}
cioè
\begin{equation}
  u_{0} = A + B\cos(\phi + \delta),
\end{equation}
con $B$ e $\delta$ costanti reali di integrazione.  Ruotando opportunamente il
sistema di riferimento si può fare in modo che $\delta = 0$ cosicché $u_{0}$
diventa
\begin{equation}
  \label{eq:sol-binet}
  u_{0}(\phi) = A + B\cos\phi.
\end{equation}
Dunque l'equazione da risolvere è
\begin{equation}
  (u_{0}'' + \epsilon v'') + (u_{0} + \epsilon v) = A + \epsilon \frac{(u_{0} +
    \epsilon v)^{2}}{A} = A + \epsilon \frac{u_{0}^{2}}{A} +
  \mathcal{O}(\epsilon^{2}).
\end{equation}
Tenendo conto della~\eqref{eq:binet-2} l'equazione per $v(\phi)$ al primo ordine
in $\epsilon$ è
\begin{equation}
  \begin{split}
    v'' + v &= \frac{u_{0}^{2}}{A} = A + 2B\cos\phi +
    \frac{B^{2}}{A}\cos^{2}\phi \\
    &= \bigg(A + \frac{B^{2}}{2A}\bigg) + 2B\cos\phi +
    \frac{B^{2}}{2A}\cos(2\phi),
\end{split}
\end{equation}
in cui abbiamo ricordato l'identità trigonometrica
$\cos^{2}\phi = (1 + \cos(2\phi))/2$.  Dunque questa è l'equazione differenziale
del secondo ordine relativa al termine che perturba l'orbita dei pianeti.
Questa equazione è lineare, quindi la soluzione può essere espressa come somma
delle soluzioni delle tre equazioni seguenti
\begin{subequations}
  \begin{align}
    v_{a}'' + v_{a} &= A + \frac{B^{2}}{2A} \implies v_{a}(\phi) = A +
    \frac{B^{2}}{2A}, \\
    v_{b}'' + v_{b} &= 2B\cos\phi \implies v_{b}(\phi) = B\phi \sin\phi, \\
    v_{c}'' + v_{c} &= \frac{B^{2}}{2A}\cos(2\phi) \implies v_{c}(\phi) =
    -\frac{B^{2}}{6A}\cos(2\phi),
  \end{align}
\end{subequations}
dunque
\begin{equation}
  v(\phi) = v_{a}(\phi) + v_{b}(\phi) + v_{c}(\phi) = A + \frac{B^{2}}{2A} +
  B\phi\sin\phi - \frac{B^{2}}{6A}\cos(2\phi)
\end{equation}
e in questo modo $u(\phi)$ è data da
\begin{equation}
  \begin{split}
    u(\phi) &= u_{0}(\phi) + \epsilon v(\phi) \\
    &= A + B\cos\phi + \epsilon A + \frac{\epsilon B^{2}}{2A} + \epsilon
    B\phi\sin\phi - \frac{\epsilon B^{2}}{6A}\cos(2\phi).
  \end{split}
\end{equation}
Al primo ordine in $\epsilon$ abbiamo
\begin{equation}
  \cos(\phi - \epsilon\phi) = \cos\phi \cos(\epsilon\phi) + \sin\phi
  \sin(\epsilon\phi) \approx \cos\phi + \epsilon\phi\sin\phi,
\end{equation}
allora
\begin{equation}
  \frac{1}{r(\phi)} = u(\phi) \approx A + B\cos(\phi - \epsilon\phi) +
  \epsilon\bigg(A + \frac{B^{2}}{2A} - \frac{B^{2}}{6A}\cos(2\phi)\bigg).
\end{equation}
L'ultimo termine, fra parentesi, induce una piccola variazione nella distanza
radiale del pianeta, ma non può essere responsabile della precessione del
perielio poiché il termine $\cos(2\phi)$ ha periodo sottomultiplo intero di
$2\pi$ (quindi dopo un'intera rivoluzione di $\var \phi = 2\pi$ assume di nuovo
lo stesso valore).  Il resto della funzione $u(\phi)$ si differenzia dalla
soluzione~\eqref{eq:sol-binet} dell'equazione di Binet per la presenza di
$-\epsilon\phi$ nell'argomento del coseno.  Il perielio si ha quando $r(\phi)$ è
minimo, vale a dire quando $u(\phi)$ è massimo e ciò si verifica quando il
coseno del secondo termine vale $1$, cioè quando
\begin{equation}
  \phi_{n}(1 - \epsilon) = 2\pi n \implies \phi_{n} = \frac{2\pi n}{1 -
    \epsilon} \approx 2\pi n(1 + \epsilon)
\end{equation}
con $n$ numero di rivoluzioni compiute dal pianeta intorno al Sole a partire da
un istante iniziale fissato.  Pertanto il perielio si ha dopo una variazione
della coordinata $\phi$ di
\begin{equation}
  \var\phi = \phi_{n+1} - \phi_{n} = 2\pi(1 + \epsilon)
\end{equation}
invece di $2\pi$ come succede per un'orbita chiusa.  La relatività generale
prevede allora una precessione del perielio dei pianeti nel loro moto intorno al
Sole, indipendentemente dalla presenza perturbatrice di altri corpi.  Stimiamo
il valore $\delta\phi$ di questa precessione per verificare se è in accordo con
i dati sperimentali.  Per i pianeti del sistema solare
$r_{\textup{S}} = 2GM_{\odot}$ è il raggio di Schwarzschild del Sole, allora
\begin{equation}
  \delta\phi = 2\pi\epsilon = 2\pi \frac{3}{4}\frac{r_{\textup{S}}^{2}}{h^{2}} =
  6\pi \frac{G^{2}M_{\odot}^{2}}{h^{2}}.
\end{equation}
In particolare per il pianeta Mercurio si ha
\begin{equation}
  \delta\phi = \ang{;;0.1036}
\end{equation}
e la precessione dopo cento anni, ricordando che il suo periodo orbitale è di
$87.8$ giorni, è data da
\begin{equation}
  \delta_{100}\phi = \frac{\delta\phi \cdot 100 \cdot 365.25}{87.8} =
  \ang{;;43}.
\end{equation}
All'inizio della sezione abbiamo ricordato che la precessione secolare di
Mercurio è di circa \ang{;;575}, possiamo ora concludere che di questi, circa
\ang{;;43} sono dovuti a effetti di relatività generale e circa \ang{;;532} alla
perturbazione indotta dalla presenza degli altri pianeti.  Mercurio è il pianeta
del sistema solare per il quale la precessione relativistica del perielio è più
evidente in quanto è il più interno e quindi il più influenzato dal campo
gravitazionale del Sole.

\subsection{Deflessione della luce}
\label{sec:deflessione-luce}

In questo paragrafo studieremo il moto di un fotone nel campo gravitazionale di
un corpo massivo, in particolare del Sole.

% TODO: non ho mai classificato gli intervalli nei tipi `luce', `spazio' o
% `tempo', quindi o lo faccio da qualche parte o riformulo la frase seguente.
% Dovrei anche spiegare cos'è la linea d'universo.
In relatività speciale, per due eventi separati da un intervallo di tipo luce la
distanza vale $\dd\tau^{2} = 0$ e assumiamo che lo stesso valga anche in
relatività generale.  Assumiamo inoltre che l'equazione della
geodetica~\eqref{eq:geodetica} sia valida anche per le particelle prive di massa
come i fotoni.  Tuttavia dobbiamo fare attenzione al fatto che, proprio in virtù
del fatto che $\dd\tau^{2} = 0$, non possiamo parametrizzare la linea d'universo
di un fotone, o di una qualsiasi particella con massa nulla, usando il tempo
proprio $\tau$.  Piuttosto dobbiamo utilizzare un differente parametro $q$ tale
che l'equazione della geodetica
\begin{equation}
  \toder[2]{x^{\mu}}{q} + \tensor{\Gamma}{^{\mu}_{\lambda\nu}}
  \toder{x^{\lambda}}{q} \toder{x^{\nu}}{q} = 0
\end{equation}
abbia senso.  A partire da questa equazione si possono ricavare nuovamente,
sempre con l'assunzione $\theta = \pi/2$, le relazioni~\eqref{eq:mercurio-phi2}
e \eqref{eq:mercurio-t2}.  La~\eqref{eq:mercurio-r2} va modificata in questo
modo tenendo conto del fatto che $\dd\tau^{2} = 0$
\begin{equation}
  0 = \bigg(1 - \frac{r_{\textup{S}}}{r}\bigg)^{-1}l^{2} - \bigg(1 -
  \frac{r_{\textup{S}}}{r}\bigg)^{-1}\dot{r}^{2} - \frac{h^{2}}{r^{2}}.
\end{equation}
Svolgendo i calcoli in maniera analoga a quanto visto nel paragrafo precedente e
ponendo sempre $u = 1/r$ si arriva all'equazione
\begin{equation}
  u'\bigg(u'' + u - \frac{3}{2}r_{\textup{S}}u^{2}\bigg) = 0.
\end{equation}
Si possono avere due situazioni
\begin{equation}
  \label{eq:deflessione1}
  u'' + u - \frac{3}{2}r_{\textup{S}}u^{2} = 0
\end{equation}
oppure
\begin{equation}
  u' = 0 \implies u(\phi) = u_{0} = \text{costante},
\end{equation}
ma quest'ultima soluzione deve essere scartata per ragioni di stabilità.
Osserviamo che si poteva giungere direttamente alla~\eqref{eq:deflessione1}
partendo dalla~\eqref{eq:mercurio1} e ponendo $\dd\tau = 0$.  Il termine
$3r_{\textup{S}}u^{2}/2$ è una piccola perturbazione rispetto a $u$, infatti
\begin{equation}
  \frac{3r_{\textup{S}}u^{2}/2}{u} = \frac{3r_{\textup{S}}u}{2} =
  \frac{3r_{\textup{S}}}{2r}.
\end{equation}
Considerando il caso del Sole, $r_{\textup{S}} = \SI{3}{\kilo\metre}$ e la
distanza $r$ rispetto alla sua origine a cui un fotone può passare è maggiore o
uguale al raggio $R_{\odot} = \SI{7e5}{\kilo\metre}$, quindi
$3r_{\textup{S}}u^{2}/(2u) \ll 1$.  Allora cerchiamo una soluzione dell'equazione
\begin{equation}
  u'' + u = \epsilon u^{2},
\end{equation}
con $\epsilon = 3r_{\textup{S}}/2$, del tipo
\begin{equation}
  u(\phi) = u_{0}(\phi) + \epsilon v(\phi).
\end{equation}
Sostituendo nell'equazione da risolvere abbiamo
\begin{equation}
  \label{eq:deflessione2}
  u_{0}'' + u_{0} + \epsilon v'' + \epsilon v = \epsilon u_{0}^{2} +
  \mathcal{O}(\epsilon^{2}).
\end{equation}
I termini di ordine zero in $\epsilon$ danno
\begin{equation}
  u_{0}'' + u_{0} = 0,
\end{equation}
la cui soluzione generale è
\begin{equation}
  u_{0}(\phi) = A\cos(\phi + \delta),
\end{equation}
con $A$ e $\delta$ costanti di integrazione.  Ruotando il sistema di coordinate
è possibile annullare $\delta$ cosicché il valore approssimato $r = 1/u_{0}$
della coordinata radiale soddisfi la relazione
\begin{equation}
  r \cos\phi = \frac{1}{A} = \text{costante}.
\end{equation}
Il prodotto $r\cos\phi$ è la coordinata cartesiana $x$, quindi questa è
l'equazione di una retta parallela all'asse $y$.  Potevamo attenderci questo
risultato: in prima approssimazione il raggio di luce non viene deviato dal
campo gravitazionale del Sole.  Inoltre la distanza $r_{0}$ di minimo
avvicinamento al corpo si ha per $\cos\phi = 1$, cioè
$r_{0} = 1/A \implies u_{0} = (\cos\phi)/r_{0}$.  Uguagliando i termini di
ordine uno in $\epsilon$ troviamo
\begin{equation}
  \label{eq:deflessione3}
  v'' + v = u_{0}^{2} = \frac{\cos^{2}\phi}{r_{0}^{2}} = \frac{1 +
    \cos(2\phi)}{2r_{0}^{2}}.
\end{equation}
Cerchiamo soluzioni del tipo
\begin{equation}
  v(\phi) = \alpha + \beta\cos(2\phi)
\end{equation}
con $\alpha$ e $\beta$ costanti di integrazione da determinare.  Sostituendo
nella~\eqref{eq:deflessione3} otteniamo
\begin{equation}
  v'' + v = \alpha - 3\beta\cos(2\phi).
\end{equation}
Confrontando con la~\eqref{eq:deflessione3} abbiamo
\begin{subequations}
  \begin{align}
    \alpha &= \frac{1}{2r_{0}^{2}}, \\
    \beta &= - \frac{1}{6r_{0}^{2}},
  \end{align}
\end{subequations}
quindi
\begin{equation}
  v(\phi) = \frac{1}{2r_{0}^{2}} - \frac{\cos(2\phi)}{6r_{0}^{2}} =
  \frac{2}{3r_{0}^{2}} - \frac{\cos^{2}\phi}{3r_{0}^{2}}.
\end{equation}
In definitiva
\begin{equation}
  \label{eq:deflessione4}
  u(\phi) = u_{0}(\phi) + v(\phi) = \frac{\cos\phi}{r_{0}} +
  \frac{2\epsilon}{3r_{0}^{2}} - \frac{\epsilon\cos^{2}\phi}{3r_{0}^{2}}.
\end{equation}
% TODO: fare una figura altrimenti si capisce piuttosto poco e riformulare il
% testo seguente in modo da adattarsi alla figura.
La traiettoria del fotone è, dunque, una retta (data dal termine
$(\cos\phi)/r_{0}$) disturbata da una piccola perturbazione dell'ordine di
$\epsilon$.  L'effetto di questa perturbazione è il seguente: il fotone si
avvicina al Sole in linea retta, quando l'influenza del campo gravitazionale
della stella non è più trascurabile il fotone viene leggermente deflesso, quindi
si allontana nuovamente in linea retta.  Calcoliamo la deflessione del fotone.
Le direzioni asintotiche seguite dal fotone si ottengono calcolando il limite
$r \to \infty$ o, il che è equivalente, $u \to 0$ nella~\eqref{eq:deflessione4}
\begin{equation}
  0 = \frac{\cos\phi}{r_{0}} + \frac{2\epsilon}{3r_{0}^{2}} -
  \frac{\epsilon\cos^{2}\phi}{3r_{0}^{2}} \implies \cos^{2}\phi -
  \frac{3r_{0}}{\epsilon}\cos\phi - 2 = 0.
\end{equation}
Le soluzioni sono
\begin{equation}
  \cos\phi = \frac{3r_{0}}{2\epsilon}\bigg(1 \pm \sqrt{1 +
    \frac{8\epsilon^{2}}{9r_{0}^{2}}}\bigg),
\end{equation}
ma l'unica accettabile è quella con il segno $-$, poiché quella con il segno $+$
rende il coseno maggiore di $1$.  Sviluppando la soluzione accettabile al primo
ordine in $\epsilon$ abbiamo
\begin{equation}
  \cos\phi = \frac{3r_{0}}{2\epsilon}\bigg(1 - \sqrt{1 +
    \frac{8\epsilon^{2}}{9r_{0}^{2}}}\bigg) \approx \frac{3r_{0}}{2\epsilon}\bigg(1 - 1 -
  \frac{4\epsilon^{2}}{9r_{0}^{2}}\bigg) = -\frac{2\epsilon}{3r_{0}} =
  -\frac{r_{\textup{S}}}{r_{0}}.
\end{equation}
Abbiamo già osservato che per i fotoni che passano nelle vicinanze del Sole
risulta $r_{\textup{S}}/r \ll 1$, quindi $\phi \approx \pm \pi/2$, allora
possiamo porre $\psi = \pi/2 + \delta$, da cui
\begin{equation}
  -\frac{r_{\textup{S}}}{r_{0}} = \cos\phi = \cos(\pi/2 + \delta) = -\sin\delta.
\end{equation}
Dal momento che $\delta$ è molto piccolo possiamo approssimare il seno
dell'angolo con l'angolo stesso e quindi
\begin{equation}
  \delta = \frac{r_{\textup{S}}}{r_{0}}.
\end{equation}
La deflessione $\var$ è il doppio di questo angolo, dunque, ripristinando il
valore della velocità della luce,
\begin{equation}
  \var = 2\delta = \frac{2r_{\textup{S}}}{r_{0}} = \frac{4GM}{c^{2}r_{0}}.
\end{equation}

Per quanto riguarda la deviazione dei raggi di luce da parte del campo
gravitazionale del Sole, supponendo che la distanza di minimo avvicinamento sia
proprio pari al raggio solare $R_{\odot}$, l'angolo di deflessione vale
\begin{equation}
  \var_{\odot} = \frac{4GM_{\odot}}{c^{2}R_{\odot}} = \ang{;;1.75}.
\end{equation}
Questa previsione teorica è stata confermata in numerosi esperimenti via via più
accurati, il primo dei quali fu realizzato da Eddington durante l'eclissi totale
di Sole del 1919.

\subsection{Ritardo dei segnali radar}
\label{sec:ritardo-radar}

L'ultimo fenomeno di cui ci interessiamo è il ritardo nel tempo di volo dei
segnali radar rispetto a quanto previsto dalla relatività speciale in uno
spazio-tempo piatto.  Ciò è dovuto al fatto che i fotoni non si muovono lungo
linee rette ma lungo le geodetiche, che hanno lunghezza maggiore rispetto.
Questo fenomeno può essere osservato inviando un segnale radar a un pianeta in
congiunzione superiore e misurando il tempo necessario affinché il segnale vada
verso il pianeta e torni sulla Terra.  Tuttavia l'esecuzione di esperimenti su
questo fenomeno presentano delle difficoltà di tipo pratico, delle quali non ci
occuperemo.

% TODO: mettere anche qui una figura e rielaborare il testo altrimenti non si
% capisce niente.  Dopo aver fatto la figura, parlare in generale di distanza di
% minimo avvicinamento piuttosto che di raggio solare.
Senza entrare nei dettagli dei conti forniamo i risultati delle previsioni
teoriche della relatività
generale.\footnote{Questo problema è trattato da
  \textcite[1103-1109]{misner:gravitation},
  \textcite[201-207]{weinberg:gravitation}.}
In un spazio-tempo piatto, il tempo $\tau(r,R_{\odot})$ necessario per un fotone
per giungere tangente sulla superficie del Sole da un punto a distanza $r$ dal
centro della stella è dato semplicemente dal rapporto fra la distanza euclidea
fra questi due punti e la velocità dei fotoni nel vuoto
\begin{equation}
  \tau(r, R_{\odot}) = \frac{\sqrt{r^{2} - R_{\odot}^{2}}}{c}.
\end{equation}
La correzione relativistica per , che tiene conto della curvatura dello spazio tempo
prodotta dal campo gravitazionale del Sole, è
\begin{equation}
  \begin{split}
    t(r,R_{\odot}) &\approx \tau(r,R_{\odot}) + \var t(r,R_{\odot}) \\
    &= \frac{\sqrt{r^{2} - R_{\odot}^{2}}}{c} + \frac{2GM_{\odot}}{c^{3}} \ln
    \frac{r + \sqrt{r^{2} - R_{\odot}^{2}}}{R_{\odot}} +
    \frac{GM_{\odot}}{c^{3}} \sqrt{\frac{r - R_{\odot}}{r + R_{\odot}}}.
  \end{split}
\end{equation}
Il tempo $t_{\textup{tot}}$ totale di volo del fotone per andare dalla Terra, a
distanza $r_{1}$ dal Sole, a un pianeta a distanza $r_{2}$ in congiunzione
superiore è
\begin{equation}
  \begin{split}
    t_{\textup{tot}} &= 2(t(r_{1},R_{\odot}) + t(r_{2},R_{\odot})) \\
    &= 2(\tau(r_{1},R_{\odot}) + \var t(r_{1},R_{\odot}) +
    \tau(r_{2},R_{\odot}) + \var t(r_{2},R_{\odot})).
  \end{split}
\end{equation}
Considerando il sistema Terra-Mercurio, il ritardo
$\var t = 2(\var t(r_{1},R_{\odot}) + \var t(r_{2},R_{\odot}))$ rispetto alla
previsione $2(\tau(r_{1},R_{\odot}) + \tau(r_{2},R_{\odot}))$ in uno
spazio-tempo piatto vale \SI{240}{\micro\second}, che corrisponde a una distanza
di \SI{72}{\kilo\metre}, su un tempo di volo totale $t_{\textup{tot}}$ di
\SI{20}{\minute}.

%%% Local Variables:
%%% mode: latex
%%% TeX-master: "../gravitazione"
%%% fill-column: 80
%%% End:

\chapter{Onde gravitazionali}
\label{cha:onde-grav}

In questo capitolo ci occuperemo della risoluzione delle equazioni di Einstein
nel caso di un campo gravitazionale debole, rilasciando l'ipotesi di simmetria
sferica e/o indipendenza dal tempo della metrica.  Trascurando nel calcolo del
tensore di Ricci i termini dell'ordine $\mathcal{O}(h^2)$, otterremo equazioni
di Einstein linearizzate analoghe all'equazione per le onde elettromagnetiche.
Dal punto di vista fisico, il processo di linearizzazione corrisponde a
trascurare l'energia trasportata dalle onde gravitazionali che si propagano
liberamente senza auto influenzarsi.

Troveremo che masse accelerate possono emettere onde gravitazionali così come
particelle cariche accelerate possono emettere onde elettromagnetiche (vedi
paragrafi~\ref{sec:onde-elettromagnetiche-vuoto} e
\ref{sec:onde-elettromagnetiche-cariche}).

L'esistenza delle onde gravitazionali è facile da intuire in base al seguente
ragionamento.  Una variazione della distribuzione delle masse in un sistema, ad
esempio una binaria molto stretta, produce una variazione del campo
gravitazionale.  Tale informazione però (per il principio di causalità), non può
propagarsi istantaneamente fino a grandi distanze, ma si diffonderà alla
velocità della luce.  La variazione del campo gravitazionale giungerà lontano
dalla sorgente con un ritardo pari al tempo necessario a un segnale luminoso per
andare dalla sorgente fino al punto considerato.  Questa propagazione della
perturbazione del campo gravitazionale, e quindi della geometria dello
spazio-tempo, è l'onda gravitazionale.

\section{Approssimazione di campo debole}
\label{sec:approx-campodebole]}

Nell'approssimazione di campo debole il tensore metrico covariante è dato da
\begin{equation}
  g_{\mu \nu}(t, \bm{x}) = \eta_{\mu \nu} + h_{\mu \nu}(t, \bm{x}) +
  \mathcal{O}(h^2)
  \label{metrica_contro}
\end{equation}
con $\abs{h_{\mu\nu}} \ll 1$, mentre il tensore metrico controvariante è
\begin{equation}
  g^{\mu\nu}(t, \bm{x}) = \eta^{\mu\nu} - h^{\mu\nu}(t, \bm{x}) +
  \mathcal{O}(h^2)
\end{equation}
poiché si ha $g^{\mu\nu} g_{\mu\lambda} = \delta^{\nu}_{\lambda}$.

Da questo punto in poi, l'innalzamento e l'abbassamento degli indici dei tensori
si esegue con il tensore metrico non perturbato $\eta_{\mu\nu}$.  Per
linearizzare le equazioni di Einsetin dobbiamo sviluppare quindi $R_{\mu\nu}$ in
serie di potenze di $h_{\mu\nu}$ e considerare solo i termini del primo ordine.

Innanzitutto osserviamo che possiamo sviluppare il tensore di Ricci in serie di
potenze di $h_{\mu\nu}$ nel seguente modo
\begin{equation}
  R_{\mu\nu} = R_{\mu\nu}^{(1)} + R_{\mu\nu}^{(2)} + \mathcal{O}(h^{3}).
\end{equation}
L'equazione di Einstein esatta, in presenza di materia può essere allora scritta
\begin{equation}
  \label{eq:einstein-lineare-esatta}
  \bigg(R_{\mu\nu}^{(1)} - \frac{1}{2}\eta_{\mu\nu}R^{(1)}\bigg) = -8\pi
  G(T_{\mu\nu} + t_{\mu\nu})
\end{equation}
in cui $R^{(1)} = \eta^{\mu\nu}R_{\mu\nu}^{(1)}$ e il tensore $t_{\mu\nu}$ è
definito da
\begin{equation}
  8\pi G t_{\mu\nu} = \left[ R_{\mu\nu} - \frac{1}{2}\eta_{\mu\nu} R \right]
  - \left[R_{\mu\nu}^{(1)} - \frac{1}{2}\eta_{\mu\nu}R^{(1)} \right] \approx
  R_{\mu\nu}^{(2)} - \frac{1}{2} \eta_{\mu\nu} R^{(2)}.
\end{equation}
Dunque $t_{\mu\nu}$ contiene solo termini superiori al primo ordine in
$h_{\mu\nu}$.  La~\eqref{eq:einstein-lineare-esatta} può essere interpretata
come un'equazione d'onda lineare in $h_{\mu\nu}$ in cui il termine di sorgente
$\tau_{\mu\nu} = T_{\mu\nu} + t_{\mu\nu}$ dipende esplicitamente dal campo
gravitazionale $h_{\mu\nu}$ stesso.  Questo suggerisce che $h_{\mu\nu}$ è
generato dal tensore $\tau_{\mu\nu}$, flusso e densità totale di energia e
momento, in cui $T_{\mu\nu}$ è il contributo della sola materia e $t_{\mu\nu}$
(che è del secondo ordine in $h_{\mu\nu}$) è associato al campo
gravitazionale.\footnote{In realtà si dimostra che $t_{\mu\nu}$ si comporta come
  uno pseudotensore.  Vedi~\textcite[367-374]{landau:campi}.  Ulteriori prove a
  sostegno di questa interpretazione sono riportate
  in~\textcite[165-171]{weinberg:gravitation}.}

Trascurare nelle equazioni di Einstein i termini superiori al primo ordine di
infinitesimo significa pertanto trascurare i contributi dell'energia del campo
gravitazionale stesso.

\section{Onde gravitazionali nel vuoto}
\label{sec:onde-grav-vuoto}

La connessione affine approssimata al primo ordine in $h_{\mu\nu}$ è
\begin{equation}
  \tensor{\Gamma}{^{\lambda}_{\mu\nu}^{(1)}} \approx \frac{1}{2}\eta^{\sigma\lambda}
  (\partial_{\mu}h_{\nu\sigma} + \partial_{\nu}h_{\mu\sigma}
  - \partial_{\sigma}h_{\mu\nu})
\end{equation}
e quindi il tensore di Ricci è dato da
\begin{equation}
  \label{eq:ricci-lineare}
  \begin{split}
    R_{\mu\nu}^{(1)} &= \partial_{\nu}
    \tensor{\Gamma}{^{\lambda}_{\mu\lambda}^{(1)}}
    - \partial_{\lambda} \tensor{\Gamma}{^{\lambda}_{\mu\nu}^{(1)}} +
    \overbrace{\tensor{\Gamma}{^{\eta}_{\mu\lambda}^{(1)}}
      \tensor{\Gamma}{^{\lambda}_{\eta\nu}^{(1)}} -
      \tensor{\Gamma}{^{\eta}_{\mu\nu}^{(1)}}
      \tensor{\Gamma}{^{\lambda}_{\eta\lambda}^{(1)}}}^{\mathcal{O}(h^{2})} \\
    &\approx \partial_{\nu}\tensor{\Gamma}{^{\lambda}_{\mu\lambda}^{(1)}}
    - \partial_{\lambda} \tensor{\Gamma}{^{\lambda}_{\mu\nu}^{(1)}} \\
    &= \frac{1}{2}\partial_{\nu} (\eta^{\lambda\sigma}(\partial_{\mu}
    h_{\sigma\lambda} + \partial_{\lambda}h_{\mu\sigma}
    - \partial_{\sigma}h_{\mu\lambda})) -
    \frac{1}{2}\partial_{\lambda}(\eta^{\lambda\sigma}(\partial_{\mu}h_{\sigma\nu}
    + \partial_{\nu}h_{\mu\sigma} - \partial_{\sigma}h_{\mu\nu})) \\
    &= \frac{1}{2} \eta^{\lambda\sigma}(\partial_{\mu}\partial_{\nu}
    h_{\sigma\lambda} + \partial_{\lambda}\partial_{\nu} h_{\mu\sigma}
    - \partial_{\sigma}\partial_{\nu} h_{\mu\lambda}
    - \partial_{\mu}\partial_{\lambda} h_{\sigma\nu}
    - \partial_{\nu}\partial_{\lambda} h_{\mu\sigma}
    + \partial_{\sigma}\partial_{\lambda} h_{\mu\nu}) \\
    &= \frac{1}{2}(\partial_{\mu}\partial_{\nu}
    \tensor{h}{^{\lambda}_{\lambda}} - \partial^{\lambda}\partial_{\nu}
    h_{\mu\lambda} - \partial^{\lambda}\partial_{\mu} h_{\nu\lambda}
    + \partial_{\lambda}\partial^{\lambda} h_{\mu\nu}) \\
    &= \frac{1}{2} (\dalamb h_{\mu\nu} + \partial_{\mu}\partial_{\nu} h
    - \partial^{\lambda}\partial_{\nu} h_{\mu\lambda}
    - \partial_{\mu}\partial^{\lambda} h_{\nu\lambda})
  \end{split}
\end{equation}
in cui $h = \tensor{h}{^{\lambda}_{\lambda}} = \eta^{\mu\lambda} h_{\mu\lambda}$
è la traccia della perturbazione.  Nel vuoto allora abbiamo
\begin{equation}
  R_{\mu\nu}^{(1)} = \frac{1}{2} (\dalamb h_{\mu\nu}
  + \partial_{\mu}\partial_{\nu} h - \partial^{\lambda}\partial_{\nu}
  h_{\mu\lambda} - \partial_{\mu}\partial^{\lambda} h_{\nu\lambda}) = 0.
\end{equation}

A questo punto vogliamo manipolare l'espressione precedente in modo che le
equazioni di Einstein assumano una forma più semplice, analoga alle equazioni di
Maxwell nel vuoto $\dalamb A_{\beta} = 0$.  Come nel caso elettromagnetico (in
cui per il potenziale si sceglie la gauge di Lorenz $\partial^{\alpha}/\partial
A_{\alpha}=0$, è possibile fare ciò scegliendo un sa gauge (un istema di
coordinate) opportuna.  In particolare conviene scegliere la
\index{gauge!armonica}\emph{gauge armonica} per cui si ha
\begin{equation}
  \label{eq:gauge-armonica}
  \partial_{\mu} \tensor{h}{^{\mu}_{\nu}} =
  \frac{1}{2}\partial_{\nu}\tensor{h}{^{\mu}_{\mu}}.
\end{equation}
Infatti usando questa gauge gli ultimi tre termini in $R_{\mu \nu}^{(1)}$
si annullano
\begin{equation}
  \begin{split}
    \partial_{\mu}\partial_{\nu} h - \partial^{\lambda}\partial_{\nu}
    h_{\mu\lambda} - \partial_{\mu}\partial^{\lambda} h_{\nu\lambda}
    &= \partial_{\mu}(2 \partial_{\lambda}\tensor{h}{^{\lambda}_{\nu}})
    - \partial_{\nu}\partial_{\lambda}\tensor{h}{^{\lambda}_{\mu}}
    - \partial_{\lambda}\partial_{\mu} \tensor{h}{^{\lambda}_{\nu}} \\
    &= \partial_{\mu}\partial_{\lambda} \tensor{h}{^{\lambda}_{\nu}}
    - \partial_{\nu}\partial_{\lambda} \tensor{h}{^{\lambda}_{\mu}} \\
    &= \partial_{\mu}\bigg(\frac{1}{2}\partial_{\nu}
    \tensor{h}{^{\lambda}_{\lambda}}\bigg)
    - \partial_{\nu}\bigg(\frac{1}{2}\partial_{\mu}
    \tensor{h}{^{\lambda}_{\lambda}}\bigg) = 0.
  \end{split}
\end{equation}
Quindi, nel vuoto  le equazioni di Einstein linearizzate sono
\begin{equation}
  \label{eq:einstein-lineare}
  \dalamb h_{\mu\nu} = 0.
\end{equation}

Osserviamo che la condizione di gauge armonica si può anche porre nella forma
$\eta^{\mu\nu}\tensor{\Gamma}{^{\lambda}_{\mu\nu}} = 0$ in quanto
\begin{equation}
  \begin{split}
    0 &=\eta^{\mu\nu} \tensor{\Gamma}{^{\lambda}_{\mu\nu}} = \frac{1}{2}
    \eta^{\lambda\sigma} (\partial^{\mu}h_{\mu\sigma} + \partial^{\mu}
    h_{\mu\sigma} - \partial_{\sigma}\tensor{h}{^{\mu}_{\mu}}) \\
    &= \frac{1}{2} (\partial^{\mu} \tensor{h}{_{\mu}^{\lambda}} + \partial^{\mu}
    \tensor{h}{_{\mu}^{\lambda}} - \partial^{\lambda} \tensor{h}{^{\mu}_{\mu}})
    \iff \partial^{\lambda} \tensor{h}{^{\mu}_{\mu}} = 2 \partial^{\mu}
    \tensor{h}{_{\mu}^{\lambda}}
  \end{split}
\end{equation}
che equivale alla~\eqref{eq:gauge-armonica}.

\textbf{Digressione}: mostriamo ora che è sempre possibile scegliere un sistema
di coordinate in cui sia valida la condizione armonica (in analogia al caso
elettromagnetico in cui se fosse $\partial^{\alpha}/ \partial A_{\alpha} \ne 0$,
la trasformazione $A_{\alpha} \to A'_{\alpha} = A_{\alpha} + \partial \Lambda(x)
/ \partial x^{\alpha}$ rende $\partial^{\beta}/ \partial A'_{\beta} = 0$ qualora
la funzione $\Lambda(x)$ fosse la soluzione dell'equazione $ \dalamb \Lambda
=0$).

Sia $x^{\mu}$ un sistema di coordinate in cui la~\eqref{eq:gauge-armonica} non è
valida.  Allora effettuiamo una trasformazione infinitesima
\begin{equation}
  x^{\mu} \to x'^{\mu} = x^{\mu} + \epsilon^{\mu}(x)
\end{equation}
con $\abs{\epsilon^{\mu}(x)} \ll 1$.  Nel nuovo sistema di coordinate $x^{'\mu}$
si ha
\begin{equation}
  g'^{\mu\nu} = g^{\lambda\rho} \parder{x'^{\mu}}{x^{\lambda}}
  \parder{x'^{\nu}}{x^{\rho}}
\end{equation}
Trascurando i termini del secondo ordine - $\mathcal{O}(\epsilon^2)$,
$\mathcal{O}(h \epsilon)$ e $\mathcal{O}(h^2)$ - abbiamo
 \begin{equation}
  \begin{split}
    g'^{\mu\nu} &=
    (\eta^{\lambda\rho} - h^{\lambda\rho}) (\tensor{\delta}{^{\mu}_{\lambda}}
    + \partial_{\lambda}\epsilon^{\mu}) (\tensor{\delta}{^{\nu}_{\rho}}
    + \partial_{\rho} \epsilon^{\nu}) \\
    &= (\eta^{\mu\rho} - h^{\mu\rho} +
    \eta^{\lambda\rho}\partial_{\lambda}\epsilon^{\mu} -
    \underbrace{h^{\lambda\rho} \partial_{\lambda}
      \epsilon^{\mu}}_{\mathcal{O}(h\epsilon)})
    (\tensor{\delta}{^{\nu}_{\rho}} + \partial_{\rho} \epsilon^{\nu}) \\
    &\approx \eta^{\mu\nu} - h^{\mu\nu} + \eta^{\lambda\nu}\partial_{\lambda}
    \epsilon^{\mu} + \eta^{\mu\rho}\partial_{\rho} \epsilon^{\nu} -
    \underbrace{h^{\mu\rho}\partial_{\rho}\epsilon^{\nu}}_{\mathcal{O}(h\epsilon)}
    + \underbrace{\partial^{\rho} \epsilon^{\mu} \partial_{\rho}
      \epsilon^{\nu}}_{\mathcal{O}(\epsilon^{2})} \\
    & \approx \eta^{\mu\nu} - h^{\mu\nu} + \partial^{\nu} \epsilon^{\mu}
    + \partial^{\mu} \epsilon^{\nu}.
  \end{split}
\end{equation}
Dunque
\begin{equation}
  h'^{\mu\nu} = h^{\mu\nu} - \partial^{\nu} \epsilon^{\mu} - \partial^{\mu}
  \epsilon^{\nu}.
\label{tras_infit_metrica}
\end{equation}
Nel nuovo sistema di riferimento calcoliamo ambo i membri
della~\eqref{eq:gauge-armonica}
\begin{equation}
  \begin{split}
    &\partial_{\mu} \tensor{{h'}}{^{\mu}_{\nu}} = \frac{1}{2}\partial_{\nu}
    \tensor{{h'}}{^{\mu}_{\mu}} \iff \\
    &\partial_{\mu} \tensor{h}{^{\mu}_{\nu}}
    - \partial_{\mu} \partial_{\nu} \epsilon^{\mu}
    - \partial_{\mu}\partial^{\mu} \epsilon_{\nu}
    = -\frac{1}{2} \partial_{\nu} \partial_{\mu} \epsilon^{\mu} -
    \frac{1}{2}\partial_{\nu} \partial_{\mu} \epsilon^{\mu} +
    \frac{1}{2} \partial_{\nu} \tensor{h}{^{\mu}_{\mu}} \iff \\
    &\partial_{\mu} \tensor{h}{^{\mu}_{\nu}}
    - \partial_{\mu}\partial^{\mu} \epsilon_{\nu}
    = \frac{1}{2} \partial_{\nu} \tensor{h}{^{\mu}_{\mu}}.
  \end{split}
\end{equation}
Quindi se scegliamo il generatore $\epsilon_{\nu}$ della trasformazione
infinitesima delle coordinate in modo tale che
\begin{equation}
  \dalamb \epsilon_{\nu} = \partial_{\mu} \tensor{h}{^{\mu}_{\nu}} -
  \frac{1}{2} \partial_{\nu} \tensor{h}{^{\mu}_{\mu}}
\end{equation}
allora nel nuovo sistema di riferimento la condizione
armonica~\eqref{eq:gauge-armonica} sarà soddisfatta.  Come osservato, questa
libertà di scelta ricorda la libertà che si ha nella gauge di Lorenz per il
potenziale elettromagnetico. (\emph{Fine della digressione})

Ritorniamo alla~\eqref{eq:einstein-lineare} e supponiamo ora soddisfatta la
condizione di gauge armonica~\eqref{eq:gauge-armonica}.  Come per il caso
elettromagnetico, la soluzione generale $h_{\mu\nu}$ di questa equazione sarà
una sovrapposizione lineare di onde piane del tipo
\begin{equation}
  h_{\mu\nu}(t,\bm{x}) = \Re\{e_{\mu\nu} \e^{\uimm k^{\lambda}x_{\lambda}} \}.
\end{equation}
Queste sono le \index{onda!gravitazionale}\emph{onde gravitazionali}.

Dall'equazione d'onda, il 4-vettore numero d'onda $k^{\lambda} \equiv (\omega,
\bm{k})$ (omettendo il simbolo $\Re$ per quanto detto nel
paragrafo~\ref{sec:onde-elettromagnetiche-vuoto}) soddisfa la condizione
\begin{equation}
  0 = \dalamb h_{\mu\nu} = \uimm k^{\lambda} \uimm k_{\lambda} e_{\mu\nu}
  \e^{\uimm k^{\lambda}x_{\lambda}} \implies k^{\lambda}k_{\lambda} = 0 \rightarrow \lambda \nu=1
\end{equation}
cioè le onde gravitazionali si muovono, come quelle elettromagnetiche, alla
velocità della luce nel vuoto, con frequenza $\nu$ uguale all'inverso della
lunghezza d'onda $\lambda$.

Inoltre, per la  gauge armonica~\eqref{eq:gauge-armonica} abbiamo
\begin{equation}
  \uimm k_{\lambda} \tensor{e}{^{\lambda}_{\nu}} \e^{\uimm k^{\lambda}
    x_{\lambda}} = \frac{\uimm}{2} k_{\nu}
  \tensor{e}{^{\lambda}_{\lambda}} \e^{\uimm k^{\lambda}x_{\lambda}} \iff
  k_{\lambda} \tensor{e}{^{\lambda}_{\nu}} = \frac{1}{2} k_{\nu}
  \tensor{e}{^{\lambda}_{\lambda}},
  \label{8.22}
\end{equation}
che in analogia la caso elettromagnetico (per cui è $k^{\alpha} A_{\alpha}=0$)
implica (come ora dimostriamo) che le onde gravitazionale sono trasversali, cioè
hanno componente nulla nella direzione di propagazione.

Il tensore $e_{\mu\nu}$ è chiamato
\index{tensore!di polarizzazione}\emph{tensore di polarizzazione} e come
$h_{\mu\nu}$ è simmetrico nello scambio degli indici.  Esso ha quindi dieci
componenti indipendenti, che si riducono a 6 per le 4 condizioni di
gauge~\eqref{eq:gauge-armonica}.  Scegliendo poi in maniera opportuna le
coordinate (usando quindi ulteriori 4 gradi di libertà) avremo alla fine che
solo 2 componenti del tensore di polarizzazione sono differenti da zero.  Allora
esisteranno solo due stati di polarizzazione indipendenti.

Scegliamo l'asse $z$ lungo la direzione di propagazione dell'onda, di modo che
il vettore d'onda abbia la forma $k^{\mu} = (k, 0, 0, k)$.  Ponendo
nell'equazione~\eqref{8.22} $\nu = 1$ abbiamo
\begin{equation}
  k_{\mu}\tensor{e}{^{\mu}_{1}} = \frac{1}{2} k_{1}\tensor{e}{^{\mu}_{\mu}}
  \equiv 0.
\end{equation}
Esplicitiamo il primo membro e tenendo conto che $k_{\nu} \equiv \eta_{\mu \nu}
k^{\mu} = (-k, 0, 0, k)$, dunque si ha
\begin{equation}
  0 = k_{0} \tensor{e}{^{0}_{1}} + k_{3} \tensor{e}{^{3}_{1}} = -k
  \tensor{e}{^{0}_{1}} + k \tensor{e}{^{3}_{1}} \implies
  \tensor{e}{^{0}_{1}} = \tensor{e}{^{3}_{1}} \implies e_{01} = -e_{31}.
\end{equation}
Analogamente per $\nu = 2$ si trova
\begin{equation}
  k_{\mu} \tensor{e}{^{\mu}_{2}} = \frac{1}{2} k_{2} \tensor{e}{^{\mu}_{\mu}} =
  0 \implies e_{02} = -e_{32}.
\end{equation}
Per $\nu = 0$ risulta
\begin{equation}
  \begin{split}
    &k_{\mu} \tensor{e}{^{\mu}_{0}} = \frac{1}{2} k_{0}\tensor{e}{^{\mu}_{\mu}}
    = -\frac{1}{2}k^{0}\tensor{e}{^{\mu}_{\mu}} \implies \\
    &k^{0} e_{00} + k^{3}e_{30} = -\frac{1}{2}k^{0}(\tensor{e}{^{1}_{1}} +
    \tensor{e}{^{2}_{2}} + \tensor{e}{^{3}_{3}} - e^0_0)
  \end{split}
\end{equation}
da cui si ottiene ($e^0_0 = - e_{00}$)
\begin{equation}
  e_{00} + e_{30} = -\frac{1}{2}(e_{11} + e_{22} + e_{33} - e_{00}).
\end{equation}
In maniera analoga, per $\nu = 3$ si trova
\begin{equation}
  e_{03} + e_{33} = \frac{1}{2}(e_{11} + e_{22} + e_{33} - e_{00}).
\end{equation}
Sommando membro a membro le ultime due equazioni abbiamo
\begin{equation}
  e_{30}= e_{03} = -\frac{1}{2}(e_{33} + e_{00})
\end{equation}
e inserendo questo risultato nell'equazione trovata per $\nu = 3$ otteniamo
\begin{equation}
  e_{22} = -e_{11}.
\end{equation}
Riepilogando, fino a questo punto abbiamo trovato che per un'onda gravitazionale
la quale si propaga lungo l'asse $z$ si ha
\begin{subequations}
  \begin{align}
    e_{01} &= -e_{31}, \\
    e_{02} &= -e_{32}, \\
    e_{03} &= -\frac{1}{2}(e_{33} + e_{00}), \\
    \label{8.31}
    e_{22} &= -e_{11}.
  \end{align}
\end{subequations}

Effettuiamo ora una trasformazione infinitesima di coordinate $x^{\mu} \to
x'^{\mu} = x^{\mu} + \epsilon^{\mu}(x)$.  Perché continui a valere la gauge
armonica, $\epsilon^{\mu}$ deve soddisfare l'equazione $\dalamb \epsilon^{\mu} =
0$, che è ancora un'equazione d'onda.  La sua soluzione sarà data dalla
sovrapposizione di onde piane monocromatiche
\begin{equation}
  \epsilon^{\mu} = \Re\{\tilde{\epsilon}^{\mu}\e^{\uimm
    k^{\lambda}x_{\lambda}}\}.
\end{equation}
Per l'equazione~\eqref{tras_infit_metrica}, la perturbazione della metrica in
questo nuovo sistema è
\begin{equation}
  h'_{\mu\nu} = \Re\{e'_{\mu\nu} \e^{\uimm k^{\lambda}x_{\lambda}}\} =
  h_{\mu\nu} - \partial_{\nu}\epsilon_{\mu} - \partial_{\mu}\epsilon_{\nu}
\end{equation}
e poiché $\partial_{\nu}\epsilon_{\mu} = \uimm k_{\nu}\tilde{\epsilon}_{\mu}
\e^{\uimm k^{\lambda}x_{\lambda}}$, si ha
\begin{equation}
  e'_{\mu\nu} = e_{\mu\nu} - \uimm k_{\nu}\tilde{\epsilon}_{\mu} - \uimm
  k_{\mu}\tilde{\epsilon}_{\nu}.
\end{equation}
Così risulta
\begin{subequations}
  \begin{align}
    e'_{11} &= e_{11} - \uimm k_{1} \tilde{\epsilon}_{1} - \uimm
    k_{1}\tilde{\epsilon}_{1} = e_{11}, \\
    e'_{12} &= e_{12} - \uimm k_{2}\tilde{\epsilon}_{1} - \uimm
    k_{1}\tilde{\epsilon}_{2} = e_{12}, \\
    e'_{13} &= e_{13} - \uimm k_{3}\tilde{\epsilon}_{1} - \uimm
    k_{1}\tilde{\epsilon}_{3} = e_{13} - \uimm k\tilde{\epsilon}_{1}  \\
    e'_{23} &= e_{23} - \uimm k_{3}\tilde{\epsilon}_{2} - \uimm
    k_{2}\tilde{\epsilon}_{3} = e_{23} - \uimm k\tilde{\epsilon}_{2}  \\
    e'_{33} &= e_{33} - 2\uimm k_{3}\tilde{\epsilon}_{3} = \\
    e'_{00} &= e_{00} - 2\uimm k_{0}\tilde{\epsilon}_{0} =
  \end{align}
\end{subequations}
Allora, se scegliamo le quattro costanti
$\tilde{\epsilon}_{1} = -\uimm e_{13}/k$,
$\tilde{\epsilon}_{2} = -\uimm e_{23}/k$,
$\tilde{\epsilon}_{3} = -\uimm e_{33}/(2k)$,
$\tilde{\epsilon}_{0} = \uimm e_{00}/(2k)$
potremo annullare 4 componenti del tensore di polarizzazione, e cioè
$e'_{13} =  e'_{23} = e'_{33} = e'_{00} = 0$.
Inoltre, per le condizioni~\eqref{8.31} avremo anche
$e'_{01} = -e'_{31} = 0$, $e'_{02} = -e'_{32} = 0$ e
$e'_{03} = -(e'_{33}+e'_{00})/2 = 0$.

In definitiva per un'onda che si propaga lungo $z$ sono diversi da zero solo le
componenti $e_{22} = -e_{11}$ e $e_{12} = e_{21}$, e quindi per l'onda
gravitazionale vi sono due stati di polarizzazione indipendenti.

Osserviamo inoltre che la scelta di coordinate determina la
\index{gauge!trasversa a traccia nulla}\emph{gauge trasversa a traccia nulla}
(gauge TT): ``a traccia nulla'' perché $\tensor{e}{^{\mu}_{\mu}} =
\tensor{e}{^{0}_{0}} + \tensor{e}{^{1}_{1}} + \tensor{e}{^{2}_{2}} +
\tensor{e}{^{3}_{3}} = \tensor{e}{^{1}_{1}} + \tensor{e}{^{2}_{2}} = 0$,
``trasversa'' perché le componenti non nulle del tensore di polarizzazione sono
sulle direzioni perpendicolari alla direzione di propagazione dell'onda.

Per un'onda che si propaga lungo $z$, $h_{\mu\nu}^{\textup{TT}}$ è dato dalla
parte reale di
\begin{equation}
  \begin{pmatrix}
    0 & 0      & 0       & 0 \\
    0 & e_{11} & e_{12}  & 0 \\
    0 & e_{12} & -e_{11} & 0 \\
    0 & 0      & 0       & 0
  \end{pmatrix}\cdot \e^{\uimm k^{\lambda}x_{\lambda}}.
\end{equation}

In una teoria dei campi quantizzata
\begin{itemize}
\item a un \emph{campo scalare} (invariato per trasformazioni arbitrarie delle
  coordinate) si associa una particella (ad esempio il pione) di spin $0$;
\item a un \emph{campo vettoriale} (che, per effetto di un cambiamento
  arbitrario delle coordinate, si trasforma come un vettore) si associa un
  particella di spin $1$;
\item a un \emph{campo tensoriale a 2 indici} (che si trasforma come un tensore
  di rango $2$ sotto un cambiamento di coordinate) si associa una particella di
  spin $2$.
\end{itemize}
Possiamo quindi osservare che i fotoni, i quanti del campo elettromagnetico,
sono descritti dal campo vettoriale $A_{\mu}$ e hanno spin $1$ e massa nulla in
quanto il modulo quadro $k^{\mu}k_{\mu}$ del vettore d'onda è nullo.  I quanti
del campo gravitazionale, i \index{gravitone}\emph{gravitoni}, sono descritti
dal campo tensoriale $h_{\mu\nu}$ e sono particelle con spin $2$ e anch'essi con
massa nulla perché $k^{\mu}k_{\mu} = 0$.

\section{Rivelazione delle onde gravitazionali}
\label{sec:rivelazione-onde}

Consideriamo due particelle di prova $A$ e $B$ molto vicine, in caduta libera
all'interno di un generico campo gravitazionale.  Le posizioni nello
spazio-tempo siano le linee d'universo $x^{\mu}_{A} = x^{\mu}(\tau)$ e
$x^{\mu}_{B} = x^{\mu}(\tau) + \delta x^{\mu}(\tau)$, cioè il vettore $\delta
x^{\mu}(\tau)$ rappresenta la separazione fra $A$ e $B$.  Dall'equazione di
deviazione geodetica sappiamo che
\begin{equation}
  \label{eq:onde-grav-dev-geo}
  \curder[2]{\delta x^{\lambda}} = \tensor{R}{^{\lambda}_{\nu\mu\rho}} \delta
  x^{\mu} \toder{x^{\nu}}{\tau} \toder{x^{\rho}}{\tau}.
\end{equation}
Poniamoci\footnote{Questa trattazione è presa da Valeria Ferrari, Leonardo
  Gualtieri, ``General Relativity'', pagine 174-176.  \textsc{url}:
  \url{http://www.roma1.infn.it/teongrav/VALERIA/TEACHING/RELATIVITA_GENERALE/AA2013_14/dispense.pdf}.}
in un sistema di riferimento localmente inerziale $x^{\alpha'}$ centrato sulla
geodetica di una delle due particelle, per esempio $A$, in modo che nelle
vicinanze di $A$ la metrica differisca da quella di Minkowski per termini di
ordine $\abs{\delta x}^{2}$: $\dd \tau^{2} = \eta_{\alpha'\beta'}\dd
x^{\alpha'}\dd x^{\beta'} + \mathcal{O}(\abs{\delta x}^{2})$.  In questo sistema
di riferimento le coordinate della particella $A$ sono $x_{A}^{i'} = 0$, le
coordinate di $B$ sono $x^{\mu'}_{B} = \delta x^{\mu'}$ e risulta
\begin{subequations}
  \begin{align}
    t_{A} &= \tau, \\
    \label{eq:foo-geodetica-A}
    \toder{x^{\mu'}}{\tau}\bigg|_{A} &= (1, \bm{0}), \\
    g_{\mu'\nu'}|_{A} &= \eta_{\mu'\nu'}, \\
    g_{\mu'\nu',\alpha'}|_{A} &= 0 \implies
    \tensor{\Gamma}{^{\alpha'}_{\mu'\nu'}}|_{A} = 0.
  \end{align}
\end{subequations}
Assumiamo che in questo sistema di riferimento sia valida la gauge
TT.\footnote{Se così non fosse, sappiamo che sarebbe sufficiente effettuare
  un'opportuna trasformazione infinitesima delle coordinate per fare in modo che
  $h_{\mu'\nu'}$ sia trasverso e a traccia nulla.}  Per semplicità di notazione,
da qui in poi omettiamo l'apice primo dalle coordinate.
Dall'equazione~\eqref{eq:foo-geodetica-A} abbiamo che solo le equazioni di
deviazione geodetica~\eqref{eq:onde-grav-dev-geo} con $\nu, \rho = 0$ sono
rilevanti e quindi la deviazione geodetica per le coordinate spaziali è
\begin{equation}
  \toder[2]{\delta x^{j}}{t} = \tensor{R}{^{j}_{0k0}} \delta x^{k}.
\end{equation}
Nella gauge TT le componenti $h_{\mu 0} = h_{0\mu}$ sono nulle.  Allora
nell'approssimazione di campo debole si hanno le seguenti relazioni tra
componenti del tensore di Riemann: $R_{j0k0} = R_{0j0k} = -R_{j00k} =
-R_{0jk0}$.  Dall'equazione~\eqref{eq:riemann-approx}, segue
\begin{equation}
  R_{j0k0}^{(1)} = R_{0j0k}^{(1)} = \frac{1}{2}(\partial_{k}\partial_{j}h_{00}
  - \partial_{k}\partial_{0}h_{j0} - \partial_{0}\partial_{j}h_{0k}
  + \partial_{0}\partial_{0}h_{jk}) = \frac{1}{2}h^{\textup{TT}}_{jk,00}
\end{equation}
e l'equazione di deviazione geodetica diventa
\begin{equation}
  \label{eq:onde-grav-dev-geo2}
  \toder[2]{\delta x^{j}}{t} =
  \frac{1}{2} \parder{\tensor{h}{^{j}_{k}}}{*{2}{t}} \delta x^{k}.
\end{equation}
Supponiamo che le due particelle sono a riposo a una distanza costante
$\delta x^{j}(0)$ prima che l'onda arrivi (cioè con $h_{jk} = 0$).  Poiché
stiamo considerando campi deboli, $h_{\mu\nu}$ è una piccola perturbazione e la
posizione relativa fra le particelle cambierà di una quantità infinitesima al
passaggio dell'onda, cioè
\begin{equation}
  \delta x^{j}(t) = \delta x_{0}^{j} + \delta x_{1}^{j}(t),
\end{equation}
in cui $\delta x_{1}^{j}(t)$ è una perturbazione infinitesima rispetto alla
posizione iniziale $\delta x_{0}^{j}$.  Sostituendo questa espressione
nell'equazione~\eqref{eq:onde-grav-dev-geo2} e ricordando che $\delta x_{0}^{j}$
è costante, troviamo che al primo ordine in $h$ l'equazione di deviazione
geodetica si semplifica in
\begin{equation}
  \toder[2]{\delta x_{1}^{j}}{t} =
  \frac{1}{2}\parder{\tensor{h}{^{j}_{k}}}{*{2}{t}} \delta x_{0}^{k}.
\end{equation}
La soluzione di questa equazione è
% Per la risoluzione di questa equazione vedi anche MTW, pagina 951
\begin{equation}
  \delta x_{1}^{j}(t) = \delta x^{j}(t) - \delta x_{0}^{j} \approx \frac{1}{2}
  \tensor{h}{^{j}_{k}}(x^{\mu}(t)) \delta x_{0}^{k}
\end{equation}
che può anche essere scritta come
\begin{equation}
  \label{eq:spostamento-onda-grav}
  \delta x^{j}(t) \approx \delta x^{k}(0) \bigg(\tensor{\delta}{^{j}_{k}} +
  \frac{1}{2}\tensor{h}{^{j}_{k}}(x^{\mu}(t))\bigg).
\end{equation}
Questo risultato ci permette di analizzare l'effetto di un'onda gravitazionale
che investe un sistema di due corpi inizialmente in quiete.

\begin{figure}
  \centering
  \subfloat[][Onda longitudinale al sistema\label{fig:onda-grav1}]{
    \begin{tikzpicture}[font=\small,scale=0.9]
      \fill[black] (3,2.5) node[label={right:$A$}] (A) {} circle (0.08) ++(0,2)
      node[label={right:$B$}] (B) {} circle (0.08);
      \draw[->,decorate,decoration=snake] (3,0.5) -- ++(0,6);% onda gravitaz
      \draw[<->] ($(A) - (0.5,0)$) -- node[above,sloped] {$\delta x^{j}$}
      ($(B) - (0.5,0)$);
      \draw[->] (0,0) -- ++(0,7) node[left] {$z$}; % asse z
      \draw[->] (0,0) -- ++(6,0) node[below] {$y$}; % asse y
    \end{tikzpicture}}
  \qquad
  \subfloat[][Onda trasversale al sistema\label{fig:onda-grav2}]{
    \begin{tikzpicture}[font=\small,scale=0.9]
      \fill[black] (1,3.5) node[label={above:$A$}] (A) {}
      circle (0.08) ++(4,0) node[label={above:$B$}] (B) {} circle (0.08);
      \draw[->,decorate,decoration=snake] (3,0.5) -- ++(0,6);% onda gravitaz
      \draw[<->] ($(A) - (0,0.5)$) -- node[below left] {$\delta x^{j}$}
      ($(B) - (0,0.5)$);
      \draw[->] (0,0) -- ++(0,7) node[left] {$z$}; % asse z
      \draw[->] (0,0) -- ++(6,0) node[below] {$y$}; % asse y
    \end{tikzpicture}}
  \caption[Onda gravitazionale incidente su un sistema di due particelle]{Onda
    gravitazionale (rappresentata da \tikz \draw[->,decorate,decoration=snake]
    (0,0) -- ++(1.5,0);) che investe un sistema costituito da due corpi}
  \label{fig:onda-grav}
\end{figure}
Consideriamo un'onda gravitazionale che si propaga lungo la direzione $z$, per
la quale allora le uniche componenti non nulle di $h_{\mu\nu}$ nella gauge TT
sono
\begin{subequations}
  \begin{align}
    h_{11}^{\textup{TT}} &= - h_{22}^{\textup{TT}} \\
    h_{12}^{\textup{TT}} &= h_{21}^{\textup{TT}}
  \end{align}
\end{subequations}
Supponiamo che i due corpi siano allineati nella direzione $z$ a una posizione
iniziale reciproca costante $\delta x^{j}(0) = (0,0,L_{0})$ come nella
figura~\ref{fig:onda-grav1}, allora dalla~\eqref{eq:spostamento-onda-grav}
abbiamo che dopo il passaggio dell'onda
\begin{equation}
  \delta x^{j}(t) = \delta x^{k}(0)\bigg(\tensor{\delta}{^{j}_{k}} + \frac{1}{2}
  \tensor{h}{^{j}_{k}}\bigg) = \delta x^{3}(0)\bigg(\tensor{\delta}{^{j}_{3}} +
  \frac{1}{2}\tensor{h}{^{j}_{3}}\bigg) = \delta x^{3}(0)
  \tensor{\delta}{^{j}_{3}},
\end{equation}
cioè la separazione rimane costante $\delta x^{j}(t) = \delta x^{j}(0)$ lungo
ogni direzione $j$.  Supponiamo ora che i due corpi siano allineati lungo una
direzione perpendicolare a $z$, per esempio $y$, con separazione iniziale
$\delta x^{j}(0) = (0,L_{0},0)$, come nella figura~\ref{fig:onda-grav2}.  Dopo
il passaggio dell'onda, dalla~\eqref{eq:spostamento-onda-grav} risulta
\begin{equation}
  \delta x^{j}(t) = \delta x^{k}(0) \bigg(\tensor{\delta}{^{j}_{k}} +
  \frac{1}{2}\tensor{h}{^{j}_{k}}(x^{\mu}(t))\bigg) = \delta
  x^{2}(0)\bigg(\tensor{\delta}{^{j}_{2}} +
  \frac{1}{2}\tensor{h}{^{j}_{2}}(x^{\mu}(t))\bigg),
\end{equation}
quindi solo le separazioni fra i due corpi lungo $x$ e lungo $y$ cambiano: le
onde gravitazionali sono trasversali anche negli effetti fisici, oltre che nella
loro descrizione matematica.

Questi fatti ci suggeriscono le possibili tecniche per la rivelazione delle onde
gravitazionali.  Osserviamo comunque che il rapporto fra la variazione $\var L$
della distanza fra i corpi, indotta dal passaggio dell'onda, e la distanza
iniziale $L_{0}$ è
\begin{equation}
  \frac{\var L}{L_{0}} \sim h
\end{equation}
che è generalmente una grandezza molto piccola ($\sim 10^{-20}$).  Quindi due
particelle di prova, inizialmente a distanza di un metro, si spostano di circa
$10^{-20}$ m, distanza minore delle dimensioni di un nucleo $\simeq 1$ fermi.
Allora, la precisione richiesta negli esperimenti che si propongono di mettere
in evidenza le onde gravitazionali è molto elevata.

Sono stati sviluppate tre tecniche per la rivelazione delle onde gravitazionali:
\begin{enumerate}
\item \emph{Antenne gravitazionali risonanti (alla Weber)}.  Sono costituite da
  un cilindro (in alluminio, di circa 3 m, sospeso opportunamente, per tagliare
  le oscillazioni indotte dall'ambiente, e raffreddato all temperatura dell'elio
  liquido, per ridure il rumore browniano dell'antenna.  Il cilindro investito
  dall'onda gravitazionale, incomincia a oscillare.  Attraverso un trasduttore,
  le variazioni di lunghezza sono trasformate in micro-correnti che
  opportunamente amplificate, sono alla fine rivelate.  Tale tecnica sviluppata
  a partire dagli anni '70, nonostante oggi abbia raggiunto la fase di maturità,
  non ha comunque dato luogo ad osservazioni completamente positive.

  Un cilindro di alluminio ha una frequenza fondamentale di oscillazione
  $\nu_0 \simeq 1500$ Hz che è la frequenza attesa per onde gravitazionali
  emesse durante la fase di collasso gravitazionale di una stella massiva (se
  $M > 3-8 M_{\odot}$ in una stella di neutroni; se $M > 8 M_{\odot}$ in buco
  nero).  Poiché il collasso avviene in un tempo
  $\Delta t \equiv 1/\nu_0 \simeq 10^{-3}$ secondi, il cilindro si comparta come
  un oscillatore forzato alla frequenza propria (con guadagno $Q$ massimo).

  Si noti che il collasso gravitazionale è accompagnato da emissione di neutrini
  prodotti attraverso reazioni adroniche della materia condensata (densità
  dell'ordine di \SI{e15}{\gram \per \centi \metre \cubed}) e successivi
  decadimenti di leptoni in neutrini.  In questo caso si dovrebbe rivelare a
  terra un flusso di neutrini in coincidenza con l'arrivo dell'onda
  gravitazionale se i neutrini hanno massa nulla, o in ritardo, se i neutrini
  hanno massa non nulla.
\item \emph{Rivelatori del tipo interferometro alla Michelson-Morley}.  Un
  segnale laser è inviato verso lo specchio semi-riflettente dove è splittato in
  due cammini ottici disposti a 90 gradi, quindi i due raggi sono riflessi
  indietro da due specchi, posti a distanza di qualche km negli esperimenti
  Virgo (PISA) e Ligo (US), e alla fine fatti interferire.  La variazione di
  distanza nei due cammini ottici indotta dall'onda gravitazionale determina lo
  spostamento delle frange di interferenza.

  La tecnica è nella fase matura, i dati sono ora analizzati, ma non ci sono
  risultati positivi.  La possibilità di riflessioni multiple dei fasci laser
  lungo i cammini ottici --- comunque limitata dall'assorbimento, stabilità di
  sospensione e allineamento degli specchi --- permette di aumentare il cammino
  ottico $L$ e quindi la variazione di distanza indotta dall'onda
  gravitazionale, $\Delta l \simeq L h$.

  La frequenza caratteristica delle onde gravitazionali in questo caso sarebbe
  dell'ordine di ?? che corrisponde (attraverso la relazione $\nu \to 1/T$) al
  periodo orbitale $T$ di sistemi binari di stelle di neutroni e/o buchi neri
  nella fase di coalescenza.  Si ritiene che i centri delle galassie contengano
  buchi neri super-massivi e allora è naturale guardare in direzione dei centri
  delle galassie e AGN.
\item \emph{Satelliti nel sistema solare}.  Sono esperimenti previsti per gli
  anni 2020.  Nel caso di LISA si pensa di posizionare 3 satelliti in punti
  lagrangiani del sistema solare in cui $\bm{g} \simeq \bm{0}$, a distanza
  reciproca $\simeq 1$ AU (150 milioni di km).  La posizione dei satelliti
  sarebbe determinata attraverso segnali radar e la variazione di distanza
  indotta da un'onda gravitazionale sarebbe così rivelata.  La frequenza
  caratteristica è nel range di frequenza ??--
\end{enumerate}

\section{Generazione di Onde gravitazionali}
\label{sec:onde-grav-masse}

Studiamo adesso il campo gravitazionale debole generato da masse in movimento.
Possiamo ottenere delle stime degli ordini di grandezza della luminosità emessa
sotto forma di onde gravitazionali ricordando che nello studio delle onde
elettromagnetiche in presenza di cariche in moto
(paragrafo~\ref{sec:radiazione-dipolo}) abbiamo trovato che i termini principali
nello sviluppo in multipoli dell'energia emessa dal sistema per unità di tempo
sono dati dalla formula~\eqref{eq:potenza-onde-em}
\begin{equation}
  I_{\textup{em}} = \underbrace{\frac{2}{3c^{3}}
    (\ddot{\bm{d}}_{\textup{em}})^{2}}_{\text{dipolo di carica}} +
  \underbrace{\frac{2}{3c^{3}} (\ddot{\bm{m}}_{\textup{em}})^{2}}_{\text{dipolo
      magnetico}} +
  \underbrace{\frac{1}{180c^{5}}
    \dddot{Q}_{ij,\textup{em}}\dddot{Q}^{ij,\textup{em}}}_{\text{quadrupolo di
      carica}} + \cdots
\end{equation}
con $\bm{d}_{\textup{em}} = \sum_{k} q_{k} \bm{r}_{k}$ momento di dipolo di
carica, $\bm{m}_{\textup{em}} = \sum_{k} q_{k} \bm{r}_{k} \times
\bm{v}_{k}/(2c)$ momento di dipolo magnetico e $Q_{ij,\textup{em}} = \sum_{k}
q_{k}(3x_{i}x_{j} -\delta_{ij}r_{k}^{2})$ momento di quadrupolo di carica.  Per
passare al caso gravitazionale possiamo effettuare la sostituzione
$\text{carica} \to \text{massa}$, vale a dire $q_{k} \to m_{k}$.  In questo modo
\begin{subequations}
  \begin{align}
    \bm{d}_{\textup{g}} &= \sum_{k} m_{k} \bm{r}_{k} = M \bm{R}_{\textup{cm}},\\
    \bm{m}_{\textup{g}} &= \sum_{k} \bm{r}_{k} \times (m_{k} \bm{v}_{k}) =
                          \bm{L}, \\
    Q_{ij,\textup{g}} &= \sum_{k} m_{k}(3x_{i}x_{j} -\delta_{ij}r_{k}^{2}),
  \end{align}
\end{subequations}
in cui $M = \sum_{k} m_{k}$ è la massa totale delle particelle del sistema,
$\bm{R}_{\textup{cm}} = \sum_{k} m_{k} \bm{r}_{k}/M$ è la posizione del centro
di massa e $\bm{L}$ è il momento angolare totale.  Per un sistema non
relativistico isolato, cioè per nel quale non agiscono forze esterne né è
presente un momento torcente esterno diverso da zero, risulta
\begin{subequations}
  \begin{align}
    \ddot{\bm{d}}_{\textup{g}} &= M \ddot{\bm{R}}_{\textup{cm}} =
    \dot{\bm{P}} = \bm{0}, \\
    \dot{\bm{m}}_{\textup{g}} &= \dot{\bm{L}} = \bm{\tau} = \bm{0},
  \end{align}
\end{subequations}
con $\bm{P}$ quantità di moto totale del sistema e $\bm{\tau}$ momento torcente
totale.  Dunque per un sistema isolato di masse in moto che emettono onde
gravitazionali i due termini di dipolo sono nulli e il termine dominante sarà
quello di quadrupolo
\begin{equation}
  I_{\textup{og}} = \frac{1}{45c^{5}}\dddot{Q}_{ij}\dddot{Q}^{ij}.
\end{equation}
Il fattore numerico $45$, invece di $180$ del caso elettromagnetico, è legato al
fatto che il campo gravitazionale ha spin $2$, mentre il campo elettromagnetico
ha spin $1$.\footnote{Vedi \textcite[974]{misner:gravitation};
  \textcite[471]{shapiro:black-holes}.}

Esaminiamo più in dettaglio il problema.  Sappiamo che nell'approssimazione di
campo debole l'equazione di Einstein in presenza della
materia~\eqref{eq:einstein-lineare-esatta} diventa
\begin{equation}
  \label{eq:einstein-lineare-materia1}
  \dalamb \bigg(h_{\mu\nu} - \frac{1}{2} \eta_{\mu\nu}h \bigg) = -\frac{16\pi
    G}{c^{4}} \tau_{\mu\nu}.
\end{equation}
Introduciamo il tensore $\psi_{\mu\nu}$ definito da
\begin{equation}
  \psi_{\mu\nu} = h_{\mu\nu} - \frac{1}{2} \eta_{\mu\nu} h.
\end{equation}
Osserviamo che nella gauge TT risulta $h = 0$ e quindi
$\psi_{\mu\nu} = h_{\mu\nu}$.  La gauge armonica~\eqref{eq:gauge-armonica} può
essere definita con riferimento a $\psi_{\mu\nu}$ come
\begin{equation}
  \label{eq:gauge-armonica2}
  \partial_{\mu}\psi^{\mu\nu} = 0
\end{equation}
e l'equazione di Einstein linearizzata~\eqref{eq:einstein-lineare-materia1} in
presenza di materia si scrive come
\begin{equation}
  \label{eq:einstein-lineare-materia2}
  \dalamb \psi_{\mu\nu} = -\frac{16\pi G}{c^{4}} \tau_{\mu\nu}.
\end{equation}
Da qui deriva inoltre che se è verificata la condizione
armonica~\eqref{eq:gauge-armonica2} risulta anche
\begin{equation}
  \label{eq:gauge-armonica-tau}
  \partial_{\mu}\tau^{\mu\nu} = 0,
\end{equation}
che nell'approssimazione lineare sostituisce l'equazione più generale
$\tensor{T}{^{\mu\nu}_{;\mu}} = 0$.

Vogliamo determinare il campo gravitazionale a grandi distanze rispetto al
sistema in esame, in particolare nella \index{zona!delle onde}zona delle onde,
cioè in una regione a distanza $R$ molto maggiore rispetto alla lunghezza
$\lambda$ delle onde irradiate dal sistema.  Dunque possiamo seguire gli stessi
metodi utilizzati nel paragrafo~\ref{sec:onde-elettromagnetiche-cariche}.

Infatti l'equazione di Einstein~\eqref{eq:einstein-lineare-materia2} in presenza
di masse in moto è formalmente uguale all'equazione di
Maxwell~\eqref{eq:maxwell-cariche} in presenza di cariche in moto.  Quindi
possiamo scrivere direttamente la soluzione
\begin{equation}
  \psi_{\mu\nu}(t, \bm{x}) = \frac{4G}{c^{4}}\int \frac{(\tau_{\mu\nu})|_{t - R/c}}{R} \dd
  V',
\end{equation}
in cui si è usata la stessa notazione del
paragrafo~\ref{sec:onde-elettromagnetiche-cariche}. Se assumiamo inoltre che le
velocità con cui si muovono i corpi sono molto più piccole di quelle della luce
nel vuoto, possiamo adottare l'approssimazione di dipolo e scrivere
\begin{equation}
  \label{eq:psi-mu-nu}
  \psi_{\mu\nu}(t, \bm{x}) = \frac{4G}{c^{4}r} \int(\tau_{\mu\nu})|_{t-r/c} \dd V'.
\end{equation}
In seguito ometteremo, per brevità, di specificare che le funzione integrande
devono essere valutate all'istante $t-r/c$.
% Nota: nel paragrafo 110 il Landau fa dei calcoli molto strani: abbassa un
% indice muto (cioè cambia la posizione verticale di un solo indice della coppia
% di indici muti), definisce il tensore D_{\alpha\beta} completamente covariante
% uguale a un oggetto che è somma di un tensore completamente controvariante e
% un altro completamente covariante.  Magari a me sfugge qualcosa ma per il
% momento questi calcoli non mi convincono.  Ho seguito invece il Gasperini,
% "Relatività generale e teoria della gravitazione", il quale fa calcoli
% analoghi ma più sensati.
Dalla~\eqref{eq:gauge-armonica-tau}, osservando che $\tau_{\mu\nu}$ è simmetrico
rispetto allo scambio degli indici,
\begin{subequations}
  \begin{align}
    \label{eq:baz1}
    \partial^{k}\tau_{ik} + \partial^{0}\tau_{i0} = 0, \\
    \label{eq:baz2}
    \partial^{k}\tau_{0k} + \partial^{0}\tau_{00} = 0.
  \end{align}
\end{subequations}
Moltiplichiamo la prima di queste equazioni per $x_{j}$ e integriamo su tutto lo spazio
\begin{equation}
  \begin{split}
    \partial^{0} \int \tau_{i0} x_{j} \dd V &= -\int \partial^{k} \tau_{ik} x_{j}
    \dd V = - \int \partial^{k}(\tau_{ik} x_{j}) \dd V + \int \tau_{ik}
    \tensor{\delta}{^{k}_{j}} \dd V = \\
    &= - \int \partial^{k}(\tau_{ik} x_{j}) \dd V + \int \tau_{ij} \dd V.
  \end{split}
\end{equation}
Il primo integrale dell'ultimo membro può essere trasformato, usando il teorema
di Gauss, in un integrale sulla superficie all'infinito di $\tau_{ik}x_{j}$, ma
poiché $\tau_{ik}$ è nullo all'infinito l'integrale è uguale a zero.  Dunque
\begin{equation}
  \int \tau_{ij} \dd V = \partial^{0} \int \tau_{i0} x_{j} \dd V =
  \frac{1}{2} \partial^{0} \int (\tau_{i0} x_{j} + \tau_{j0}x_{i}) \dd V.
\end{equation}
Abbiamo simmetrizzato l'integrale a secondo membro perché $\tau_{ij}$ è
simmetrico.  Svolgendo analoghe considerazioni, moltiplichiamo
la~\eqref{eq:baz2} per $x_{i}x_{j}$ e integriamo su tutto lo spazio
\begin{equation}
  \begin{split}
    \partial^{0} \int \tau_{00} x_{i} x_{j} \dd V &= -\int \partial^{k}
    \tau_{0k} x_{i} x_{j} \dd V \\
    &= -\int \partial^{k}(\tau_{0k} x_{i} x_{j}) \dd V + \int \tau_{0k}
    \tensor{\delta}{^{k}_{i}} x_{j} \dd V + \int \tau_{0k} x_{i}
    \tensor{\delta}{^{k}_{j}} \dd V \\
    &= \int (\tau_{0j}x_{i} + \tau_{0i}x_{j}) \dd V.
  \end{split}
\end{equation}
Mettendo insieme i due risultati ottenuti abbiamo
\begin{equation}
  \begin{split}
    \int \tau_{ij} \dd V &= \frac{1}{2} \partial^{0} \int (\tau_{0j}x_{i} +
    \tau_{0i}x_{j}) \dd V = \frac{1}{2}(\partial^{0})^{2} \int \tau_{00} x_{i}
    x_{j} \dd V \\
    &= \frac{1}{2c^{2}} \parder{}{*{2}{t}} \int \tau_{00} x_{i} x_{j} \dd V.
  \end{split}
\end{equation}

Questa equazione ci dice che tutti gli integrali di volume delle componenti
$\tau_{ij}$ possono essere espressi come integrali della componente
$\tau_{00}=T_{00}+t_{00}$ che approssimiamo con $T_{00}$.  Quindi $\tau_{00} =
\rho c^{2}$, con $\rho$ densità di massa a riposo.  Sostituiamo
nella~\eqref{eq:psi-mu-nu}
\begin{equation}
  \psi_{ij} = \frac{2G}{c^{4}r} \parder{}{*{2}{t}}\int \rho x_{i} x_{j} \dd V.
\end{equation}
Come abbiamo visto nel caso elettromagnetico, nella zona delle onde, cioè per
$R \gg \lambda \gg a$ con $a$ dimensione del sistema, le onde possono essere
trattate come piane in piccole regioni dello spazio.  Adottando inoltre la gauge
TT, nella quale $\psi_{ij} = h_{ij}$, abbiamo
\begin{equation}
  h_{ij} = \frac{2G}{c^{4}r} \parder{}{*{2}{t}}\int \rho x_{i} x_{j} \dd V =
  \frac{2G}{3c^{4}r} \bigg(\ddot{Q}_{ij} + \delta_{ij}\parder{}{*{2}{t}} \int
  \rho r^{2} \dd V\bigg),
  \label{67}
\end{equation}
in cui
\begin{equation}
  Q_{ij} = \int \rho(3x_{i} x_{j} - r^{2}\delta_{ij})
  \dd V
\end{equation}
è il \index{momento!di quadrupolo di massa}
\emph{momento di quadrupolo delle masse} e il punto indica la derivazione
rispetto al tempo.

Si può dimostrare\footnote{Vedi~\textcite[449]{landau:campi}.} che
\begin{equation}
  t_{\mu\nu} = \frac{c^{4}}{32\pi G} \partial_{\mu}
  \tensor{h}{^{\lambda}_{\rho}} \partial_{\nu} \tensor{h}{^{\rho}_{\lambda}}
\end{equation}
Inoltre per un'onda piana che si propaga in una direzione $x^{i}$, il flusso di
energia è diretto lungo lo stesso asse $x^{i}$ ed è determinato da $ct^{0i}$.

In particolare, supponendo che l'onda gravitazionale si propaghi lungo $x^{1}$,
le uniche componenti non nulle di $h_{\mu\nu}$ nella gauge TT sono
$h_{22} = -h_{33}$ e $h_{23} = h_{32}$ e il flusso di energia è dato da
\begin{equation}
  ct^{01} = \frac{c^{3}}{16\pi G}\bigg((\dot{h}_{23})^{2} +
  \frac{1}{4}(\dot{h}_{22} - \dot{h}_{33})^{2}\bigg).
  \label{70}
\end{equation}
Dall'equazione~\eqref{67} si ha
\begin{subequations}
  \begin{align}
    h_{23} &= \frac{2G}{3c^{4}r} \ddot{Q}_{23}, \\
    h_{22} - h_{33} &= \frac{2G}{3c^{4}r}(\ddot{Q}_{22} - \ddot{Q}_{33}),
   \end{align}
\end{subequations}
da cui il flusso di energia nella direzione Ox diventa
\begin{equation}
  ct^{01} = \frac{G}{36\pi c^{5}r}\bigg((\dddot{Q}_{23})^{2} +
  \frac{1}{4}(\dddot{Q}_{22} - \dddot{Q}_{33})^{2}\bigg).
\end{equation}

Per ottenere l'energia totale $I_{og}$ emessa dal sistema nell'unità di tempo,
bisogna moltiplicare la media di $ct^{01}$ su tutti gli stati di polarizzazione
per $r^{2}\dd\Omega$ e integrare su tutto l'angolo solido.  Sviluppando i
calcoli si ottiene il seguente
risultato\footnote{Vedi~\textcite[460-461]{landau:campi}.}
\begin{equation}
  I = \frac{G}{45c^{5}} \langle \dddot{Q}_{ij} \dddot{Q}^{ij}\rangle.
\end{equation}

Quindi si può avere emissioni di radiazione gravitazionale solo da parte di
distribuzioni di massa in movimento con momento di quadrupolo di massa
dipendente dal tempo.  Le masse che costituiscono il sistema sono singolarmente
in moto accelerato, anche se il centro di massa del sistema ha accelerazione
nulla.  Un esempio di questo tipo è un sistema binario: ciascuna stella è in
moto di rivoluzione attorno al comune centro di massa, ma il centro di massa,
non essendo soggetto a forze o momenti esterni, si muove con velocità costante.
Osserviamo che l'energia emessa dal sistema è proporzionale $G/c^{5}$, che è un
fattore estremamente piccolo. Quindi l'efficienza di emissione di radiazione da
parte di masse in movimento è, generalmente, molto bassa.

\section{Sorgenti di onde gravitazionali}
\label{sec:sorgenti-onde-grav}


\subsection{Stima della potenza emessa}
\label{sec:potenza-emessa}

Innanzitutto determiniamo l'ordine di grandezza della luminosità (energia
irradiata nell'unità di tempo) da un sistema isolato che emette onde
gravitazionali.

Consideriamo un sistema, costituito da $N$ corpi di uguale massa, con massa
totale $M$ e dimensioni $\simeq a$.  Possiamo stimare il tempo $T \simeq a/v$
necessario a che un generico corpo attraversi con velocit\'a $v$ l'intero
sistema.

Approssimando le componenti del tensore quadrupolo di massa con
\begin{equation}
  Q_{ij} \approx M a^{2}
\end{equation}
si può stimare la sua derivata temporale terza con
\begin{equation}
  \dddot{Q}_{ij} \approx \frac{M a^{2}}{T^{3}} \simeq \frac{Mv^{3}}{a}.
\end{equation}

Introduciamo la luminosità di scala $L_{0} = c^{5}/G = 3.6 \times 10^{59}$ erg
s$^{-1}$, allora
\begin{equation}
  I = \frac{G}{45c^{5}} \dddot{Q}_{ij} \dddot{Q}^{ij} \approx
  L_{0} \bigg(\frac{r_{\textup{S}}}{a}\bigg)^{2} \bigg(\frac{v}{c}\bigg)^{6}.
\end{equation}

Possiamo eliminare $v$ usando il teorema del viriale $2K + U = 0$, cioè $v^{2}
\simeq GM/a$, da cui segue
\begin{equation}
  \frac{v^{2}}{c^{2}} \approx \frac{GM}{ac^{2}} \sim   \frac{r_{\textup{S}}}{a}.
\end{equation}
In definitiva
\begin{equation}
  I \sim L_{0} \bigg(\frac{r_{\textup{S}}}{a}\bigg)^{5},
\end{equation}
quindi per poter avere un'emissione efficiente di onde gravitazionali bisogna
avere oggetti collassati, cioè corpi per i quali $a \simeq r_{\textup{S}}$.

\subsection{Sorgenti di laboratorio}
\label{sec:sorgenti-laboratorio}

Osserviamo innanzitutto che in sistemi a terra è impossibile produrre onde
gravitazionali con una luminosità che non sia completamente trascurabile.

Onde gravitazionali potrebbero essere prodotte in laboratorio usando una trave
in rotazione intorno al suo punto medio.  Massimizzando l'esperimento, potremo
usare una trave di acciaio con raggio $r = \SI{1}{\metre}$, lunga $l =
\SI{20}{\metre}$, di densità $\rho = \SI{7.8}{\gram\per\centi\metre\cubed}$ e
massa $M = \SI{490}{\tonne}$, carico di rottura $t =
\SI{3e9}{\dyne\per\centi\metre\squared}$.  La velocità di rotazione massima
$\omega_{max}$ si ottiene bilanciando la forza centrifuga con il carico di
rottura
\begin{equation}
  \omega_{max} = \sqrt{\frac{8t}{\rho l^{2}}} = \SI{28}{\radian\per\second}.
\end{equation}
Per questo sistema, la potenza emessa in onde gravitazionali è data
da\footnote{Vedi~\textcite[980]{misner:gravitation}.}
\begin{equation}
  I = \frac{2}{45} M^{2} l^{2} \omega^{6} = \SI{2.2e-22}{\erg\per\second},
\end{equation}
che è una luminosità vicina a zero.

\subsection{Sistema binario}
\label{sec:sistema-binario}

Per un sistema binario di corpi in orbita circolare di raggio $r$, con massa
ridotta $\mu = m_{1}m_{2}/(m_{1} + m_{2})$, si trova\footnote{Vedi
  \textcite[461]{landau:campi}, \textcite[986]{misner:gravitation},
  \textcite[476]{shapiro:black-holes}.}  che l'energia $I$ emessa nell'unità di
tempo sotto forma di onde gravitazionali è data da
\begin{equation}
  \label{eq:potenza-onde-sistema-binario}
  -\toder{E}{t} = I = \frac{32}{5} \frac{G}{c^{5}} \mu^{2} \omega^{6} r^{4} =
  \frac{32}{5} \frac{G^{4} m_{1}^{2} m_{2}^{2} (m_{1} + m_{2})}{c^{5}r^{5}}
\end{equation}
con
\begin{equation}
  \omega^{2} = \frac{G(m_{1} + m_{2})}{r^{3}}.
\end{equation}

Poiché il sistena binario perde energia a causa dell'emissione di onde
gravitazionali, la sua energia totale $E$ (che è negativa) diminuisce.  Questo
fatto determina un avvicinamento dei due corpi, così che il sistema binario
diventa via via più stretto fino a coalescere.  L'emissione di onde
gravitazionali è massima durante la fase finale che precede la collisione tra i
costituenti.

Nel caso di orbite circolari $E = -Gm_{1}m_{2}/(2r)$, la variazione della
distanza reciproca $r$ fra i due corpi varia, nell'unità di tempo, come
\begin{equation}
  \toder{r}{t} = -\frac{Gm_{1}m_{2}}{2} \toder{(1/E)}{t} =
  \frac{2r^{2}}{Gm_{1}m_{2}} \toder{E}{t} = -\frac{64}{5}\frac{G^{3}}{c^{5}}
  \frac{m_{1}m_{2}(m_{1} + m_{2})}{r^{3}}.
\end{equation}

Consideriamo il sistema composto dal Sole e dal pianeta Giove.  Poiché la massa
di Giove $M = \SI{2e30}{\gram}$ è molto più piccola della massa del Sole
$M_{\odot} = \SI{2e33}{\gram}$, la massa ridotta $\mu$ del sistema può essere
approssimata con quella di Giove.  Inserendo i dati orbitali del pianeta
$\omega = \SI[per-mode=reciprocal]{2e-8}{\per\second}$,
$r = \SI{8e13}{\centi\metre}$ nella~\eqref{eq:potenza-onde-sistema-binario}
abbiamo che la potenza emessa da Giove sotto forma di onde gravitazionali è
\begin{equation}
  I_{\textup{og}} = \frac{32}{5} \frac{G}{c^{5}} M^{2} \omega^{6} r^{4} =
  \SI{5.3e10}{\erg\per\second}.
\end{equation}
Per confronto, la radiazione gravitazionale
emessa\footnote{Vedi~\textcite[266]{weinberg:gravitation}.}  dalle collisioni
termiche del plasma solare è dell'ordine di \SI{e15}{\erg\per\second}.  Allora
sono necessari tempi scala molto più grandi dell'età del sistema solare per
poter osservare degli effetti sull'orbita di Giove dovuti alla perdita di
energia per radiazione gravitazionale.

Comunque, a circa \SI{7.3}{\kilo\parsec} di distanza da noi esiste un sistema
binario molto stretto costituito da due stelle di neutroni, di cui una è la
pulsar PSR B1913+16 (scoperta da Hulse e Taylor nel 1974), in orbita l'una
attorno all'altra con un periodo di $\simeq 8$ ore.  Le masse delle due stelle
sono $\simeq \SI{1.4}{\solarmass}$, la distanza fra le stelle è $\simeq
\SI{1.9e11}{\centi\metre} = \SI{1.9e6}{\kilo\metre}$ e la pulsar si muove alla
velocità di \SI{300}{\kilo\metre\per\second}.  Questo sistema è importante
perché fornisce una forte evidenza sperimentale, seppur indiretta, della
emissione di onde gravitazionali e della validità della Teoria della Relatività
Generale.  Grazie alle regolari emissioni di impulsi elettromagnetici da parte
delle pulsar, è stato possibile misurare i parametri orbitali del sistema, in
particolare la precessione del perielio del sistema e la dilatazione
gravitazionale del tempo (red-shift gravitazionale).

\subsection{Collasso gravitazionale}
\label{sec:collasso-grav}

Un'altra possibile sorgente di onde gravitazionali è il collasso gravitazionale
di un corpo massivo.  Poiché un processo di collasso sfericamente simmetrico,
per il teorema di Birkhoff, darebbe luogo a un campo statico, è necessario
ipotizzare che il corpo non sia dotato di simmetria sferica.  Un esempio di
questo fenomeno è quello di una stella che collassa in una stella di neutroni
(NS), visibile nel radio poi come pulsar.  La massa massima di una NS è di circa
qualche massa solare.  Approssimiamo la stella originaria con un ellissoide di
ellitticità $e$
\begin{equation}
  e = \frac{\text{differenza fra i raggi equatoriali}}{\text{raggio equatoriale
      medio}}  = \frac{a-b}{(a+b)/2} = 10^{-4},
\end{equation}
con raggi equatoriali $a$ e $b$ dell'ellissoide dell'ordine di $R_{0} \sim
\SI{7e5}{\kilo\metre}$.  Dopo il collasso la NS avrà dimensioni $R \simeq 10$ km
e a causa della conservazione del momento angolare $L \simeq M \omega R^2$,
ruoterà con velocità angolare $\omega \sim
\SI[per-mode=reciprocal]{e4}{\per\second}$ (si parla in questi casi di pulsar a
milli-secondi).  La potenza emessa dalla NS in rotazione sotto forma di onde
gravitazionali\footnote{Vedi \textcite[488]{shapiro:black-holes};
  \textcite[272]{weinberg:gravitation}.} è
\begin{equation}
  I_{\textup{og}} = \frac{32}{5} \frac{G}{c^{5}} \mathcal{I}^{2} \omega^{6}
  e^{2} \sim \SI{e47}{\erg\per\second}
\end{equation}
con $\mathcal{I} = \text{momento di inerzia} \sim MR^{2}$.  L'energia
rotazionale alla formazione della pulsar è
\begin{equation}
  E = \frac{1}{2} \mathcal{I} \omega^{2} \sim \SI{e53}{\erg}.
\end{equation}
Pertanto, se la luminosità $I_{\textup{og}}$ rimanesse costante nel tempo (in
particolare se $e=\text{cost}$), l'energia rotazionale iniziale dovrebbe
esaurirsi in un tempo $\var t = E/I$ dell'ordine di 1 anno (dopo un anno la
stella dovrebbe smettere di ruotare).  In realtà noi oggi osserviamo pulsar
molto vecchie (la Crab è esplosa nel 1054).  Quindi siamo costretti a ipotizzare
che la NS dopo il collasso circolarizzi velocemente ($e \to 0$) con la
conseguente soppressione della potenza emessa in onde gravitazionali.

Comunque, la NS continua a emettere radiazione elettromagnetica se è dotata di
un momento di dipolo magnetico $\bm{m}$ non allineato con l'asse di rotazione
(modello di Pacini).

%%% Local Variables:
%%% mode: latex
%%% TeX-master: "../gravitazione"
%%% fill-column: 80
%%% End:

\chapter{\completare{Cosmologia}}
\label{cha:cosmologia}

\section{Equazioni di Friedmann}

\begin{itemize}
\item La geometria dello spazio tempo sia descritta dall metrica R-W:
  \begin{subequations}
    \begin{align}
      \label{rw1}
      g_{tt} &= -1, \\
      g_{it} &= 0, \\
      g_{ij} &= R^2(t) ~ \tilde{g}_{ij}(x^i)
    \end{align}
  \end{subequations}
  dove $t$ è il tempo cosmico, $x^{i} = (r,\theta,\phi)$ sono coordinate
  coomoving e $\tilde{g}_{ij}(x^i)$ è la metrica dello spazio 3-D omogeneo ed
  isotropo
  \begin{subequations}
    \begin{align}
      \label{rw2}
      \tilde{g}_{rr} &= (1-kr^2)^{-1}, \\
      \tilde{g}_{\theta \theta} & = r^2, \\
      \tilde{g}_{\phi \phi} &= r^2 \sin^2 \phi, \\
      \tilde{g}_{ij} &= 0 \qquad \text{per $i \ne j$}.
    \end{align}
  \end{subequations}
\item Universo è omogeneo ed isotropo: $\implies$ tensore energia impulso per
  l'Universo ha la forma (fluido perfetto)
  \begin{equation}
    T_{\mu\nu} = p g_{\mu\nu} + (p+\rho) U_{\mu} U_{\nu}
    \label{fp}
  \end{equation}
  con pressione $p(t)$ e densità di energia $\rho(t)$ funzioni del tempo e
  quadrivelocità dei costituenti (oggi le galassie) dell'Universo
  $U^{\mu}=(1,\bm{0})$.
\item La dinamica dell'Universo determinata dalle equazioni di Einstein
  \begin{subequations}
    \begin{align}
      \label{EE}
      R_{\mu\nu} &= -8 \pi G S_{\mu \nu}, \\
      S_{\mu \nu} &= T_{\mu \nu} - (1/2) g_{\mu \nu}
      \tensor{T}{^{\lambda}_{\lambda}}
    \end{align}
  \end{subequations}
\end{itemize}
Il tensore di Ricci diventa (il punto indica derivazione rispetto al tempo):
\begin{subequations}
  \begin{align}
    R_{tt} &= \frac{ 3 \ddot{R}}{R} \\
    R_{ti} &= 0 \\
    R_{ij} &= -(R\ddot{R}+2\dot{R}^2+2k) \tilde{g}_{ij}.
  \end{align}
\end{subequations}
Osserva:
\begin{subequations}
  \begin{align}
    \tensor{T}{^{\lambda}}{_{\lambda}} &= g^{\lambda\nu} T_{\lambda\nu} = p
    \tensor{\delta}{^{\lambda}}_{{\lambda}} + (p+\rho) (-1) = 3p - \rho \\
    S_{\mu \nu} &= (1/2) (\rho-p) g_{\mu\nu}+(p+\rho)U_{\mu} U_{\nu}
  \end{align}
\end{subequations}
In particolare
\begin{subequations}
  \begin{align}
    S_{tt} &= (1/2) (\rho-p)(-1)+(p+\rho)(+1) = (1/2)(\rho+3p) \\
    \label{sij}
    S_{ij} &= (1/2)(\rho-p) g_{ij}= (1/2)(\rho-p) R^2(t) \tilde{g_{ij}}
  \end{align}
\end{subequations}

Dalle equazioni di Einstein per la componente $tt$:
\begin{equation}
  \frac{3 \ddot{R}}{R} = - 4\pi G (\rho+3p)
  \label{e1}
\end{equation}
e per la componente $ij$:
\begin{equation}
  \label{e2}
  \frac{\ddot{R}}{R} +\frac{2\dot{R}^2}{R^2}+\frac{2k}{R^2}= 4 \pi G (\rho-p)
\end{equation}
In aggiunta, si ha l'equazione di conservazione
$\tensor{T}{^{\mu \nu}}{_{;\mu}} = 0$, e per $\nu =0$
\begin{equation}
  \dot{p} R^3 = \toder{}{t}[R^3(\rho+p)]
  \label{moto1}
\end{equation}
o, equivalentemente, (poiché $\ltoder{}{t} = \dot{R} \ltoder{}{R}$)
\begin{equation}
  \toder{(\rho R^3)}{R} = -3pR^2
  \label{moto2}
\end{equation}
{\bf Osserva} che dalle ultime 2 equazioni segue
\begin{subequations}
  \begin{gather}
    \text{se $p=0$ (era materia)} \implies \rho \propto R^{-3} \\
    \text{se $p=\rho/3$ (era radiazione)} \implies \rho \propto R^{-4} .
  \end{gather}
\end{subequations}
Il set di Le equazioni \eqref{e1}, \eqref{e2} e \eqref{moto1} non sono
indipendenti.  Eliminando $\ddot{R}$ si ottine un'equazione differenziale del
primo ordine
\begin{equation}
  \dot{R}^2 +k = \frac{8 \pi G}{3} \rho R^2
  \label{fri}
\end{equation}
Le equazioni \eqref{fri} e \eqref{moto1} insieme all'equazione di stato
\begin{equation}
  p=p(\rho)
  \label{stato}
\end{equation}
permettono di determinare le 3 funzioni incognite $R(t)$, $p(t)$ e $\rho(t)$.
Si otterranno 3 differenti soluzioni a seconda del valore di $k=-1,0,+1$.  I
modelli cosmologici basati sulla metrica R-W, sulle equazioni di Einstein sono
noti come modelli di Friedmann.

{\bf Osservazione}: senza conoscere esplicitamente l'equazione di stato è
possibile ricavare importanti informazioni sull'età $t_0$ dell'Universo e
sull'andamento del fattore di scala cosmico $R(t)$.  Infatti:
\begin{itemize}
\item $R(t)$ definito positivo
\item l'eq. \eqref{e1} mostra che $ \ddot{R} \propto - (\rho+3p) <0$ ad ogni
  $t$.  Quindi $R(t)$ ha concavità rivolta verso il basso per ogni tempo cosmico
  poiché per materia e radiazione $\rho(t)>0$ e $p(t)>0$
\item all'istante attuale l'Universo è in espansione $\dot{R}(t_0) >0$ ed
  inoltre $p_0=0$ (era materia).
\end{itemize}
Allora per $t>t_0$ ($t<t_0$), il termine a destra nell'eq. \ref{fri} va a zero
(cresce) almeno come $1/R(t)$.

Da queste considerazioni segue:
\begin{itemize}
\item età dell'Universo
  \begin{equation}
    t_0 < H_0^{-1}  = \frac {R(t_0)}{\dot{R}(t_0)} =  \num{13e9}
    \left(\frac{\SI[per-mode=reciprocal]{75}{\kilo\metre\per\second\per\mega\parsec}}{H_0}\right)
    \si{yr}.
  \end{equation}
\end{itemize}
L'uguaglianza si avrebbe nel caso $\ddot{R} =0$ (per cui $R(t)=t \dot{R}$).
Comportamento di $R(t)$ per $t>t_0$:
\begin{itemize}
\item se $k=-1$, per $t \to \infty$, $\dot{R} \to 1$, e quindi $R(t) \propto t$;
\item se $k=0$, per $t \to \infty$, $\dot{R} \propto R^{-1/2}$, e quindi
  $R(t) \propto t^{2/3}$;
\item se $k=-1$, $\dot{R}$ decresce dal valore attuale (positivo), ed esiste un
  istante $\tilde{t}$ a cui si annulla; quindi pet $t>\tilde{t}$ ${R}$ decresce
  con il tempo fino ad un nuova singolarità poiché $\ddot {R} <0$ sempre.
\end{itemize}

\section{Densità e pressione dell'Universo presente}

All'istante attuale densità e pressione sono date dalle equazioni
\begin{subequations}
  \begin{align}
    \rho_0  &= \frac{3}{8 \pi G} \left(\frac{k}{R_0^2} + H_0^2 \right)
    \label{rho0} \\
    p_0 &= -\frac{1}{8 \pi G} \left[\frac {k}{R_0^2}+H_0^2(1-2q_0)\right]
    \label{p0}
  \end{align}
\end{subequations}
Posto
\begin{equation}
  \rho_c = \frac{3 H_0^2}{8 \pi G} = \num{1.1e29}
  \left(
    \frac{H_0}{\SI[per-mode=reciprocal]{75}{\kilo\metre \per \second \per
        \mega\parsec}}
  \right)^2
  \si{\gram\per\centi\metre\cubed}
  \label{rhoc}
\end{equation}
l'equazione \eqref{rho0} diventa
\begin{equation}
  \rho_0-\rho_c =  \frac{3 k}{8 \pi G R_0^2}
  \label{temp1}
\end{equation}
pertanto l'Universo ha curvatura spaziale positiva o negativa a seconda che
$\rho_0 > \rho_c$ oppure $\rho_0 < \rho_c$.  Possiamo inoltre mettere in
relazione la densità $\rho_0$ con il parametro $q_0$.  Infatti per l'Universo
presente (dominato dalla materia) $p_0=0$, dall'eq. \eqref{p0} si ha
$k/R_0^2 = H_0^2 (2q_0-1)$, e sostituendo nella eq. \eqref{temp1}
\begin{equation}
  \frac {\rho_0} {\rho_c}=2 q_0
  \label{rho0surhoc}
\end{equation}
pertanto
\begin{itemize}
\item se $q_0 > 1/2 \implies \rho_0>\rho_c$ e $k=+1$
\item se $q_0 < 1/2 \implies \rho_0<\rho_c$ e $k=-1$.
\end{itemize}

\section{Era dominata dalla materia}

All'istante attuale $\rho_m(t_0) = \SI{e-34}{\gram\per\centi\metre\cubed}$,
$\rho_r(t_0)= \SI{e-34}{\gram\per\centi\metre\cubed}$; inoltre ad ogni istante
$\rho_m \propto R^{-3}$ e $\rho_r \propto R^{-4}$. Allora ($1+z=R_0/R$)
\begin{equation}
  \frac {\rho_m(t)} {\rho_r(t)} = 10^{3} \frac {R(t)} {R_0}= 10^3 \frac{1}{1+z}
\end{equation}
da cui risulta che l'Universo è dominato dalla materia ($\rho_m(t) > \rho_r(t)$)
dall'istante attuale ($z=0$) indietro nel tempo fino a red-shift $z = 1000$ (le
galassie più distanti e i QSOs osservati hanno $z<3$).  Nelle equazioni di
Friedmann porremo $p(t)=0$ ed assumeremo $\rho = \rho_0 [R_0/R(t)]^{3}$.
Riprendiamo in esame l'equazione \eqref{fri} in cui è conveniente esprimere
$\rho_0$ e $k/R^2$ in termini di $q_0$ ed $H_0$.  Dall'eq. \eqref{p0} segue
\begin{equation}
  \frac{k}{R_0^2}=(2q_0-1) H_0^2
\end{equation}
sostituendo questa espressione nell'eq. \eqref{rho0} si ha
\begin{equation}
  \frac{8\pi G}{3} \rho_0 = 2q_0 H_0^2
\end{equation}
Con le posizioni preceenti l'eq. \eqref{fri} di Friedmann diventa:
\begin{equation}
  \dot { R } ^2 + (2 q_0-1) H_0^2 R_0^2 = \frac{8\pi G}{3} \rho_0 \frac{R_0^3}{R}
\end{equation}
o equivalentemente
\begin{equation}
  \left(\frac{\dot R}{R_0}\right)^2 = H_0^2 \left(1-2q_0+2 q_0
    \frac{R_0}{R}\right)
\end{equation}
Posto $x=R(t)/R_0$, l'eq. precedente diventa
\begin{equation}
  \frac{dx}{dt}= H_0 \left(1-2q_0+ \frac {2q_0}{x}\right)^{1/2}
\end{equation}
la cui soluzione formale è
\begin{equation}
  t= H_0^{-1} \int_0^{R/R_0} dx \left(1-2q_0+\frac{2q_0}{x}\right)^{-1/2}
  \label{efa}
\end{equation}
Si osservi che abbiamo definito come $t=0$ l'inizio dell'era della materia.  In
effetti l'era della materia ininiza a $t \simeq \SI{e5}{yr}$.  L'età
dell'Universo è ottenuta dalla relazione precedente integrando sino ad $x=1$
\begin{equation}
  t_0= H_0^{-1} \int_0^{1} dx \left(1-2q_0+\frac{2q_0}{x}\right)^{-1/2}
\end{equation}
e quindi risulta sempre $t_0< H_0^{-1}$.  L'eq. \eqref{efa} può essere risolta
analiticamente nei tre casi $k=+1,0,-1$.

\subsubsection{Caso A $q_0 > 1/2,  k=+1 , \rho_0> \rho_c$}

Definiamo l'angolo $\theta$ dalla relazione:
\begin{equation}
  1-\cos \theta = \frac{2q_0-1 }{q_0} x
  \label{posizioneiniziale}
\end{equation}
da cui segue
\begin{equation}
  \dd x = \frac {q_0}{2q_0-1} \sin \theta \dd \theta =  \frac{- q_0}{2q_0-1} \dd \cos \theta
\end{equation}
L'integrale in eq. (\ref{efa}) diventa:
\begin{equation}
  \begin{split}
    t H_0 &= \int _1^{\cos \theta} \left[1-2q_0+2q_0 \left(\frac{2q_0-1}{q_0}
      \right) \left(\frac{1}{1-\cos \theta} \right) \right]^{-1/2} \dd\cos\theta
    \left(\frac{-q_0}{2q_0-1} \right) \\
    &=\frac{-q_0}{2q_0-1} \int _1^{\cos \theta} \left[
      \frac{1-\cos \theta - 2q_0 + 2q_0 \cos \theta + 4q_0 -2} {1-\cos\theta}
    \right]^{-1/2} \dd\cos\theta \\
    &=\frac {-q_0}{2q_0-1} \int _1^{\cos \theta} \left[ \frac {1-\cos \theta}
      {\cos \theta (2q_0-1) + (2q_0-1)} \right]^{1/2} \dd\cos\theta \\
    &=\frac {-q_0}{(2q_0-1)^{3/2}} \int_1^{\cos \theta} \sqrt \frac
    {1-\cos \theta} {1+\cos \theta} \dd \cos \theta
  \end{split}
\end{equation}
Nota che
\begin{equation}
  \int \sqrt \frac {1-y}{1+y} \dd y = \sqrt {1-y} + \arcsin y
\end{equation}
allora abbiamo
\begin{equation}
  t H_0 = \frac {-q_0}{(2q_0-1)^{3/2}}
  \left[ \sqrt {1-\cos\theta^2} + \arcsin \cos \theta - \arcsin 1 \right]
\end{equation}
Nota che $\cos \theta = \sin (\pi/2-\theta)$ ed inoltre
$\arcsin \sin \theta = \theta$. Allora
\begin{equation}
  t H_0= \frac {-q_0}{(2q_0-1)^{3/2}} \left[ \sin \theta + {\pi}/{2} - \theta -
    {\pi}/{2}\right]
\end{equation}
da cui
\begin{equation}
  t = \frac {-q_0}{H_0(2q_0-1)^{3/2}} ( \theta - \sin \theta )
  \label{sema}
\end{equation}
Questa relazione insieme alla posizione iniziale in eq. \eqref{posizioneiniziale}
(con $x=R(t)/R_0$)
\begin{equation}
  R(t)= \frac {R_0 q_0} {2q_0-1} (1-\cos \theta)
\end{equation}
mostra che $R(t)$ è una cicloide: $R(t)$ aumenta dal valore zero a $\theta=0$
($t=0$) fino al valore massimo $R_{max}= {2 q_0 R_0 }/({2q_0-1})$ a
$t_{max}= \pi q_0/(2q_0-1)^{3/2}$ e dopo ritorna a zero per $\theta =2 \pi$ e
$t=2t_{max}$.  L'istante attuale è ottenuto ponendo $R(t)=R_0$ ($x=1$)
\begin{subequations}
  \begin{align}
    \cos\theta_0 &= \frac {1} {q_0} -1 \\
    t_0 &= \frac {q_0}{H_0 (2q_0-1)^{3/2}} \left(\arccos(\frac {1}{q_0}-1) -
      \frac {1}{q_0}(2q_0-1)^{1/2} \right)
  \end{align}
\end{subequations}
Nota che
$\sin \theta_0= \sqrt{1-\cos^2\theta_0}= \sqrt{1-(q_0^{-1}-1)^2}=
{\sqrt{2q_0-1}}/{q_0}$
Se poniamo $q_0=1$ ($\cos \theta_0=0$), allora
$t_0 = H_0^{-1} (\pi /2-1) = \SI{7.5e5}{yr}$, $t_{max} = \SI{40e9}{yr}$,
l'intero ciclo si compierebbe in \SI{80e9}{yr}.  Qui abbiamo usato i valori
$H_0^{-1} = \SI{13e9}{yr}$,
$H_0 = \SI{75}{\kilo\metre \per \second \per \mega\parsec}$.

\subsubsection{Caso B: $q_0 = 1/2,  k=0, \rho_0=  \rho_c$}

Questo è il caso Einstein-de Sitter.  L'eq. \eqref{efa} diventa
\begin{equation}
  t = H_0^{-1} \int_0^{R/R_0} \sqrt x dx = H_0^{-1} \frac{2} {3} x^{3/2}
\end{equation}
da cui segue
\begin{equation}
  \frac{R(t)}{R_0} = \left( \frac {3H_0 t}{2} \right)^{2/3}
\end{equation}
Allora $R(t)$ aumenta senza limiti; l'istante attuale $t_0=2H_0^{-1}/3$ risulta
pari a $t_0= \SI{9e9}{yr}$
($H_0= \SI{75}{\kilo\metre \per \second \per \mega\parsec}$.)

\subsubsection{Caso C: $0< q_0 < 1/2,  k=-1, \rho_0 <  \rho_c$ }

Con la posizione $\theta = i \phi$ si ottengono risultati analoghi al caso A.
In particolare:
\begin{gather}
  t H_0= \frac {q_0}{(-2q_0+1)^{3/2}} \left( \sinh \phi \ - \phi \right)
  \label{topen} \\
  \cosh \phi -1 = \frac {1-2q_0}{q_0} \frac {R(t)}{R_0}
  \label{cosht}
\end{gather}
Per $t \to \infty$ si ha
$R(t)/R_0 \simeq {q_0}/{(1-2q_0)} {\exp{\phi}} \simeq \sqrt{1-2q_0} H_0 t$, da
cui si vede che $R(t) $ cresce senza limiti come $t$ stesso.  L'istante attuale
è definito ponendo $R(t_0)= R_0$ nell'eq. \eqref{cosht}.  Si ha
$\cosh \phi_0= q_0^{-1}-1$, e sostituendo nella eq. \eqref{topen}
($ \sinh^2 \phi - \cosh^2 \phi =1$) si ha:
\begin{equation}
  \begin{split}
    t_0 &= \frac {q_0}{H_0 (1-2q_0)^{3/2}}
    \left[- \cosh^{-1} (q_0^{-1}-1) + \sqrt{1+(q_0^{-1}-1)} \right] \\
    &= H_0^{-1} \left[ (1-2q_0)^{-1} - q_0 (1-2q_0)^{-3/2} \cosh^{-1}
      (q_0^{-1}-1) \right]
  \end{split}
\end{equation}
Ad esempio se prendiamo $q_0 = 0.014$ ($\rho_0=\rho^*_{~Gal}$) si ha
$t_0 \simeq 0.96~ H_0^{-1} \simeq \SI{13e9}{yr}$.  Nota che dallo studio delle
abbondanze degli isotopi degli elementi radioattivi segue che l'età della Terra
$\simeq \SI{4.5e9}{yr}$, e l'età della Galassia $\simeq \SI{7e9}{yr}$.  Comunque
gli oggetti più vecchi nell'Universo sono gli ammassi globulari la cui età
risulta $\simeq \SI{14e9}{yr}$.

\section{Universo di Friedmann con Costante Cosmologica}

Osserviamo che dall'eq. \eqref{moto1} con $p=-\rho$ segue
$\rho(t)=\text{costante}$, cioè si ha un Universo stazionario.  Prima della
scoperta dell'espansione cosmologica (Hubble 1929), Einstein modifica le
equazioni di campo introducendo un termine costante
\begin{equation}
  R_{\mu \nu} - \frac{1}{2} g_{\mu \nu} R - \Lambda g_{\mu \nu} =
  - 8 \pi G T_{\mu \nu}
  \label{EECC1}
\end{equation}
Possiamo scrivere l'eq. \eqref{EECC1} in un'altra forma.  Moltiplicando per
$g^{\mu \nu}$ segue: $R - (1/2) 4 R - 4\Lambda = -8 \pi G T^{\mu}_{mu}$, cioè
$R=8 \pi G \tensor{T}{^{\mu}}{_{mu}} -4 \Lambda$.  Sostituendo e portando a
destra il termine $\Lambda g_{\mu \nu}$
\begin{equation}
  R_{\mu \nu} = - 8 \pi G
  \left(
    T_{\mu \nu} - \frac {1}{2} g_{\mu \nu} T + \rho_{\Lambda}
  \right)
  \label{EECC2}
\end{equation}
con
\begin{equation}
  \rho_{\Lambda} = \frac {\Lambda c^2} {8 \pi G}
\end{equation}
Ho quindi un termine di sorgente $S_{\mu \nu}$ che nel caso $\mu=\nu=0$ diventa:
\begin{equation}
  S_{00} = \frac {1}{2} (\rho+3p) - \rho_{\Lambda}
\end{equation}
Uguagliando tale termine alla componente $tt$ del tensore di Ricci
($R_{tt}= 3 \ddot{R}/R$), in sostituzione dell'eq. \eqref{e1}, si ha ora:
\begin{equation}
  \frac{3 \ddot{R}}{R} = - 8\pi G \left[ \frac{1}{2}  (\rho+3p) - \rho_{\Lambda} \right] R
\end{equation}
e con la sostituzione $\rho_c=3H_0^2/(8\pi G)$
\begin{equation}
  \frac {\ddot{R}} {R} = - \frac {H_0^2} {\rho_c}
  \left[
    \frac {1}{2}  (\rho+3p) - \rho_{\Lambda}
  \right]
  \label{ecc1}
\end{equation}
ed infine
\begin{equation}
  \label{ddRcc}
  \frac {\ddot{R}} {R} = H_0^2
  \left[
    \Omega_{\Lambda} -\frac {\Omega_m (1+z)^3} {2 }  - \Omega_r (1+z)^4
  \right]
\end{equation}
con
\begin{subequations}
  \begin{align}
    \Omega_{\Lambda} & = \frac{\rho_{\Lambda}}{\rho_c}, \\
    \Omega_m         & = \frac{\rho_m(0)}{\rho_c}, \\
    \Omega_r         & = \frac{\rho_r(0)}{\rho_c}
  \end{align}
\end{subequations}
In maniera analoga il termine $S_{ij}$ diventa
\begin{equation}
  S_{ij}= \left[ \frac {1}{2}(\rho-p)  + \rho_{\Lambda} \right] R^2(t) \tilde{g_{ij}}
  \label{sijcc}
\end{equation}
Eq. \eqref{e2}  è ora sostituita da
\begin{equation}
  \frac {\ddot{R}}{R}+\frac{2\dot{R}^2}{R^2}+\frac{2k}{R^2}=
  8 \pi G \left[ \frac{1}{2}(\rho-p) +\rho_{\Lambda} \right]
  \label{ecc1}
\end{equation}
Eliminando $\ddot R$ con l'uso dell'eq. (\ref{ecc1})
\begin{equation}
  \frac  {\dot{R}^2} {R^2} +\frac {k}{R^2} = \frac {8 \pi G}{3} (\rho+\rho_{\Lambda})
\end{equation}
da cui segue:
\begin{equation}
  H(z)=H_0 \left[\Omega_{\Lambda}+\Omega_{k}(1+z)^2+\Omega_m (1+z)^3+\Omega_r (1+z)^4\right]^{1/2}
  \label{hzcc}
\end{equation}
con
\begin{equation}
  \Omega_k = \frac {k}{R_0^2 H_0^2}
\end{equation}
Osserva che se $k=0$ allora (per $t=t_0$ , $z=0$)
\begin{equation}
  \Omega_{\Lambda} + \Omega_m  + \Omega_r = 1
\end{equation}
Inoltre per il parametro di decelerazione si ha:
\begin{equation}
  q_0= -\frac {\ddot R_0 R_0}{\dot R_0^2} = \frac{\Omega_m}{2} + \Omega_r -\Omega_{\Lambda}
  \label{q0cc}
\end{equation}
{\bf Osservazione}: la dipendenza $H=H(t)$ influenza la relazione
distanza-redshift.  Dalla definizione $z=R_0/R(t) -1$, segue
\begin{equation}
  \dd z = -\frac {R_0}{R(t)} H(t) \dd t
\end{equation}
Per segnali luminosi ($d \tau=0$)
\begin{equation}
  \int_t^{t_0} \frac {\dd t}{R(t)}= \int_0^{r} \frac {\dd r}{\sqrt1-kr^2} = r +
  \mathcal{O}(r^3)
\end{equation}
Dalle precedenti segue:
\begin{equation}
  r= \int_0^z \frac{\dd z}{R_0 ~ H(t)}
\end{equation}
per cui la distanza di luminosit\'a $d_L= r R_0 (1+z)$ diventa
\begin{equation}
  d_L = (1+z) \int_0^z \frac{\dd z}{H(z)}.
\end{equation}
In conclusione lo studio del diagramma $m-z$ per oggetti molto lontani coinvolge
i 4 parametri: $\Omega_{\Lambda}$, $\Omega_{k}$, $\Omega_{m}$, $\Omega_{r}$.  La
vecchia relazione era
\begin{equation}
  d_L = \frac {1}{H_0} \left[ z+ \frac{1}{2}(1-q_0)z^2\right].
\end{equation}

{\bf Osservazione:} L'introduzione della costante cosmologica nelle equazioni di
Einstein equivale all'introduzione tra i costituenti dell'Universo di un fluido
con densità $\rho_{\Lambda}$ e pressione $p_{\Lambda}=-\rho_{\Lambda}$.
Indichiamo con X una generica componente dell'Universo con equazione di stato
\begin{equation}
  p_X = w \rho_X
\end{equation}
Ora, poiché per ogni componente disaccoppiata dalla radiazione vale l'equazione
di conservazione:
\begin{equation}
  \toder{(\rho_X R^3)}{R} = -3 p_X R^2
\end{equation}
si ha quindi
\begin{equation}
  \rho_X \propto \frac {1}{R^{-3(1+w)}}
\end{equation}
Pertanto $\rho_{X}=\text{costante}$ implica $w=-1$.  Questo dimostra appunto che
da $\rho_{\Lambda}=\text{costante}$ si ha $p_{\Lambda}=-\rho_{\Lambda}$.
Notiamo ancora che il rapporto
$\rho_{X}/\rho_m \propto R^{-3w} \propto (1+z)^{3w}$.  Quindi per il caso $w<0$,
si ha che per $t \to 0$ ($z \to \infty$), $\rho_{X}/\rho_m \to 0$, cioè
l'importanza della componente X diminuisce indietro nel tempo.  Allora la
cosmologia di Friedmann non è modificata per $z>>1$ dall'introduzione della
costante cosmologicanel caso $w<0$.

Allora invece di introdurre nelle equazioni di Einstein la costante cosmologica
(modificare la geometria) potrei aggiungere nel termine di sorgente una
componente dark che chiamiamo dark energy.  In generale si assume che la dark
energy abbia equazione di stato $p_X=w\rho_X$ con $w<0$, il caso $w=-1$ include
la costante cosmologica. La relazione in eq. \eqref{hzcc} risulterà così
modificata:
\begin{equation}
  H(z)=H_0 \left[ \Omega_{X} (1+z)^{3(1+w)} + \Omega_{k} (1+z)^2+ \Omega_m (1+z)^3 + \Omega_r (1+z)^4 \right]^{1/2}
  \label{hzccX}
\end{equation}
e per l'istante attuale
\begin{equation}
  1= \left[ \Omega_{X} + \Omega_{k} + \Omega_m  + \Omega_r \right]
 \end{equation}
Inoltre l'eq. (\ref{ddRcc}) diventa
\begin{equation}
  \frac {\ddot{R}} {R} = \frac{H_0^2}{2}
  \left[ \Omega_{X} (1+3w) (1+z)^{3(1+w)}
    +\frac {\Omega_m (1+z)^3} {2 }  +
    \Omega_r (1+z)^4 \right]
  \label {ddRw}
\end{equation}
da cui segue per il parametro di decelerazione
\begin{equation}
  q_0= \frac{\Omega_X}{2} (1+3w) +  \frac{\Omega_m}{2} + \Omega_r - \Omega_{k}
\end{equation}
Nel caso $k=0$,
\begin{equation}
  q_0= \frac {1+3w \Omega_{X}}{2}
\end{equation}
vedremo che $\Omega_{X}=0.7$ ed $\Omega_{m}=0.3$, pertanto
$q_0=0.5+w \propto - \ddot{R}_0$ quindi se $w<0.5$ Universo accelera.

Tornando indietro alla relazione $d_L(z)$ osserviamo che nel caso $\Omega_{k}=0$
ed $\Omega_{\Lambda}=0$
\begin{equation}
d_L =
\end{equation}
mentre se prendo $\Omega_{k}=0$ ed  $\Omega_{m}=0$
\begin{equation}
d_L = (1+z) \int_0^z \frac{dz}{H_0 (1+z)^{3(1+w)/2}}
\end{equation}
e per $w=-1$ segue $d_L=H_0^{-1} z (1+z)$. Quindi a parità di $z$ oggetti
appaiono più o meno distanti a seconda del valore dei parametri cosmologici
$\Omega$.  Vedi figura da cui si ottiene (assumendo $k=0$ $\Omega_{m}=0.32$ ed
$\Omega_{\Lambda}=0.68$)

L'introduzione della costante cosmologica pu\'o essere interpretata in due modi:
caso (a) equivale a cambiare la lagrangiana della materia (descritta da
coordinate generalizzate $\phi$) in accordo a:
\begin{equation}
  L_{\textup{matter}} \to L'_{\textup{matter}} = L_{\textup{matter}} - \frac{\Lambda}{8 \pi G}.
\end{equation}
In questo caso l'azione diventa
\begin{equation}
  S_{tot}=S_{g}+S_{m}= \frac {1}{16 \pi G}
  \int R \sqrt{-g} \dd^4x + \int \left( L_{m}- \frac {\Lambda}{8 \pi G} \right) \sqrt{-g} \dd^4 x
\end{equation}
Osserva che le equazioni del moto per la materia (descritta dal campo $\phi$),
ottenute da $\delta S/\delta \phi=0$, rimangono inalterate poiché
$\Lambda=\text{costante}$.  In questo caso l'introduzione del termine con
$\Lambda$ introduce uno shift nello zero point dell'energia della materia.
Questo shift non influenza la dinamica della materia, mentre la gravità che
accoppia alla totale energia del sistema ne è influenzata.  Il caso (b)
corrisponde a riscrivere l'azione totale come:
\begin{equation}
  S_{tot}= \frac {1}{16 \pi G}
  \int (R-2\Lambda)  \sqrt{-g} \dd^4x + \int  L_{m} \sqrt{-g} \dd^4 x
\end{equation}
Allora la gravità descritta da due costanti $G$ e $\Lambda$; in questo caso lo
spazio-tempo è curvo anche in assenza di materia (eq. di Einstein nel vuoto non
ammettono la soluzione flat space.)

L'equazione di stato $\rho=-p$ ha un'altra importante implicazione in relatività
generale.  La parte spaziale $\bm{g}$ dell'accelerazione geogtiche (che misura
la relativa accelerazione di due geodetiche vicine) soddisfa la seguente
equazione:
\begin{equation}
  \nabla \cdot \bm{g} = - 4 \pi G (\rho+3p)
\end{equation}
allora la sorgente dell'accelerazione geodetica è $\rho+3p$; allora se
$\rho+3p<0$ la gravità è repulsiva.  Questo appunto accade se
$\rho \simeq \rho_{\Lambda}$ poiché
$\rho_{\Lambda}+p_{\Lambda}= -2 \rho{\Lambda}$.  Ovviamente la transizione dalla
gravità attrattiva a quella repulsiva accade quando $\rho_m$ diventa minore di
$\rho_{\Lambda}$

Quanto vale $\Lambda$?  La velocità della luce $c$ e l'istante attuale $t_0$
definiscono una lunghezza caratteristica
$l_0 = c t_0 \simeq c/H_0 \simeq \SI{3000}{\mega\parsec}$.  Quindi
$l_0 \simeq \SI{10e28}{\centi\metre}$.  La costante cosmologica deve soddisfare
la condizione
$\Lambda < \SI[per-mode=reciprocal]{e-45}{\per\centi\metre\squared}$ in modo
tale che $1/\Lambda > l_0^2$.

%%% Local Variables:
%%% mode: latex
%%% TeX-master: "../gravitazione"
%%% End:


\appendix{}
\cleardoublepage
\chapter{Il teorema di Gauss nello spazio di Minkowski}
\label{cha:teorema-gauss}

Il teorema di Gauss in $\M$ mette in relazione l'integrale di volume
(spaziotemporale) della quadridivergenza di un quadrivettore $V^{\alpha}(x)$ con
un integrale di ipersuperficie di $V^{\alpha}(x)$
\begin{equation}
  \int\limits_{\Omega} \partial_{\alpha}V^{\alpha}(x) \dd^{4} x =
  \int\limits_{\partial\Omega} V^{\alpha}(x) \dd\Sigma_{\alpha},
\end{equation}
in cui $\partial\Omega$ è l'ipersuperficie di contorno del volume
quadrimensionale $\Omega$. Il quadrivettore $\dd\Sigma_{\alpha}$ è l'elemento di
ipersuperficie
\begin{equation}
  \dd\Sigma_{\alpha} = (\dd x^{1}\dd x^{2}\dd x^{3}, \dd x^{0}\dd x^{2}\dd
  x^{3}, \dd x^{0}\dd x^{1}\dd x^{3}, \dd x^{0}\dd x^{1}\dd x^{2})
\end{equation}
che ha come direzione la normale all'ipersuperficie e come modulo la sua area
$\dd\Sigma$.

% TODO: fare un semplice disegno come quello che ha fatto Ingrosso a lezione
% (11/04/2012) per aiutare a capire la geometria del sistema.
Questo teorema ha un'utile applicazione per quanto riguarda i quadrivettori
$V^{\alpha}(x)$ con quadridivergenza nulla
\begin{equation}
  \partial_{\alpha}V^{\alpha}(x) = 0.
\end{equation}
Integriamo $\partial_{\alpha}V^{\alpha}(x)$ sul volume $\Omega$ delimitato da
due ipersuperfici a tempo costante $\Sigma_{1}$ e $\Sigma_{2}$ (cioè individuate
rispettivamente dalle equazioni $t = t_{1}$ e $t = t_{2}$, con $t_{1}$ e $t_{2}$
costanti) e da un'ipersuperficie $\Sigma_{\infty}$ posta all'infinito spaziale.
Per il teorema di Gauss risulta
\begin{equation}
  0 = \int\limits_{\Omega} \partial_{\alpha}V^{\alpha}(x) \dd^{4}x =
  \int\limits_{\Sigma_{2}} V^{\alpha}(x) \dd\Sigma_{\alpha} -
  \int\limits_{\Sigma_{1}} V^{\alpha}(x) \dd\Sigma_{\alpha} +
  \int\limits_{\Sigma_{\infty}} V^{\alpha}(x) \dd\Sigma_{\alpha}.
\end{equation}
Il segno meno davanti all'integrale su $\Sigma_{1}$ è dovuto al fatto che
abbiamo orientato entrambe le normali a $\Sigma_{1}$ e $\Sigma_{2}$ nel verso
crescente del tempo.  Se $V^{\alpha}(x)$ tende abbastanza rapidamente a $0$
all'infinito spaziale, il terzo integrale nell'equazione precedente è nullo e
quindi
\begin{equation}
  \int\limits_{\Sigma_{2}} V^{\alpha}(x) \dd\Sigma_{\alpha} =
  \int\limits_{\Sigma_{1}} V^{\alpha}(x) \dd\Sigma_{\alpha}.
\end{equation}
Per l'arbitrarietà della scelta delle ipersuperfici $\Sigma_{1}$ e $\Sigma_{2}$,
ciascuna con la coordinata temporale fissata, dobbiamo concludere che
l'integrale
\begin{equation}
  I(\Sigma) = \int\limits_{\Sigma} V^{\alpha}(x) \dd\Sigma_{\alpha}
\end{equation}
non dipende dall'ipersuperficie $\Sigma$, cioè è una costante rispetto al tempo.
In particolare, se scegliamo un sistema di riferimento nel quale
$\dd\Sigma_{\alpha} = (\dd^{3} x, 0, 0, 0)$ allora abbiamo mostrato che
\begin{equation}
  \partial_{\alpha}V^{\alpha}(x) = 0 \implies \int V^{0}(x) \dd^{3} x =
  \text{costante in $t$}
\end{equation}
se l'integrale è esteso a una regione spaziale sufficientemente grande in modo
che $V^{\alpha}(x)$ si annulla sul suo bordo.

%%% Local Variables: 
%%% mode: latex
%%% TeX-master: "../astrofisica-teorica"
%%% fill-column: 80
%%% End: 

% Non l'abbiamo fatto a lezione, però introdurre brevemente lo spin serve per la
% precessione di De Sitter e mi sembra che Weinberg lo spieghi abbastanza bene.
\chapter{Spin}
\label{cha:spin}

Il momento angolare totale di un sistema è dato dalla somma del
\emph{momento angolare orbitale}, dovuto al suo moto attorno a un asse fisso, e
un momento angolare intrinseco, detto \emph{spin}.  Naturalmente non tutti i
sistemi hanno necessariamente sia un momento angolare orbitale sia uno spin.
Nel sistema di riferimento del centro di massa, il sistema è a riposo e il
momento angolare orbitale è nullo ma è presente sempre lo stesso spin, da qui
deriva l'aggettivo ``intrinseco''.  Per esempio, la Terra possiede un momento
angolare orbitale dovuto al moto di rivoluzione del pianeta intorno al Sole, e
uno spin, dovuto al moto di rotazione attorno a un proprio asse.  Un altro
esempio di corpo dotato di spin è un giroscopio che orbita intorno alla Terra.
Usando il tensore energia-impulso $T^{\alpha\beta}$, introdotto nel
paragrafo~\ref{sec:tensore-energia-impulso}, possiamo definire in maniera
precisa queste quantità nella relatività speciale.

Definiamo il quadritensore
\begin{equation}
  M^{\gamma\alpha\beta} = x^{\alpha}T^{\beta\gamma} - x^{\beta}T^{\alpha\gamma}.
\end{equation}
Questo tensore è antisimmetrico per lo scambio degli ultimi due indici:
$M^{\gamma\alpha\beta} = - M^{\gamma\beta\alpha}$.  Il tensore energia-impulso è
simmetrico e inoltre soddisfa la relazione di conservazione
$\lparder{T^{\alpha\beta}}{x^{\beta}} = 0$, per questi motivi anche
$M^{\gamma\alpha\beta}$ si conserva
\begin{equation}
  \parder{M^{\gamma\alpha\beta}}{x^{\gamma}} = \parder{x^{\alpha}}{x^{\gamma}}
  T^{\beta\gamma} - \parder{x^{\beta}}{x^{\gamma}} T^{\alpha\gamma} =
  T^{\beta\alpha} - T^{\alpha\beta} = 0.
\end{equation}
Allora per il teorema di Gauss (vedi l'appendice~\ref{cha:teorema-gauss})
possiamo definire il seguente quadritensore $J^{\alpha\beta}$ di rango $2$ e
costante nel tempo
\begin{equation}
  J^{\alpha\beta} = \int M^{0\alpha\beta} \dd^{3} \bm{x} = -\int
  M^{0\beta\alpha} \dd^{3} \bm{x} = -J^{\beta\alpha}.
\end{equation}
Questo tensore prende il nome di \index{momento!angolare totale}
\emph{momento angolare totale}.  Possiamo giustificare questo nome scrivendo
esplicitamente le componenti spaziali
\begin{equation}
  J^{ij} = \int (x^{i}T^{j0} - x^{j}T^{i0})\dd^{3} \bm{x}.
\end{equation}
Abbiamo interpretato le componenti $T^{i0}$ del tensore energia-impulso come la
densità della componente $i$ della quantità di moto, dunque riconosciamo che
$\bm{L} = (J^{23}, J^{31}, J^{12}) = \bm{r} \times \bm{p}$ è il vettore
tridimensionale momento angolare.  Per le altre componenti non nulle
\begin{equation}
  J^{0i} = tp^{i} - \int x^{i}T^{00} \dd^{3} \bm{x}
\end{equation}
non esiste un'interpretazione fisica in termini di variabili dinamiche.  Queste
componenti possono peraltro essere annullate ponendo
$\int x^{i}T^{00} \dd^{3} \bm{x} = 0$ al tempo $t = 0$.

Il tensore momento angolare totale $J^{\alpha\beta}$ si comporta come un normale
tensore di rango $2$ rispetto alle trasformazioni di Lorentz omogenee
$x^{\alpha} \to x'^{\alpha} = \tensor{\Lambda}{^{\alpha}_{\beta}} x^{\beta}$, ma
per una traslazione $x^{\alpha} \to x'^{\alpha} = x^{\alpha} + a^{\alpha}$
abbiamo
\begin{equation}
  J^{\alpha\beta} \to J'^{\alpha\beta} = J^{\alpha\beta} + a^{\alpha}p^{\beta} -
  a^{\beta}p^{\alpha}.
\end{equation}
Ciò è dovuto al fatto che $J^{\alpha\beta}$ comprende anche il momento angolare
orbitale, definito in relazione a un asse di rotazione.  Per individuare la
parte ``intrinseca'' del momento angolare totale $J^{\alpha\beta}$, cioè una
parte che non cambi per effetto di una traslazione, definiamo il quadrivettore
di \index{spin}\index{momento!angolare intrinseco|see{spin}}\emph{spin}
\begin{equation}
  S_{\alpha} = \frac{1}{2}\epsilon_{\alpha\beta\gamma\delta} J^{\beta\gamma} U^{\delta},
\end{equation}
in cui $U^{\alpha}$ è il quadrivettore velocità del sistema.  Grazie a questa
definizione, il quadrivettore di spin non cambia per effetto di una traslazione.
È intuitivo capire che per una particella libera il quadrivettore di spin è
costante
\begin{equation}
  \toder{S_{\alpha}}{t} = 0.
\end{equation}
Nel sistema di riferimento del centro di massa del sistema
$U'^{\alpha} = (1, \bm{0})$, quindi
\begin{subequations}
  \begin{align}
    S'_{0} &= \epsilon_{0\beta\gamma\delta} J^{\alpha\beta} U'^{\gamma} =
    \epsilon_{0\beta\gamma 0} J^{\alpha\beta} = 0, \\
    S'_{i} &= \frac{1}{2}\epsilon_{ijk} J^{jk} = \int \epsilon_{ijk} x^{j}T^{0k}
    \dd^{3} \bm{x},
  \end{align}
\end{subequations}
cioè $S'_{\alpha} = (0, J^{23}, J^{31}, J^{12})$.  Nel sistema di riferimento
del centro di massa il momento angolare orbitale è nullo, pertanto è corretta
l'identificazione del quadrivettore di spin con il momento angolare intrinseco.
Anche quando la velocità del sistema non è nulla, $(U^{i}) \neq \bm{0}$, il
quadrivettore di spin $S_{\alpha}$ ha solo tre componenti indipendenti perché
il quadrivettore di spin e la quadrivelocità sono sempre ortogonali
\begin{equation}
  S_{\alpha}U^{\alpha} = \frac{1}{2}
  \epsilon_{\alpha\beta\gamma\delta}J^{\beta\gamma}U^{\delta}U^{\alpha} =
  -\frac{1}{2}
  \epsilon_{\delta\beta\gamma\alpha}J^{\beta\gamma}U^{\delta}U^{\alpha} = 0.
\end{equation}


%%% Local Variables:
%%% mode: latex
%%% TeX-master: "../gravitazione"
%%% TeX-PDF-mode: t
%%% End:

\cleardoublepage
\chapter{Dimostrazioni di alcune relazioni}
\label{cha:dimostrazioni}

\section{Dimostrazione della
  relazione~\texorpdfstring{\eqref{eq:correzione-metrica-controvariante}}
  {(2.33)}}
\label{sec:dimostr-correz-metrica-controvariante}

Nel caso di campo gravitazionale debole, la correzione al tensore metrico
covariante al primo ordine negli infinitesimi è
\begin{equation}
  \label{eq:correzione-metrica-covariante}
  g_{\mu\nu} = \eta_{\mu\nu} + h_{\mu\nu},
\end{equation}
con $\abs{h_{\mu\nu}} \ll 1$.  Anche $g^{\mu\nu}$ dovrà differire di poco dal
tensore di Minkowski controvariante
\begin{equation}
    \label{eq:correzione-metrica-controvariante2}
  g^{\mu\nu} = \eta^{\mu\nu} + c^{\mu\nu}
\end{equation}
ma le correzioni $c^{\mu\nu}$ alle componenti del tensore metrico controvariante
non sono semplicemente
$c^{\mu\nu} = h^{\mu\nu} = \eta^{\mu\lambda}\eta^{\nu\sigma} h_{\lambda\sigma}$
perché $g^{\mu\nu}$ e $g_{\mu\nu}$ sono legati dalla relazione
\begin{equation}
  g_{\mu\nu}g^{\nu\lambda} = \tensor{\delta}{_{\mu}^{\lambda}}.
\end{equation}
Infatti sostituendo le espressioni~\eqref{eq:correzione-metrica-covariante} e
\eqref{eq:correzione-metrica-controvariante2} dei tensori metrici nell'equazione
precedente abbiamo
\begin{equation}
  \tensor{\delta}{_{\mu}^{\lambda}} = (\eta_{\mu\nu} +
  h_{\mu\nu})(\eta^{\nu\lambda} + c^{\nu\lambda}) =
  \tensor{\delta}{_{\mu}^{\lambda}} + \tensor{c}{_{\mu}^{\lambda}} +
  \tensor{h}{_{\mu}^{\lambda}} + h_{\mu\nu}c^{\nu\lambda},
\end{equation}
quindi deve risultare
\begin{equation}
  0 = \tensor{c}{_{\mu}^{\lambda}} + \tensor{h}{_{\mu}^{\lambda}} +
  h_{\mu\nu}c^{\nu\lambda} \implies \tensor{c}{_{\mu}^{\lambda}} =
  -\tensor{h}{_{\mu}^{\lambda}} - h_{\mu\nu}c^{\nu\lambda}.
\end{equation}
Innalzando in ambo i membri l'indice libero $\mu$ ricaviamo
\begin{equation}
  c^{\mu\lambda} = -h^{\mu\lambda} - \tensor{h}{^{\mu}_{\nu}}c^{\nu\lambda} =
  -h^{\mu\lambda} -\tensor{h}{^{\mu}_{\nu}}(-h^{\nu\lambda} -
  \tensor{h}{^{\nu}_{\rho}} c^{\rho\lambda}) = -h^{\mu\lambda} +
  \tensor{h}{^{\mu}_{\nu}}h^{\nu\lambda} + \mathcal{O}(h^{3}).
\end{equation}
Quindi trascurando termini infinitesimi superiori al terzo ordine, il tensore
metrico controvariante nell'approssimazione di campo debole è dato da
\begin{equation}
  g^{\mu\nu} = \eta^{\mu\nu} + c^{\mu\nu} \approx \eta^{\mu\nu} - h^{\mu\lambda}
  + \tensor{h}{^{\mu}_{\nu}}h^{\nu\lambda}.
\end{equation}
Fermandosi al primo ordine della perturbazione si ha
$g^{\mu\nu} \approx \eta^{\mu\nu} - h^{\mu\nu}$.

\section{Dimostrazione delle
  relazioni~\texorpdfstring{\eqref{eq:g_mu_rho-determinante}}{(3.51)}}
\label{sec:dimostr-determinante}

Sia $A$ una matrice quadrata reale di ordine $n$
\begin{equation}
  A =
  \begin{pmatrix}
    a_{11} & a_{12} & \cdots & a_{1n} \\
    a_{21} & a_{22} & \cdots & a_{2n} \\
    \vdots & \vdots & \ddots & \vdots \\
    a_{n1} & a_{n2} & \cdots & a_{nn}
  \end{pmatrix},
\end{equation}
il suo determinante $D$ si può calcolare, avendo fissato una riga $i$, con lo
\emph{sviluppo di Laplace}
\begin{equation}
  D = \det A = \sum_{j=1}^{n} a_{ij}A_{ij}
\end{equation}
in cui $A_{ij}$ è il \emph{complemento algebrico} dell'elemento $a_{ij}$, cioè
il determinante della matrice che si ottiene da $A$ sopprimendo l'$i$-esima riga
e la $j$-esima colonna moltiplicato per $+1$ se $i+j$ è pari, per $-1$ se $i+j$
è dispari.  Lo sviluppo di Laplace può anche essere svolto fissando la colonna
$j$ e sommando sull'indice di riga $i$.  Dunque, la derivata di $D$ fatta
rispetto all'elemento $a_{ij}$ è il suo coefficiente nello sviluppo di Laplace,
cioè il complemento algebrico corrispondente
\begin{equation}
  \label{eq:deriv-determinante}
  \parder{D}{a_{ij}} = A_{ij}.
\end{equation}

La matrice $B = A^{-1}$ inversa di $A$ con determinante non nullo è quella
matrice di ordine $n$ tale che
\begin{equation}
  A B = B A = I_{n},
\end{equation}
dove $I_{n} = (\delta_{ij})_{i,j=1}^{n}$ è la matrice identità di ordine
$n$. Per la regola di Cramer gli elementi $b_{ij}$ della matrice inversa sono
dati dal rapporto $A_{ji}/D$, cioè la matrice $B$ può essere così scritta
\begin{equation}
  B = A^{-1} =
  \begin{pmatrix}
    b_{11} & b_{12} & \cdots & b_{1n} \\
    b_{21} & b_{22} & \cdots & b_{2n} \\
    \vdots & \vdots & \ddots & \vdots \\
    b_{n1} & b_{n2} & \cdots & b_{nn}
  \end{pmatrix} =
  \begin{pmatrix}
    \dfrac{A_{11}}{D} & \dfrac{A_{21}}{D} & \cdots & \dfrac{A_{n1}}{D} \\[2.0ex]
    \dfrac{A_{11}}{D} & \dfrac{A_{22}}{D} & \cdots & \dfrac{A_{n2}}{D} \\[2.0ex]
    \vdots            & \vdots            & \ddots & \vdots            \\[2.0ex]
    \dfrac{A_{1n}}{D} & \dfrac{A_{2n}}{D} & \cdots & \dfrac{A_{nn}}{D}
  \end{pmatrix}.
\end{equation}
Per la regola di Binet risulta $\det B = 1/\det A = 1/D$.  Inoltre, se la
matrice $A$ è diagonale, si ricava subito che anche $B$ è diagonale con elementi
$b_{ii}=1/a_{ii}$.

Il tensore metrico covariante $g_{\mu\nu}$ è un insieme di $4^{2}$ quantità
reali che possono essere raccolte in una matrice $4 \times 4$
\begin{equation}
  g_{\mu\nu} =
  \begin{pmatrix}
    g_{00} & g_{01} & g_{02} & g_{03} \\
    g_{10} & g_{11} & g_{12} & g_{13} \\
    g_{20} & g_{21} & g_{22} & g_{23} \\
    g_{30} & g_{31} & g_{32} & g_{33}
  \end{pmatrix}.
\end{equation}
Dall'equazione~\eqref{eq:deriv-determinante} abbiamo che la derivata di
$g = -\det(g_{\mu\nu})$ rispetto all'elemento $g_{\mu\nu}$ è
\begin{equation}
  \parder{g}{g_{\mu\nu}} = -\Delta_{\mu\nu},
\end{equation}
in cui $\Delta_{\mu\nu}$ è il complemento algebrico dell'elemento $g_{\mu\nu}$.
Il tensore metrico controvariante $g^{\mu\nu}$ è la matrice inversa di
$g_{\mu\nu}$, quindi, sfruttando la simmetria del tensore metrico, abbiamo
\begin{equation}
  g^{\mu\nu} = g^{\nu\mu} = \frac{\Delta_{\mu\nu}}{-g} =
  \frac{1}{g} \parder{g}{g_{\mu\nu}}.
\end{equation}
Applicando lo stesso ragionamento a $g^{\mu\nu}$ troviamo
($G = -\det(g^{\mu\nu}) = 1/\det(g_{\mu\nu}) = 1/g$)
\begin{equation}
  g_{\mu\nu} = \frac{1}{G} \parder{G}{g^{\mu\nu}} =
  \frac{1}{G} \parder{G}{g} \parder{g}{g^{\mu\nu}} = g \bigg(- \frac{1}{g^{2}}
  \bigg) \parder{g}{g^{\mu\nu}} = - \frac{1}{g} \parder{g}{g^{\mu\nu}}.
\end{equation}

\section{Dimostrazione dell'identità di Bianchi}
\label{sec:dimostr-identita-bianchi}

Per verificare \index{identità!di Bianchi}l'identità di
Bianchi~\eqref{eq:bianchi} conviene verificarla in un sistema di coordinate
localmente inerziali, quindi sarà valida in tutto lo spazio dato il carattere
tensoriale di $R_{\lambda\mu\nu\kappa}$.  In questo sistema di coordinate la
connessione affine si annulla e la derivata covariante coincide con la derivata
ordinaria, dunque dalla~\eqref{eq:riemann-covariante} abbiamo
\begin{subequations}
  \begin{align}
    R_{\lambda\mu\nu\kappa;\eta} &=
    \frac{1}{2} \partial_{\eta}(\partial_{\kappa} \partial_{\mu} g_{\lambda\nu}
    - \partial_{\kappa} \partial_{\lambda} g_{\mu\nu}
    - \partial_{\nu} \partial_{\mu} g_{\lambda\kappa}
    + \partial_{\nu} \partial_{\lambda} g_{\mu\kappa}), \\
    R_{\lambda\mu\kappa\eta;\nu} &=
    \frac{1}{2} \partial_{\nu}(\partial_{\eta} \partial_{\mu} g_{\lambda\kappa}
    - \partial_{\nu} \partial_{\lambda} g_{\mu\kappa}
    - \partial_{\kappa} \partial_{\mu} g_{\lambda\eta}
    + \partial_{\kappa} \partial_{\lambda} g_{\mu\eta}), \\
    R_{\lambda\mu\eta\nu;\kappa} &= \frac{1}{2} \partial_{\kappa}(\partial_{\nu}
    \partial_{\mu} g_{\lambda\eta} - \partial_{\nu} \partial_{\lambda}
    g_{\mu\eta} - \partial_{\eta} \partial_{\mu} g_{\lambda\nu}
    + \partial_{\eta} \partial_{\lambda} g_{\mu\nu}).
  \end{align}
\end{subequations}
Sommando membro a membro queste tre relazioni si ottiene l'identità di Bianchi
\begin{equation}
  R_{\lambda\mu\nu\kappa;\eta} + R_{\lambda\mu\kappa\eta;\nu} +
  R_{\lambda\mu\eta\nu;\kappa} = 0.
\end{equation}

\section{Dimostrazione dell'identità di Palatini}
\label{sec:dimostr-palatini}

% vedi Weinberg, pagina 290
Se si fa variare la metrica $g_{\mu\nu} \to g_{\mu\nu} + \delta g_{\mu\nu}$, con
$\delta g_{\mu\nu}$ piccolo, la variazione del tensore di
Ricci~\eqref{eq:tens-ricci} al primo ordine in $\delta g_{\mu\nu}$ è data da
\begin{equation}
  \delta R_{\mu\nu}
  = \parder{\delta\tensor{\Gamma}{^{\lambda}_{\mu\lambda}}}{x^{\nu}}
  - \parder{\delta\tensor{\Gamma}{^{\lambda}_{\mu\nu}}}{x^{\lambda}} +
  \delta\tensor{\Gamma}{^{\eta}_{\mu\lambda}}
  \tensor{\Gamma}{^{\lambda}_{\eta\nu}} +
  \delta\tensor{\Gamma}{^{\lambda}_{\eta\nu}}
  \tensor{\Gamma}{^{\eta}_{\mu\lambda}} -
  \delta\tensor{\Gamma}{^{\eta}_{\mu\nu}}
  \tensor{\Gamma}{^{\lambda}_{\eta\lambda}} -
  \delta\tensor{\Gamma}{^{\lambda}_{\eta\lambda}}
  \tensor{\Gamma}{^{\eta}_{\mu\nu}},
\end{equation}
in cui $\delta\tensor{\Gamma}{^{\lambda}_{\mu\nu}}$ è la variazione della
connessione affine
\begin{equation}
  \delta\tensor{\Gamma}{^{\lambda}_{\mu\nu}} = -g^{\lambda\rho}\delta
  g_{\rho\sigma} \tensor{\Gamma}{^{\sigma}_{\mu\nu}} +
  \frac{1}{2}g^{\lambda\rho} \bigg(\parder{\delta g_{\rho\mu}}{x^{n}}
  + \parder{\delta g_{\rho\nu}}{x^{\mu}} - \parder{\delta
    g_{\mu\nu}}{x^{\rho}}\bigg).
\end{equation}
Le quantità $\delta \tensor{\Gamma}{^{\lambda}_{\mu\nu}}$ sono dei tensori,
essendo differenze fra due connessioni affini, quindi possiamo calcolare le loro
derivate covarianti (l'operazione di derivazione covarianti è calcolata usando
la connessione affine imperturbata $\tensor{\Gamma}{^{\lambda}_{\mu\nu}}$).  In
particolare risulta
\begin{subequations}
  \begin{align}
    (\delta \tensor{\Gamma}{^{\lambda}_{\mu\lambda}})_{;\nu} &= (\delta
    \tensor{\Gamma}{^{\lambda}_{\mu\lambda}})_{,\nu} +
    \delta\tensor{\Gamma}{^{\eta}_{\mu\lambda}}
    \tensor{\Gamma}{^{\lambda}_{\eta\nu}} -
    \delta\tensor{\Gamma}{^{\lambda}_{\eta\lambda}}
    \tensor{\Gamma}{^{\eta}_{\mu\nu}} -
    \delta\tensor{\Gamma}{^{\lambda}_{\mu\eta}}
    \tensor{\Gamma}{}^{\eta}_{\nu\lambda},  \\
    (\delta \tensor{\Gamma}{^{\lambda}_{\mu\nu}})_{;\lambda} &= (\delta
    \tensor{\Gamma}{^{\lambda}_{\mu\nu}})_{,\lambda} +
    \delta\tensor{\Gamma}{^{\eta}_{\mu\nu}}
    \tensor{\Gamma}{^{\lambda}_{\eta\lambda}} -
    \delta\tensor{\Gamma}{^{\lambda}_{\eta\nu}}
    \tensor{\Gamma}{^{\eta}_{\mu\lambda}} -
    \delta\tensor{\Gamma}{^{\lambda}_{\mu\eta}}
    \tensor{\Gamma}{}^{\eta}_{\nu\lambda}.
  \end{align}
\end{subequations}
È immediato verificare che la variazione del tensore di Ricci
$\delta R_{\mu\nu}$ è data dalla differenza di queste due quantità
\begin{equation}
  \delta R_{\mu\nu} = (\delta \tensor{\Gamma}{^{\lambda}_{\mu\lambda}})_{;\nu}
  - (\delta \tensor{\Gamma}{^{\lambda}_{\mu\nu}})_{;\lambda}.
\end{equation}
Questa è l'\index{identità!di Palatini}\emph{identità di Palatini}.  È stata
ottenuta con la solita convenzione dei segni $(+--)$
di~\textcite{weinberg:gravitation}.  Se si invece adotta la convenzione $(+--)$,
come \textcites{barone:relativita,landau:campi}, si ottiene
\begin{equation}
  \delta R_{\mu\nu} = (\delta \tensor{\Gamma}{^{\lambda}_{\mu\nu}})_{;\lambda} -
  (\delta \tensor{\Gamma}{^{\lambda}_{\mu\lambda}})_{;\nu}.
\end{equation}

%%% Local Variables:
%%% mode: latex
%%% TeX-master: "../gravitazione"
%%% fill-column: 80
%%% TeX-PDF-mode: t
%%% End:


\backmatter{}

\cleardoublepage{}
\phantomsection
\addcontentsline{toc}{chapter}{\refname}
\nocite{*}
\printbibliography

\cleardoublepage{}
\phantomsection
\addcontentsline{toc}{chapter}{\indexname}
\printindex{}

\end{document}

%%% Local Variables:
%%% mode: latex
%%% TeX-master: t
%%% fill-column: 80
%%% End:
