\cleardoublepage{}

\chapter{Calcolo tensoriale nella relatività generale}
\label{cha:calcolo-tensoriale}

\emph{Nota: da questo capitolo in poi useremo, per brevità, i termini
  \emph{tensore} e \emph{vettore} in luogo di, rispettivamente,
  \emph{quadritensore} e \emph{quadrivettore} se questo non dovesse generare
  ambiguità.}

Nel paragrafo~\ref{sec:calcolo-tensoriale-minkowski} abbiamo introdotto, nel
contesto della relatività speciale, i tensori, specificando come si modificano
per effetto di una trasformazione di Lorentz, la quale è una trasformazione
lineare fra sistemi di riferimento inerziali.  Nella definizione dei tensori
intervengono le componenti $\tensor{\Lambda}{^{\alpha}_{\beta}}$ e
$\tensor{\Lambda}{_{\alpha}^{\beta}}$ delle matrici di Lorentz associate alle
specifiche trasformazioni considerate.

Anche nella relatività generale è utile introdurre i tensori, ma in questa
teoria si considerano trasformazioni arbitrarie, quindi non necessariamente
lineari, fra sistemi di riferimento anche non inerziali.  Abbiamo già notato
che, nella relatività speciale, gli elementi delle matrici delle trasformazioni
di Lorentz $x^{\alpha} \to x'^{\alpha}$ sono dati da
\begin{subequations}
  \begin{align}
    \tensor{\Lambda}{^{\alpha}_{\beta}} &= \parder{x'^{\alpha}}{x^{\beta}}, \\
    \tensor{\Lambda}{_{\beta}^{\alpha}} &= \parder{x^{\alpha}}{x'^{\beta}}.
  \end{align}
\end{subequations}
Definiremo i tensori, in relatività generale, in maniera analoga a quanto fatto
all'interno della relatività speciale, a patto di effettuare le sostituzioni
\begin{subequations}
  \label{eq:sostituzioni}
  \begin{align}
    \tensor{\Lambda}{^{\alpha}_{\beta}} &\to \parder{x'^{\alpha}}{x^{\beta}}, \\
    \tensor{\Lambda}{_{\beta}^{\alpha}} &\to \parder{x^{\alpha}}{x'^{\beta}}.
  \end{align}
\end{subequations}

\section{Tensori}
\label{sec:tensori}

Un \emph{tensore}
$\tensor*{A}{^{\alpha_{1}\dots\alpha_{p}}_{\beta_{1}\dots\beta_{q}}}$ (con
$p + q = k$) di rango $k$ è un insieme di $4^{k}$ quantità che sotto una
trasformazione arbitraria delle coordinate $x^{\alpha} \to x'^{\alpha}$ si
trasformano come
\begin{equation}
  \tensor*{A}{^{\alpha_{1}\dots\alpha_{p}}_{\beta_{1}\dots\beta_{q}}} \to
  \tensor*{{A'}}{^{\alpha_{1}\dots\alpha_{p}}_{\beta_{1}\dots\beta_{q}}} =
  \parder{x'^{\alpha_{1}}}{x^{\mu_{1}}}
  \cdots \parder{x'^{\alpha_{p}}}{x^{\mu_{p}}} \parder{x^{\nu_{1}}}{x'^{\beta_{1}}}
  \cdots \parder{x^{\nu_{q}}}{x'^{\beta_{q}}}
  \tensor*{A}{^{\mu_{1}\dots\mu_{p}}_{\nu_{1}\dots\nu_{q}}}.
\end{equation}
Il tensore $\tensor*{A}{^{\alpha_{1}\dots\alpha_{p}}_{\beta_{1}\dots\beta_{q}}}$
è un tensore misto $p$ volte controvariante e $q$ volte covariante, oppure di
tipo $(p,q)$.  Un tensore di rango $k$ completamente controvariante è del tipo
$\tensor{A}{^{\alpha_{1}\dots\alpha_{k}}}$ e si trasforma come
\begin{equation}
  \tensor{A}{^{\alpha_{1}\dots\alpha_{k}}} \to
  \tensor{{A'}}{^{\alpha_{1}\dots\alpha_{k}}} =
  \parder{x'^{\alpha_{1}}}{x^{\mu_{1}}}
  \cdots \parder{x'^{\alpha_{k}}}{x^{\mu_{k}}}
  \tensor{A}{^{\mu_{1}\dots\mu_{k}}}.
\end{equation}
Un quadritensore di rango $k$ completamente covariante è del tipo
$\tensor{A}{_{\alpha_{1}\dots\alpha_{k}}}$ e si trasforma come
\begin{equation}
  \tensor{A}{_{\alpha_{1}\dots\alpha_{k}}} \to
  \tensor{{A'}}{_{\alpha_{1}\dots\alpha_{k}}} =
  \parder{x^{\mu_{1}}}{x'^{\alpha_{1}}}
  \cdots \parder{x^{\mu_{k}}}{x'^{\alpha_{k}}}
  \tensor{A}{_{\mu_{1}\dots\mu_{k}}}.
\end{equation}
In particolare, i tensori di rango $0$, che quindi non hanno indici, sono
chiamati \emph{scalari} e sono invarianti sotto trasformazioni arbitrarie di
coordinate, i tensori di rango $1$ sono i \emph{vettori}.  Distinguiamo fra
vettori controvarianti $V^{\alpha}$, i quali si trasformano, per cambianti di
coordinate $x^{\alpha} \to x'^{\alpha}$, come
\begin{equation}
  V^{\alpha} \to V'^{\alpha} = \parder{x'^{\alpha}}{x^{\mu}}V^{\mu},
\end{equation}
e vettori covarianti $U_{\alpha}$, che si trasformano come
\begin{equation}
  U_{\alpha} \to U'_{\alpha} = \parder{x^{\mu}}{x'^{\alpha}} U_{\mu}.
\end{equation}
Le trasformazioni inverse per i vettori controvarianti e covarianti sono
\begin{gather}
  V'^{\alpha} \to V^{\alpha} = \parder{x^{\alpha}}{x'^{\mu}} V'^{\mu}, \\
  U'_{\alpha} \to U_{\alpha} = \parder{x'^{\mu}}{x^{\alpha}} U'_{\mu}.
\end{gather}

Uno \index{pseudotensore}pseudotensore
$\tensor*{A}{^{\alpha_{1}\dots\alpha_{p}}_{\beta_{1}\dots\beta_{q}}}$ (con
$p+q=k$) di rango $k$ e peso $w$ è un insieme di $4^{k}$ quantità che per
effetto di un'arbitraria trasformazione delle coordinate
$x^{\alpha} \to x'^{\alpha}$ si trasformano come
\begin{equation}
  \tensor*{A}{^{\alpha_{1}\dots\alpha_{p}}_{\beta_{1}\dots\beta_{q}}} \to
  \tensor*{{A'}}{^{\alpha_{1}\dots\alpha_{p}}_{\beta_{1}\dots\beta_{q}}}
  = \abs*{\parder{x'}{x}}^{w} \parder{x'^{\alpha_{1}}}{x^{\mu_{1}}}
  \cdots \parder{x'^{\alpha_{p}}}{x^{\mu_{p}}} \parder{x^{\nu_{1}}}{x'^{\beta_{1}}}
  \cdots \parder{x^{\nu_{q}}}{x'^{\beta_{q}}}
  \tensor*{A}{^{\mu_{1}\dots\mu_{p}}_{\nu_{1}\dots\nu_{q}}},
\end{equation}
in cui $\abs{\lparder{x'}{x}}$ è lo jacobiano della matrice della trasformazione
delle coordinate.

I differenziali delle coordinate controvarianti sono vettori controvarianti,
infatti
\begin{equation}
  \dd x'^{\alpha} = \parder{x'^{\alpha}}{x^{\mu}} \dd x^{\mu}.
\end{equation}

La delta di Kronecker definita nella~\eqref{eq:delta-kronecker} è l'unico
tensore le cui componenti sono uguali in qualsiasi sistema di riferimento, oltre
agli scalari e al tensore nullo.

\subsection{Tensore metrico}
\label{sec:tensore-metrico}

In un generico sistema di coordinate $x^{\mu}$, il tensore metrico è dato per
definizione da
\begin{equation}
  g_{\mu\nu} =
  \eta_{\alpha\beta} \parder{\xi^{\alpha}}{x^{\mu}} \parder{\xi^{\beta}}{x^{\nu}},
\end{equation}
in cui $\xi^{\alpha}$ è un sistema di coordinate localmente inerziale.  In un
altro sistema di coordinate $x'^{\mu}$ risulta
\begin{equation}
  \label{eq:trasf-tensore-metrico}
  \begin{split}
    g'_{\mu\nu} &=
    \eta_{\alpha\beta} \parder{\xi^{\alpha}}{x'^{\mu}}
    \parder{\xi^{\beta}}{x'^{\nu}} =
    \eta_{\alpha\beta} \parder{\xi^{\alpha}}{x^{\rho}}
    \parder{x^{\rho}}{x'^{\mu}} \parder{\xi^{\beta}}{x^{\sigma}}
    \parder{x^{\sigma}}{x'^{\nu}} \\
    &= g_{\rho\sigma} \parder{x^{\rho}}{x'^{\mu}} \parder{x^{\sigma}}{x'^{\nu}}.
  \end{split}
\end{equation}
Osserviamo che, a differenza del caso dello spazio di Minkowski piatto, nello
spazio-tempo curvo della relatività generale il tensore metrico $g_{\mu\nu}$ non
è invariante per trasformazioni delle coordinate.  Definiamo il tensore metrico
controvariante $g^{\mu\nu}$ come quel tensore tale che
\begin{equation}
  g_{\mu\rho}g^{\mu\sigma} = \tensor{\delta}{^{\sigma}_{\rho}}.
\end{equation}
Si dimostra che $g^{\mu\nu}$ è effettivamente un tensore controvariante di rango
$2$.

L'opposto del determinante della matrice che rappresenta il tensore metrico
\begin{equation}
  g = -\det(g_{\mu\nu})
\end{equation}
è uno pseudoscalare di peso $-2$.  Infatti, la
relazione~\eqref{eq:trasf-tensore-metrico} può essere scritta in forma
matriciale come
\begin{equation}
  g'_{\mu\nu} = \Big( \parder{x}{x'}\Big)^{\textup{T}} g_{\mu\nu}
  \Big( \parder{x}{x'} \Big),
\end{equation}
da cui, passando ai determinanti,
\begin{equation}
  \det(g'_{\mu\nu}) = \det(g_{\mu\nu}) \abs*{\parder{x}{x'}}^{2} \implies g' = g
  \abs*{\parder{x}{x'}}^{2} = g\abs*{\parder{x'}{x}}^{-2}.
\end{equation}
In particolare $\sqrt{g}$ è uno pseudoscalare di peso $-1$
\begin{equation}
  \sqrt{g'} = \sqrt{g}\abs*{\parder{x'}{x}}^{-1}.
\end{equation}
Possiamo usare $\sqrt{g}$ per costruire un elemento infinitesimo di ipervolume
quadrimensionale invariante per arbitrarie trasformazione di coordinate.
Infatti osserviamo che mentre nella relatività speciale $\dd^{4} x$ è uno
scalare di Lorentz, nella relatività generale è uno pseudoscalare di peso $1$
\begin{equation}
  \dd^{4} x' = \dd^{4} x \abs*{\parder{x'}{x}}.
\end{equation}
Dunque l'elemento infinitesimo di ipersuperficie invariante è
$\sqrt{g}\dd^{4} x$
\begin{equation}
  \sqrt{g'}\dd^{4} x' = \sqrt{g}\abs*{\parder{x'}{x}}^{-1} \dd^{4} x
  \abs*{\parder{x'}{x}} = \sqrt{g}\dd^{4} x.
\end{equation}


Anche nella relatività generale il tensore metrico può essere usato per
innalzare a abbassare gli indici dei tensori, per esempio
\begin{gather}
  V^{\alpha} \to V_{\alpha} = g_{\alpha\beta} V^{\beta}, \\
  U_{\alpha} \to U^{\alpha} = g^{\alpha\beta} U_{\beta}.
\end{gather}

\section{Operazioni con i tensori}
\label{sec:operazioni-tensori}

In questo paragrafo riporteremo le operazioni, già analizzate nel
paragrafo~\ref{sec:operazioni-tensori-minkowski}, che si possono effettuare con
i tensori anche all'interno della relatività generale.  Ometteremo le
dimostrazioni delle proprietà che sono state già svolte per lo spazio di
Minkowski poiché è sufficiente effettuare le
sostituzioni~\eqref{eq:sostituzioni}.

\begin{description}
\item[Combinazione lineare] La combinazione lineare
  \begin{equation}
    a \tensor*{R}{^{\alpha_{1}\dots\alpha_{p}}_{\beta_{1}\dots\beta_{q}}} + b
    \tensor*{S}{^{\alpha_{1}\dots\alpha_{p}}_{\beta_{1}\dots\beta_{q}}},
  \end{equation}
  con $a$ e $b$ scalari, di due tensori dello stesso tipo è un tensore
  $\tensor*{T}{^{\alpha_{1}\dots\alpha_{p}}_{\beta_{1}\dots\beta_{q}}}$ ancora
  dello stesso tipo.
\item[Prodotto diretto] Il \index{prodotto!diretto}prodotto
  $\tensor*{A}{^{\alpha_{1}\dots\alpha_{p}}_{\beta_{1}\dots\beta_{q}}}
  \tensor*{B}{^{\alpha_{1}\dots\alpha_{r}}_{\beta_{1}\dots\beta_{s}}}$
  di un tensore di tipo $(p,q)$ e di uno di tipo $(r,s)$ è un tensore
  $\tensor*{T}{^{\alpha_{1}\dots\alpha_{p+r}}_{\beta_{1}\dots\beta_{q+s}}}$ di
  tipo $(p+r,q+s)$.
\item[Contrazione degli indici] \index{contrazione!degli indici}Se in un tensore
  di tipo $(p,q)$ si contraggono $n$ indici controvarianti con $n$ indici
  covarianti si ottiene un tensore di tipo $(p-n,q-n)$.  Si può effettuare la
  contrazione degli indici anche nel prodotto di due tensori: il prodotto di un
  tensore di tipo $(p,q)$ e di uno di tipo $(r,s)$ è un tensore di tipo
  $(p+r,q+s)$, ma se si contraggono $n$ indici si ottiene un tensore di tipo
  $(p+r-n,q+s-n)$.
\item[Prodotto scalare] Il prodotto scalare fra due vettori $V$ e $U$ è definito
  come
  \begin{equation}
    V\cdot U = V^{\alpha}U_{\alpha} = g_{\alpha\beta}V^{\alpha}U^{\beta}.
  \end{equation}
  Il prodotto scalare di due vettori è un caso particolare di contrazione degli
  indici nel prodotto fra due tensori.  Dunque il loro prodotto è un tensore di
  rango $0$, cioè uno scalare e quindi una quantità che non varia per effetto di
  una trasformazione delle coordinate.
\end{description}

Non abbiamo ricordato fra le operazioni la derivazione poiché la derivazione
ordinaria nella relatività generale non ha le stesse importanti proprietà che
possiede nella relatività speciale.  Approfondiremo questo discorso nel
paragrafo~\ref{sec:derivazione-covariante}.

\section{Connessione affine}
\label{sec:connessione-affine}

Abbiamo definito la connessione affine come
\begin{equation}
  \tensor{\Gamma}{^{\lambda}_{\mu\nu}}
  = \parder{x^{\lambda}}{\xi^{\alpha}} \parder{\xi^{\alpha}}{x^{\mu},x^{\nu}},
\end{equation}
in cui $x^{\alpha}$ è un generico sistema di coordinate e
$\xi^{\alpha} = \xi^{\alpha}(x^{\alpha})$ un sistema localmente inerziale.
Considerando un altro sistema di coordinate $x'^{\mu} = x'^{\mu}(\xi^{\alpha})$
abbiamo
\begin{equation}
  \begin{split}
    \tensor{{\Gamma{}'}}{^{\lambda}_{\mu\nu}}
    &= \parder{x'^{\lambda}}{\xi^{\alpha}} \parder{}{x'^{\mu}}
    \Big( \parder{\xi^{\alpha}(x)}{x'^{\nu}} \Big)
    = \parder{x'^{\lambda}}{x^{\sigma}} \parder{x^{\sigma}}{\xi^{\alpha}}
    \parder{}{x'^{\mu}} \Big( \parder{x^{\sigma}}{x'^{\nu}}
    \parder{\xi^{\alpha}}{x^{\sigma}} \Big) \\
    &=  \parder{x'^{\lambda}}{x^{\rho}} \parder{x^{\rho}}{\xi^{\alpha}}
    \Big( \parder{x^{\sigma}}{x'^{\nu}}
    \parder{\xi^{\alpha}}{x'^{\mu},x^{\sigma}} +
    \parder{x^{\sigma}}{x'^{\mu},x'^{\nu}} \parder{\xi^{\alpha}}{x^{\sigma}}
    \Big) \\
    &= \parder{x'^{\lambda}}{x^{\rho}} \parder{x^{\rho}}{\xi^{\alpha}}
    \Big( \parder{x^{\sigma}}{x'^{\nu}} \parder{x^{\tau}}{x'^{\mu}}
    \parder{\xi^{\alpha}}{x^{\tau},x^{\sigma}} +
    \parder{x^{\sigma}}{x'^{\mu},x'^{\nu}} \parder{\xi^{\alpha}}{x^{\sigma}}
    \Big) \\
    &= \parder{x'^{\lambda}}{x^{\rho}} \parder{x^{\tau}}{x'^{\mu}}
    \parder{x^{\sigma}}{x'^{\nu}} \tensor{\Gamma}{^{\rho}_{\tau\sigma}}
    + \parder{x'^{\lambda}}{x^{\rho}} \parder{x^{\rho}}{x'^{\mu},x'^{\nu}}.
  \end{split}
\end{equation}
Questa relazione mostra che la connessione affine non è un tensore: se non ci
fosse il secondo termine nell'ultimo membro avremmo la formula di trasformazione
di un tensore di tipo $(1,2)$.

Riscriviamo il secondo termine dell'ultimo membro dell'equazione precedente in
una forma diversa che ci sarà utile in seguito.  Derivando rispetto a $x'^{\mu}$
ambo i membri dell'identità
\begin{equation}
  \tensor{\delta}{^{\lambda}_{\nu}} = \parder{x'^{\lambda}}{x'^{\nu}}
  = \parder{x'^{\lambda}}{x^{\rho}} \parder{x^{\rho}}{x'^{\nu}}
\end{equation}
abbiamo
\begin{equation}
  0 = \parder{\tensor{\delta}{^{\lambda}_{\nu}}}{x'^{\mu}}
  = \parder{x'^{\lambda}}{x^{\rho}} \parder{x^{\rho}}{x'^{\mu}, x'^{\nu}}
  + \parder{x^{\rho}}{x'^{\nu}} \parder{x^{\sigma}}{x'^{\mu}}
  \parder{x'^{\lambda}}{x^{\rho},x^{\sigma}}.
\end{equation}
In questo modo la trasformazione della connessione affine può essere riscritta
come
\begin{equation}
  \tensor{{\Gamma{}'}}{^{\lambda}_{\mu\nu}} =
  -\parder{x'^{\lambda}}{x^{\rho}} \parder{x^{\rho}}{x'^{\mu}, x'^{\nu}}
  \tensor{\Gamma}{^{\rho}_{\tau\sigma}}
  + \parder{x'^{\lambda}}{x^{\rho}} \parder{x^{\rho}}{x'^{\mu},x'^{\nu}}.
\end{equation}

\section{Derivazione covariante}
\label{sec:derivazione-covariante}

Nella relatività ristretta, la derivazione ordinaria di un tensore di rango $k$
per una coordinata $x^{\lambda}$ produceva un tensore di rango $k+1$.
All'interno della teoria della relatività ristretta la situazione è differente:
la derivazione ordinaria di un tensore di rango $k$ per una coordinata
$x^{\lambda}$ produce un ente che non è, in generale, un tensore di rango $k+1$.
Per esempio, consideriamo un vettore controvariante $V^{\mu}$ il quale, per
definizione, si trasformerà, per effetto di un cambiamento di coordinate, come
\begin{equation}
  V^{\mu} \to V'^{\mu} = \parder{x'^{\mu}}{x^{\nu}} V^{\nu}.
\end{equation}
Derivando rispetto a $x'^{\lambda}$ abbiamo
\begin{equation}
  \begin{split}
    \parder{V'^{\mu}}{x'^{\lambda}}
    &= \parder{x'^{\mu}}{x^{\nu}} \parder{V^{\nu}}{x'^{\lambda}}
    + \parder{x'^{\mu}}{x^{\nu},x^{\nu}} V^{\nu} \\
    &= \parder{x'^{\mu}}{x^{\rho}} \parder{x^{\rho}}{x'^{\nu}}
    \parder{V^{\nu}}{x'^{\lambda}}
    + \parder{x'^{\mu}}{x^{\nu},x^{\rho}} \parder{x^{\rho}}{x'^{\lambda}}
    V^{\nu}.
  \end{split}
\end{equation}
Se nell'ultimo membro ci fosse solo il primo termine avremmo che
$\lparder{V'^{\mu}}{x'^{\lambda}}$ sarebbe un tensore di rango $2$, tuttavia il
secondo termine è, in generale, diverso da zero.  Osserviamo che per
trasformazioni delle coordinate $x^{\mu} \to x'^{\mu}$ lineari, come le
trasformazioni di Lorentz, il secondo termine si annulla.

Vediamo come si trasforma per effetto di un cambiamento delle coordinate la
quantità $\tensor{\Gamma}{^{\mu}_{\nu\kappa}} V^{\kappa}$
\begin{equation}
  \begin{split}
    \tensor{{\Gamma{}'}}{^{\mu}_{\lambda\kappa}} V'^{\kappa} &=
    \Big( \parder{x'^{\mu}}{x^{\nu}} \parder{x^{\sigma}}{x'^{\kappa}}
    \parder{x^{\rho}}{x'^{\lambda}} \tensor{\Gamma}{^{\nu}_{\rho\sigma}}
    -\parder{x^{\rho}}{x'^{\nu}} \parder{x^{\sigma}}{x'^{\kappa}}
    \parder{x'^{\mu}}{x^{\rho},x^{\sigma}} \Big) \parder{x'^{\kappa}}{x^{\eta}}
    V^{\eta} \\
    &= \parder{x'^{\mu}}{x^{\nu}}
    \underbrace{\parder{x^{\sigma}}{x^{\eta}}}_{\tensor{\delta}{^{\sigma}_{\eta}}}
    \parder{x^{\rho}}{x'^{\lambda}} \tensor{\Gamma}{^{\sigma}_{\rho\sigma}}
    V^{\eta} - \parder{x^{\rho}}{x'^{\nu}}
    \underbrace{\parder{x^{\sigma}}{x^{\eta}}}_{\tensor{\delta}{^{\sigma}_{\eta}}}
    \parder{x'^{\mu}}{x^{\rho},x^{\sigma}} V^{\eta} \\
    &= \parder{x'^{\mu}}{x^{\nu}} \parder{x^{\rho}}{x'^{\lambda}}
    \tensor{\Gamma}{^{\nu}_{\rho\sigma}} V^{\sigma}
    - \parder{x^{\rho}}{x'^{\nu}} \parder{x'^{\mu}}{x^{\rho},x^{\sigma}}
    V^{\sigma}.
  \end{split}
\end{equation}
Anche in questo caso, se nell'ultimo membro ci fosse solo il primo termine
$\tensor{\Gamma}{^{\mu}_{\lambda\kappa}} V^{\kappa}$ si trasformerebbe come un
tensore misto di rango $2$, tuttavia il termine addizionale è uguale e opposto
al termine addizionale che compare nella trasformazione di
$\lparder{V^{\mu}}{x^{\lambda}}$, quindi la somma delle due quantità si
trasforma come un tensore misto di rango $2$, infatti
\begin{equation}
  \parder{V'^{\mu}}{x'^{\lambda}} + \tensor{{\Gamma{}'}}{^{\mu}_{\lambda\kappa}}
  V'^{\kappa} = \parder{x'^{\mu}}{x^{\nu}} \parder{x^{\rho}}{x'^{\lambda}} \Big(
  \parder{V^{\nu}}{x^{\rho}} + \tensor{\Gamma}{^{\nu}_{\rho\sigma}}
  V'^{\sigma} \Big).
\end{equation}
Definiamo allora la \index{derivata!covariante}\emph{derivata covariante}
$\tensor{V}{^{\mu}_{;\lambda}}$ di un vettore controvariante $V^{\mu}$ come
\begin{equation}
  \tensor{V}{^{\mu}_{;\lambda}} = \tensor{V}{^{\mu}_{,\lambda}} +
  \tensor{\Gamma}{^{\mu}_{\kappa\lambda}} V^{\kappa}
\end{equation}
e per un cambiamento delle coordinate $x^{\mu} \to x'^{\mu}$ si ha
\begin{equation}
  \tensor{V}{^{\mu}_{;\lambda}} \to \tensor{{V'}}{^{\mu}_{;\lambda}}
  = \parder{x'^{\mu}}{x^{\nu}} \parder{x^{\rho}}{x'^{\lambda}}
  \tensor{V}{^{\nu}_{;\rho}}
\end{equation}
cioè la derivata covariante di un vettore è un tensore di rango $2$, in analogia
alla derivazione ordinaria di un vettore nello spazio di Minkowski.

Ripetendo gli stessi calcoli per un vettore covariante si giunge a definire la
\index{derivata covariante}derivata covariante $V_{\mu;\nu}$ di un vettore
covariante $V_{\mu}$ come
\begin{equation}
  V_{\mu;\nu} = V_{\mu,\nu} - \tensor{\Gamma}{^{\nu}_{\mu\nu}} V_{\nu}
\end{equation}
e per una trasformazione delle coordinate $x^{\mu} \to x'^{\mu}$ risulta
\begin{equation}
  V_{\mu;\nu} \to V'_{\mu;\nu}
  = \parder{x^{\lambda}}{x'^{\mu}} \parder{x^{\rho}}{x'^{\nu}} V_{\lambda;\rho}.
\end{equation}

Più in generale è possibile definire la \index{derivata covariante}derivata
covariante di un tensore.  Vediamo il caso particolare di un tensore misto di
rango $3$
\begin{subequations}
  \begin{align}
    \tensor{T}{^{\mu\sigma}_{\lambda;\rho}} &=
    \tensor{T}{^{\mu\sigma}_{\lambda,\rho}} + \tensor{\Gamma}{^{\mu}_{\rho\nu}}
    \tensor{T}{^{\nu\sigma}_{\lambda}} + \tensor{\Gamma}{^{\sigma}_{\rho\nu}}
    \tensor{T}{^{\mu\nu}_{\lambda}} - \tensor{\Gamma}{^{\nu}_{\rho\lambda}}
    \tensor{T}{^{\mu\sigma}_{\nu}}, \\
    \tensor{{T'}}{^{\mu\sigma}_{\lambda;\rho}}
    &= \parder{x'^{\mu}}{x^{\omega}} \parder{x'^{\sigma}}{x^{\tau}}
    \parder{x^{\alpha}}{x'^{\lambda}} \parder{x^{\beta}}{x'^{\rho}}
    \tensor{T}{^{\omega\tau}_{\alpha;\beta}}.
  \end{align}
\end{subequations}

Grazie alle proprietà algebriche dei tensori viste nel
paragrafo~\ref{sec:operazioni-tensori} si possono dimostrare le seguenti
proprietà della derivata covariante
\begin{enumerate}
\item la derivata covariante della combinazione lineare di tensori è uguale alla
  combinazione lineare delle derivate dei tensori
  \begin{equation}
    (\alpha \tensor*{A}{^{\mu_{1}\cdots\mu_{p}}_{\nu_{1}\cdots\nu_{q}}} + \beta
    \tensor*{B}{^{\mu_{1}\cdots\mu_{p}}_{\nu_{1}\cdots\nu_{q}}})_{;\lambda} =
    \alpha \tensor*{A}{^{\mu_{1}\cdots\mu_{p}}_{\nu_{1}\cdots\nu_{q};\lambda}} +
    \beta \tensor*{B}{^{\mu_{1}\cdots\mu_{p}}_{\nu_{1}\cdots\nu_{q};\lambda}},
  \end{equation}
  con $\alpha$ e $\beta$ scalari;
\item la derivata covariante del \index{prodotto!diretto}prodotto diretto di due
  tensori segue la \index{regola!di Leibniz}\emph{regola di Leibniz}
  \begin{equation}
    (\tensor*{A}{^{\mu_{1}\cdots\mu_{p}}_{\nu_{1}\cdots\nu_{q}}}
    \tensor*{B}{^{\mu_{1}\cdots\mu_{r}}_{\nu_{1}\cdots\nu_{s}}})_{;\lambda} =
    \tensor*{A}{^{\mu_{1}\cdots\mu_{p}}_{\nu_{1}\cdots\nu_{q};\lambda}}
    \tensor*{B}{^{\mu_{1}\cdots\mu_{r}}_{\nu_{1}\cdots\nu_{s}}} +
    \tensor*{A}{^{\mu_{1}\cdots\mu_{p}}_{\nu_{1}\cdots\nu_{q}}}
    \tensor*{B}{^{\mu_{1}\cdots\mu_{r}}_{\nu_{1}\cdots\nu_{s};\lambda}};
  \end{equation}
\item la derivata covariante della \index{contrazione!degli indici}contrazione
  di un tensore è uguale alla contrazione della derivata covariante del
  tensore.  Per esempio
  \begin{equation}
    \tensor{T}{^{\mu\lambda}_{\lambda;\rho}} =
    \tensor{T}{^{\mu\lambda}_{\lambda,\rho}} +
    \tensor{\Gamma}{^{\mu}_{\sigma\rho}}\tensor{T}{^{\sigma\lambda}_{\lambda}}.
  \end{equation}
  Infatti
  \begin{equation}
    \begin{split}
      \tensor{T}{^{\mu\lambda}_{\lambda;\rho}} &=
      \tensor{T}{^{\mu\lambda}_{\lambda,\rho}} +
      \tensor{\Gamma}{^{\mu}_{\sigma\rho}}\tensor{T}{^{\sigma\lambda}_{\lambda}}
      + \tensor{\Gamma}{^{\lambda}_{\sigma\rho}}
      \tensor{T}{^{\mu\sigma}_{\lambda}} -
      \tensor{\Gamma}{^{\sigma}_{\lambda\rho}}
      \tensor{T}{^{\mu\lambda}_{\sigma}} \\
      &= \tensor{T}{^{\mu\lambda}_{\lambda,\rho}} +
      \tensor{\Gamma}{^{\mu}_{\sigma\rho}}\tensor{T}{^{\sigma\lambda}_{\lambda}}
      + \tensor{\Gamma}{^{\sigma}_{\lambda\rho}}
      \tensor{T}{^{\mu\lambda}_{\sigma}} -
      \tensor{\Gamma}{^{\sigma}_{\lambda\rho}}
      \tensor{T}{^{\mu\lambda}_{\sigma}} \\
      &= \tensor{T}{^{\mu\lambda}_{\lambda,\rho}} +
      \tensor{\Gamma}{^{\mu}_{\sigma\rho}}\tensor{T}{^{\sigma\lambda}_{\lambda}}
    \end{split}
  \end{equation}
  in cui nel penultimo passaggio abbiamo scambiato gli indici muti $\lambda$ e
  $\sigma$ del terzo termine.
\end{enumerate}

La derivata covariante del tensore metrico $g_{\mu\nu}$ è nulla poiché in un
sistema di coordinate localmente inerziale il tensore metrico è uguale al
tensore di Minkowski $\eta_{\mu\nu}$ e la derivata covariante del tensore di
Minkowski è uguale al tensore nullo.  Poiché il tensore nullo non cambia per
trasformazioni arbitrarie delle coordinate, la derivata covariante del tensore
metrico è nulla in qualsiasi sistema di coordinate.  Formalmente questo può
essere fatto vedere calcolando direttamente la derivata covariante di
$g_{\mu\nu}$
\begin{equation}
  g_{\mu\nu;\lambda} = g_{\mu\nu,\lambda} -
  \tensor{\Gamma}{^{\rho}_{\lambda\mu}}g_{\rho\nu} -
  \tensor{\Gamma}{^{\rho}_{\lambda\nu}}g_{\rho\mu}
\end{equation}
e dalla~\eqref{eq:foo} abbiamo proprio $g_{\mu\nu;\lambda} = 0$.  Questo
risultato significa che l'operazione di abbassamento degli indici (o
innalzamento nel caso di $g^{\mu\nu}$) e derivazione covariante commutano, cioè
\begin{subequations}
  \begin{gather}
    (g_{\mu\nu}V^{\nu})_{;\lambda}=g_{\mu\nu}(\tensor{V}{^{\nu}_{;\lambda}}), \\
    (g^{\mu\nu}V_{\nu})_{;\lambda} = g^{\mu\nu}(V_{\nu;\lambda}).
  \end{gather}

\end{subequations}

%%% Local Variables:
%%% mode: latex
%%% TeX-master: "../astrofisica-teorica"
%%% fill-column: 80
%%% End:
