\cleardoublepage{}

\chapter{Calcolo tensoriale nella relatività generale}
\label{cha:calcolo-tensoriale}

\emph{Nota: da questo capitolo in poi useremo, per brevità, i termini
  \emph{tensore} e \emph{vettore} in luogo di, rispettivamente,
  \emph{quadritensore} e \emph{quadrivettore} se questo non dovesse generare
  ambiguità.}

Nel paragrafo~\ref{sec:calcolo-tensoriale-minkowski} abbiamo introdotto, nel
contesto della relatività speciale, i tensori, specificando come si modificano
per effetto di una trasformazione di Lorentz, la quale è una trasformazione
lineare fra sistemi di riferimento inerziali.  Nella definizione dei tensori
intervengono le componenti $\tensor{\Lambda}{^{\alpha}_{\beta}}$ e
$\tensor{\Lambda}{_{\alpha}^{\\beta}}$ delle matrici di Lorentz associate alle
specifiche trasformazioni considerate.

Anche nella relatività generale è utile introdurre i tensori, ma in questa
teoria si considerano trasformazioni arbitrarie, quindi non necessariamente
lineari, fra sistemi di riferimento anche non inerziali.  Abbiamo già notato
che, nella relatività speciale, gli elementi delle matrici delle trasformazioni
di Lorentz $x^{\alpha} \to x'^{\alpha}$ sono dati da
\begin{subequations}
  \begin{align}
    \tensor{\Lambda}{^{\alpha}_{\beta}} &= \parder{x'^{\alpha}}{x^{\beta}}, \\
    \tensor{\Lambda}{_{\beta}^{\alpha}} &= \parder{x^{\alpha}}{x'^{\beta}}.
  \end{align}
\end{subequations}
Definiremo i tensori, in relatività generale, in maniera analoga a quanto fatto
all'interno della relatività speciale, a patto di effettuare le sostituzioni
\begin{subequations}
  \label{eq:sostituzioni}
  \begin{align}
    \tensor{\Lambda}{^{\alpha}_{\beta}} &\to \parder{x'^{\alpha}}{x^{\beta}}, \\
    \tensor{\Lambda}{_{\beta}^{\alpha}} &\to \parder{x^{\alpha}}{x'^{\beta}}.
  \end{align}
\end{subequations}

Un \emph{tensore}
$\tensor*{A}{^{\alpha_{1}\dots\alpha_{p}}_{\beta_{1}\dots\beta_{q}}}$ (con
$p + q = k$) di rango $k$ è un insieme di $4^{k}$ quantità che sotto una
trasformazione arbitraria delle coordinate $x^{\alpha} \to x'^{\alpha}$ si
trasformano come
\begin{equation}
  \tensor*{A}{^{\alpha_{1}\dots\alpha_{p}}_{\beta_{1}\dots\beta_{q}}} \to
  \tensor*{{A'}}{^{\alpha_{1}\dots\alpha_{p}}_{\beta_{1}\dots\beta_{q}}} =
  \parder{x'^{\alpha_{1}}}{x^{\mu_{1}}}
  \cdots \parder{x'^{\alpha_{p}}}{x^{\mu_{p}}} \parder{x^{\nu_{1}}}{x'^{\beta_{1}}}
  \cdots \parder{x^{\nu_{q}}}{x'^{\beta_{q}}}
  \tensor*{A}{^{\mu_{1}\dots\mu_{p}}_{\nu_{1}\dots\nu_{q}}}.
\end{equation}
Il tensore $\tensor*{A}{^{\alpha_{1}\dots\alpha_{p}}_{\beta_{1}\dots\beta_{q}}}$
è un tensore misto $p$ volte controvariante e $q$ volte covariante, oppure di
tipo $(p,q)$.  Un tensore di rango $k$ completamente controvariante è del tipo
$\tensor{A}{^{\alpha_{1}\dots\alpha_{k}}}$ e si trasforma come
\begin{equation}
  \tensor{A}{^{\alpha_{1}\dots\alpha_{k}}} \to
  \tensor{{A'}}{^{\alpha_{1}\dots\alpha_{k}}} =
  \parder{x'^{\alpha_{1}}}{x^{\mu_{1}}}
  \cdots \parder{x'^{\alpha_{k}}}{x^{\mu_{k}}}
  \tensor{A}{^{\mu_{1}\dots\mu_{k}}}.
\end{equation}
Un quadritensore di rango $k$ completamente covariante è del tipo
$\tensor{A}{_{\alpha_{1}\dots\alpha_{k}}}$ e si trasforma come
\begin{equation}
  \tensor{A}{_{\alpha_{1}\dots\alpha_{k}}} \to
  \tensor{{A'}}{_{\alpha_{1}\dots\alpha_{k}}} =
  \parder{x^{\mu_{1}}}{x'^{\alpha_{1}}}
  \cdots \parder{x^{\mu_{k}}}{x'^{\alpha_{k}}}
  \tensor{A}{_{\mu_{1}\dots\mu_{k}}}.
\end{equation}
In particolare, i tensori di rango $0$, che quindi non hanno indici, sono
chiamati \emph{scalari} e sono invarianti sotto trasformazioni arbitrarie di
coordinate, i tensori di rango $1$ sono i \emph{vettori}.  Distinguiamo fra
vettori controvarianti $V^{\alpha}$, i quali si trasformano, per cambianti di
coordinate $x^{\alpha} \to x'^{\alpha}$, come
\begin{equation}
  V^{\alpha} \to V'^{\alpha} = \parder{x'^{\alpha}}{x^{\mu}}V^{\mu},
\end{equation}
e vettori covarianti $U_{\alpha}$, che si trasformano come
\begin{equation}
  U_{\alpha} \to U'_{\alpha} = \parder{x^{\mu}}{x'^{\alpha}} U_{\mu}.
\end{equation}
Le trasformazioni inverse per i vettori controvarianti e covarianti sono
\begin{gather}
  V'^{\alpha} \to V^{\alpha} = \parder{x^{\alpha}}{x'^{\mu}} V'^{\mu}, \\
  U'_{\alpha} \to U_{\alpha} = \parder{x'^{\mu}}{x^{\alpha}} U'_{\mu}.
\end{gather}

I differenziali delle coordinate controvarianti sono vettori controvarianti,
infatti
\begin{equation}
  \dd x'^{\alpha} = \parder{x'^{\alpha}}{x^{\mu}} \dd x^{\mu}.
\end{equation}

La delta di Kronecker definita nella~\eqref{eq:delta-kronecker} è l'unico
tensore le cui componenti sono uguali in qualsiasi sistema di riferimento, oltre
agli scalari e al tensore nullo.

In un generico sistema di coordinate $x^{\mu}$, il tensore metrico è dato per
definizione da
\begin{equation}
  g_{\mu\nu} =
  \eta_{\alpha\beta} \parder{\xi^{\alpha}}{x^{\mu}} \parder{\xi^{\beta}}{x^{\nu}},
\end{equation}
in cui $\xi^{\alpha}$ è un sistema di coordinate localmente inerziale.  In un
altro sistema di coordinate $x'^{\mu}$ risulta
\begin{equation}
  \begin{split}
    g'_{\mu\nu} &=
    \eta_{\alpha\beta} \parder{\xi^{\alpha}}{x'^{\mu}}
    \parder{\xi^{\beta}}{x'^{\nu}} =
    \eta_{\alpha\beta} \parder{\xi^{\alpha}}{x^{\rho}}
    \parder{x^{\rho}}{x'^{\mu}} \parder{\xi^{\beta}}{x^{\sigma}}
    \parder{x^{\sigma}}{x'^{\nu}} \\
    &= g_{\rho\sigma} \parder{x^{\rho}}{x'^{\mu}} \parder{x^{\sigma}}{x'^{\nu}}.
  \end{split}
\end{equation}
Osserviamo che, a differenza del caso dello spazio di Minkowski piatto, nello
spazio-tempo curvo della relatività generale il tensore metrico $g_{\mu\nu}$ non
è invariante per trasformazioni delle coordinate.  Definiamo il tensore metrico
controvariante $g^{\mu\nu}$ come quel tensore tale che
\begin{equation}
  g_{\mu\rho}g^{\mu\sigma} = \tensor{\delta}{^{\sigma}_{\rho}}.
\end{equation}
Si dimostra che $g^{\mu\nu}$ è effettivamente un tensore controvariante di rango
$2$.

Anche nella relatività generale il tensore metrico può essere usato per
innalzare a abbassare gli indici dei tensori, per esempio
\begin{gather}
  V^{\alpha} \to V_{\alpha} = g_{\alpha\beta} V^{\beta}, \\
  U_{\alpha} \to U^{\alpha} = g^{\alpha\beta} U_{\beta}.
\end{gather}

% TODO: rivedere tutta la sezione, ho (quasi) solo copia-incollato il
% corrispondente paragrafo dello spazio di Minkowski.
\section{\completare{Operazioni con i tensori}}
\label{sec:operazioni-tensori}

In questo paragrafo riporteremo le operazioni che si possono effettuare con i
tensori all'interno della relatività generale.  Nei casi in cui è sufficiente
solo effettuare le sostituzioni~\eqref{eq:sostituzioni}, ometteremo le
dimostrazioni delle proprietà che sono state già svolte nel
paragrafo~\ref{sec:operazioni-tensori-minkowski} per lo spazio di Minkowski.

\subsubsection{Combinazione lineare}
\label{sec:combinazione-lineare}

La combinazione lineare
\begin{equation}
  \tensor*{T}{^{\alpha_{1}\dots\alpha_{p}}_{\beta_{1}\dots\beta_{q}}} = a
  \tensor*{R}{^{\alpha_{1}\dots\alpha_{p}}_{\beta_{1}\dots\beta_{q}}} + b
  \tensor*{S}{^{\alpha_{1}\dots\alpha_{p}}_{\beta_{1}\dots\beta_{q}}},
\end{equation}
con $a$ e $b$ scalari, di due tensori dello stesso tipo è un tensore ancora
dello stesso tipo.

\subsubsection{Prodotto diretto}
\label{sec:prodotto-diretto}

Il prodotto
$\tensor*{A}{^{\alpha_{1}\dots\alpha_{p}}_{\beta_{1}\dots\beta_{q}}}
\tensor*{B}{^{\alpha_{1}\dots\alpha_{r}}_{\beta_{1}\dots\beta_{s}}}$
di un tensore di tipo $(p,q)$ e di uno di tipo $(r,s)$ è un tensore
$\tensor*{T}{^{\alpha_{1}\dots\alpha_{p+r}}_{\beta_{1}\dots\beta_{q+s}}}$ di
tipo $(p+r,q+s)$.

\subsubsection{Contrazione degli indici}
\label{sec:contrazione-indice}

Se in un tensore di tipo $(p,q)$ $n$ indici controvarianti sono uguali a $n$
indici covarianti e si somma su tali indici si ottiene un tensore di tipo
$(p-n,q-n)$.  Usando la convenzione di Einstein sulla somma si omette il simbolo
di sommatoria.  Questa operazione viene detta di contrazione degli indici.  Per
esempio, il tensore $\tensor{T}{^{\alpha}_{\beta}^{\gamma\delta}}$ è un tensore
di tipo $(3,1)$, ma se si contraggono l'indice covariante $\beta$ con il terzo
indice controvariante $\delta$ si ottiene il tensore
$\tensor{T}{^{\alpha}_{\beta}^{\gamma\beta}} =
\tensor{T}{^{\alpha}_{0}^{\gamma 0}} + \tensor{T}{^{\alpha}_{1}^{\gamma 1}} +
\tensor{T}{^{\alpha}_{2}^{\gamma 2}} + \tensor{T}{^{\alpha}_{3}^{\gamma 3}}$
che è completamente controvariante di rango $2$.

Si può effettuare la contrazione degli indici anche nel prodotto di due tensori:
il prodotto di un tensore di tipo $(p,q)$ e di uno di tipo $(r,s)$ è un tensore
di tipo $(p+r,q+s)$, ma se si contraggono $n$ indici si ottiene un tensore di
tipo $(p+r-n,q+s-n)$.  Per esempio, il tensore
$\tensor{A}{^{\alpha}_{\beta}^{\gamma\delta}}$ è di tipo $(3,1)$,
$\tensor{B}{_{\epsilon\zeta}}$ è di tipo $(0,2)$, il prodotto
$\tensor{A}{^{\alpha}_{\beta}^{\gamma\delta}} \tensor{B}{_{\epsilon\zeta}}$ è un
tensore di tipo $(3,3)$, ma il prodotto
$\tensor{A}{^{\alpha}_{\beta}^{\gamma\delta}} \tensor{B}{_{\alpha\delta}}$ con
due indici contratti è un tensore di tipo $(1,1)$.

\subsubsection{Prodotto scalare di due vettori}
\label{sec:prodotto-scalare-vettori-minkowski}

Il \emph{prodotto scalare} fra due vettori $V$ e $U$ è definito come
\begin{equation}
  \begin{split}
    V\cdot U &= V^{\alpha}U_{\alpha} =
    \tensor{\eta}{_{\alpha\beta}}V^{\alpha}U^{\beta} = V^{0}U_{0} + V^{1}U_{1} +
    V^{2}U_{2} + V^{3}U_{3} \\
    &= -V^{0}U^{0} + V^{1}U^{1} + V^{2}U^{2} + V^{3}U^{3} = -V^{0}U^{0} +
    \bm{V}\cdot\bm{U}.
  \end{split}
\end{equation}

Il prodotto scalare di due vettori è un caso particolare di contrazione degli
indici nel prodotto fra due tensori.  Dunque il loro prodotto è un tensore di
rango $0$, cioè uno scalare e quindi una quantità che non varia per effetto di
una trasformazione delle coordinate.

Si può definire anche il modulo quadro di un vettore $V$ come il prodotto
scalare di $V$ per se stesso, vale a dire
\begin{equation}
  \begin{split}
    V\cdot V &= V^{\alpha}V_{\alpha} =
    \tensor{\eta}{_{\alpha\beta}}V^{\alpha}V^{\beta} = V^{0}V_{0} + V^{1}V_{1} +
    V^{2}V_{2} + V^{3}V_{3} \\
    &= -(V^{0})^{2} + (V^{1})^{2} + (V^{2})^{2} + (V^{3})^{2} = -(V^{0})^{2} +
    \bm{V}^{2}.
  \end{split}
\end{equation}
Anche il modulo quadro di un vettore è uno scalare.

%%% Local Variables:
%%% mode: latex
%%% TeX-master: "../astrofisica-teorica"
%%% fill-column: 80
%%% End:
