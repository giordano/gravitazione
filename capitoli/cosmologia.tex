\chapter{\completare{Cosmologia}}
\label{cha:cosmologia}

\section{Equazioni di Friedmann}

\begin{itemize}
\item La geometria dello spazio tempo sia descritta dall metrica R-W:
  \begin{subequations}
    \begin{align}
      \label{rw1}
      g_{tt} &= -1, \\
      g_{it} &= 0, \\
      g_{ij} &= R^2(t) ~ \tilde{g}_{ij}(x^i)
    \end{align}
  \end{subequations}
  dove $t$ è il tempo cosmico, $x^{i} = (r,\theta,\phi)$ sono coordinate
  coomoving e $\tilde{g}_{ij}(x^i)$ è la metrica dello spazio 3-D omogeneo ed
  isotropo
  \begin{subequations}
    \begin{align}
      \label{rw2}
      \tilde{g}_{rr} &= (1-kr^2)^{-1}, \\
      \tilde{g}_{\theta \theta} & = r^2, \\
      \tilde{g}_{\phi \phi} &= r^2 \sin^2 \phi, \\
      \tilde{g}_{ij} &= 0 \qquad \text{per $i \ne j$}.
    \end{align}
  \end{subequations}
\item Universo è omogeneo ed isotropo: $\implies$ tensore energia impulso per
  l'Universo ha la forma (fluido perfetto)
  \begin{equation}
    T_{\mu\nu} = p g_{\mu\nu} + (p+\rho) U_{\mu} U_{\nu}
    \label{fp}
  \end{equation}
  con pressione $p(t)$ e densità di energia $\rho(t)$ funzioni del tempo e
  quadrivelocità dei costituenti (oggi le galassie) dell'Universo
  $U^{\mu}=(1,\bm{0})$.
\item La dinamica dell'Universo determinata dalle equazioni di Einstein
  \begin{subequations}
    \begin{align}
      \label{EE}
      R_{\mu\nu} &= -8 \pi G S_{\mu \nu}, \\
      S_{\mu \nu} &= T_{\mu \nu} - (1/2) g_{\mu \nu}
      \tensor{T}{^{\lambda}_{\lambda}}
    \end{align}
  \end{subequations}
\end{itemize}
Il tensore di Ricci diventa (il punto indica derivazione rispetto al tempo):
\begin{subequations}
  \begin{align}
    R_{tt} &= \frac{ 3 \ddot{R}}{R} \\
    R_{ti} &= 0 \\
    R_{ij} &= -(R\ddot{R}+2\dot{R}^2+2k) \tilde{g}_{ij}.
  \end{align}
\end{subequations}
Osserva:
\begin{subequations}
  \begin{align}
    \tensor{T}{^{\lambda}}{_{\lambda}} &= g^{\lambda\nu} T_{\lambda\nu} = p
    \tensor{\delta}{^{\lambda}}_{{\lambda}} + (p+\rho) (-1) = 3p - \rho \\
    S_{\mu \nu} &= (1/2) (\rho-p) g_{\mu\nu}+(p+\rho)U_{\mu} U_{\nu}
  \end{align}
\end{subequations}
In particolare
\begin{subequations}
  \begin{align}
    S_{tt} &= (1/2) (\rho-p)(-1)+(p+\rho)(+1) = (1/2)(\rho+3p) \\
    \label{sij}
    S_{ij} &= (1/2)(\rho-p) g_{ij}= (1/2)(\rho-p) R^2(t) \tilde{g}_{ij}
  \end{align}
\end{subequations}

Dalle equazioni di Einstein per la componente $tt$:
\begin{equation}
  \frac{3 \ddot{R}}{R} = - 4\pi G (\rho+3p)
  \label{e1}
\end{equation}
e per la componente $ij$:
\begin{equation}
  \label{e2}
  \frac{\ddot{R}}{R} +\frac{2\dot{R}^2}{R^2}+\frac{2k}{R^2}= 4 \pi G (\rho-p)
\end{equation}
In aggiunta, si ha l'equazione di conservazione
$\tensor{T}{^{\mu \nu}}{_{;\mu}} = 0$, e per $\nu =0$
\begin{equation}
  \dot{p} R^3 = \toder{}{t}[R^3(\rho+p)]
  \label{moto1}
\end{equation}
o, equivalentemente, (poiché $\ltoder{}{t} = \dot{R} \ltoder{}{R}$)
\begin{equation}
  \toder{(\rho R^3)}{R} = -3pR^2
  \label{moto2}
\end{equation}
{\bf Osserva} che dalle ultime 2 equazioni segue
\begin{subequations}
  \begin{gather}
    \text{se $p=0$ (era materia)} \implies \rho \propto R^{-3} \\
    \text{se $p=\rho/3$ (era radiazione)} \implies \rho \propto R^{-4} .
  \end{gather}
\end{subequations}
Il set di Le equazioni \eqref{e1}, \eqref{e2} e \eqref{moto1} non sono
indipendenti.  Eliminando $\ddot{R}$ si ottine un'equazione differenziale del
primo ordine
\begin{equation}
  \dot{R}^2 +k = \frac{8 \pi G}{3} \rho R^2
  \label{fri}
\end{equation}
Le equazioni \eqref{fri} e \eqref{moto1} insieme all'equazione di stato
\begin{equation}
  p=p(\rho)
  \label{stato}
\end{equation}
permettono di determinare le 3 funzioni incognite $R(t)$, $p(t)$ e $\rho(t)$.
Si otterranno 3 differenti soluzioni a seconda del valore di $k=-1,0,+1$.  I
modelli cosmologici basati sulla metrica R-W, sulle equazioni di Einstein sono
noti come modelli di Friedmann.

{\bf Osservazione}: senza conoscere esplicitamente l'equazione di stato è
possibile ricavare importanti informazioni sull'età $t_0$ dell'Universo e
sull'andamento del fattore di scala cosmico $R(t)$.  Infatti:
\begin{itemize}
\item $R(t)$ definito positivo
\item l'eq. \eqref{e1} mostra che $ \ddot{R} \propto - (\rho+3p) <0$ ad ogni
  $t$.  Quindi $R(t)$ ha concavità rivolta verso il basso per ogni tempo cosmico
  poiché per materia e radiazione $\rho(t)>0$ e $p(t)>0$
\item all'istante attuale l'Universo è in espansione $\dot{R}(t_0) >0$ ed
  inoltre $p_0=0$ (era materia).
\end{itemize}
Allora per $t>t_0$ ($t<t_0$), il termine a destra nell'eq. \ref{fri} va a zero
(cresce) almeno come $1/R(t)$.

Da queste considerazioni segue:
\begin{itemize}
\item età dell'Universo
  \begin{equation}
    t_0 < H_0^{-1}  = \frac {R(t_0)}{\dot{R}(t_0)} =  \num{13e9}
    \left(\frac{\SI[per-mode=reciprocal]{75}{\kilo\metre\per\second\per\mega\parsec}}{H_0}\right)
    \si{yr}.
  \end{equation}
\end{itemize}
L'uguaglianza si avrebbe nel caso $\ddot{R} =0$ (per cui $R(t)=t \dot{R}$).
Comportamento di $R(t)$ per $t>t_0$:
\begin{itemize}
\item se $k=-1$, per $t \to \infty$, $\dot{R} \to 1$, e quindi $R(t) \propto t$;
\item se $k=0$, per $t \to \infty$, $\dot{R} \propto R^{-1/2}$, e quindi
  $R(t) \propto t^{2/3}$;
\item se $k=-1$, $\dot{R}$ decresce dal valore attuale (positivo), ed esiste un
  istante $\tilde{t}$ a cui si annulla; quindi pet $t>\tilde{t}$ ${R}$ decresce
  con il tempo fino ad un nuova singolarità poiché $\ddot {R} <0$ sempre.
\end{itemize}

\section{Densità e pressione dell'Universo presente}

All'istante attuale densità e pressione sono date dalle equazioni
\begin{subequations}
  \begin{align}
    \rho_0  &= \frac{3}{8 \pi G} \left(\frac{k}{R_0^2} + H_0^2 \right)
    \label{rho0} \\
    p_0 &= -\frac{1}{8 \pi G} \left[\frac {k}{R_0^2}+H_0^2(1-2q_0)\right]
    \label{p0}
  \end{align}
\end{subequations}
Posto
\begin{equation}
  \rho_c = \frac{3 H_0^2}{8 \pi G} = \num{1.1e29}
  \left(
    \frac{H_0}{\SI[per-mode=reciprocal]{75}{\kilo\metre \per \second \per
        \mega\parsec}}
  \right)^2
  \si{\gram\per\centi\metre\cubed}
  \label{rhoc}
\end{equation}
l'equazione \eqref{rho0} diventa
\begin{equation}
  \rho_0-\rho_c =  \frac{3 k}{8 \pi G R_0^2}
  \label{temp1}
\end{equation}
pertanto l'Universo ha curvatura spaziale positiva o negativa a seconda che
$\rho_0 > \rho_c$ oppure $\rho_0 < \rho_c$.  Possiamo inoltre mettere in
relazione la densità $\rho_0$ con il parametro $q_0$.  Infatti per l'Universo
presente (dominato dalla materia) $p_0=0$, dall'eq. \eqref{p0} si ha
$k/R_0^2 = H_0^2 (2q_0-1)$, e sostituendo nella eq. \eqref{temp1}
\begin{equation}
  \frac {\rho_0} {\rho_c}=2 q_0
  \label{rho0surhoc}
\end{equation}
pertanto
\begin{itemize}
\item se $q_0 > 1/2 \implies \rho_0>\rho_c$ e $k=+1$
\item se $q_0 < 1/2 \implies \rho_0<\rho_c$ e $k=-1$.
\end{itemize}

\section{Era dominata dalla materia}

All'istante attuale $\rho_m(t_0) = \SI{e-34}{\gram\per\centi\metre\cubed}$,
$\rho_r(t_0)= \SI{e-34}{\gram\per\centi\metre\cubed}$; inoltre ad ogni istante
$\rho_m \propto R^{-3}$ e $\rho_r \propto R^{-4}$. Allora ($1+z=R_0/R$)
\begin{equation}
  \frac {\rho_m(t)} {\rho_r(t)} = 10^{3} \frac {R(t)} {R_0}= 10^3 \frac{1}{1+z}
\end{equation}
da cui risulta che l'Universo è dominato dalla materia ($\rho_m(t) > \rho_r(t)$)
dall'istante attuale ($z=0$) indietro nel tempo fino a red-shift $z = 1000$ (le
galassie più distanti e i QSOs osservati hanno $z<3$).  Nelle equazioni di
Friedmann porremo $p(t)=0$ ed assumeremo $\rho = \rho_0 [R_0/R(t)]^{3}$.
Riprendiamo in esame l'equazione \eqref{fri} in cui è conveniente esprimere
$\rho_0$ e $k/R^2$ in termini di $q_0$ ed $H_0$.  Dall'eq. \eqref{p0} segue
\begin{equation}
  \frac{k}{R_0^2}=(2q_0-1) H_0^2
\end{equation}
sostituendo questa espressione nell'eq. \eqref{rho0} si ha
\begin{equation}
  \frac{8\pi G}{3} \rho_0 = 2q_0 H_0^2
\end{equation}
Con le posizioni preceenti l'eq. \eqref{fri} di Friedmann diventa:
\begin{equation}
  \dot { R } ^2 + (2 q_0-1) H_0^2 R_0^2 = \frac{8\pi G}{3} \rho_0 \frac{R_0^3}{R}
\end{equation}
o equivalentemente
\begin{equation}
  \left(\frac{\dot R}{R_0}\right)^2 = H_0^2 \left(1-2q_0+2 q_0
    \frac{R_0}{R}\right)
\end{equation}
Posto $x=R(t)/R_0$, l'eq. precedente diventa
\begin{equation}
  \frac{dx}{dt}= H_0 \left(1-2q_0+ \frac {2q_0}{x}\right)^{1/2}
\end{equation}
la cui soluzione formale è
\begin{equation}
  t= H_0^{-1} \int_0^{R/R_0} dx \left(1-2q_0+\frac{2q_0}{x}\right)^{-1/2}
  \label{efa}
\end{equation}
Si osservi che abbiamo definito come $t=0$ l'inizio dell'era della materia.  In
effetti l'era della materia ininiza a $t \simeq \SI{e5}{yr}$.  L'età
dell'Universo è ottenuta dalla relazione precedente integrando sino ad $x=1$
\begin{equation}
  t_0= H_0^{-1} \int_0^{1} dx \left(1-2q_0+\frac{2q_0}{x}\right)^{-1/2}
\end{equation}
e quindi risulta sempre $t_0< H_0^{-1}$.  L'eq. \eqref{efa} può essere risolta
analiticamente nei tre casi $k=+1,0,-1$.

\subsubsection{Caso A $q_0 > 1/2$, $k=+1$, $\rho_0> \rho_c$}

Definiamo l'angolo $\theta$ dalla relazione:
\begin{equation}
  1-\cos \theta = \frac{2q_0-1 }{q_0} x
  \label{posizioneiniziale}
\end{equation}
da cui segue
\begin{equation}
  \dd x = \frac {q_0}{2q_0-1} \sin \theta \dd \theta =  \frac{- q_0}{2q_0-1} \dd \cos \theta
\end{equation}
L'integrale in eq. (\ref{efa}) diventa:
\begin{equation}
  \begin{split}
    t H_0 &= \int _1^{\cos \theta} \left[1-2q_0+2q_0 \left(\frac{2q_0-1}{q_0}
      \right) \left(\frac{1}{1-\cos \theta} \right) \right]^{-1/2} \dd\cos\theta
    \left(\frac{-q_0}{2q_0-1} \right) \\
    &=\frac{-q_0}{2q_0-1} \int _1^{\cos \theta} \left[
      \frac{1-\cos \theta - 2q_0 + 2q_0 \cos \theta + 4q_0 -2} {1-\cos\theta}
    \right]^{-1/2} \dd\cos\theta \\
    &=\frac {-q_0}{2q_0-1} \int _1^{\cos \theta} \left[ \frac {1-\cos \theta}
      {\cos \theta (2q_0-1) + (2q_0-1)} \right]^{1/2} \dd\cos\theta \\
    &=\frac {-q_0}{(2q_0-1)^{3/2}} \int_1^{\cos \theta} \sqrt \frac
    {1-\cos \theta} {1+\cos \theta} \dd \cos \theta
  \end{split}
\end{equation}
Nota che
\begin{equation}
  \int \sqrt \frac {1-y}{1+y} \dd y = \sqrt {1-y} + \arcsin y
\end{equation}
allora abbiamo
\begin{equation}
  t H_0 = \frac {-q_0}{(2q_0-1)^{3/2}}
  \left[ \sqrt {1-\cos\theta^2} + \arcsin \cos \theta - \arcsin 1 \right]
\end{equation}
Nota che $\cos \theta = \sin (\pi/2-\theta)$ ed inoltre
$\arcsin \sin \theta = \theta$. Allora
\begin{equation}
  t H_0= \frac {-q_0}{(2q_0-1)^{3/2}} \left[ \sin \theta + {\pi}/{2} - \theta -
    {\pi}/{2}\right]
\end{equation}
da cui
\begin{equation}
  t = \frac {-q_0}{H_0(2q_0-1)^{3/2}} ( \theta - \sin \theta )
  \label{sema}
\end{equation}
Questa relazione insieme alla posizione iniziale in eq. \eqref{posizioneiniziale}
(con $x=R(t)/R_0$)
\begin{equation}
  R(t)= \frac {R_0 q_0} {2q_0-1} (1-\cos \theta)
\end{equation}
mostra che $R(t)$ è una cicloide: $R(t)$ aumenta dal valore zero a $\theta=0$
($t=0$) fino al valore massimo $R_{max}= {2 q_0 R_0 }/({2q_0-1})$ a
$t_{max}= \pi q_0/(2q_0-1)^{3/2}$ e dopo ritorna a zero per $\theta =2 \pi$ e
$t=2t_{max}$.  L'istante attuale è ottenuto ponendo $R(t)=R_0$ ($x=1$)
\begin{subequations}
  \begin{align}
    \cos\theta_0 &= \frac {1} {q_0} -1 \\
    t_0 &= \frac {q_0}{H_0 (2q_0-1)^{3/2}} \left(\arccos(\frac {1}{q_0}-1) -
      \frac {1}{q_0}(2q_0-1)^{1/2} \right)
  \end{align}
\end{subequations}
Nota che
$\sin \theta_0= \sqrt{1-\cos^2\theta_0}= \sqrt{1-(q_0^{-1}-1)^2}=
{\sqrt{2q_0-1}}/{q_0}$
Se poniamo $q_0=1$ ($\cos \theta_0=0$), allora
$t_0 = H_0^{-1} (\pi /2-1) = \SI{7.5e5}{yr}$, $t_{max} = \SI{40e9}{yr}$,
l'intero ciclo si compierebbe in \SI{80e9}{yr}.  Qui abbiamo usato i valori
$H_0^{-1} = \SI{13e9}{yr}$,
$H_0 = \SI{75}{\kilo\metre \per \second \per \mega\parsec}$.

\subsubsection{Caso B: $q_0 = 1/2$, $k=0$, $\rho_0=  \rho_c$}

Questo è il caso Einstein-de Sitter.  L'eq. \eqref{efa} diventa
\begin{equation}
  t = H_0^{-1} \int_0^{R/R_0} \sqrt x dx = H_0^{-1} \frac{2} {3} x^{3/2}
\end{equation}
da cui segue
\begin{equation}
  \frac{R(t)}{R_0} = \left( \frac {3H_0 t}{2} \right)^{2/3}
\end{equation}
Allora $R(t)$ aumenta senza limiti; l'istante attuale $t_0=2H_0^{-1}/3$ risulta
pari a $t_0= \SI{9e9}{yr}$
($H_0= \SI{75}{\kilo\metre \per \second \per \mega\parsec}$.)

\subsubsection{Caso C: $0< q_0 < 1/2$, $k=-1$, $\rho_0 <  \rho_c$ }

Con la posizione $\theta = i \phi$ si ottengono risultati analoghi al caso A.
In particolare:
\begin{gather}
  t H_0= \frac {q_0}{(-2q_0+1)^{3/2}} \left( \sinh \phi \ - \phi \right)
  \label{topen} \\
  \cosh \phi -1 = \frac {1-2q_0}{q_0} \frac {R(t)}{R_0}
  \label{cosht}
\end{gather}
Per $t \to \infty$ si ha
$R(t)/R_0 \simeq {q_0}/{(1-2q_0)} {\exp{\phi}} \simeq \sqrt{1-2q_0} H_0 t$, da
cui si vede che $R(t) $ cresce senza limiti come $t$ stesso.  L'istante attuale
è definito ponendo $R(t_0)= R_0$ nell'eq. \eqref{cosht}.  Si ha
$\cosh \phi_0= q_0^{-1}-1$, e sostituendo nella eq. \eqref{topen}
($ \sinh^2 \phi - \cosh^2 \phi =1$) si ha:
\begin{equation}
  \begin{split}
    t_0 &= \frac {q_0}{H_0 (1-2q_0)^{3/2}}
    \left[- \cosh^{-1} (q_0^{-1}-1) + \sqrt{1+(q_0^{-1}-1)} \right] \\
    &= H_0^{-1} \left[ (1-2q_0)^{-1} - q_0 (1-2q_0)^{-3/2} \cosh^{-1}
      (q_0^{-1}-1) \right]
  \end{split}
\end{equation}
Ad esempio se prendiamo $q_0 = 0.014$ ($\rho_0=\rho^*_{~Gal}$) si ha
$t_0 \simeq 0.96~ H_0^{-1} \simeq \SI{13e9}{yr}$.  Nota che dallo studio delle
abbondanze degli isotopi degli elementi radioattivi segue che l'età della Terra
$\simeq \SI{4.5e9}{yr}$, e l'età della Galassia $\simeq \SI{7e9}{yr}$.  Comunque
gli oggetti più vecchi nell'Universo sono gli ammassi globulari la cui età
risulta $\simeq \SI{14e9}{yr}$.

\section{Universo di Friedmann con Costante Cosmologica}

Osserviamo che dall'eq. \eqref{moto1} con $p=-\rho$ segue
$\rho(t)=\text{costante}$, cioè si ha un Universo stazionario.  Prima della
scoperta dell'espansione cosmologica (Hubble 1929), Einstein modifica le
equazioni di campo introducendo un termine costante
\begin{equation}
  R_{\mu \nu} - \frac{1}{2} g_{\mu \nu} R - \Lambda g_{\mu \nu} =
  - 8 \pi G T_{\mu \nu}
  \label{EECC1}
\end{equation}
Possiamo scrivere l'eq. \eqref{EECC1} in un'altra forma.  Moltiplicando per
$g^{\mu \nu}$ segue: $R - (1/2) 4 R - 4\Lambda = -8 \pi G T^{\mu}_{mu}$, cioè
$R=8 \pi G \tensor{T}{^{\mu}}{_{mu}} -4 \Lambda$.  Sostituendo e portando a
destra il termine $\Lambda g_{\mu \nu}$
\begin{equation}
  R_{\mu \nu} = - 8 \pi G
  \left(
    T_{\mu \nu} - \frac {1}{2} g_{\mu \nu} T + \rho_{\Lambda}
  \right)
  \label{EECC2}
\end{equation}
con
\begin{equation}
  \rho_{\Lambda} = \frac {\Lambda c^2} {8 \pi G}
\end{equation}
Ho quindi un termine di sorgente $S_{\mu \nu}$ che nel caso $\mu=\nu=0$ diventa:
\begin{equation}
  S_{00} = \frac {1}{2} (\rho+3p) - \rho_{\Lambda}
\end{equation}
Uguagliando tale termine alla componente $tt$ del tensore di Ricci
($R_{tt}= 3 \ddot{R}/R$), in sostituzione dell'eq. \eqref{e1}, si ha ora:
\begin{equation}
  \frac{3 \ddot{R}}{R} = - 8\pi G \left[ \frac{1}{2}  (\rho+3p) - \rho_{\Lambda}
  \right]
  \label{ecc1}
\end{equation}
e con la sostituzione $\rho_c=3H_0^2/(8\pi G)$
\begin{equation}
  \frac {\ddot{R}} {R} = - \frac {H_0^2} {\rho_c}
  \left[
    \frac {1}{2}  (\rho+3p) - \rho_{\Lambda}
  \right]
\end{equation}
ed infine
\begin{equation}
  \label{ddRcc}
  \frac {\ddot{R}} {R} = H_0^2
  \left[
    \Omega_{\Lambda} -\frac {\Omega_m (1+z)^3} {2 }  - \Omega_r (1+z)^4
  \right]
\end{equation}
con
\begin{subequations}
  \begin{align}
    \Omega_{\Lambda} & = \frac{\rho_{\Lambda}}{\rho_c}, \\
    \Omega_m         & = \frac{\rho_m(0)}{\rho_c}, \\
    \Omega_r         & = \frac{\rho_r(0)}{\rho_c}
  \end{align}
\end{subequations}
In maniera analoga il termine $S_{ij}$ diventa
\begin{equation}
  S_{ij}= \left[ \frac {1}{2}(\rho-p)  + \rho_{\Lambda} \right] R^2(t) \tilde{g}_{ij}
  \label{sijcc}
\end{equation}
Eq. \eqref{e2}  è ora sostituita da
\begin{equation}
  \frac {\ddot{R}}{R}+\frac{2\dot{R}^2}{R^2}+\frac{2k}{R^2}=
  8 \pi G \left[ \frac{1}{2}(\rho-p) +\rho_{\Lambda} \right]
\end{equation}
Eliminando $\ddot R$ con l'uso dell'eq. (\ref{ecc1})
\begin{equation}
  \frac  {\dot{R}^2} {R^2} +\frac {k}{R^2} = \frac {8 \pi G}{3} (\rho+\rho_{\Lambda})
\end{equation}
da cui segue:
\begin{equation}
  H(z)=H_0 \left[\Omega_{\Lambda}+\Omega_{k}(1+z)^2+\Omega_m (1+z)^3+\Omega_r (1+z)^4\right]^{1/2}
  \label{hzcc}
\end{equation}
con
\begin{equation}
  \Omega_k = \frac {k}{R_0^2 H_0^2}
\end{equation}
Osserva che se $k=0$ allora (per $t=t_0$ , $z=0$)
\begin{equation}
  \Omega_{\Lambda} + \Omega_m  + \Omega_r = 1
\end{equation}
Inoltre per il parametro di decelerazione si ha:
\begin{equation}
  q_0= -\frac {\ddot R_0 R_0}{\dot R_0^2} = \frac{\Omega_m}{2} + \Omega_r -\Omega_{\Lambda}
  \label{q0cc}
\end{equation}
{\bf Osservazione}: la dipendenza $H=H(t)$ influenza la relazione
distanza-redshift.  Dalla definizione $z=R_0/R(t) -1$, segue
\begin{equation}
  \dd z = -\frac {R_0}{R(t)} H(t) \dd t
\end{equation}
Per segnali luminosi ($d \tau=0$)
\begin{equation}
  \int_t^{t_0} \frac {\dd t}{R(t)}= \int_0^{r} \frac {\dd r}{\sqrt1-kr^2} = r +
  \mathcal{O}(r^3)
\end{equation}
Dalle precedenti segue:
\begin{equation}
  r= \int_0^z \frac{\dd z}{R_0 ~ H(t)}
\end{equation}
per cui la distanza di luminosit\'a $d_L= r R_0 (1+z)$ diventa
\begin{equation}
  d_L = (1+z) \int_0^z \frac{\dd z}{H(z)}.
\end{equation}
In conclusione lo studio del diagramma $m-z$ per oggetti molto lontani coinvolge
i 4 parametri: $\Omega_{\Lambda}$, $\Omega_{k}$, $\Omega_{m}$, $\Omega_{r}$.  La
vecchia relazione era
\begin{equation}
  d_L = \frac {1}{H_0} \left[ z+ \frac{1}{2}(1-q_0)z^2\right].
\end{equation}

{\bf Osservazione:} L'introduzione della costante cosmologica nelle equazioni di
Einstein equivale all'introduzione tra i costituenti dell'Universo di un fluido
con densità $\rho_{\Lambda}$ e pressione $p_{\Lambda}=-\rho_{\Lambda}$.
Indichiamo con X una generica componente dell'Universo con equazione di stato
\begin{equation}
  p_X = w \rho_X
\end{equation}
Ora, poiché per ogni componente disaccoppiata dalla radiazione vale l'equazione
di conservazione:
\begin{equation}
  \toder{(\rho_X R^3)}{R} = -3 p_X R^2
\end{equation}
si ha quindi
\begin{equation}
  \rho_X \propto \frac {1}{R^{-3(1+w)}}
\end{equation}
Pertanto $\rho_{X}=\text{costante}$ implica $w=-1$.  Questo dimostra appunto che
da $\rho_{\Lambda}=\text{costante}$ si ha $p_{\Lambda}=-\rho_{\Lambda}$.
Notiamo ancora che il rapporto
$\rho_{X}/\rho_m \propto R^{-3w} \propto (1+z)^{3w}$.  Quindi per il caso $w<0$,
si ha che per $t \to 0$ ($z \to \infty$), $\rho_{X}/\rho_m \to 0$, cioè
l'importanza della componente X diminuisce indietro nel tempo.  Allora la
cosmologia di Friedmann non è modificata per $z>>1$ dall'introduzione della
costante cosmologicanel caso $w<0$.

Allora invece di introdurre nelle equazioni di Einstein la costante cosmologica
(modificare la geometria) potrei aggiungere nel termine di sorgente una
componente dark che chiamiamo dark energy.  In generale si assume che la dark
energy abbia equazione di stato $p_X=w\rho_X$ con $w<0$, il caso $w=-1$ include
la costante cosmologica. La relazione in eq. \eqref{hzcc} risulterà così
modificata:
\begin{equation}
  H(z)=H_0 \left[ \Omega_{X} (1+z)^{3(1+w)} + \Omega_{k} (1+z)^2+ \Omega_m (1+z)^3 + \Omega_r (1+z)^4 \right]^{1/2}
  \label{hzccX}
\end{equation}
e per l'istante attuale
\begin{equation}
  1= \left[ \Omega_{X} + \Omega_{k} + \Omega_m  + \Omega_r \right]
 \end{equation}
Inoltre l'eq. (\ref{ddRcc}) diventa
\begin{equation}
  \frac {\ddot{R}} {R} = \frac{H_0^2}{2}
  \left[ \Omega_{X} (1+3w) (1+z)^{3(1+w)}
    +\frac {\Omega_m (1+z)^3} {2 }  +
    \Omega_r (1+z)^4 \right]
  \label {ddRw}
\end{equation}
da cui segue per il parametro di decelerazione
\begin{equation}
  q_0= \frac{\Omega_X}{2} (1+3w) +  \frac{\Omega_m}{2} + \Omega_r - \Omega_{k}
\end{equation}
Nel caso $k=0$,
\begin{equation}
  q_0= \frac {1+3w \Omega_{X}}{2}
\end{equation}
vedremo che $\Omega_{X}=0.7$ ed $\Omega_{m}=0.3$, pertanto
$q_0=0.5+w \propto - \ddot{R}_0$ quindi se $w<0.5$ Universo accelera.

Tornando indietro alla relazione $d_L(z)$ osserviamo che nel caso $\Omega_{k}=0$
ed $\Omega_{\Lambda}=0$
\begin{equation}
d_L =
\end{equation}
mentre se prendo $\Omega_{k}=0$ ed  $\Omega_{m}=0$
\begin{equation}
d_L = (1+z) \int_0^z \frac{dz}{H_0 (1+z)^{3(1+w)/2}}
\end{equation}
e per $w=-1$ segue $d_L=H_0^{-1} z (1+z)$. Quindi a parità di $z$ oggetti
appaiono più o meno distanti a seconda del valore dei parametri cosmologici
$\Omega$.  Vedi figura da cui si ottiene (assumendo $k=0$ $\Omega_{m}=0.32$ ed
$\Omega_{\Lambda}=0.68$)

L'introduzione della costante cosmologica pu\'o essere interpretata in due modi:
caso (a) equivale a cambiare la lagrangiana della materia (descritta da
coordinate generalizzate $\phi$) in accordo a:
\begin{equation}
  L_{\textup{matter}} \to L'_{\textup{matter}} = L_{\textup{matter}} - \frac{\Lambda}{8 \pi G}.
\end{equation}
In questo caso l'azione diventa
\begin{equation}
  S_{tot}=S_{g}+S_{m}= \frac {1}{16 \pi G}
  \int R \sqrt{-g} \dd^4x + \int \left( L_{m}- \frac {\Lambda}{8 \pi G} \right) \sqrt{-g} \dd^4 x
\end{equation}
Osserva che le equazioni del moto per la materia (descritta dal campo $\phi$),
ottenute da $\delta S/\delta \phi=0$, rimangono inalterate poiché
$\Lambda=\text{costante}$.  In questo caso l'introduzione del termine con
$\Lambda$ introduce uno shift nello zero point dell'energia della materia.
Questo shift non influenza la dinamica della materia, mentre la gravità che
accoppia alla totale energia del sistema ne è influenzata.  Il caso (b)
corrisponde a riscrivere l'azione totale come:
\begin{equation}
  S_{tot}= \frac {1}{16 \pi G}
  \int (R-2\Lambda)  \sqrt{-g} \dd^4x + \int  L_{m} \sqrt{-g} \dd^4 x
\end{equation}
Allora la gravità descritta da due costanti $G$ e $\Lambda$; in questo caso lo
spazio-tempo è curvo anche in assenza di materia (eq. di Einstein nel vuoto non
ammettono la soluzione flat space.)

L'equazione di stato $\rho=-p$ ha un'altra importante implicazione in relatività
generale.  La parte spaziale $\bm{g}$ dell'accelerazione geogtiche (che misura
la relativa accelerazione di due geodetiche vicine) soddisfa la seguente
equazione:
\begin{equation}
  \nabla \cdot \bm{g} = - 4 \pi G (\rho+3p)
\end{equation}
allora la sorgente dell'accelerazione geodetica è $\rho+3p$; allora se
$\rho+3p<0$ la gravità è repulsiva.  Questo appunto accade se
$\rho \simeq \rho_{\Lambda}$ poiché
$\rho_{\Lambda}+p_{\Lambda}= -2 \rho{\Lambda}$.  Ovviamente la transizione dalla
gravità attrattiva a quella repulsiva accade quando $\rho_m$ diventa minore di
$\rho_{\Lambda}$

Quanto vale $\Lambda$?  La velocità della luce $c$ e l'istante attuale $t_0$
definiscono una lunghezza caratteristica
$l_0 = c t_0 \simeq c/H_0 \simeq \SI{3000}{\mega\parsec}$.  Quindi
$l_0 \simeq \SI{10e28}{\centi\metre}$.  La costante cosmologica deve soddisfare
la condizione
$\Lambda < \SI[per-mode=reciprocal]{e-45}{\per\centi\metre\squared}$ in modo
tale che $1/\Lambda > l_0^2$.

%%% Local Variables:
%%% mode: latex
%%% TeX-master: "../gravitazione"
%%% End:
