\cleardoublepage
\chapter{Il principio di equivalenza}
\label{cha:principio-equivalenza}

\completare{Esperimento ideale di Einstein, uguaglianza massa inerziale e
  gravitazionale.}

\index{principio!di equivalenza}\emph{principio di equivalenza}:
\emph{per ogni evento dello spazio-tempo in un arbitrario campo gravitazionale è
  possibile scegliere un sistema di riferimento \emph{localmente inerziale} (in
  caduta libera nel campo gravitazionale), tale che in un intorno
  sufficientemente piccolo dell'evento gli effetti della gravità siano assenti e
  le leggi della natura assumano la stessa forma che hanno in un sistema di
  riferimento inerziale}.
Dunque in un sistema di riferimento localmente inerziale la fisica segue le
leggi della relatività speciale.

\section{Equazione del moto di una particella in un campo gravitazionale}
\label{sec:equazione-moto}

Consideriamo una particella in caduta libera all'interno di un campo
gravitazionale $\bm{g}$.  In un sistema di riferimento, di coordinate
$\xi^{\alpha}$, localmente inerziale in cui la particella si muove di moto
rettilineo uniforme nello spazio-tempo, l'equazione del moto è
\begin{equation}
  \label{eq:moto-caduta-libera}
  \toder[2]{\xi^{\alpha}}{\tau} = 0,
\end{equation}
in cui $\dd^{2} \tau = -\eta_{\alpha\beta}\dd\xi^{\alpha}\dd\xi^{\beta}$ è
l'intervallo di tempo proprio.  Le coordinate $\xi^{\alpha}$ possono essere
espresse come funzione di un altro qualsiasi sistema di coordinate $x^{\mu}$ e
viceversa
\begin{align}
  \xi^{\alpha} &= \xi^{\alpha}(x^{\mu}), \\
  x^{\mu} &= x^{\mu}(\xi^{\alpha}).
\end{align}
% Poiché la trasformazione è invertibile deve risultare
% \begin{equation}
%   \abs*{\parder{x^{\mu}}{\xi^{\alpha}}} \neq 0.
% \end{equation}
Inoltre le coordinate $x^{\mu}$ dipendono anche dal tempo proprio $\tau$, allora
dalla~\eqref{eq:moto-caduta-libera} abbiamo
\begin{equation}
  \label{eq:moto1}
  0 = \toder{}{\tau}\toder{\xi^{\alpha}}{\tau} = \toder{}{\tau}
  \Big( \parder{\xi^{\alpha}}{x^{\mu}}\toder{x^{\mu}}{\tau} \Big)
  = \parder{\xi^{\alpha}}{x^{\mu},x^{\nu}} \toder{x^{\nu}}{\tau}
  \toder{x^{\mu}}{\tau} + \parder{\xi^{\alpha}}{x^{\mu}}
  \toder[2]{x^{\mu}}{\tau}.
\end{equation}
Supponendo che ciascuna delle quattro coordinate $x^{\mu}$ sia indipendente
dalle altre possiamo scrivere le relazioni
\begin{equation}
  \parder{x^{\mu}}{x^{\nu}} = \tensor{\delta}{^{\mu}_{\nu}}
  \implies \parder{x^{\mu}}{\xi^{\alpha}} \parder{\xi^{\alpha}}{x^{\nu}} =
  \tensor{\delta}{^{\mu}_{\nu}},
\end{equation}
quindi moltiplicando la~\eqref{eq:moto1} per
$\lparder{x^{\lambda}}{\xi^{\alpha}}$ l'equazione del moto diventa
\begin{equation}
  \tensor{\delta}{^{\lambda}_{\mu}} \toder[2]{x^{\mu}}{\tau} +
  \Big( \parder{\xi^{\alpha}}{x^{\mu},x^{\nu}} \parder{x^{\lambda}}{\xi^{\alpha}}
  \Big) \toder{x^{\mu}}{\tau} \toder{x^{\nu}}{\tau} = 0.
\end{equation}
Definendo la \index{connessione!affine}\emph{connessione affine}
$\tensor{\Gamma}{^{\lambda}_{\mu\nu}}$ come
\begin{equation}
  \label{eq:connessione-affine}
  \tensor{\Gamma}{^{\lambda}_{\mu\nu}}
  = \parder{\xi^{\alpha}}{x^{\mu},x^{\nu}} \parder{x^{\lambda}}{\xi^{\alpha}}
\end{equation}
l'equazione del moto può essere riscritta nel seguente modo
\begin{equation}
  \label{eq:geodetica}
  \toder[2]{x^{\lambda}}{\tau} + \tensor{\Gamma}{^{\lambda}_{\mu\nu}}
  \toder{x^{\mu}}{\tau} \toder{x^{\nu}}{\tau} = 0,
\end{equation}
oppure, utilizzando la quadrivelocità $u^{\alpha} = \ltoder{x^{\alpha}}{\tau}$,
\begin{equation}
  \toder{u^{\lambda}}{\tau} + \tensor{\Gamma}{^{\lambda}_{\mu\nu}}
  u^{\nu} u^{\nu} = 0.
\end{equation}
La~\eqref{eq:geodetica} è chiamata \emph{equazione della geodetica}.
$\tensor{\Gamma}{^{\lambda}_{\mu\nu}}$ è un insieme di $4^{3} = 64$ quantità che
tuttavia, come vedremo, % TODO: farlo vedere oppure non dire "come vedremo" :-P
non si trasformano, in generale, come un quadritensore di rango $3$.  Dalla
definizione di connessione affine abbiamo inoltre che questa è simmetrica
rispetto agli indici inferiori, cioè
$\tensor{\Gamma}{^{\lambda}_{\mu\nu}} = \tensor{\Gamma}{^{\lambda}_{\nu\mu}}$.

L'intervallo di tempo proprio
$\dd\tau^{2} = -\eta_{\alpha\beta} \dd\xi^{\alpha} \dd\xi^{\beta}$ può essere
espresso rispetto alle coordinate $x^{\mu}$ come
\begin{equation}
  \begin{split}
    \dd\tau^{2} &= -\eta_{\alpha\beta} \parder{\xi^{\alpha}}{x^{\mu}}\dd
    x^{\mu} \parder{\xi^{\beta}}{x^{\nu}}\dd x^{\nu} \\
    &= -g_{\mu\nu}\dd x^{\mu}\dd x^{\nu},
  \end{split}
\end{equation}
dove
\begin{equation}
  \label{eq:tensore-metrico}
  g_{\mu\nu} =
  \eta_{\alpha\beta} \parder{\xi^{\alpha}}{x^{\mu}} \parder{\xi^{\beta}}{x^{\nu}}
\end{equation}
è il \index{tensore!metrico}\emph{tensore metrico}.  Osserviamo che $g_{\mu\nu}$
è simmetrico rispetto allo scambio degli indici in quanto il tensore metrico di
Minkowski $\eta_{\alpha\beta}$ è simmetrico, infatti
\begin{equation}
  g_{\nu\mu} =
  \eta_{\alpha\beta} \parder{\xi^{\alpha}}{x^{\nu}}\parder{\xi^{\beta}}{x^{\mu}}
  =
  \eta_{\beta\alpha} \parder{\xi^{\beta}}{x^{\mu}} \parder{\xi^{\alpha}}{x^{\nu}}
  = g_{\mu\nu}.
\end{equation}

\section{Relazione fra tensore metrico e connessione affine}
\label{sec:relazione-g-Gamma}

Derivando rispetto a $x^{\lambda}$ l'espressione del tensore
metrico~\eqref{eq:tensore-metrico} abbiamo
\begin{equation}
  \partial_{\lambda}g_{\mu\nu}
  = \parder{\xi^{\alpha}}{x^{\lambda},x^{\mu}} \parder{\xi^{\beta}}{x^{\nu}}
  \eta_{\alpha\beta}
  + \parder{\xi^{\alpha}}{x^{\mu}} \parder{\xi^{\beta}}{x^{\lambda},x^{\nu}}
  \eta_{\alpha\beta}.
\end{equation}
Osserviamo che dalla definizione di connessione
affine~\eqref{eq:connessione-affine} risulta
\begin{equation}
  \parder{\xi^{\alpha}}{x^{\lambda}}\tensor{\Gamma}{^{\lambda}_{\mu\nu}}
  = \parder{\xi^{\alpha}}{x^{\mu},x^{\nu}},
\end{equation}
quindi
\begin{equation}
  \begin{split}
    \partial_{\lambda}g_{\mu\nu} &=
    \tensor{\Gamma}{^{\rho}_{\lambda\mu}} \parder{\xi^{\alpha}}{x^{\rho}}
    \parder{\xi^{\beta}}{x^{\nu}} \eta_{\alpha\beta} +
    \tensor{\Gamma}{^{\rho}_{\lambda\nu}} \parder{\xi^{\alpha}}{x^{\mu}}
    \parder{\xi^{\beta}}{x^{\rho}}
    \eta_{\alpha\beta} \\
    &= \tensor{\Gamma}{^{\rho}_{\lambda\mu}}g_{\rho\nu} +
    \tensor{\Gamma}{^{\rho}_{\lambda\nu}}g_{\rho\mu}.
  \end{split}
\end{equation}
Calcolando in maniera $\partial_{\mu}g_{\lambda\nu}$ e
$\partial_{\nu}g_{\mu\lambda}$ si giunge all'equazione
\begin{equation}
  \begin{split}
    \partial_{\lambda}g_{\mu\nu} + \partial_{\mu}g_{\lambda\nu}
    - \partial_{\nu}g_{\mu\lambda} &=
    \tensor{\Gamma}{^{\rho}_{\lambda\mu}}g_{\rho\nu} +
    \tensor{\Gamma}{^{\rho}_{\lambda\nu}}g_{\rho\mu} +
    \tensor{\Gamma}{^{\rho}_{\mu\lambda}}g_{\rho_{\nu}} +
    \tensor{\Gamma}{^{\rho}_{\mu\nu}}g_{\rho\lambda} \\
    &- \tensor{\Gamma}{^{\rho}_{\nu\mu}}g_{\rho\lambda} -
    \tensor{\Gamma}{^{\rho}_{\nu\lambda}}g_{\rho\mu} \\
    &= 2g_{\rho\nu} \tensor{\Gamma}{^{\rho}_{\lambda\mu}},
  \end{split}
\end{equation}
ricordando che $g_{\mu\nu}$ e $\tensor{\Gamma}{^{\rho}_{\mu\nu}}$ sono
simmetrici per scambi degli indici $\mu$ e $\nu$.  Definiamo la matrice
$g^{\nu\sigma}$ come l'inversa di $g_{\nu\sigma}$, cioè
$g^{\nu\sigma}g_{\rho\nu} = \tensor{\delta}{^{\sigma}_{\rho}}$, allora,
moltiplicando l'equazione precedente per $g^{\nu\sigma}$ otteniamo la seguente
relazione fra il tensore metrico e la connessione affine
\begin{equation}
  \begin{aligned}
    &g^{\nu\sigma}(\partial_{\lambda}g_{\mu\nu} + \partial_{\mu}g_{\lambda\nu}
    - \partial_{\nu}g_{\mu\lambda}) = 2\tensor{\Gamma}{^{\sigma}_{\lambda\mu}}
    \iff \\
    &\tensor{\Gamma}{^{\sigma}_{\lambda\mu}} =
    \frac{1}{2} g^{\nu\sigma} (\partial_{\lambda}g_{\mu\nu}
    + \partial_{\mu}g_{\lambda\nu} - \partial_{\nu}g_{\lambda\mu}).
  \end{aligned}
\end{equation}

La geodetica è, per definizione, la curva più breve fra due punti, allora
l'equazione della geodetica~\eqref{eq:geodetica} per una particella in caduta
libera può essere ricavata applicando il principio variazionale alla distanza
$S$ fra due eventi $A$ e $B$ nello spazio-tempo
\begin{equation}
  S = \int_{A}^{B} \dd\tau.
\end{equation}
Conviene parametrizzare la linea d'universo $x^{\mu}$ della particella con un
arbitrario parametro $p$, cioè $x^{\mu} = x^{\mu}(p)$, quindi
\begin{equation}
  S = \int_{A}^{B} \dd\tau = \int_{A}^{\beta} \toder{\tau}{p}\dd p =
  \int_{A}^{B} \Big( -g_{\mu\nu}\toder{x^{\mu}}{p}\toder{x^{\nu}}{p} \Big)^{1/2}
  \dd p.
\end{equation}
Per ricavare l'equazione del moto dobbiamo imporre che $S$ sia stazionario
rispetto a variazioni infinitesime $\delta x^{\mu}$ delle coordinate $x^{\mu}$,
con la condizione che le variazioni sia nulle nei punti $A$ e $B$
\begin{equation}
  \label{eq:condizione-variazione}
  \delta x_{A}^{\mu} = \delta x_{B}^{\mu} = 0.
\end{equation}
La variazione della distanza fra i due eventi è
\begin{equation}
  \begin{split}
    \delta S &= \int_{A}^{B} \frac{1}{2}\Big( -g_{\mu\nu} \toder{x^{\mu}}{p}
    \toder{x^{\nu}}{p} \Big)^{-1/2} \Big( -\parder{g_{\mu\nu}}{x^{\lambda}}
    \delta x^{\lambda} \toder{x^{m}}{p} \toder{x^{n}}{p} \\
    &- g_{\mu\nu} \toder{\delta x^{\mu}}{p} \toder{x^{\nu}}{p} - g_{\mu\nu}
    \toder{x^{\mu}}{p} \toder{\delta x^{\nu}}{p} \Big) \dd p \\
    &= \int_{A}^{B} \frac{1}{2}\Big( -g_{\mu\nu} \toder{x^{\mu}}{p}
    \toder{x^{\nu}}{p} \Big)^{-1/2} \Big( -\parder{g_{\mu\nu}}{x^{\lambda}}
    \delta x^{\lambda} \toder{x^{\mu}}{p} \toder{x^{\nu}}{p} -
    2g_{\mu\nu}\toder{\delta x^{\mu}}{p} \toder{x^{\nu}}{p} \Big) \dd p.
  \end{split}
\end{equation}
Nell'ultimo passaggio abbiamo sfruttato la simmetria di del tensore metrico
$g_{\mu\nu}$.  Osserviamo che il primo fattore dell'integrale è
$\ltoder{p}{\tau}$, così che risulta
\begin{equation}
  \delta S = -\int_{A}^{B} \Big( \frac{1}{2}\parder{g_{\mu\nu}}{x^{\lambda}}
  \delta  x^{\lambda} \toder{x^{\mu}}{\tau} \toder{x^{\nu}}{\tau} + g_{\mu\nu}
  \toder{\delta x^{\mu}}{\tau} \toder{x^{\nu}}{\tau} \Big) \dd \tau.
\end{equation}
Integriamo il secondo termine per parti ricordando la
condizione~\eqref{eq:condizione-variazione}
\begin{equation}
  \begin{split}
    \int_{A}^{\beta} g_{\lambda\nu} \toder{\delta x^{\lambda}}{\tau}
    \toder{x^{\nu}}{\tau} \dd \tau &= g_{\lambda\nu} \delta x^{\lambda}
    \toder{x^{\nu}}{\tau} \bigg|_{A}^{B} - \int_{A}^{B} \delta x^{\lambda}
    \toder{}{\tau} \Big( g_{\lambda\nu} \toder{x^{\nu}}{\tau} \Big) \dd \tau \\
    &= -\int_{A}^{B} \delta x^{\lambda} \Big( \toder{g_{\lambda\nu}}{\tau}
    \toder{x^{\nu}}{\tau} + g_{\lambda\nu}\toder[2]{x^{\nu}}{\tau} \Big) \dd
    \tau \\
    &= -\int_{A}^{B} \delta x^{\lambda}
    \Big( \parder{g_{\lambda\nu}}{x^{\mu}} \toder{x^{\mu}}{\tau}
    \toder{x^{\nu}}{\tau} + g_{\lambda\nu}\toder[2]{x^{\nu}}{\tau} \Big) \dd
    \tau.
  \end{split}
\end{equation}
Allora
\begin{equation}
  \delta S = -\int_{A}^{B} \delta x^{\lambda} \Big(
  \frac{1}{2} \parder{g_{\mu\nu}}{x^{\lambda}} \toder{x^{\mu}}{\tau}
  \toder{x^{\nu}}{\tau} - \parder{g_{\lambda\nu}}{x^{\mu}}
  \toder{x^{\mu}}{\tau} \toder{x^{\nu}}{\tau} -
  g_{\lambda\nu}\toder[2]{x^{\nu}}{\tau} \Big) \dd \tau.
\end{equation}
Affinché sia $\delta S = 0$, per l'arbitrarietà della variazione $\delta
x^{\lambda}$ deve annullarsi la quantità fra parentesi, cioè
\begin{equation}
  \Big(\frac{1}{2} \partial_{\lambda} g_{\mu\nu} - \partial_{\mu}g_{\lambda\nu}
  \Big) \toder{x^{\mu}}{\tau} \toder{x^{\nu}}{\tau}
  - g_{\lambda\nu}\toder[2]{x^{\nu}}{\tau} = 0.
\end{equation}
Dal momento che
\begin{equation}
  \partial_{\mu}g_{\lambda\nu} \toder{x^{\mu}}{\tau} \toder{x^{\nu}}{\tau} =
  \frac{1}{2} (\partial_{\mu}g_{\lambda\nu} + \partial_{\nu}g_{\lambda\mu})
  \toder{x^{\mu}}{\tau} \toder{x^{\nu}}{\tau}
\end{equation}
possiamo scrivere
\begin{equation}
  \frac{1}{2} (\partial_{\lambda} g_{\mu\nu} - \partial_{\mu}g_{\lambda\nu}
  - \partial_{\nu}g_{\lambda\mu}) \toder{x^{\mu}}{\tau}
  \toder{x^{\nu}}{\tau} - g_{\lambda\nu}\toder[2]{x^{\nu}}{\tau} = 0
\end{equation}
e poiché
$\partial_{\lambda} g_{\mu\nu} - \partial_{\mu}g_{\lambda\nu}
- \partial_{\nu}g_{\lambda\mu} = 2g_{\rho\lambda}
\tensor{\Gamma}{^{\rho}_{\nu\mu}}$ abbiamo
\begin{equation}
  g_{\rho\lambda} \tensor{\Gamma}{^{\rho}_{\nu\mu}} \toder{x^{\mu}}{\tau}
  \toder{x^{\nu}}{\tau} - g_{\lambda\nu}\toder[2]{x^{\nu}}{\tau} = 0.
\end{equation}
Moltiplicando infine ambo i membri per $g^{\lambda\sigma}$ e osservando che
$g_{\alpha\beta}g^{\alpha\gamma} = \tensor{\delta}{^{\gamma}_{\beta}}$ giungiamo
all'equazione della geodetica
\begin{equation}
  \toder[2]{x^{\sigma}}{\tau} + \tensor{\Gamma}{^{\sigma}_{\nu\mu}}
  \toder{x^{\mu}}{\tau} \toder{x^{\nu}}{\tau} = 0.
\end{equation}



\section{Limite newtoniano dell'equazione del moto}
\label{sec:limite-newtoniano}

Vogliamo vedere come l'equazione del moto~\eqref{eq:geodetica} di una particella
all'interno di un campo gravitazionale si riduce considerando il limite
newtoniano, vale a dire nelle condizioni
\begin{itemize}
\item il campo gravitazionale è debole, cioè può essere considerato come una
  perturbazione dello spazio-tempo piatto;
\item il campo gravitazionale inoltre è stazionario;
\item la velocità della particella è molto più piccola di quella della luce.
\end{itemize}
La prima condizione può essere formalizzata dicendo che il tensore metrico
$g_{\mu\nu}$ deve differire di poco dal tensore di Minkowski dello spazio-tempo
piatto
\begin{equation}
  g_{\mu\nu} = \eta_{\mu\nu} + h_{\mu\nu},
\end{equation}
con $\abs{h_{\mu\nu}} \ll 1$.  Calcoliamo la connessione affine al primo ordine
della perturbazione $h_{\mu\nu}$
\begin{equation}
  \label{eq:christoffel-approx}
  \tensor{\Gamma}{^{\sigma}_{\mu\nu}} = \frac{1}{2}g^{\lambda\sigma}
  (\partial_{\mu}g_{\nu\lambda} + \partial_{\nu}g_{\mu\lambda}
  - \partial_{\lambda}g_{\mu\nu}) \approx \frac{1}{2}\eta^{\lambda\sigma}
  (\partial_{\mu}h_{\nu\lambda} + \partial_{\nu}h_{\mu\lambda}
  - \partial_{\lambda}h_{\mu\nu}).
\end{equation}
La terza condizione può essere formalizzata come
\begin{equation}
  \toder{x^{i}}{\tau} \ll \toder{t}{\tau},
\end{equation}
quindi l'equazione della geodetica si riduce a
\begin{equation}
  \label{eq:geodetica-approx}
  \begin{split}
    0 &= \toder[2]{x^{\sigma}}{\tau} + \tensor{\Gamma}{^{\sigma}_{\lambda\mu}}
    \toder{x^{\lambda}}{\tau} \toder{x^{\mu}}{\tau} \\
    &= \toder[2]{x^{\sigma}}{\tau} + \tensor{\Gamma}{^{\sigma}_{00}} \Big(
    \toder{t}{\tau} \Big)^{2} + 2\tensor{\Gamma}{^{\sigma}_{0i}} \toder{t}{\tau}
    \toder{x^{i}}{\tau} + \tensor{\Gamma}{^{\sigma}_{ij}} \toder{x^{i}}{\tau}
    \toder{x^{j}}{\tau} \\
    &\approx \toder[2]{x^{\sigma}}{\tau} + \tensor{\Gamma}{^{\sigma}_{00}} \Big(
    \toder{t}{\tau} \Big)^{2}.
  \end{split}
\end{equation}
Dalla~\eqref{eq:christoffel-approx} risulta
\begin{equation}
  \tensor{\Gamma}{^{\sigma}_{00}} \approx
  \frac{1}{2}\eta^{\lambda\sigma}(\partial_{0}h_{0\lambda}
  + \partial_{0}h_{0\lambda} - \partial_{\lambda}h_{00}) =
  -\frac{1}{2}\eta^{\lambda\sigma} \partial_{\lambda}h_{00},
\end{equation}
in quanto le derivate temporali di $h_{\mu\nu}$ sono nulle in base alla seconda
condizione che descrive il limite newtoniano.  Notiamo che per $\sigma=0$ si ha
$\tensor{\Gamma}{^{0}_{00}}=0$.  La~\eqref{eq:geodetica-approx} corrisponde, al
variare di $\sigma$, alle equazioni
\begin{subequations}
  \begin{align}
    \toder[2]{t}{\tau} &= 0 &\text{per $\sigma = 0$}, \\
    \toder[2]{x^{i}}{\tau} &= \frac{1}{2} \Big( \toder{t}{\tau} \Big)^{2}
    (\nabla h_{00})_{i} &\text{per $\sigma = i$}.
  \end{align}
\end{subequations}
La seconda equazione ci dice che $\ltoder{t}{\tau}$ è una costante, mentre dalla
prima, dividendo per $(\ltoder{t}{\tau})^{2}$, abbiamo
\begin{equation}
  \toder[2]{\bm{x}}{t} = \frac{1}{2}\nabla h_{00}.
\end{equation}
L'equazione newtoniana del moto è
\begin{equation}
  \bm{a}_{\textup{N}} = \toder[2]{\bm{x}}{t} = -\nabla\phi,
\end{equation}
in cui $\phi$ è il potenziale newtoniano del campo gravitazionale, che a
distanza $r$ da un corpo sferico di massa $M$ assume l'espressione
$\phi = -GM/r$.  Confrontando le due equazioni precedenti troviamo
\begin{equation}
  h_{00} = -2\phi + \text{costante}.
\end{equation}
A grandi distanze dal corpo, cioè nel limite $r \to \infty$, il potenziale
$\phi$ si annulla, il campo gravitazionale è assente e il tensore metrico
$g_{\mu\nu}$ deve tendere al tensore di Minkowski $\eta_{\mu\nu}$ dello
spazio-tempo piatto, cioè si devono annullare tutte le componenti di
$h_{\mu\nu}$.  Pertanto ricaviamo che la costante deve essere uguale a $0$ e
$h_{00} = -2\phi$ o, ripristinando la velocità $c$ della luce  nel vuoto,
\begin{equation}
  h_{00} = -\frac{2\phi}{c^{2}} = \frac{2GM}{rc^{2}} = \frac{r_{\textup{g}}}{r},
\end{equation}
in cui
\begin{equation}
  r_{\textup{g}} = \frac{2GM}{c^{2}}
\end{equation}
è il \index{raggio!di Schwarzschild}\emph{raggio di Schwarzschild} del corpo che
genera il campo gravitazionale.  Dunque
\begin{equation}
  g_{00} = \eta_{00} + h_{00} = - \Big(1 - \frac{2GM}{rc^{2}} \Big)
\end{equation}
e la quantità $h_{00} = r_{\textup{g}}/r$ è una misura delle deviazioni della
relatività generale rispetto alle previsioni della gravità newtoniana.

\begin{table}
  \centering
  \caption[Valori del raggio di Schwarzschild per diversi corpi]{Valori del
    raggio di Schwarzschild $r_{\textup{g}}$ per diversi oggetti e valore della
    correzione $r_{\textup{g}}/r$ sulla superficie dei corpi.  $r$ è il raggio
    degli oggetti}
  \label{tab:Schwarzschild}
  \begin{tabular}{lSSSS}
    \toprule
    corpo & {$r_{\textup{g}}$ (\si{\metre})} & {$r$ (\si{\metre})} &
    {$r_{\textup{g}}/r$} \\
    \midrule
    Terra              & 8.7e-3 & 6.4e6 & 1.4e-9 \\
    Sole               & 3e3    & 7e8   & 4.3e-6 \\
    nana bianca        & 3e3    & 1e6   & 3e-3   \\
    stella di neutroni & 3e3    & 1e4   & 0.3    \\
    buco nero stellare & 3e4    & 3e4   & 1      \\
    \bottomrule
  \end{tabular}
\end{table}
Nella tabella~\ref{tab:Schwarzschild} sono riportati i valori delle deviazioni
$r_{\textup{g}}/r$ dalla gravitazione newtoniana sulla superficie di alcuni
corpi.  Mentre per la Terra, il Sole e le nane bianche queste correzioni sono
molto piccole, non è possibile descrivere le vicinanze di corpi come le stelle
di neutroni o i buchi neri utilizzando la fisica newtoniana ma bisogna tenere
necessariamente in considerazione la relatività generale.

\section{\completare{Campo gravito-magnetico}}
\label{sec:campo-gravito-magnetico}

\subsection{\completare{Effetto Lense-Thirring}}
\label{sec:effetto-lense-thirring}

\section{\completare{Red-shift gravitazionale}}
\label{sec:red-shift-gravitazionale}



%%% Local Variables:
%%% mode: latex
%%% TeX-master: "../astrofisica-teorica"
%%% fill-column: 80
%%% End:
