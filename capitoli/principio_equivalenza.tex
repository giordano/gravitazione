\cleardoublepage
\chapter{Principio di equivalenza}
\label{cha:principio-equivalenza}

\completare{Esperimento ideale di Einstein, uguaglianza massa inerziale e
  gravitazionale.}

\index{principio!di equivalenza}\emph{principio di equivalenza}:
\emph{per ogni evento dello spazio-tempo in un arbitrario campo gravitazionale è
  possibile scegliere un sistema di riferimento \emph{localmente inerziale} (in
  caduta libera nel campo gravitazionale), tale che in un intorno
  sufficientemente piccolo dell'evento gli effetti della gravità siano assenti e
  le leggi della natura assumano la stessa forma che hanno in un sistema di
  riferimento inerziale}.
Dunque in un sistema di riferimento localmente inerziale la fisica segue le
leggi della relatività speciale.

\section{Equazione del moto di una particella in un campo gravitazionale}
\label{sec:equazione-moto}

Consideriamo una particella in caduta libera all'interno di un campo
gravitazionale.  In un sistema di riferimento, di coordinate $\xi^{\alpha}$,
localmente inerziale in cui la particella si muove di moto rettilineo uniforme
nello spazio-tempo, l'equazione del moto è
\begin{equation}
  \label{eq:moto-caduta-libera}
  \toder[2]{\xi^{\alpha}}{\tau} = 0,
\end{equation}
in cui $\dd \tau^{2} = -\eta_{\alpha\beta}\dd\xi^{\alpha}\dd\xi^{\beta}$ è
l'intervallo di tempo proprio.  Le coordinate $\xi^{\alpha}$ possono essere
espresse come funzione di un altro qualsiasi sistema di coordinate $x^{\mu}$ e
viceversa
\begin{align}
  \xi^{\alpha} &= \xi^{\alpha}(x^{\mu}), \\
  x^{\mu} &= x^{\mu}(\xi^{\alpha}).
\end{align}
% Poiché la trasformazione è invertibile deve risultare
% \begin{equation}
%   \abs*{\parder{x^{\mu}}{\xi^{\alpha}}} \neq 0.
% \end{equation}
Inoltre le coordinate $x^{\mu}$ dipendono anche dal tempo proprio $\tau$, allora
dalla~\eqref{eq:moto-caduta-libera} abbiamo
\begin{equation}
  \label{eq:moto1}
  0 = \toder{}{\tau}\toder{\xi^{\alpha}}{\tau} = \toder{}{\tau}
  \bigg( \parder{\xi^{\alpha}}{x^{\mu}}\toder{x^{\mu}}{\tau} \bigg)
  = \parder{\xi^{\alpha}}{x^{\mu},x^{\nu}} \toder{x^{\nu}}{\tau}
  \toder{x^{\mu}}{\tau} + \parder{\xi^{\alpha}}{x^{\mu}}
  \toder[2]{x^{\mu}}{\tau}.
\end{equation}
Supponendo che ciascuna delle quattro coordinate $x^{\mu}$ sia indipendente
dalle altre possiamo scrivere le relazioni
\begin{equation}
  \parder{x^{\lambda}}{x^{\mu}} = \tensor{\delta}{^{\lambda}_{\mu}}
  \implies \parder{x^{\lambda}}{\xi^{\alpha}} \parder{\xi^{\alpha}}{x^{\mu}} =
  \tensor{\delta}{^{\lambda}_{\mu}},
\end{equation}
quindi moltiplicando la~\eqref{eq:moto1} per
$\lparder{x^{\lambda}}{\xi^{\alpha}}$ l'equazione del moto diventa
\begin{equation}
  \tensor{\delta}{^{\lambda}_{\mu}} \toder[2]{x^{\mu}}{\tau} +
  \bigg(\parder{\xi^{\alpha}}{x^{\mu},x^{\nu}}\parder{x^{\lambda}}{\xi^{\alpha}}
  \bigg) \toder{x^{\mu}}{\tau} \toder{x^{\nu}}{\tau} = 0.
\end{equation}
Definendo la \index{connessione!affine}\emph{connessione affine}
$\tensor{\Gamma}{^{\lambda}_{\mu\nu}}$ come
\begin{equation}
  \label{eq:connessione-affine}
  \tensor{\Gamma}{^{\lambda}_{\mu\nu}}
  = \parder{\xi^{\alpha}}{x^{\mu},x^{\nu}} \parder{x^{\lambda}}{\xi^{\alpha}}
\end{equation}
l'equazione del moto può essere riscritta nel seguente modo
\begin{equation}
  \label{eq:geodetica}
  \toder[2]{x^{\lambda}}{\tau} + \tensor{\Gamma}{^{\lambda}_{\mu\nu}}
  \toder{x^{\mu}}{\tau} \toder{x^{\nu}}{\tau} = 0,
\end{equation}
oppure, utilizzando la quadrivelocità $u^{\alpha} = \ltoder{x^{\alpha}}{\tau}$,
\begin{equation}
  \label{eq:geodetica2}
  \toder{u^{\lambda}}{\tau} + \tensor{\Gamma}{^{\lambda}_{\mu\nu}}
  u^{\mu} u^{\nu} = 0.
\end{equation}
La~\eqref{eq:geodetica} è chiamata
\index{equazione!della geodetica}\emph{equazione della geodetica}.
$\tensor{\Gamma}{^{\lambda}_{\mu\nu}}$ è un insieme di $4^{3} = 64$ quantità che
tuttavia, come vedremo nel paragrafo~\ref{sec:connessione-affine}, \emph{non} si
trasformano, in generale, come un quadritensore di rango $3$.  Nonostante ciò è
utile indicarla con il simbolo di un tensore.  Dalla definizione di connessione
affine abbiamo inoltre che questa è simmetrica rispetto agli indici inferiori,
cioè
$\tensor{\Gamma}{^{\lambda}_{\mu\nu}} = \tensor{\Gamma}{^{\lambda}_{\nu\mu}}$.

L'intervallo di tempo proprio
$\dd\tau^{2} = -\eta_{\alpha\beta} \dd\xi^{\alpha} \dd\xi^{\beta}$ può essere
espresso rispetto alle coordinate $x^{\mu}$ come
\begin{equation}
  \begin{split}
    \dd\tau^{2} &= -\eta_{\alpha\beta} \parder{\xi^{\alpha}}{x^{\mu}}\dd
    x^{\mu} \parder{\xi^{\beta}}{x^{\nu}}\dd x^{\nu} \\
    &= -g_{\mu\nu}\dd x^{\mu}\dd x^{\nu},
  \end{split}
\end{equation}
dove
\begin{equation}
  \label{eq:tensore-metrico}
  g_{\mu\nu} =
  \eta_{\alpha\beta} \parder{\xi^{\alpha}}{x^{\mu}} \parder{\xi^{\beta}}{x^{\nu}}
\end{equation}
è il \index{tensore!metrico}\emph{tensore metrico}.  Il tensore metrico, a
differenza del tensore di Minkowski, in generale non è diagonale ed è funzione
delle coordinate poiché il valore delle derivate varia punto per punto.  Quindi
dovremmo scrivere $g_{\mu\nu}(x)$ ma per brevità ometteremo di specificare la
dipendenza dal punto.  Osserviamo che $g_{\mu\nu}$ è simmetrico rispetto allo
scambio degli indici in quanto il tensore metrico di Minkowski
$\eta_{\alpha\beta}$ è simmetrico, infatti
\begin{equation}
  g_{\nu\mu} =
  \eta_{\alpha\beta} \parder{\xi^{\alpha}}{x^{\nu}}\parder{\xi^{\beta}}{x^{\mu}}
  =
  \eta_{\beta\alpha} \parder{\xi^{\beta}}{x^{\mu}} \parder{\xi^{\alpha}}{x^{\nu}}
  = g_{\mu\nu}.
\end{equation}

Dall'equazione della geodetica risulta che il moto di una particella in un campo
gravitazionale è determinato dalla connessione affine.  Inoltre
$\ltoder{u^{\lambda}}{\tau}$ è la quadriaccelerazione del corpo, allora possiamo
interpretare la quantità
$-m\tensor{\Gamma}{^{\lambda}_{\mu\nu}} u^{\mu} u^{\nu}$ come la ``quadriforza''
agente sul corpo all'interno del campo gravitazionale.  Come vedremo nel
paragrafo~\ref{sec:relazione-g-Gamma}, le derivate di $g_{\mu\nu}$ definiscono
il campo $\tensor{\Gamma}{^{\lambda}_{\mu\nu}}$ e quindi il tensore metrico può
essere visto come il ``potenziale'' del campo gravitazionale.  In effetti nel
paragrafo~\ref{sec:limite-newtoniano} vedremo che nel limite newtoniano la
componente $g_{00}$ del tensore metrico è strettamente legata al potenziale
gravitazionale classico.  Chiaramente queste interpretazioni della connessione
affine e del tensore metrico sono solo un'analogia fisica derivante dal
confronto con le equazioni newtoniane del moto
$\bm{F} = m \bm{a} = - \nabla\phi$.

\section{Relazione fra tensore metrico e connessione affine}
\label{sec:relazione-g-Gamma}

Derivando rispetto a $x^{\lambda}$ l'espressione del tensore
metrico~\eqref{eq:tensore-metrico} abbiamo
\begin{equation}
  \partial_{\lambda}g_{\mu\nu}
  = \parder{\xi^{\alpha}}{x^{\lambda},x^{\mu}} \parder{\xi^{\beta}}{x^{\nu}}
  \eta_{\alpha\beta}
  + \parder{\xi^{\alpha}}{x^{\mu}} \parder{\xi^{\beta}}{x^{\lambda},x^{\nu}}
  \eta_{\alpha\beta}.
\end{equation}
Osserviamo che dalla definizione di connessione
affine~\eqref{eq:connessione-affine} risulta
\begin{equation}
  \parder{\xi^{\alpha}}{x^{\lambda}}\tensor{\Gamma}{^{\lambda}_{\mu\nu}}
  = \parder{\xi^{\alpha}}{x^{\mu},x^{\nu}},
\end{equation}
quindi
\begin{equation}
  \label{eq:foo}
  \begin{split}
    \partial_{\lambda}g_{\mu\nu} &=
    \tensor{\Gamma}{^{\rho}_{\lambda\mu}} \parder{\xi^{\alpha}}{x^{\rho}}
    \parder{\xi^{\beta}}{x^{\nu}} \eta_{\alpha\beta} +
    \tensor{\Gamma}{^{\rho}_{\lambda\nu}} \parder{\xi^{\alpha}}{x^{\mu}}
    \parder{\xi^{\beta}}{x^{\rho}}
    \eta_{\alpha\beta} \\
    &= \tensor{\Gamma}{^{\rho}_{\lambda\mu}}g_{\rho\nu} +
    \tensor{\Gamma}{^{\rho}_{\lambda\nu}}g_{\rho\mu}.
  \end{split}
\end{equation}
Calcolando in maniera $\partial_{\mu}g_{\lambda\nu}$ e
$\partial_{\nu}g_{\mu\lambda}$ si giunge all'equazione
\begin{equation}
  \begin{split}
    \partial_{\lambda}g_{\mu\nu} + \partial_{\mu}g_{\lambda\nu}
    - \partial_{\nu}g_{\mu\lambda} &=
    \tensor{\Gamma}{^{\rho}_{\lambda\mu}}g_{\rho\nu} +
    \tensor{\Gamma}{^{\rho}_{\lambda\nu}}g_{\rho\mu} +
    \tensor{\Gamma}{^{\rho}_{\mu\lambda}}g_{\rho\nu} +
    \tensor{\Gamma}{^{\rho}_{\mu\nu}}g_{\rho\lambda} \\
    &- \tensor{\Gamma}{^{\rho}_{\nu\mu}}g_{\rho\lambda} -
    \tensor{\Gamma}{^{\rho}_{\nu\lambda}}g_{\rho\mu} \\
    &= 2g_{\rho\nu} \tensor{\Gamma}{^{\rho}_{\lambda\mu}},
  \end{split}
\end{equation}
avendo ricordato che $g_{\mu\nu}$ e $\tensor{\Gamma}{^{\rho}_{\mu\nu}}$ sono
simmetrici per scambi degli indici $\mu$ e $\nu$.  Definiamo la matrice
$g^{\nu\sigma}$ come l'inversa di $g_{\nu\sigma}$, cioè
$g^{\nu\sigma}g_{\rho\nu} = \tensor{\delta}{^{\sigma}_{\rho}}$, allora,
moltiplicando primo e ultimo membro dell'equazione precedente per
$g^{\nu\sigma}$
\begin{equation}
  g^{\nu\sigma}(\partial_{\lambda}g_{\mu\nu} + \partial_{\mu}g_{\lambda\nu}
  - \partial_{\nu}g_{\mu\lambda}) = 2\tensor{\Gamma}{^{\sigma}_{\lambda\mu}}
\end{equation}
otteniamo la seguente relazione fra il \index{tensore!metrico}tensore metrico e
la \index{connessione!affine}connessione
affine\footnote{Presentiamo un utile metodo per ricordare la relazione fra
  tensore metrico e connessione affine.  Osservando che il tensore metrico è
  simmetrico, la connessione affine è simmetrica nei suoi due indici inferiori e
  che nella relazione compaiono due derivate con il segno $+$ e una con il segno
  $-$ deduciamo che nell'unica derivata con il segno $-$ il tensore metrico deve
  avere gli stessi indici inferiori della connessione affine e la derivata è
  fatta rispetto all'indice muto, nelle derivate con il segno $+$ il tensore
  metrico ha sempre uno degli indici uguale all'indice muto e l'altro uguale a
  uno degli indici inferiori della connessione affine mentre la derivata è fatta
  rispetto all'altro indice inferiore della connessione affine.  Infine, il
  tensore metrico controvariante che compare davanti alle parentesi avrà un
  indice uguale all'unico indice superiore della connessione affine e l'altro
  uguale all'indice muto.}
\begin{equation}
  \label{eq:connessione-metrica}
  \tensor{\Gamma}{^{\sigma}_{\lambda\mu}} =
  \frac{1}{2} g^{\nu\sigma} (\partial_{\lambda}g_{\mu\nu}
  + \partial_{\mu}g_{\lambda\nu} - \partial_{\nu}g_{\lambda\mu}).
\end{equation}

L'equazione della geodetica~\eqref{eq:geodetica} per una particella in caduta
libera può essere ricavata applicando il principio variazionale alla distanza
$S$ fra due eventi $A$ e $B$ nello spazio-tempo
\begin{equation}
  S = \int_{A}^{B} \dd\tau.
\end{equation}
Conviene parametrizzare la linea d'universo $x^{\mu}$ della particella con un
arbitrario parametro $p$, cioè $x^{\mu} = x^{\mu}(p)$, quindi
\begin{equation}
  S = \int_{A}^{B} \dd\tau = \int_{A}^{\beta} \toder{\tau}{p}\dd p =
  \int_{A}^{B} \bigg( -g_{\mu\nu}\toder{x^{\mu}}{p}\toder{x^{\nu}}{p}
  \bigg)^{1/2} \dd p.
\end{equation}
Per ricavare l'equazione del moto dobbiamo imporre che $S$ sia stazionario
rispetto a variazioni infinitesime $\delta x^{\mu}$ delle coordinate $x^{\mu}$,
con la condizione che le variazioni sia nulle nei punti $A$ e $B$
\begin{equation}
  \label{eq:condizione-variazione}
  \delta x_{A}^{\mu} = \delta x_{B}^{\mu} = 0.
\end{equation}
La variazione della distanza fra i due eventi è
\begin{equation}
  \begin{split}
    \delta S &= \int_{A}^{B} \frac{1}{2}\bigg( -g_{\mu\nu} \toder{x^{\mu}}{p}
    \toder{x^{\nu}}{p} \bigg)^{-1/2} \bigg( -\parder{g_{\mu\nu}}{x^{\lambda}}
    \delta x^{\lambda} \toder{x^{m}}{p} \toder{x^{n}}{p} \\
    &- g_{\mu\nu} \toder{\delta x^{\mu}}{p} \toder{x^{\nu}}{p} - g_{\mu\nu}
    \toder{x^{\mu}}{p} \toder{\delta x^{\nu}}{p} \bigg) \dd p \\
    &= \int_{A}^{B} \frac{1}{2}\bigg( -g_{\mu\nu} \toder{x^{\mu}}{p}
    \toder{x^{\nu}}{p} \bigg)^{-1/2} \bigg( -\parder{g_{\mu\nu}}{x^{\lambda}}
    \delta x^{\lambda} \toder{x^{\mu}}{p} \toder{x^{\nu}}{p} -
    2g_{\mu\nu}\toder{\delta x^{\mu}}{p} \toder{x^{\nu}}{p} \bigg) \dd p.
  \end{split}
\end{equation}
Nell'ultimo passaggio abbiamo sfruttato la simmetria del tensore metrico
$g_{\mu\nu}$.  Osserviamo che il primo fattore dell'integrale è
$\ltoder{p}{\tau}$, così che risulta
\begin{equation}
  \delta S = -\int_{A}^{B} \bigg( \frac{1}{2}\parder{g_{\mu\nu}}{x^{\lambda}}
  \delta  x^{\lambda} \toder{x^{\mu}}{\tau} \toder{x^{\nu}}{\tau} + g_{\mu\nu}
  \toder{\delta x^{\mu}}{\tau} \toder{x^{\nu}}{\tau} \bigg) \dd \tau.
\end{equation}
Integriamo il secondo termine per parti ricordando la
condizione~\eqref{eq:condizione-variazione}
\begin{equation}
  \begin{split}
    \int_{A}^{B} g_{\lambda\nu} \toder{\delta x^{\lambda}}{\tau}
    \toder{x^{\nu}}{\tau} \dd \tau &= \bigg[g_{\lambda\nu} \delta x^{\lambda}
    \toder{x^{\nu}}{\tau} \bigg]_{A}^{B} - \int_{A}^{B} \delta x^{\lambda}
    \toder{}{\tau} \bigg( g_{\lambda\nu} \toder{x^{\nu}}{\tau} \bigg) \dd\tau \\
    &= -\int_{A}^{B} \delta x^{\lambda} \bigg( \toder{g_{\lambda\nu}}{\tau}
    \toder{x^{\nu}}{\tau} + g_{\lambda\nu}\toder[2]{x^{\nu}}{\tau} \bigg) \dd
    \tau \\
    &= -\int_{A}^{B} \delta x^{\lambda}
    \bigg( \parder{g_{\lambda\nu}}{x^{\mu}} \toder{x^{\mu}}{\tau}
    \toder{x^{\nu}}{\tau} + g_{\lambda\nu}\toder[2]{x^{\nu}}{\tau} \bigg) \dd
    \tau.
  \end{split}
\end{equation}
Allora
\begin{equation}
  \delta S = -\int_{A}^{B} \delta x^{\lambda} \bigg(
  \frac{1}{2} \parder{g_{\mu\nu}}{x^{\lambda}} \toder{x^{\mu}}{\tau}
  \toder{x^{\nu}}{\tau} - \parder{g_{\lambda\nu}}{x^{\mu}}
  \toder{x^{\mu}}{\tau} \toder{x^{\nu}}{\tau} -
  g_{\lambda\nu}\toder[2]{x^{\nu}}{\tau} \bigg) \dd \tau.
\end{equation}
Affinché sia $\delta S = 0$, per l'arbitrarietà della variazione
$\delta x^{\lambda}$ deve annullarsi la quantità fra parentesi, cioè
\begin{equation}
  \bigg(\frac{1}{2} \partial_{\lambda} g_{\mu\nu} - \partial_{\mu}g_{\lambda\nu}
  \bigg) \toder{x^{\mu}}{\tau} \toder{x^{\nu}}{\tau}
  - g_{\lambda\nu}\toder[2]{x^{\nu}}{\tau} = 0.
\end{equation}
Sfruttando il fatto che gli indici muti possono essere scambiati possiamo
scrivere
\begin{equation}
  \partial_{\mu}g_{\lambda\nu} \toder{x^{\mu}}{\tau} \toder{x^{\nu}}{\tau} =
  \frac{1}{2} (\partial_{\mu}g_{\lambda\nu} + \partial_{\nu}g_{\lambda\mu})
  \toder{x^{\mu}}{\tau} \toder{x^{\nu}}{\tau},
\end{equation}
quindi
\begin{equation}
  \frac{1}{2} (\partial_{\lambda} g_{\mu\nu} - \partial_{\mu}g_{\lambda\nu}
  - \partial_{\nu}g_{\lambda\mu}) \toder{x^{\mu}}{\tau}
  \toder{x^{\nu}}{\tau} - g_{\lambda\nu}\toder[2]{x^{\nu}}{\tau} = 0
\end{equation}
e poiché
$(\partial_{\lambda} g_{\mu\nu} - \partial_{\mu}g_{\lambda\nu}
- \partial_{\nu}g_{\lambda\mu})/2 = g_{\rho\lambda}
\tensor{\Gamma}{^{\rho}_{\nu\mu}}$ abbiamo
\begin{equation}
  g_{\rho\lambda} \tensor{\Gamma}{^{\rho}_{\nu\mu}} \toder{x^{\mu}}{\tau}
  \toder{x^{\nu}}{\tau} - g_{\lambda\nu}\toder[2]{x^{\nu}}{\tau} = 0.
\end{equation}
Moltiplicando infine ambo i membri per $g^{\lambda\sigma}$ e ricordando che
$g_{\alpha\beta}g^{\alpha\gamma} = \tensor{\delta}{^{\gamma}_{\beta}}$ giungiamo
all'\index{equazione!della geodetica}equazione della geodetica
\begin{equation}
  \toder[2]{x^{\sigma}}{\tau} + \tensor{\Gamma}{^{\sigma}_{\nu\mu}}
  \toder{x^{\mu}}{\tau} \toder{x^{\nu}}{\tau} = 0.
\end{equation}
Poiché abbiamo ricavato questa equazione determinando il percorso che deve
seguire un corpo in caduta libera in un campo gravitazionale in maniera tale che
la distanza percorsa sia un estremo (e in particolare si tratta spesso di un
minimo), possiamo dare un'interpretazione geometrica \emph{a posteriori}
dell'equazione del moto: un corpo in caduta libera all'interno di un campo
gravitazionale si muoverà lungo un percorso il più breve (o lungo) possibile fra
due eventi nello spazio-tempo deformato dalla presenza del campo e la distanza è
misurata dal tempo proprio.  Un percorso simile è chiamato \emph{geodetica}, da
cui il nome dell'equazione~\eqref{eq:geodetica}.

\section{Limite newtoniano dell'equazione del moto}
\label{sec:limite-newtoniano}

Vogliamo vedere come l'equazione del moto~\eqref{eq:geodetica} di una particella
all'interno di un campo gravitazionale si riduce considerando il limite
newtoniano, vale a dire nelle condizioni
\begin{itemize}
\item il campo gravitazionale è debole, cioè può essere considerato come una
  perturbazione dello spazio-tempo piatto;
\item il campo gravitazionale inoltre è stazionario;
\item la velocità della particella è molto più piccola di quella della luce.
\end{itemize}
La prima condizione può essere formalizzata dicendo che il
\index{tensore!metrico}tensore metrico $g_{\mu\nu}$ deve differire di poco dal
tensore di Minkowski dello spazio-tempo piatto
\begin{equation}
  g_{\mu\nu} = \eta_{\mu\nu} + h_{\mu\nu},
\end{equation}
in cui $\abs{h_{\mu\nu}} \ll 1$ sono piccole correzioni che determinano il campo
gravitazionale.  Poiché sia $g_{\mu\nu}$ sia $\eta_{\mu\nu}$ sono simmetrici
nello scambio degli indici, tale dovrà essere anche $h_{\mu\nu}$.  Osserviamo
che
\begin{equation}
  \tensor{h}{^{\mu}_{\nu}} = g_{\nu\lambda}h^{\mu\lambda} =
  \eta_{\nu\lambda}h^{\mu\lambda} + h_{\nu\lambda}h^{\mu\lambda} =
  \eta_{\nu\lambda}h^{\mu\lambda} + \mathcal{O}(h^{2}),
\end{equation}
cioè al primo ordine della perturbazione della metrica le operazioni di
innalzamento e abbassamento degli indici di $h_{\mu\nu}$ vengono eseguite con il
tensore metrico non perturbato $\eta_{\mu\nu}$ invece che con il tensore metrico
``completo'' $g_{\mu\nu}$.  Nel seguito, trattando campi deboli in cui la
metrica è approssimabile con $g_{\mu\nu} = \eta_{\mu\nu} + h_{\mu\nu}$,
seguiremo sempre questa convenzione.  Inoltre la correzione al primo ordine
negli infinitesimi per il tensore metrico controvariante è data da (vedi
l'appendice~\ref{sec:dimostr-correz-metrica-controvariante})
\begin{equation}
  \label{eq:correzione-metrica-controvariante}
  g^{\mu\nu} = \eta^{\mu\nu} - h^{\mu\nu},
\end{equation}
dove $h^{\mu\nu} = \eta^{\mu\lambda}\eta^{\nu\sigma} h_{\lambda\sigma}$.
Calcoliamo la connessione affine al primo ordine della perturbazione
$h_{\mu\nu}$
\begin{equation}
  \label{eq:christoffel-approx}
  \tensor{\Gamma}{^{\sigma}_{\mu\nu}} = \frac{1}{2}g^{\lambda\sigma}
  (\partial_{\mu}g_{\nu\lambda} + \partial_{\nu}g_{\mu\lambda}
  - \partial_{\lambda}g_{\mu\nu}) \approx \frac{1}{2}\eta^{\lambda\sigma}
  (\partial_{\mu}h_{\nu\lambda} + \partial_{\nu}h_{\mu\lambda}
  - \partial_{\lambda}h_{\mu\nu}).
\end{equation}
Nel paragrafo~\ref{sec:equazione-moto} abbiamo interpretato il tensore metrico
come il potenziale gravitazionale.  Nello studio del limite newtoniano
$h_{\mu\nu}$ rappresenta la deviazione rispetto al caso di assenza di gravità
($g_{\mu\nu} = \eta_{\mu\nu}$), quindi ci aspettiamo che il potenziale
gravitazionale newtoniano sia contenuto interamente in questo termine.

La terza condizione può essere formalizzata come
\begin{equation}
  \toder{x^{i}}{\tau} \ll \toder{t}{\tau},
\end{equation}
quindi l'equazione della geodetica si riduce a
\begin{equation}
  \label{eq:geodetica-approx}
  \begin{split}
    0 &= \toder[2]{x^{\sigma}}{\tau} + \tensor{\Gamma}{^{\sigma}_{\lambda\mu}}
    \toder{x^{\lambda}}{\tau} \toder{x^{\mu}}{\tau} \\
    &= \toder[2]{x^{\sigma}}{\tau} + \tensor{\Gamma}{^{\sigma}_{00}} \bigg(
    \toder{t}{\tau} \bigg)^{2} + 2\tensor{\Gamma}{^{\sigma}_{0i}}\toder{t}{\tau}
    \toder{x^{i}}{\tau} + \tensor{\Gamma}{^{\sigma}_{ij}} \toder{x^{i}}{\tau}
    \toder{x^{j}}{\tau} \\
    &\approx \toder[2]{x^{\sigma}}{\tau} + \tensor{\Gamma}{^{\sigma}_{00}}\bigg(
    \toder{t}{\tau} \bigg)^{2}.
  \end{split}
\end{equation}
Dalla~\eqref{eq:christoffel-approx} risulta
\begin{equation}
  \tensor{\Gamma}{^{\sigma}_{00}} \approx
  \frac{1}{2}\eta^{\lambda\sigma}(\partial_{0}h_{0\lambda}
  + \partial_{0}h_{0\lambda} - \partial_{\lambda}h_{00}) =
  -\frac{1}{2}\eta^{\lambda\sigma} \partial_{\lambda}h_{00},
\end{equation}
in quanto le derivate temporali di $h_{\mu\nu}$ sono nulle in base alla seconda
condizione che descrive il limite newtoniano.  Notiamo che per $\sigma=0$ si ha
$\tensor{\Gamma}{^{0}_{00}}=0$.  Allora la~\eqref{eq:geodetica-approx}
corrisponde, al variare di $\sigma$, alle equazioni
\begin{subequations}
  \begin{align}
    \toder[2]{t}{\tau} &= 0 &\text{per $\sigma = 0$}, \\
    \toder[2]{x^{i}}{\tau} &= \frac{1}{2} \bigg( \toder{t}{\tau} \bigg)^{2}
    (\nabla h_{00})_{i} &\text{per $\sigma = i$}.
  \end{align}
\end{subequations}
La prima equazione ci dice che $\ltoder{t}{\tau}$ è una costante, mentre dalla
seconda, dividendo per $(\ltoder{t}{\tau})^{2}$, abbiamo
\begin{equation}
  \toder[2]{\bm{x}}{t} = \frac{1}{2}\nabla h_{00}.
\end{equation}
L'equazione newtoniana del moto è
\begin{equation}
  \bm{a}_{\textup{N}} = \toder[2]{\bm{x}}{t} = -\nabla\phi,
\end{equation}
in cui $\phi$ è il potenziale newtoniano del campo gravitazionale, legato alla
distribuzione di massa $\rho$
\index{equazione!di Poisson!per il campo gravitazionale}dall'equazione di
Poisson $\nabla^{2} \phi = 4\pi G\rho$.  Confrontando le due equazioni
precedenti troviamo
\begin{equation}
  h_{00} = -2\phi + \text{costante}.
\end{equation}
A grandi distanze dal corpo, cioè nel limite $r \to \infty$, il potenziale
newtoniano $\phi$ si annulla, il campo gravitazionale è assente e il tensore
metrico $g_{\mu\nu}$ deve tendere al tensore di Minkowski $\eta_{\mu\nu}$ dello
spazio-tempo piatto, cioè si devono annullare tutte le componenti di
$h_{\mu\nu}$.  Pertanto ricaviamo che la costante deve essere uguale a $0$ e
$h_{00} = -2\phi$ o, ripristinando la velocità $c$ della luce nel vuoto,
$h_{00} = -2\phi/c^{2}$.  Nel caso particolare di un campo gravitazionale
prodotto da un corpo sferico di massa $M$ e raggio $R$, l'espressione del
potenziale a distanza $r > R$ è $\phi = -GM/r$, quindi
\begin{equation}
  \label{eq:h00}
  h_{00} = -\frac{2\phi}{c^{2}} = \frac{2GM}{rc^{2}} = \frac{r_{\textup{S}}}{r},
\end{equation}
in cui
\begin{equation}
  r_{\textup{S}} = \frac{2GM}{c^{2}}
\end{equation}
è il \index{raggio!di Schwarzschild}\emph{raggio di Schwarzschild} (o
\emph{raggio gravitazionale}) del corpo che genera il campo gravitazionale.
Dunque abbiamo trovato che $h_{00}$ è esattamente il potenziale gravitazionale
newtoniano, a meno di fattori moltiplicativi costanti, coerentemente con quanto
previsto all'inizio della discussione del problema.  La componente $g_{00}$ del
tensore metrico è
\begin{equation}
  g_{00} = \eta_{00} + h_{00} = - \bigg(1 - \frac{2GM}{rc^{2}} \bigg).
\end{equation}
Poiché abbiamo ottenuto questi risultati nell'ipotesi di campo debole, cioè
$\abs{h_{\mu\nu}} \ll 1$, la quantità $h_{00} = r_{\textup{S}}/r$ dà una misura
delle deviazioni della relatività generale rispetto alle previsioni della fisica
newtoniana.

\begin{table}
  \centering
  \caption[Valori del raggio di Schwarzschild per diversi corpi]{Valori del
    raggio di Schwarzschild $r_{\textup{S}}$ per diversi oggetti e valore della
    correzione $r_{\textup{S}}/r$ sulla superficie dei corpi.  $r$ è il raggio
    degli oggetti.  Questi valori sono approssimativi, sono rilevanti gli ordini
    di grandezza}
  \label{tab:Schwarzschild}
  \begin{tabular}{lSSSS}
    \toprule
    corpo & {$r_{\textup{S}}$ (\si{\metre})} & {$r$ (\si{\metre})} &
    {$r_{\textup{S}}/r$} \\
    \midrule
    Terra              & 8.7e-3 & 6.4e6 & 1.4e-9 \\
    Sole               & 3e3    & 7e8   & 4.3e-6 \\
    nana bianca        & 3e3    & 1e6   & 3e-3   \\
    stella di neutroni & 3e3    & 1e4   & 0.3    \\
    buco nero stellare & 3e4    & 3e4   & 1      \\
    \bottomrule
  \end{tabular}
\end{table}
Nella tabella~\ref{tab:Schwarzschild} sono riportati i valori delle deviazioni
$r_{\textup{S}}/r$ dalla gravitazione newtoniana sulla superficie di alcuni
corpi.  Mentre per la Terra, il Sole e le nane bianche queste correzioni sono
molto piccole, non è possibile descrivere le vicinanze di corpi come le stelle
di neutroni o i buchi neri utilizzando la fisica newtoniana ma bisogna tenere
necessariamente in considerazione la relatività generale.

\section{Campo gravitomagnetico}
\label{sec:campo-gravitomagnetico}

Nella teoria newtoniana della gravitazione la velocità di una particella in
orbita attorno a un corpo di massa $M$ a distanza $r$ è data, per il teorema del
viriale $2K = -U \iff mv^{2} = GMm/r$, da $v = \sqrt{GM/r}$.  Nel paragrafo
precedente abbiamo studiato l'equazione del moto sotto le ipotesi di campo
debole ($\abs{h_{\mu\nu}} \sim GM/r \sim v^{2} \ll 1$), statico e velocità non
relativistiche ($v \ll 1$) trascurando i termini dell'ordine di
$\abs{h_{\mu\nu}}v^{2} \sim v^{4}$ e $\abs{h_{\mu\nu}}v \sim v^{3}$.  In questo
paragrafo studieremo l'equazione della geodetica con un livello di
approssimazione minore: continueremo a trascurare i termini dell'ordine di
$\abs{h_{\mu\nu}}v^{2}$ ma non quelli dell'ordine di $\abs{h_{\mu\nu}}v$.

Nelle ipotesi illustrate, le componenti $k$ dell'equazione della
geodetica~\eqref{eq:geodetica} sono date da
\begin{equation}
    \begin{split}
    0 &= \toder[2]{x^{k}}{\tau} + \tensor{\Gamma}{^{k}_{\lambda\mu}}
    \toder{x^{\lambda}}{\tau} \toder{x^{\mu}}{\tau} \\
    &= \toder[2]{x^{k}}{\tau} + \tensor{\Gamma}{^{k}_{00}} \bigg(
    \toder{t}{\tau} \bigg)^{2} + 2\tensor{\Gamma}{^{k}_{0i}}\toder{t}{\tau}
    \toder{x^{i}}{\tau} + \tensor{\Gamma}{^{k}_{ij}} \toder{x^{i}}{\tau}
    \toder{x^{j}}{\tau} \\
    &\approx \toder[2]{x^{\sigma}}{\tau} + \tensor{\Gamma}{^{\sigma}_{00}}\bigg(
    \toder{t}{\tau} \bigg)^{2} + 2\tensor{\Gamma}{^{k}_{0i}}\toder{t}{\tau}
    \toder{x^{i}}{\tau}.
  \end{split}
\end{equation}
Poiché consideriamo ancora velocità non relativistiche poniamo
$\dd \tau \approx \dd t$ e
$\ltoder{x^{k}}{\tau} \approx \ltoder{x^{k}}{t} = v^{k}$
\begin{equation}
  \label{eq:gem1}
  \toder[2]{x^{k}}{t} + \tensor{\Gamma}{^{k}_{00}} +
  2\tensor{\Gamma}{^{k}_{0i}}v^{i} = 0.
\end{equation}
Ricordando che stiamo trattando campo gravitazionali statici, le componenti
della connessione affine che ci servono sono
\begin{subequations}
  \begin{align}
    \tensor{\Gamma}{^{k}_{00}} &=
    \frac{1}{2}\eta^{k\sigma}(\partial_{0}h_{0\sigma} + \partial_{0}h_{0\sigma}
    - \partial_{\sigma}h_{00}) = - \frac{1}{2}\partial^{k}h_{00}, \\
    \tensor{\Gamma}{^{k}_{0i}} &= \tensor{\Gamma}{^{k}_{i0}} =
    \frac{1}{2}\eta^{k\sigma} (\partial_{0}h_{i\sigma} + \partial_{i}h_{0\sigma}
    - \partial_{\sigma}h_{0i}) = \frac{1}{2}(\partial_{i}\tensor{h}{_{0}^{k}}
    - \partial^{k}h_{0i}).
  \end{align}
\end{subequations}
Abbassando gli indici liberi nell'equazione~\eqref{eq:gem1} così abbiamo
\begin{equation}
  \label{eq:gem2}
  \toder{v^{k}}{t} = \frac{1}{2}\partial_{k}h_{00} + (\partial_{k}h_{0i}
  - \partial_{k}h_{0k})v^{i}.
\end{equation}
Definiamo la matrice $4 \times 4$ antisimmetrica di componenti $f_{\alpha\beta}$
date da
\begin{equation}
  f_{\alpha\beta} = \frac{1}{2}(h_{0\beta,\alpha} - h_{0\alpha,\beta}) =
  -f_{\beta\alpha}.
\end{equation}
Osserviamo che non si tratta di un vero tensore di rango $2$ perché questa
quantità si trasforma come le componenti $0\alpha\beta$ di un tensore di rango
$3$.  Definiamo il vettore tridimensionale $\bm{g}$ di componenti
\begin{equation}
  g_{k} = -f_{0k} = \frac{1}{2}h_{00,k} = \frac{1}{2}(\nabla h_{00})_{k} =
  -(\nabla \phi)_{k},
\end{equation}
in cui abbiamo ricordato il risultato $h_{00} = -2\phi$ trovato nel paragrafo
precedente.  Osserviamo che $\bm{g} = -\nabla \phi$, quindi $\bm{g}$ è proprio
il potenziale gravitazionale newtoniano.  Con queste posizioni l'equazione della
geodetica~\eqref{eq:gem2} diventa
\begin{equation}
  \label{eq:gem3}
  \toder{v_{k}}{t} = g_{k} + 2f_{ki}v^{i}.
\end{equation}

Inoltre definiamo il vettore
tridimensionale $\bm{b}$ di componenti
\begin{subequations}
  \begin{align}
    b_{1} &= b_{x} = h_{03,2} - h_{02,3} = -2f_{32} = 2f_{23}, \\
    b_{2} &= b_{y} = h_{01,3} - h_{03,1} = -2f_{13} = 2f_{31}, \\
    b_{3} &= b_{z} = h_{02,1} - h_{01,2} = -2f_{21} = 2f_{12}.
  \end{align}
\end{subequations}
In maniera più compatta possiamo indicare ciascuna componente del vettore
$\bm{b}$ con (si somma su $j$ e $k$)
\begin{equation}
  b_{i} = \epsilon_{ijk}h_{0k,j} = \epsilon_{ijk}\partial_{j}h_{k},
\end{equation}
dove $\epsilon^{ijk}$ è il simbolo di Levi-Civita tridimensionale e abbiamo
definito il vettore tridimensionale
$\bm{h} = (h_{1}, h_{2}, h_{3}) = (h_{01}, h_{02}, h_{03})$.  Questo ci permette
di scrivere $\bm{b}$ come il rotore di $\bm{h}$
\begin{equation}
  \bm{b} = \nabla \times \bm{h}.
\end{equation}
La matrice $f_{\alpha\beta}$ è allora data da
\begin{equation}
  f_{\alpha\beta} =
  \begin{pmatrix}
    0     & -g_{x}   & -g_{y}   & -g_{z}   \\
    g_{x} & 0        & b_{z}/2  & -b_{y}/2 \\
    g_{y} & -b_{z}/2 & 0        & b_{x}/2  \\
    g_{z} & b_{y}/2  & -b_{x}/2 & 0
  \end{pmatrix}.
\end{equation}
Questa matrice è molto simile al tensore
elettromagnetico~\eqref{eq:tensore-elettromagnetico}, con il campo
gravitazionale newtoniano $\bm{g}$ che svolge il ruolo del campo elettrico
$\bm{E}$ e $\bm{b}/2$ quello del campo magnetico $\bm{B}$.  Data questa
similitudine, $\bm{b}$ viene chiamato \emph{campo gravitomagnetico}.  Una sfera
carica ferma genera solo un campo elettrico, una sfera in rotazione genera anche
un campo magnetico.  In analogia, una massa in rotazione genera un campo
gravitomagnetico non nullo.

Il parallelo con il campo elettromagnetico non si ferma qui.  Innanzitutto
notiamo che
\begin{equation}
  \begin{split}
    \epsilon_{kij}b^{j} &= \epsilon_{kij}\epsilon^{jlm}h_{0m,l} =
    -\epsilon_{jik}\epsilon^{jlm} h_{0m,l} = (\tensor{\delta}{_{i}^{m}}
    \tensor{\delta}{_{k}^{l}} - \tensor{\delta}{_{i}^{l}}
    \tensor{\delta}{_{k}^{m}}) h_{0m,l} = h_{0i,k} - h_{0k,i} \\
    &= 2f_{ki}.
  \end{split}
\end{equation}
Quindi la componente $k$ del prodotto vettoriale fra il vettore velocità
$\bm{v}$ e il campo gravitomagnetico $\bm{b}$ è
\begin{equation}
  (\bm{v}\times\bm{b})_{k} = \epsilon_{ijk}v^{i}b^{j} = 2f_{ki}v^{i}.
\end{equation}
Allora possiamo scrivere l'equazione della geodetica~\eqref{eq:gem3} nella
seguente forma vettoriale
\begin{equation}
  \toder{\bm{v}}{t} = \bm{g} + \bm{v}\times\bm{b}.
\end{equation}
Quindi $\bm{g}$ e $\bm{b}$ soddisfano un'equazione analoga a quella della forza
di Lorentz.  Inoltre usando la definizione di $f_{\alpha\beta}$ si verifica che
risulta
\begin{equation}
  f_{\alpha\beta,\gamma} + f_{\gamma\alpha,\beta} + f_{\beta\gamma,\alpha} = 0,
\end{equation}
che è un'equazione analoga all'equazione di Maxwell senza
sorgenti~\eqref{eq:maxwell-cov22}.  Poiché da questa equazione derivano le
equazioni di Maxwell $\nabla \cdot \bm{B} = 0$ e
$\nabla\times\bm{E} = - \lparder{\bm{B}}{t}$, deduciamo immediatamente che per
$\bm{g}$ e $\bm{b}$ deve inoltre risultare
\begin{subequations}
  \begin{align}
    \nabla \cdot \bm{b} &= 0, \\
    \nabla\times\bm{g} &= -\frac{1}{2}\parder{\bm{b}}{t}.
  \end{align}
\end{subequations}
Nel limite newtoniano il potenziale gravitazionale $\phi$ soddisfa
\index{equazione!di Poisson!per il campo gravitazionale}l'equazione di Poisson
$\nabla^{2} \phi = 4\pi G \rho$, quindi ricordando che $\bm{g} = -\nabla \phi$
otteniamo l'ulteriore relazione
\begin{equation}
  \nabla \cdot \bm{g} = -4\pi G \rho.
\end{equation}
Per trovare l'ultima ``equazione di Maxwell'' per il campo gravitomagnetico
dobbiamo sfruttare la seguente equazione valida per un campo gravitazionale
debole e indipendente dal
tempo\footnote{Per una dimostrazione di questa relazione
  vedi~\textcite[139]{ohanian:gravitazione}.}
% Nota: l'Ohanian sbaglia un segno.  Per determinare correttamente l'equazione
% ho usato l'espressione lineare (primo ordine di h) di R_{\mu\nu} calcolata nel
% paragrafo sulle onde gravitazionali nel vuoto.  Si trova subito che la
% componente 0k è quella che ho scritto qui.
\begin{equation}
  \nabla^{2}h^{0k} - \partial_{l}\partial^{k} h^{0l} = -16\pi GT^{0k},
\end{equation}
in cui $T^{0k} = \rho v^{k}$ è la componente $0k$ del tensore energia-impulso,
quindi rappresenta la densità di flusso di energia nella direzione $k$.  Questa
equazione può anche essere scritta come
\begin{equation}
  -16\pi GT^{0k} = \partial_{l}(h^{0k,l} - h^{0l,k}) = \partial_{l}f^{lk} =
  \epsilon^{lkj}\partial_{l}b_{j} = -\epsilon^{klj}\partial_{l}b_{j} = -(\nabla
  \times \bm{b})^{k},
\end{equation}
oppure nella seguente forma vettoriale
\begin{equation}
  \nabla \times \bm{b} = 16\pi G \bm{S},
\end{equation}
in cui $\bm{S} = (T^{01}, T^{02}, T^{03})$ è l'analogo del
\index{vettore!di Poynting}vettore di Poynting $\bm{E}\times\bm{B}/4\pi$ per il
campo gravitomagnetico.

\subsection{\completare{Effetto Lense-Thirring}}
\label{sec:effetto-lense-thirring}

Il Sole e la Terra hanno un raggio di Schwarzschild molto più piccolo del loro
raggio fisico, quindi i campi gravitazionali al loro esterno sono deboli,
inoltre sono in lenta (non relativistica) rotazione, quindi il moto di
particelle nel campo gravitazionale di questi corpi può essere descritto usando
il campo gravitomagnetico.  Sono in corso degli esperimenti volti a verificare
le previsioni della teoria che abbiamo sviluppato nella sezione precedente.  In
particolare si prova a misura la precessione di un giroscopio in orbita intorno
al nostro pianeta dovuta al moto di rotazione della Terra.  Questo fenomeno
prende il nome di \emph{effetto Lense-Thirring}.

Ricordiamo che in elettrodinamica classica dalle equazioni
\begin{subequations}
  \begin{align}
    \nabla\times\bm{B} &= 4\pi \bm{J}, \\
    \bm{B} &= \nabla\times\bm{A},
  \end{align}
\end{subequations}
insieme alla condizione di Lorenz $\nabla\cdot\bm{A} = 0$ si ottiene
\begin{equation}
  \nabla^{2} \bm{A} = -4\pi\bm{J}.
\end{equation}
Una soluzione di questa equazione è data da
\begin{equation}
  \bm{A}(\bm{r}) = \int \frac{\bm{J}(\bm{r}') \dd^{3} \bm{r}'}{\norm{\bm{r} -
      \bm{r}'}}.
\end{equation}
Se la distribuzione di corrente $\bm{J}$ è localizzata in una regione piccola
rispetto alle dimensioni dell'intero spazio, cioè
$\norm{\bm{r}} \ll \norm{\bm{r}'}$, allora possiamo sviluppare il denominatore
della funzione integranda in serie di potenze
\begin{equation}
  \frac{1}{\norm{\bm{r} - \bm{r}'}} = \frac{1}{\norm{\bm{r}}} +
  \frac{\bm{r}\cdot\bm{r}'}{\norm{\bm{r}}^{3}} + \cdots
\end{equation}
e si trova che il primo termine dello sviluppo in multipoli del potenziale
vettore è
\begin{equation}
  \bm{A}(\bm{r}) = \frac{\bm{m} \times \bm{r}}{\norm{\bm{r}}^{3}},
\end{equation}
in cui
\begin{equation}
  \bm{m} = \frac{1}{2} \int \bm{r} \times \bm{J}(\bm{r}) \dd^{3} \bm{r}
\end{equation}
è il momento di dipolo magnetico.  Da qui si ha anche
\begin{equation}
  \bm{B} = \nabla \times \bm{A} = \frac{3 \versor{r}(\versor{r} \cdot \bm{m}) -
    \bm{m}}{\norm{\bm{r}}^{3}}.
\end{equation}

Per il campo gravitomagnetico possiamo trovare relazioni analoghe.  Abbiamo
visto che $\nabla \times \bm{b} = 16 \pi G \bm{S}$ e
$\bm{b} = \nabla \times \bm{h}$, allora imponendo una condizione analoga a
quella di Lorenz $\nabla \cdot \bm{h} = 0$ abbiamo
\begin{equation}
  \nabla \times \nabla \times \bm{h} = \nabla (\nabla \cdot \bm{h}) - \nabla^{2}
  \bm{h} = - \nabla^{2} \bm{h} = 16\pi G \bm{S}.
\end{equation}
Una soluzione sarà quindi data da
\begin{equation}
  % Nota: il segno di h è diverso da quello riportato sull'Ohanian ma è
  % concorde con l'espressione di h_{\mu\nu} trovata nel paragrafo sulle onde
  % gravitazionali in presenza di materia, quando abbiamo trovato la soluzione
  % col potenziale ritardato.  Inoltre quando l'Ohanian fa i calcoli per le
  % onde gravitazionali ottiene di nuovo h_{\mu\nu} = - \int ···, quindi c'è
  % sempre una discordanza di segni, tutto questo mi fa sperare che i miei
  % calcoli siano corretti.  L'articolo arXiv:gr-qc/9912027 ha tutti i segni
  % concordi ai miei (lui fa il calcolo con il potenziale ritardato, il suo B è
  % il mio b/2), compreso il segno, alla fine, di b e della velocità angolare di
  % precessione di Lense-Thirring.
  \bm{h}(\bm{r}) = 4 G \int \frac{\bm{S}(\bm{r}') \dd^{3} \bm{r}'}{\norm{\bm{r}
      - \bm{r}'}}
\end{equation}
e per sistemi localizzati definiamo l'analogo gravitazionale del momento di
dipolo magnetico
\begin{equation}
  \bm{m}_{\textup{g}} = \frac{1}{2} \int \bm{r} \times \bm{S} \dd^{3} \bm{r} =
  \frac{1}{2} \int \bm{r} \times \rho \bm{v} \dd^{3} \bm{r} = \frac{1}{2} \bm{L},
\end{equation}
in cui $\bm{L} = \int \bm{r} \times \rho \bm{v} \dd^{3} \bm{r}$ è per
definizione il momento angolare totale classico.  Quindi abbiamo anche
\begin{subequations}
  \begin{align}
    \bm{h} &= 4G \frac{\bm{m}_{\textup{g}} \times \bm{r}}{\norm{\bm{r}}^{3}} =
    2G \frac{\bm{L} \times \bm{r}}{\norm{\bm{r}}^{3}}, \\[1ex]
    \bm{b} &= \nabla \times \bm{h} = 2G\frac{3\versor{r}(\versor{r} \cdot
      \bm{L}) - \bm{L}}{\norm{\bm{r}}^{3}}.
  \end{align}
\end{subequations}



% TODO: scrivere questa sezione.  Lezione del 20/04/2012
\section{\completare{Red-shift gravitazionale}}
\label{sec:red-shift-gravitazionale}

% TODO: completare questa sezione.
\section{\completare{Principio di generale covarianza}}
\label{sec:generale-covarianza}

Dal \index{principio!di equivalenza}principio di equivalenza si può dedurre un
altro principio, il \index{principio!di generale covarianza}
\emph{principio di generale covarianza} che afferma che
\emph{un'equazione fisica è valida in qualsiasi campo gravitazionale se
  \begin{itemize}
  \item l'equazione è valida in assenza di gravità, quindi soddisfa le leggi
    della relatività speciale;
  \item l'equazione è \emph{generalmente covariante}, cioè è invariante in forma
    per trasformazioni arbitrarie delle coordinate $x^{\mu} \to x'^{\mu}$.
  \end{itemize}
}
Questo principio è applicabile solo in regioni di spazio-tempo sufficientemente
piccole, in particolare molto più piccole delle distanze alle quali agisce il
campo gravitazionale considerato, poiché solo in queste regioni il principio di
equivalenza, dal quale deriva il principio di generale covarianza, assicura la
possibilità di individuare un sistema localmente inerziale.

%%% Local Variables:
%%% mode: latex
%%% TeX-master: "../gravitazione"
%%% fill-column: 80
%%% End:
