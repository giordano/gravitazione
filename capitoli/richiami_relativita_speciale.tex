\chapter{Richiami di relatività speciale}
\label{cha:relativita-speciale}

\section{Trasformazioni di Lorentz}
\label{sec:trasformazioni-lorentz}

\index{coordinata controvariante}Coordinata controvariante
\begin{equation}
  \label{eq:coord-controvariante}
  x^{\alpha} = (x^{0},x^{i}) = (t, \bm{x}),
\end{equation}
con $\alpha$ che varia fra $0$, $1$, $2$ e $3$ e $i$ fra $1$, $2$ e $3$.

Nello spazio ordinario si defisce
\begin{equation}
  \dd l^{2}   = \dd x^{2} + \dd y^{2} + \dd z^{2} = \sum_{i} (\dd x^{i})^{2} =
  \delta_{ij}\dd x^{i} \dd x^{j},
\end{equation}
con
\begin{equation}
  \delta_{ij} =
  \begin{pmatrix}
    1 & 0 & 0 \\
    0 & 1 & 0 \\
    0 & 0 & 1
  \end{pmatrix}.
\end{equation}
Il trasformato della coordinata controvariante~\eqref{eq:coord-controvariante} è
\begin{equation}
  x'^{i} = \tensor{R}{^{i}_{j}}x^{j} + a^{i}
\end{equation}
e la matrice $R$ è univocamente definita imponendo la condizione che $\dd l^{2}$
sia invariante cambiando sistema di riferimento.  Si scopre che $R$ è la matrice
di rotazione.

Nello spazio di Minkowski definiamo
\begin{equation}
  \dd \tau^{2} = \dd t^{2} - \dd \bm{x}^{2} = -\tensor{\eta}{_{\alpha\beta}}\dd
  x^{\alpha} \dd x^{\beta},
\end{equation}
in cui $\tensor{\eta}{_{\alpha\beta}}$ è il
\index{tensore!metrico covariante di
  Minkowski}\emph{tensore metrico covariante di Minkowski}
\begin{equation}
  \label{eq:tensore-metrico-controvariante}
  \tensor{\eta}{_{\alpha\beta}} =
  \begin{pmatrix}
   \eta_{00} & \eta_{01} & \eta_{02} & \eta_{03} \\
   \eta_{10} & \eta_{11} & \eta_{12} & \eta_{13} \\
   \eta_{20} & \eta_{21} & \eta_{22} & \eta_{23} \\
   \eta_{30} & \eta_{31} & \eta_{32} & \eta_{33}
  \end{pmatrix}
  =
  \begin{pmatrix}
    -1 & 0 & 0 & 0 \\
    0 & 1 & 0 & 0 \\
    0 & 0 & 1 & 0 \\
    0 & 0 & 0 & 1
  \end{pmatrix}.
\end{equation}
Questo tensore definisce una metrica pseudo euclidea con segnatura $(-+++)$.
Imponendo che $\dd \tau$ sia invariante per trasformazioni di Lorentz possiamo
passare da un sistema di riferimento inerziale $O$ a un altro sistema di
riferimento inerziale $O'$ in moto rispetto a $O$.  Consideriamo due eventi
$E_{1}$ ed $E_{2}$ relativi a un fotone
\begin{gather}
  E_{1}: \quad (0,\bm{0}), \\
  E_{2}: \quad (\dd t, \dd \bm{x})
\end{gather}
\begin{equation}
  \dd \tau^{2} = \dd t^{2}\Big( 1 - \toder{\bm{x}^{2}}{t^{2}}\Big) = \dd t^{2}
  (1 - 1^{2}) = 0
\end{equation}
perché per un fotone si ha
\begin{equation}
  \bm{v}^{2} = \Big(\toder{\bm{x}}{t}\Big)^{2} = c^{2} = 1.
\end{equation}


\begin{equation}
  \dd \tau'^{2} = -\tensor{\eta}{_{\alpha\beta}}\dd x'^{\alpha}\dd x'^{\beta}
\end{equation}
Richiediamo che la trasformazione $x^{\alpha} \to x'^{\alpha}$ sia lineare
per rispettare la condizione di omogeneità dello spazio (la lunghezza di una
sbarra non deve dipendere dalla posizione degli estremi), del tempo (la
lunghezza della sbarra non deve dipendere dall'istante di tempo in cui la
misuriamo) e di isotropia dello spazio (la lunghezza della sbarra non deve
dipendere neanche dalla direzione in cui la guardiamo).  La più generica
trasformazione lineare è
\begin{equation}
  \label{eq:trasformazione-lineare}
  x'^{\alpha} = \tensor{\Lambda}{^{\alpha}_{\beta}}x^{\beta} + a^{\alpha}.
\end{equation}
Imponiamo che l'intervallo $\dd \tau^{2}$ non vari passando da un sistema di
riferimento all'altro, cioè che $\dd \tau^{2} = \dd \tau'^{2}$, così abbiamo
\begin{equation}
  \dd \tau'^{2} = -\tensor{\eta}{_{\alpha\beta}}
  \tensor{\Lambda}{^{\alpha}_{\gamma}} \dd x^{\gamma}
  \tensor{\Lambda}{^{\beta}_{\delta}} \dd x^{\delta} = \dd \tau^{2} =
  -\tensor{\eta}{_{\alpha\beta}} \dd x^{\alpha} \dd x^{\beta},
\end{equation}
da cui, semplificando negli ultimi due membri $\dd x^{\gamma}\dd x^{\delta}$,
otteniamo la relazione fondamentale del tensore metrico
\begin{equation}
  \label{eq:rel-fond-tensore-metrico}
  \tensor{\eta}{_{\alpha\beta}} \tensor{\Lambda}{^{\alpha}_{\gamma}}
  \tensor{\Lambda}{^{\beta}_{\delta}} = \tensor{\eta}{_{\gamma\delta}}.
\end{equation}
Questa è la condizione che deve essere soddisfatta da una
trasformazione~\eqref{eq:trasformazione-lineare} affinché sia una trasformazione
di Lorentz.  L'equazione~\eqref{eq:rel-fond-tensore-metrico} impone $10$
condizioni indipendenti in quanto il tensore metrico
$\tensor{\eta}{_{\gamma\delta}}$ è simmetrico quindi solo $10$ delle sue $16$
componenti sono indipendenti.

L'equazione matriciale equivalente alla~\eqref{eq:rel-fond-tensore-metrico} è
\begin{equation}
  \Lambda^{\textup{T}}\eta\Lambda = \eta
\end{equation}
dalla quale abbiamo
\begin{equation}
  \det\eta = \det(\Lambda^{\textup{T}}\eta\Lambda) = \det\eta
  \det(\Lambda^{\textup{T}}\Lambda) = \det\eta (\det\Lambda)^{2}
\end{equation}
e quindi deve valere la condizione
\begin{equation}
  \det\Lambda = \pm 1.
\end{equation}
Le trasformazioni di Lorentz con $\det\Lambda = +1$ sono chiamate
\emph{proprie}, quelle con $\det\Lambda = -1$ sono chiamate \emph{improprie}.

Consideriamo una particella a riposo nell'origine del sistema di riferimento $O$
(quindi $\dd \bm{x} = \bm{0}$) che si muove con velocità
$\bm{v} = (v^{1},v^{2},v^{3})$ rispetto a un altro sistema di riferimento
inerziale $O'$
% TODO: probabilmente questi calcoli andranno spezzati in più ambienti
% `equation' spiegando che passaggi vengono fatti in mezzo
\begin{gather}
  \dd x'^{\alpha} = \tensor{\Lambda}{^{\alpha}_{\beta}} \dd x^{\beta} =
  \tensor{\Lambda}{^{\alpha}_{0}} \dd x^{0} + \tensor{\Lambda}{^{\alpha}_{i}}\dd
  x^{i} \\
  \dd x'^{i} = \tensor{\Lambda}{^{i}_{0}}\dd t \\
  \dd x'^{0} = \tensor{\Lambda}{^{0}_{0}}\dd t.
\end{gather}
Dividendo membro a membro abbiamo
\begin{equation}
  \toder{x'^{i}}{x'^{0}} = \toder{x'^{i}}{t'} = v_{i} =
  \frac{\tensor{\Lambda}{^{i}_{0}}}{\tensor{\Lambda}{^{0}_{0}}},
\end{equation}
da cui
\begin{equation}
  \tensor{\Lambda}{^{i}_{0}} = v_{i}\tensor{\Lambda}{^{0}_{0}}.
\end{equation}
Ponendo $\gamma = \delta = 0$ nella~\eqref{eq:rel-fond-tensore-metrico} risulta
\begin{equation}
  \begin{split}
    \tensor{\eta}{_{00}} &= -1 = \tensor{\eta}{_{\alpha\beta}}
    \tensor{\Lambda}{^{\alpha}_{0}} \tensor{\Lambda}{^{\beta}_{0}} =
    \tensor{\eta}{_{\alpha\beta}}(\tensor{\Lambda}{^{0}_{0}})^{2} +
    \tensor{\eta}{_{ij}} \tensor{\Lambda}{^{i}_{0}} \tensor{\Lambda}{^{j}_{0}}
    \\
    &= -(\tensor{\Lambda}{^{0}_{0}})^{2} + \delta_{ij}
    \tensor{\Lambda}{^{i}_{0}} \tensor{\Lambda}{^{j}_{0}} =
    -(\tensor{\Lambda}{^{0}_{0}})^{2} + \sum_{i}(\tensor{\Lambda}{^{i}_{0}})^{2},
  \end{split}
\end{equation}
quindi
\begin{equation}
  (\tensor{\Lambda}{^{0}_{0}})^{2} = 1 +
  \sum_{i}(\tensor{\Lambda}{^{i}_{0}})^{2} = 1 +
  \sum_{i}(v_{i}\tensor{\Lambda}{^{0}_{0}})^{2} \geq 1,
\end{equation}
cioè deve risultare
\begin{equation}
  \tensor{\Lambda}{^{0}_{0}} \geq 1 \text{ oppure } \tensor{\Lambda}{^{0}_{0}}
  \leq -1.
\end{equation}
Le trasformazioni con $\tensor{\Lambda}{^{0}_{0}} \geq +1$ sono chiamate
\emph{ortocrone}, quelle con $\tensor{\Lambda}{^{0}_{0}} \leq -1$
\emph{anticrone}.  Le trasformazioni di Lorentz proprie e ortocrone sono dette
\emph{trasformazioni di Lorentz ristrette} o \emph{speciali}.  Considerando una
trasformazione ristretta inoltre abbiamo
\begin{equation}
  \tensor{\Lambda}{^{0}_{0}} = \frac{1}{\sqrt{1-\sum_{i}(v_{i})^{2}}} =
  \frac{1}{\sqrt{1-v^{2}}} = \gamma.
\end{equation}
Quindi risulta anche
\begin{equation}
  \tensor{\Lambda}{^{i}_{0}} = v_{i}\tensor{\Lambda}{^{0}_{0}} = v_{i}\gamma.
\end{equation}
Le altre componenti $\tensor{\Lambda}{^{\alpha}_{\beta}}$ non sono univocamente
determinate.  Una scelta conveniente che permette di soddisfare
la~\eqref{eq:trasformazione-lineare} è
\begin{subequations}
  \begin{align}
    \tensor{\Lambda}{^{0}_{i}} &= \tensor{\Lambda}{^{i}_{0}} = v_{i}\gamma, \\
    \tensor{\Lambda}{^{i}_{j}} &= \delta_{ij} +
    v_{i}v_{j}\frac{\gamma-1}{v^{2}}.
  \end{align}
\end{subequations}

% TODO: scrivere qualcosa!
\section{Conseguenze delle trasformazioni di Lorentz}
\label{sec:conseguenze-lorentz}

\subsection{Dilatazione degli intervalli temporali}
\label{sec:dilatazione-tempo}

\subsection{Effetto Doppler}
\label{sec:effetto-doppler}

\section{Calcolo tensoriale nello spazio di Minkowski}
\label{sec:calcolo-tensoriale}

Un \index{quadrivettore!controvariante}\emph{quadrivettore controvariante}
$V^{\alpha}$ è un insieme di quattro quantità $V^{0}$, $V^{1}$ e $V^{2}$,
$V^{3}$ che, mediante una trasformazione di Lorentz ristretta
$x^{\alpha} \to x'^{\alpha} = \tensor{\Lambda}{^{\alpha}_{\beta}}x^{\beta}$, si
trasformano come
\begin{equation}
  V^{\alpha} \to V'^{\alpha} = \tensor{\Lambda}{^{\alpha}_{\beta}}V^{\beta}.
\end{equation}
Un quadrivettore controvariante può anche essere rappresentato come
\begin{equation}
  V^{\alpha} = (V^{0}, V^{1}, V^{2}, V^{3}) = (V^{0}, V^{i}) = (V^{0}, \bm{V}),
\end{equation}
dove $\bm{V} = (V^{1}, V^{2}, V^{3})$.

Un \index{quadrivettore!covariante}\emph{quadrivettore covariante} $U_{\alpha}$
è un insieme di quattro quantità $U_{0}$, $U_{1}$, $U_{2}$ e $U_{3}$ che, sotto
una trasformazione di Lorentz ristretta, si trasformano come
\begin{equation}
  U_{\alpha} \to U'_{\alpha} = \tensor{\Lambda}{_{\alpha}^{\beta}}U_{\beta}
\end{equation}
dove
\begin{equation}
  \tensor{\Lambda}{_{\alpha}^{\beta}} = \tensor{\eta}{_{\alpha\gamma}}
  \tensor{\eta}{^{\beta\delta}} \tensor{\Lambda}{^{\gamma}_{\delta}}.
\end{equation}
La matrice $\tensor{\eta}{^{\alpha\beta}}$ qui introdotta è il
\index{tensore!metrico controvariante di
  Minkowski}\emph{tensore
  metrico controvariante di Minkowski}
ed è numericamente uguale a $\tensor{\eta}{_{\alpha\beta}}$, vale a dire
\begin{equation}
  \tensor{\eta}{^{\alpha\beta}} = \tensor{\eta}{_{\alpha\beta}}.
\end{equation}
Osserviamo che
\begin{equation}
  \tensor{\eta}{^{\alpha\gamma}} \tensor{\eta}{_{\beta\gamma}} =
  \tensor{\delta}{^{\alpha}_{\beta}} = \tensor{\delta}{_{\alpha}^{\beta}} =
  \begin{cases}
    1 & \text{se } \alpha = \beta, \\
    0 & \text{se } \alpha \neq \beta.
  \end{cases}
\end{equation}
in cui $\tensor{\delta}{^{\alpha}_{\beta}} = \tensor{\delta}{_{\alpha}^{\beta}}$
è la
\index{delta!di Kronecker quadrimensionale}delta di Kronecker quadrimensionale.
Dunque $\tensor{\Lambda}{_{\alpha}^{\beta}}$ è la matrice inversa di
$\tensor{\Lambda}{^{\alpha}_{\beta}}$.

\section{Dinamica particellare}
\label{sec:dinamica-particellare}

Data una particella il cui moto è descritto dalla coordinata controvariante
$x^{\alpha}$, definiamo la sua \index{quadrivelocità}\emph{quadrivelocità} come
la derivata di $x^{\alpha}$ rispetto al tempo proprio $\tau$:
\begin{equation}
  v^{\alpha} = (v^{0}, v^{i}) =\toder{x^{\alpha}}{\tau}
\end{equation}
e poiché risulta
\begin{equation}
  \dd \tau = \sqrt{\dd t^{2} - \dd \bm{x}^{2}} = \dd t \sqrt{1 -
    \Big(\toder{\bm{x}}{t}\Big)^{2}} = \dd t\sqrt{1 - v^{2}} = \toder{\dd
    t}{\gamma}
\end{equation}
abbiamo che la quadrivelocità è uguale a
\begin{equation}
  v^{\alpha} = \toder{x^{\alpha}}{t}\toder{t}{\tau} = \toder{x^{\alpha}}{t}
  \gamma = \gamma \Big(\toder{x^{0}}{t}, \toder{\bm{x}}{t}\Big) =
  \gamma(1,\bm{v}) = (\gamma, \gamma\bm{v}).
\end{equation}
Il \index{quadrimomento}\emph{quadrimomento} della particella è definito come il
prodotto fra la massa a riposo $m$ della particella e la quadrivelocità
\begin{equation}
  p^{\alpha} = (p^{0},p^{i}) = m v^{\alpha} = m (v^{0}, v^{i}) = (m\gamma,
  m\gamma\bm{v}).
\end{equation}
Ricordando che l'energia totale $E$ di una particella è definita come $m\gamma$
abbiamo che la prima componente del quadrivettore momento è proprio l'energia
totale, mentre le componenti $p^{i}$ sono le componenti del momento
$\bm{p} = m\gamma\bm{v}$
\begin{equation}
  p^{\alpha} = (E, \bm{p}).
\end{equation}
La \index{quadriforza}\emph{quadriforza} agente sulla particella è definita come
la derivata rispetto al tempo proprio del quadrimomento
\begin{equation}
  f^{\alpha} = (f^{0},f^{i}) =\toder{p^{\alpha}}{\tau} =
  m\toder{v^{\alpha}}{\tau} = mw^{\alpha}
\end{equation}
in cui
\begin{equation}
  w^{\alpha} = (w^{0},w^{i}) =\toder{v^{\alpha}}{\tau}
\end{equation}
è la \index{quadriaccelerazione}\emph{quadriaccelerazione}.  In un sistema di
riferimento $O$ in cui la particella è a riposo $\dd \bm{x} = \bm{0}$ quindi
$\dd \tau = \dd t$ e $f^{\alpha} = F^{\alpha}$, in cui $F^{i}$ sono le
componenti cartesiane della forza non relativistica $\bm{F}$ e $F^{0} = 0$:
\begin{equation}
  f^{\alpha} = (0, \bm{F}).
\end{equation}
Per calcolare la forza a cui è soggetta la particella in un altro sistema di
riferimento inerziale $O'$ in moto rispetto a $O$ con velocità $\bm{v}$ dobbiamo
effettuare la trasformazione di Lorentz
\begin{equation}
  f'^{\alpha} = \tensor{\Lambda}{^{\alpha}_{\beta}}f^{\beta}.
\end{equation}
Sviluppando i calcoli si ottiene
\begin{subequations}
  \begin{align}
    f^{0} &= \gamma\bm{v}\cdot\bm{F} = \bm{v}\cdot\bm{f}, \\
    \bm{f} &= \bm{F} + (\gamma - 1)\bm{v}\frac{\bm{v}\cdot\bm{F}}{v^{2}}.
  \end{align}
\end{subequations}
% TODO: mettere la relazione di mass shell se non è stata messa prima


%%% Local Variables:
%%% mode: latex
%%% TeX-master: "../astrofisica-teorica"
%%% fill-column: 80
%%% End:
