\cleardoublepage
\chapter{Richiami di relatività speciale}
\label{cha:relativita-speciale}


\section{Trasformazioni di Lorentz}
\label{sec:trasformazioni-lorentz}

La relatività speciale, o ristretta, si basa su due postulati:
\begin{enumerate}
\item\emph{tutte le leggi fisiche assumono la stessa forma in qualsiasi sistema
    di riferimento inerziale (\index{principio!di relatività}principio di
    relatività)};
\item \emph{la velocità della luce è indipendente dal moto della sorgente}.
\end{enumerate}

L'ambiente in cui si svolgono i fenomeni fisici descritti dalla relatività
speciale è uno spazio quadrimensionale piatto, composto da una coordinata
temporale e tre spaziali, chiamato \index{spazio!di Minkowski}
\emph{spazio di Minkowski} e che indicheremo con il simbolo $\M$.  Un elemento
dello spazio di Minkowski è chiamato \index{evento}\emph{evento}.  

Nello spaziotempo di Minkowski definiamo le
\index{coordinata!controvariante}\emph{coordinate controvarianti} di un evento
che avviene al tempo $t$ nel punto $\bm{x}$ dello spazio tridimensionale come
\begin{equation}
  \label{eq:coord-controvariante}
  x^{\alpha} = (x^{0},x^{i}) = (t, \bm{x});
\end{equation}
la \index{metrica!nello spazio di Minkowski}\emph{metrica} è data da
\begin{equation}
  \dd \tau^{2} = \dd t^{2} - \dd \bm{x}^{2} = -\tensor{\eta}{_{\alpha\beta}}\dd
  x^{\alpha} \dd x^{\beta},
\end{equation}
in cui $\tensor{\eta}{_{\alpha\beta}}$ è il
\index{tensore!metrico di Minkowski}\emph{tensore metrico covariante di
  Minkowski}
\begin{equation}
  \label{eq:tensore-metrico-controvariante}
  \tensor{\eta}{_{\alpha\beta}} =
  \begin{pmatrix}
   \eta_{00} & \eta_{01} & \eta_{02} & \eta_{03} \\
   \eta_{10} & \eta_{11} & \eta_{12} & \eta_{13} \\
   \eta_{20} & \eta_{21} & \eta_{22} & \eta_{23} \\
   \eta_{30} & \eta_{31} & \eta_{32} & \eta_{33}
  \end{pmatrix}
  =
  \begin{pmatrix}
    -1 & 0  & 0  & 0 \\
    0  & +1 & 0  & 0 \\
    0  & 0  & +1 & 0 \\
    0  & 0  & 0  & +1
  \end{pmatrix}.
\end{equation}
Questo tensore definisce una metrica pseudo euclidea con segnatura
$(-+++)$,\footnote{Nelle presenti note abbiamo adottato questa segnatura, ma non
  è l'unica possibile nello spazio di Minkowski.  Un'altra segnatura largamente
  usata è $(+---)$.}.  
  
Una \index{trasformazione!di Lorentz}\emph{trasformazione di Lorentz} è
una trasformazione lineare nello spazio di Minkowski da un sistema di
riferimento inerziale $x^{\alpha}$ a un altro sistema di
riferimento inerziale $x'^{\alpha}$ 
\begin{equation}
  \label{eq:trasformazione-lineare}
  x'^{\alpha} = \tensor{\Lambda}{^{\alpha}_{\beta}}x^{\beta} + a^{\alpha},
\end{equation}
($\tensor{\Lambda}{^{\alpha}_{\beta}}$ dipendenti dalla velocità relativa dei
due sistemi di coordinate, $a^{\alpha}$ costanti), che mantenga invariata la
distanza fra due eventi $\dd\tau = \dd\tau'$.\footnote{La dimostrazione di
  questa proprietà è immediata nel caso di eventi connessi da segnali luminosi
  (in questo caso è $\dd\tau=0$); per il caso generale
  vedi~\textcite[34-37]{barone:relativita}.}

Richiediamo che la trasformazione $x^{\alpha} \to x'^{\alpha}$ sia lineare per
rispettare la condizione di omogeneità dello spazio (la lunghezza di una sbarra
non deve dipendere dalla posizione degli estremi), del tempo (la lunghezza della
sbarra non deve dipendere dall'istante di tempo in cui la misuriamo) e di
isotropia dello spazio (la lunghezza della sbarra non deve dipendere neanche
dalla direzione in cui la guardiamo).  In questo modo un moto rettilineo
uniforme in un sistema di riferimento inerziale è ancora un moto rettilineo
uniforme anche nell'altro sistema di riferimento inerziale.  

I differenziali delle coordinate nel nuovo sistema si ricavano
dalla~\eqref{eq:trasformazione-lineare}
\begin{equation}
  \label{eq:trasformazione-differenziali}
  \dd x'^{\alpha} = \tensor{\Lambda}{^{\alpha}_{\beta}}\dd x^{\beta}.
\end{equation}
Dall'equazione precedente abbiamo inoltre
\begin{equation}
  \tensor{\Lambda}{^{\alpha}_{\beta}} = \parder{x'^{\alpha}}{x^{\beta}}.
\end{equation}

Affinché una trasformazione lineare~\eqref{eq:trasformazione-lineare} delle
coordinate sia di Lorentz, cioè affinché conservi la metrica, deve valere
\begin{equation}
  \begin{split}
    \dd \tau'^{2} &= -\tensor{\eta}{_{\alpha\beta}}\dd x'^{\alpha}\dd x'^{\beta}
    = -\tensor{\eta}{_{\alpha\beta}} \tensor{\Lambda}{^{\alpha}_{\gamma}} \dd
    x^{\gamma} \tensor{\Lambda}{^{\beta}_{\delta}} \dd x^{\delta} \\
    &= \dd \tau^{2} = -\tensor{\eta}{_{\gamma\delta}} \dd x^{\gamma} \dd x^{\delta},
  \end{split}
\end{equation}
da cui, confrontando terzo e quinto membro, otteniamo la relazione fondamentale
del tensore metrico
\begin{equation}
  \label{eq:rel-fond-tensore-metrico}
  \tensor{\eta}{_{\alpha\beta}} \tensor{\Lambda}{^{\alpha}_{\gamma}}
  \tensor{\Lambda}{^{\beta}_{\delta}} = \tensor{\eta}{_{\gamma\delta}}.
\end{equation}
Questa relazione impone $10$ condizioni indipendenti in quanto il tensore
metrico $\tensor{\eta}{_{\gamma\delta}}$ è simmetrico quindi solo $10$ delle sue
$16$ componenti sono indipendenti.

Ciascuna trasformazione di Lorentz è rappresentabile con una specifica
\emph{matrice di Lorentz} $\Lambda$, di dimensione $4 \times 4$, i cui elementi
sono le $\tensor{\Lambda}{^{\alpha}_{\beta}}$
\begin{equation}
  \tensor{\Lambda}{^{\alpha}_{\beta}} =
  \begin{pmatrix}
    \tensor{\Lambda}{^{0}_{0}} & \tensor{\Lambda}{^{0}_{1}} &
    \tensor{\Lambda}{^{0}_{2}} & \tensor{\Lambda}{^{0}_{3}} \\
    \tensor{\Lambda}{^{1}_{0}} & \tensor{\Lambda}{^{1}_{1}} &
    \tensor{\Lambda}{^{1}_{2}} & \tensor{\Lambda}{^{1}_{3}} \\
    \tensor{\Lambda}{^{2}_{0}} & \tensor{\Lambda}{^{2}_{1}} &
    \tensor{\Lambda}{^{2}_{2}} & \tensor{\Lambda}{^{2}_{3}} \\
    \tensor{\Lambda}{^{3}_{0}} & \tensor{\Lambda}{^{3}_{1}} &
    \tensor{\Lambda}{^{3}_{2}} & \tensor{\Lambda}{^{3}_{3}}
  \end{pmatrix}.
\end{equation}
Possiamo scrivere la~\eqref{eq:rel-fond-tensore-metrico} nella seguente forma
matriciale
\begin{equation}
  \Lambda^{\textup{T}}\eta\Lambda = \eta
\end{equation}
dalla quale abbiamo
\begin{equation}
  \det\eta = \det(\Lambda^{\textup{T}}\eta\Lambda) = \det\eta
  \det(\Lambda^{\textup{T}}\Lambda) = \det\eta (\det\Lambda)^{2}
\end{equation}
e quindi deve valere la condizione
\begin{equation}
  \det\Lambda = \pm 1.
\end{equation}
Le trasformazioni di Lorentz con $\det\Lambda = +1$ sono chiamate
\emph{proprie}, quelle con $\det\Lambda = -1$ sono chiamate \emph{improprie}.

\begin{figure}
  \centering
  \tdplotsetmaincoords{70}{100}
  \begin{tikzpicture}[tdplot_main_coords,font=\footnotesize,scale=3]
    \node[shape=coordinate] (O1) at (0,0,0) [label=above right:$O$] {};
    \node[shape=coordinate] (O2) at (-1,0.5,0.3) [label=above right:$O'$] {};
    \draw[->] (O1) -- ++(1,0,0) node[anchor=north east] {$x$}; % asse x
    \draw[->] (O1) -- ++(0,1,0) node[anchor=north west] {$y$}; % asse y
    \draw[->] (O1) -- ++(0,0,1) node[anchor=south] (Z1) {$z$}; % asse z
    \node at (Z1) [label=below right:$\mathcal{S}$] {};% sistema di riferimento
    \draw[->] (O2) -- ++(1,0,0) node[anchor=north east] {$x'$}; % asse x'
    \draw[->] (O2) -- ++(0,1,0) node[anchor=north west] {$y'$}; % asse y'
    \draw[->] (O2) -- ++(0,0,1) node[anchor=south] (Z2) {$z'$}; % asse z'
    \node at (Z2) [label=below right:$\mathcal{S}'$] {};% sistema di riferimento
    \draw[->,thick] (O1) -- node[above] {$\bm{v}$} ++(0.3,-0.4,0.3); % velocità
  \end{tikzpicture}
  \caption[Sistemi di riferimento inerziali in moto l'uno rispetto
  all'altro]{Sistemi di riferimento inerziali, con gli assi corrispondenti
    paralleli, in moto l'uno rispetto all'altro}
  \label{fig:trasformazioni-lorentz}
\end{figure}

Vediamo come si trasformano le coordinate $x^{\alpha}$ di un evento in un
sistema di riferimento per effetto di una \emph{trasformazione di velocità}, o
\index{boost}\emph{boost}, vale a dire una trasformazione tra due sistemi di
riferimento inerziali con gli assi cartesiani corrispondenti paralleli e in moto
l'uno rispetto all'altro con velocità $\bm{v}$.  Vogliamo quindi
determinare la forma esplicita della matrice di Lorentz per il caso particolare
di un boost.  Consideriamo una particella a riposo in un sistema di riferimento
$\mathcal{S}$ che si muove con velocità $\bm{v} = (v_{1},v_{2},v_{3})$ rispetto
a un altro sistema di riferimento inerziale $\mathcal{S}'$, come nella
figura~\ref{fig:trasformazioni-lorentz}.\footnote{Si noti che il sistema di
  riferimento $\mathcal{S}$ vede il sistema $\mathcal{S}'$ muoversi con velocità
  $-\bm{v}$.  Numerosi autori considerano il sistema di riferimento
  $\mathcal{S}'$ in moto rispetto al sistema $\mathcal{S}$ con velocità
  $\bm{v}$, questo comporterà una differenza di segno in numerose equazioni
  rispetto a quelle ricavate da questi autori.}
Poiché nel sistema $\mathcal{S}$ per la particella risulta $\dd\bm{x} = \bm{0}$,
si ricava
\begin{subequations}
  \begin{align}
    \dd x'^{i} &= \tensor{\Lambda}{^{i}_{\alpha}}\dd x^{\alpha} =
    \tensor{\Lambda}{^{i}_{0}}\dd x^{0} + \tensor{\Lambda}{^{i}_{j}}\dd x^{j} =
    \tensor{\Lambda}{^{i}_{0}}\dd t \\
    \dd x'^{0} &= \tensor{\Lambda}{^{0}_{\alpha}}\dd x^{\alpha} =
    \tensor{\Lambda}{^{0}_{0}}\dd x^{0} + \tensor{\Lambda}{^{0}_{j}}\dd x^{j} =
    \tensor{\Lambda}{^{0}_{0}}\dd t.
  \end{align}
\end{subequations}
Dividendo membro a membro abbiamo
\begin{equation}
  \toder{x'^{i}}{x'^{0}} = \toder{x'^{i}}{t'} = v_{i} =
  \frac{\tensor{\Lambda}{^{i}_{0}}}{\tensor{\Lambda}{^{0}_{0}}},
\end{equation}
in quanto $\ltoder{x'^{i}}{t'}$ è la componente $v_{i}$ della velocità della
particella vista dal sistema $\mathcal{S}'$.  Allora
\begin{equation}
  \tensor{\Lambda}{^{i}_{0}} = v_{i}\tensor{\Lambda}{^{0}_{0}}.
\end{equation}
Ponendo $\gamma = \delta = 0$ nella~\eqref{eq:rel-fond-tensore-metrico} risulta
\begin{equation}
  \begin{split}
    -1 &= \tensor{\eta}{_{00}} = \tensor{\eta}{_{\alpha\beta}}
    \tensor{\Lambda}{^{\alpha}_{0}} \tensor{\Lambda}{^{\beta}_{0}} =
    \tensor{\eta}{_{00}}(\tensor{\Lambda}{^{0}_{0}})^{2} + \tensor{\eta}{_{ij}}
    \tensor{\Lambda}{^{i}_{0}} \tensor{\Lambda}{^{j}_{0}}
    \\
    &= -(\tensor{\Lambda}{^{0}_{0}})^{2} + \delta_{ij}
    \tensor{\Lambda}{^{i}_{0}} \tensor{\Lambda}{^{j}_{0}} =
    -(\tensor{\Lambda}{^{0}_{0}})^{2}+\sum_{i}(\tensor{\Lambda}{^{i}_{0}})^{2},
  \end{split}
\end{equation}
quindi
\begin{equation}
  (\tensor{\Lambda}{^{0}_{0}})^{2} = 1 +
  \sum_{i}(\tensor{\Lambda}{^{i}_{0}})^{2} \geq 1,
\end{equation}
cioè per ogni trasformazione di Lorentz deve aversi
\begin{equation}
  \tensor{\Lambda}{^{0}_{0}} \geq 1 \text{ oppure } \tensor{\Lambda}{^{0}_{0}}
  \leq -1.
\end{equation}
Le trasformazioni con $\tensor{\Lambda}{^{0}_{0}} \geq +1$ sono chiamate
\emph{ortocrone}, quelle con $\tensor{\Lambda}{^{0}_{0}} \leq -1$
\emph{anticrone}.  Le trasformazioni di Lorentz proprie e ortocrone sono dette
\emph{trasformazioni di Lorentz ristrette} o \emph{speciali}.  Richiedendo che
il boost sia una trasformazione ristretta abbiamo
\begin{equation}
  \tensor{\Lambda}{^{0}_{0}} = \frac{1}{\sqrt{1-\sum_{i}(v_{i})^{2}}} =
  \frac{1}{\sqrt{1-v^{2}}} = \gamma_{v},
\end{equation}
con $v^{2} = \sum_{i}(v_{i})^{2}$ modulo quadro del vettore $\bm{v}$.  Abbiamo
introdotto la quantità $\gamma_{v} = 1/\sqrt{1-v^{2}}$ chiamata
\index{fattore!di Lorentz}\emph{fattore di Lorentz} associato al boost $\bm{v}$.
Quando non ci sarà pericolo di ambiguità indicheremo il fattore di Lorentz
associato a un certo boost semplicemente con $\gamma$.  Risulta anche
\begin{equation}
  \tensor{\Lambda}{^{i}_{0}} = v_{i}\tensor{\Lambda}{^{0}_{0}} = v_{i}\gamma.
\end{equation}
Le altre componenti $\tensor{\Lambda}{^{\alpha}_{\beta}}$ non sono univocamente
determinate.  Una scelta conveniente che permette di soddisfare
la relazione fondamentale~\eqref{eq:rel-fond-tensore-metrico} è
\begin{subequations}
  \begin{align}
    \tensor{\Lambda}{^{0}_{i}} &= \tensor{\Lambda}{^{i}_{0}} = v_{i}\gamma, \\
    \tensor{\Lambda}{^{i}_{j}} &= \delta_{ij} +
    v_{i}v_{j}\frac{\gamma-1}{v^{2}}.
  \end{align}
\end{subequations}
così che la matrice $\Lambda$ può essere scritta, per un \index{boost}boost
$\bm{v}$, come
\begin{equation}
  \Lambda =
  \begin{pmatrix}
    \gamma & \gamma v_{1} & \gamma v_{2} & \gamma v_{3} \\[2.0ex]
    \gamma v_{1} & 1 + (v_{1})^{2}\dfrac{\gamma-1}{v^{2}} &
    v_{1}v_{2}\dfrac{\gamma-1}{v^{2}} & v_{1} v_{3}\dfrac{\gamma-1}{v^{2}}
    \\[2.0ex]
    \gamma v_{2} & v_{2}v_{1}\dfrac{\gamma-1}{v^{2}} & 1 +
    (v_{2})^{2}\dfrac{\gamma-1}{v^{2}} & v_{2}v_{3}\dfrac{\gamma-1}{v^{2}}
    \\[2.0ex]
    \gamma v_{3} & v_{3}v_{1}\dfrac{\gamma-1}{v^{2}} &
    v_{3}v_{2}\dfrac{\gamma-1}{v^{2}} & 1 + (v_{3})^{2}\dfrac{\gamma-1}{v^{2}}
  \end{pmatrix}.
\end{equation}

Osserviamo ora che le trasformazioni di Lorentz formano un gruppo poichè
1) esiste la trasformazione identica: 
$\tensor{\Lambda}{^{\alpha}_{\beta}} =   \tensor{\delta}{^{\alpha}_{\beta}}$
2) la composizione di due trasformazioni di Lorentz è ancora una trasformazione di
Lorentz: 
$\tensor{\Lambda}{^{\alpha}_{1 \beta}} \tensor{\Lambda}{^{\beta}_{2 \gamma}} =
\tensor{\Lambda}{^{\alpha}_{\gamma}}$; 
2) esiste la trasformazione inversa definita dalla relazione
\begin{equation}
  \tensor{\Lambda}{_{\alpha}^{\beta}} = \tensor{\eta}{_{\alpha\gamma}}
  \tensor{\eta}{^{\beta\delta}} \tensor{\Lambda}{^{\gamma}_{\delta}}.
\end{equation}
La matrice $\tensor{\eta}{^{\alpha\beta}}$ è il
\index{tensore!metrico di Minkowski}\emph{tensore metrico controvariante di
  Minkowski}
ed è numericamente uguale a $\tensor{\eta}{_{\alpha\beta}}$, vale a dire
\begin{equation}
  \tensor{\eta}{^{\alpha\beta}} = \tensor{\eta}{_{\alpha\beta}}.
\end{equation}
Osserviamo che
\begin{equation}
  \label{eq:delta-kronecker}
  \eta^{\alpha\gamma} \eta_{\beta\gamma} = 
  \tensor{\delta}{^{\alpha}_{\beta}} = 
  \begin{cases}
    1 & \text{se } \alpha = \beta, \\
    0 & \text{se } \alpha \neq \beta,
  \end{cases}
\end{equation}
in cui $\tensor{\delta}{^{\alpha}_{\beta}}$
è la \index{delta!di Kronecker quadrimensionale}\emph{delta di Kronecker
  quadrimensionale}. Usando la relazione fondamentale delle trasformazioni di
Lorentz~\eqref{eq:rel-fond-tensore-metrico} abbiamo
\begin{equation}
  \tensor{\Lambda}{_{\alpha}^{\gamma}} \tensor{\Lambda}{^{\alpha}_{\beta}} =
  \tensor{\eta}{_{\alpha\delta}} \tensor{\eta}{^{\gamma\epsilon}}
  \tensor{\Lambda}{^{\delta}_{\epsilon}} \tensor{\Lambda}{^{\alpha}_{\beta}} =
  \tensor{\eta}{_{\epsilon\beta}} \tensor{\eta}{^{\gamma\epsilon}} =
  \tensor{\delta}{^{\gamma}_{\beta}}.
\end{equation}
Dunque $\tensor{\Lambda}{_{\alpha}^{\beta}}$ è la matrice inversa di
$\tensor{\Lambda}{^{\alpha}_{\beta}}$.

Infatti, se moltiplichiamo ambo i membri della 
\eqref{eq:trasformazione-lineare} per $\tensor{\Lambda}{_{\alpha}^{\gamma}}$ 
risulta che
\begin{equation}
  x'^{\alpha} \tensor{\Lambda}{_{\alpha}^{\gamma}} =
  \tensor{\Lambda}{^{\alpha}_{\beta}} \tensor{\Lambda}{_{\alpha}^{\gamma}}
  x^{\beta} + \tensor{\Lambda}{_{\alpha}^{\gamma}} a^{\alpha} =
  \tensor{\delta}{^{\gamma}_{\beta}} x^{\beta} + a'^{\gamma} = x^{\gamma}
  +a'^{\gamma}
\end{equation}
da cui ricaviamo la trasformazione, rinominando gli indici,
\begin{equation}
  \label{eq:trasformazione-inversa}
  x'^{\alpha} \to x^{\alpha} = \tensor{\Lambda}{_{\beta}^{\alpha}}x'^{\beta} -
  a'^{\alpha}.
\end{equation}
Osserviamo inoltre che risulta
\begin{equation}
  \tensor{\Lambda}{_{\beta}^{\alpha}} = \parder{x^{\alpha}}{x'^{\beta}}.
\end{equation}
 
Le trasformazioni di Lorentz comportano alcune conseguenze, come la contrazione
delle lunghezze, la dilatazione degli intervalli temporali e l'effetto Doppler.
Questi argomenti possono essere rivisti
in~\textcite[29-31]{weinberg:gravitation}.

\section{Conseguenze delle trasformazioni di Lorentz: dilatazione del tempo}
\label{sec:conseguenze-lorentz}

Si consideri un osservatore $O$ che osservi due eventi $A$ e $B$ accadere nello
stesso posto ($\dd x^i=0$), separati dalla coordinata temporale $\var t$.
L'intervallo di tempo proprio tra i due eventi è $\dd\tau=\var t$.  Un secondo
osservatore $O'$ in moto con velocità $\bm{v}$, vedrà i due eventi accadere in
posizioni diverse $\dd x'^{i}\ne 0$, separati dalla coordinata temporale
$\dd t'$.  L'intervallo di tempo proprio tra i due eventi è ora
$\dd\tau = \dd t' (1-v^2)^{1/2}=\dd t'/\gamma$; quindi $\dd t'= \gamma \var t$,
il che implica l'aumento del tempo che intercorre tra i due eventi misurato da
$O'$.  Questo è un effetto reale che può essere verificato osservando arrivare a
terra particelle instabili (con vita media estremamente piccola,
$\var t = \SI{2.6e-8}{\second}$ per un $\pi^{0}$) prodotte al top dell'atmosfera
(a circa 10 km di altezza) attraverso interazioni adroniche tra raggi cosmici di
alta energia e gli atomi che compongono l'atmosfera.  I pioni viaggiano a
velocità relativistiche e quindi rispetto a noi vivono un tempo molto maggiore
della loro vita media a riposo e questo permette loro di percorre senza decadere
la distanza che li separa da terra (osserva che lo spazio percorso in circa
\SI{3e-8}{\second} a velocità $c$ sarebbe circa \SI{1}{\metre}).

\subsection{Effetto Doppler}
\label{sec:effetto-doppler}
Se il nostro orologio in movimento è una sorgente di luce di frequenza
$\nu=1/\var t$.  $\var t$ può essere considerato come la distanza temporale tra
l'emissione di due successive creste di un'onda elettromagnetica piana, misurata
dall'osservatore solidale con la sorgente.  Questo intervallo di tempo apparirà
essere pari a $\dd t'= \gamma \dd t$ ad un osservatore $O'$ che vede la sorgente
muoversi con velocità $\bm{v}$.  Comunque, durante questo intervallo di tempo la
distanza della sorgente $S$ dall'osservatore $O'$ sarà aumentata (se $v_r>0$) o
diminuita (se $v_r<0$) della quantità $v_r \dd t'$, dove $v_r$ è la componente
della velocità della sorgente lungo la linea di vista da $O'$ ad $S$.  Quindi il
periodo $\dd t_0$ tra la ricezione delle due successive creste d'onda da parte
di $O'$ sarà $\dd t_0 = \dd t'+ v_r \dd t' = (1+v_r)/\sqrt{1-v^2}$: in termini
delle frequenze (osservata $\nu_0$ ed emessa $\nu$) si ha:
$\nu_0/\nu=\sqrt{1-v^2}/(1+v_r)$. Quindi red shift se $S$ si allontana, blu
shift se la sorgente si avvicina.  Se la velocità della sorgente è lungo la
linea di vista si ha $v_r=v$ e la formula Doppler diventa:
$\nu_0/\nu= \sqrt{(1-v)/(1+v)}$, e per velocità $v\ll 1$ si ottiene
$\Delta \nu / \nu = (\nu_e - \nu_o)/\nu_e = v/c = z$.


\section{Calcolo tensoriale nello spazio di Minkowski}
\label{sec:calcolo-tensoriale-minkowski}

Il calcolo tensoriale è lo strumento matematico alla base della teoria della
relatività.  In questo paragrafo daremo dei cenni relativi al calcolo tensoriale
nello spazio di Minkowski.  Si noti che alcuni dei risultati che presenteremo
qui sono validi solo nello spazio piatto della relatività speciale e non in
quello curvo della relatività generale.

Un \index{quadrivettore!controvariante}\emph{quadrivettore controvariante}
$V^{\alpha} = (V^{0}, V^{1}, V^{2}, V^{3})$ è un insieme di quattro quantità
che, per effetto di una trasformazione di Lorentz ristretta omogenea
$x^{\alpha} \to x'^{\alpha} = \tensor{\Lambda}{^{\alpha}_{\beta}}x^{\beta}$, si
trasformano come le componenti di $\dd
x^{\alpha}$~\eqref{eq:trasformazione-differenziali}, vale a dire
\begin{equation}
  V^{\alpha} \to V'^{\alpha} = \tensor{\Lambda}{^{\alpha}_{\beta}}V^{\beta}.
\end{equation}
Un quadrivettore controvariante può anche essere rappresentato come
\begin{equation}
  V^{\alpha} = (V^{0}, V^{i}) = (V^{0}, \bm{V}),
\end{equation}
dove $\bm{V} = (V^{1}, V^{2}, V^{3})$.

Un \index{quadrivettore!covariante}\emph{quadrivettore covariante}
$U_{\alpha} = (U_{0}, U_{1}, U_{2}, U_{3})$ è un insieme di quattro quantità
che, sotto una trasformazione di Lorentz ristretta, si trasformano come
\begin{equation}
  U_{\alpha} \to U'_{\alpha} = \tensor{\Lambda}{_{\alpha}^{\beta}}U_{\beta}
\end{equation}

Possiamo passare da un quadrivettore controvariante
$V^{\alpha} = (V^{0}, V^{i})$ al corrispondente quadrivettore covariante
$V_{\alpha} = (V_{0}, V_{i})$ moltiplicandolo per il tensore di Minkowski
covariante
\begin{equation}
  \label{eq:abbassamento-indice}
  \begin{split}
    V^{\alpha} \to V_{\alpha} &= \tensor{\eta}{_{\alpha\beta}}V^{\beta} =
    \tensor{\eta}{_{\alpha\beta}} (V^{0}, V^{i}) =
    (\tensor{\eta}{_{0\beta}}V^{0}, \tensor{\eta}{_{i\beta}}V^{i}) \\
    &= (\tensor{\eta}{_{00}}V^{0}, \delta_{i\beta}V^{\beta}) = (-V^{0}, V^{i}).
  \end{split}
\end{equation}
Pertanto, mediante questa operazione, chiamata a volte
\emph{abbassamento degli indici}, la componente temporale di un quadrivettore
cambia segno mentre quelle spaziali rimangono immutate.  Analogamente si può
effettuare l'\emph{innalzamento degli indici} di un quadrivettore covariante
moltiplicandolo per il tensore di Minkowski controvariante
\begin{equation}
  \label{eq:innalzamento-indice}
  U_{\alpha} \to U^{\alpha} = \tensor{\eta}{^{\alpha\beta}}U_{\beta}.
\end{equation}
Verifichiamo che il quadrivettore $V_{\alpha}$ definito
nella~\eqref{eq:abbassamento-indice} si trasforma effettivamente come un
quadrivettore covariante
\begin{equation}
  V_{\alpha} \to V'_{\alpha} = \tensor{\eta}{_{\alpha\beta}}V'^{\beta} =
  \tensor{\eta}{_{\alpha\beta}} \tensor{\Lambda}{^{\beta}_{\gamma}} V^{\gamma} =
  \tensor{\eta}{_{\alpha\beta}} \tensor{\Lambda}{^{\beta}_{\gamma}}
  \tensor{\eta}{^{\gamma\delta}}V_{\delta} =
  \tensor{\Lambda}{_{\alpha}^{\delta}} V_{\delta}.
\end{equation}
Allo stesso modo si può far vedere che il quadrivettore $U^{\alpha}$ definito
nella~\eqref{eq:innalzamento-indice} si trasforma come un quadrivettore
controvariante.

Possiamo generalizzare le definizioni date finora introducendo il concetto di
quadritensore.  Un
\index{quadritensore}\emph{quadritensore}\footnote{A rigore si dovrebbe parlare
  di quadritensore di componenti
  $\tensor*{A}{^{\alpha_{1}\dots\alpha_{p}}_{\beta_{1}\dots\beta_{q}}}$,
  tuttavia attraverso tutte queste note useremo, commettendo un abuso di
  notazione, lo stesso simbolo per indicare un quadritensore e le sue
  componenti, come abbiamo in realtà già fatto finora.}
$\tensor*{A}{^{\alpha_{1}\dots\alpha_{p}}_{\beta_{1}\dots\beta_{q}}}$ (con
$p + q = k$) di rango $k$ è un insieme di $4^{k}$ quantità che sotto una
trasformazione di Lorentz ristretta omogenea
$x^{\alpha} \to x'^{\alpha} = \tensor{\Lambda}{^{\alpha}_{\beta}}x^{\beta}$ si
trasformano come
\begin{equation}
  \tensor*{A}{^{\alpha_{1}\dots\alpha_{p}}_{\beta_{1}\dots\beta_{q}}} \to
  \tensor*{{A'}}{^{\alpha_{1}\dots\alpha_{p}}_{\beta_{1}\dots\beta_{q}}} =
  \tensor{\Lambda}{^{\alpha_{1}}_{\mu_{1}}} \cdots
  \tensor{\Lambda}{^{\alpha_{p}}_{\mu_{p}}}
  \tensor{\Lambda}{_{\beta_{1}}^{\nu_{1}}} \cdots
  \tensor{\Lambda}{_{\beta_{q}}^{\nu_{q}}}
  \tensor*{A}{^{\mu_{1}\dots\mu_{p}}_{\nu_{1}\dots\nu_{q}}}
\end{equation}
Il quadritensore
$\tensor*{A}{^{\alpha_{1}\dots\alpha_{p}}_{\beta_{1}\dots\beta_{q}}}$ è un
quadritensore misto $p$ volte controvariante e $q$ volte covariante, oppure di
tipo $(p,q)$.  Un quadritensore di rango $k$ completamente controvariante è del
tipo $\tensor{A}{^{\alpha_{1}\dots\alpha_{k}}}$ e si trasforma come
\begin{equation}
  \tensor{A}{^{\alpha_{1}\dots\alpha_{k}}} \to
  \tensor{{A'}}{^{\alpha_{1}\dots\alpha_{k}}} =
  \tensor{\Lambda}{^{\alpha_{1}}_{\mu_{1}}} \cdots
  \tensor{\Lambda}{^{\alpha_{k}}_{\mu_{k}}} \tensor{A}{^{\mu_{1}\dots\mu_{k}}},
\end{equation}
similmente un quadritensore di rango $k$ completamente covariante è un
quadritensore del tipo $\tensor{A}{_{\alpha_{1}\dots\alpha_{k}}}$ che si
trasforma come
\begin{equation}
  \tensor{A}{_{\alpha_{1}\dots\alpha_{k}}} \to
  \tensor{{A'}}{_{\alpha_{1}\dots\alpha_{k}}} =
  \tensor{\Lambda}{_{\alpha_{1}}^{\mu_{1}}} \cdots
  \tensor{\Lambda}{_{\alpha_{k}}^{\mu_{k}}} \tensor{A}{_{\mu_{1}\dots\mu_{k}}}.
\end{equation}
I quadritensori di rango $1$ sono i \index{quadrivettore}\emph{quadrivettori}, i
quadritensori di rango $0$, che quindi non hanno indici, sono chiamati
\index{scalare}\emph{scalari di Lorentz} (o brevemente \emph{scalari}) e sono
invarianti sotto le trasformazioni di Lorentz.

Uno \index{pseudotensore}\emph{pseudotensore} (o
\index{densità!tensoriale|see{pseudotensore}}\emph{densità tensoriale})
$\tensor*{A}{^{\alpha_{1}\dots\alpha_{p}}_{\beta_{1}\dots\beta_{q}}}$ (con
$p+q=k$) di rango $k$ e peso $w$ è un insieme di $4^{k}$ quantità che per
effetto di una generica trasformazione di Lorentz ortocrona $\Lambda$, ma non
necessariamente propria, si trasformano come
\begin{equation}
  \tensor*{A}{^{\alpha_{1}\dots\alpha_{p}}_{\beta_{1}\dots\beta_{q}}} \to
  \tensor*{{A'}}{^{\alpha_{1}\dots\alpha_{p}}_{\beta_{1}\dots\beta_{q}}}
  = (\det \Lambda)^{w} \tensor{\Lambda}{^{\alpha_{1}}_{\mu_{i}}} \cdots
  \tensor{\Lambda}{^{\alpha_{p}}_{\mu_{p}}}
  \tensor{\Lambda}{_{\beta_{1}}^{\nu_{1}}} \cdots
  \tensor{\Lambda}{_{\beta_{q}}^{\nu_{q}}}
  \tensor*{A}{^{\mu_{1}\dots\mu_{p}}_{\nu_{1}\dots\nu_{q}}}.
\end{equation}
Le differenze fra tensori e pseudotensori si hanno quando il peso dello
pseudotesore è dispari e si considerano trasformazioni di Lorentz che
comprendono un'inversione spaziale, per le quali risulta $\det\Lambda = -1$.
Gli pseudotensori di rango $0$ sono chiamati \emph{pseudoscalari}, gli
pseudotensori di rango $1$ sono chiamati \emph{pseudovettori}.

\subsection{Operazioni con i quadritensori}
\label{sec:operazioni-tensori-minkowski}

\subsubsection{Combinazione lineare}
\label{sec:combinazione-lineare-minkowski}

La combinazione lineare
\begin{equation}
  a \tensor*{R}{^{\alpha_{1}\dots\alpha_{p}}_{\beta_{1}\dots\beta_{q}}} + b
  \tensor*{S}{^{\alpha_{1}\dots\alpha_{p}}_{\beta_{1}\dots\beta_{q}}},
\end{equation}
con $a$ e $b$ scalari, di due quadritensori dello stesso tipo è un quadritensore
$\tensor*{T}{^{\alpha_{1}\dots\alpha_{p}}_{\beta_{1}\dots\beta_{q}}}$ ancora
dello stesso tipo.  Per esempio, verifichiamo che
\begin{equation}
  \tensor{T}{^{\alpha}_{\beta}} = a \tensor{R}{^{\alpha}_{\beta}} + b
  \tensor{S}{^{\alpha}_{\beta}}
\end{equation}
è un quadritensore di rango $2$ una volta controvariante e una volta covariante
come lo sono $\tensor{R}{^{\alpha}_{\beta}}$ e
$\tensor{S}{^{\alpha}_{\beta}}$. Abbiamo
\begin{equation}
  \begin{split}
    \tensor{T}{^{\alpha}_{\beta}} \to \tensor{{T'}}{^{\alpha}_{\beta}} &= a
    \tensor{{R'}}{^{\alpha}_{\beta}} + b \tensor{{S'}}{^{\alpha}_{\beta}} = a
    \tensor{\Lambda}{^{\alpha}_{\gamma}} \tensor{\Lambda}{_{\beta}^{\delta}}
    \tensor{R}{^{\gamma}_{\delta}} + b \tensor{\Lambda}{^{\alpha}_{\gamma}}
    \tensor{\Lambda}{_{\beta}^{\delta}} \tensor{S}{^{\gamma}_{\delta}} \\
    &= \tensor{\Lambda}{^{\alpha}_{\gamma}}
    \tensor{\Lambda}{_{\beta}^{\delta}}(a \tensor{R}{^{\gamma}_{\delta}} + b
    \tensor{S}{^{\gamma}_{\delta}}) = \tensor{\Lambda}{^{\alpha}_{\gamma}}
    \tensor{\Lambda}{_{\beta}^{\delta}} \tensor{T}{^{\gamma}_{\delta}}.
  \end{split}
\end{equation}

\subsubsection{Prodotto diretto}
\label{sec:prodotto-diretto-minkowski}

Il \index{prodotto!diretto}prodotto
$\tensor*{A}{^{\alpha_{1}\dots\alpha_{p}}_{\beta_{1}\dots\beta_{q}}}
\tensor*{B}{^{\alpha_{1}\dots\alpha_{r}}_{\beta_{1}\dots\beta_{s}}}$
di un quadritensore di tipo $(p,q)$ e di uno di tipo $(r,s)$ è un quadritensore
$\tensor*{T}{^{\alpha_{1}\dots\alpha_{p+r}}_{\beta_{1}\dots\beta_{q+s}}}$ di
tipo $(p+r,q+s)$.  Consideriamo per esempio
\begin{equation}
  \tensor{T}{^{\alpha}_{\beta}^{\delta}} = \tensor{A}{^{\alpha}_{\beta}}
  \tensor{B}{^{\gamma}}.
\end{equation}
Abbiamo
\begin{equation}
  \tensor{T}{^{\alpha}_{\beta}^{\delta}} \to
  \tensor{{T'}}{^{\alpha}_{\beta}^{\delta}} = \tensor{{A'}}{^{\alpha}_{\beta}}
  \tensor{{B'}}{^{\gamma}} = \tensor{\Lambda}{^{\alpha}_{\delta}}
  \tensor{\Lambda}{_{\beta}^{\epsilon}} \tensor{A}{^{\delta}_{\epsilon}}
  \tensor{\Lambda}{^{\gamma}_{\zeta}} \tensor{B}{^{\zeta}} =
  \tensor{\Lambda}{^{\alpha}_{\delta}} \tensor{\Lambda}{_{\beta}^{\epsilon}}
  \tensor{\Lambda}{^{\gamma}_{\zeta}} \tensor{T}{^{\delta}_{\epsilon}^{\zeta}}.
\end{equation}

\subsubsection{Contrazione degli indici}
\label{sec:contrazione-indice-minkowski}

Se in un quadritensore di tipo $(p,q)$ $n$ indici controvarianti sono uguali a
$n$ indici covarianti e si somma su tali indici si ottiene un tensore di tipo
$(p-n,q-n)$.  Usando la convenzione di Einstein sulla somma si omette il simbolo
di sommatoria.  Questa operazione viene detta di
\index{contrazione!degli indici}\emph{contrazione degli indici}.  La contrazione
di un tensore di rango superiore a $2$ non è univoca, si può scegliere di
contrarre differenti coppie di indici ottenendo di volta in volta un risultato,
in generale, diverso.  Per esempio, il tensore
$\tensor{T}{^{\alpha}_{\beta}^{\gamma\delta}}$ è un tensore di tipo $(3,1)$, ma
se si contraggono l'indice covariante $\beta$ con il terzo indice controvariante
$\delta$ si ottiene il tensore
$\tensor{T}{^{\alpha}_{\beta}^{\gamma\beta}} =
\tensor{T}{^{\alpha}_{0}^{\gamma 0}} + \tensor{T}{^{\alpha}_{1}^{\gamma 1}} +
\tensor{T}{^{\alpha}_{2}^{\gamma 2}} + \tensor{T}{^{\alpha}_{3}^{\gamma 3}}$
che è completamente controvariante di rango $2$.  Infatti
\begin{equation}
  \begin{split}
    \tensor{T}{^{\alpha}_{\beta}^{\gamma\beta}} \to
    \tensor{{T'}}{^{\alpha}_{\beta}^{\gamma\beta}} &=
    \tensor{\Lambda}{^{\alpha}_{\omega}} \tensor{\Lambda}{_{\beta}^{\mu}}
    \tensor{\Lambda}{^{\gamma}_{\nu}} \tensor{\Lambda}{^{\beta}_{\zeta}}
    \tensor{T}{^{\omega}_{\mu}^{\nu\zeta}} =
    \tensor{\Lambda}{^{\alpha}_{\omega}} \tensor{\Lambda}{^{\gamma}_{\nu}}
    \tensor{\delta}{^{\mu}_{\zeta}} \tensor{T}{^{\omega}_{\mu}^{\nu\zeta}} \\
    &= \tensor{\Lambda}{^{\alpha}_{\omega}} \tensor{\Lambda}{^{\gamma}_{\nu}}
    \tensor{T}{^{\omega}_{\mu}^{\nu\mu}}.
  \end{split}
\end{equation}
Ciascuna operazione di contrazione si effettua moltiplicando per il tensore di
Minkowski che ha per indici i due indici che si vogliono contrarre, per esempio
$\tensor{T}{_{\gamma\beta}^{\gamma\beta}} = \eta_{\alpha\gamma}
\eta^{\beta\delta} \tensor{T}{^{\alpha}_{\beta}^{\gamma}_{\delta}}$.

Due indici contratti possono essere scambiati di posto: l'indice covariante può
diventare controvariante e, allo stesso tempo, l'indice controvariante diventare
covariante, senza modificare il risultato.  Sfrutteremo spesso questa proprietà.
Per esempio, quindi, abbiamo
$A^{\mu\nu}\tensor{B}{_{\mu}^{\lambda}} =
\tensor{A}{_{\mu}^{\nu}}B^{\mu\lambda}$.  Mostriamolo
\begin{equation}
  A^{\mu\nu}\tensor{B}{_{\mu}^{\lambda}} =
  \eta^{\mu\alpha}\tensor{A}{_{\alpha}^{\nu}} \eta_{\mu\beta}B^{\beta\lambda} =
  \eta^{\mu\alpha}\eta_{\mu\beta} \tensor{A}{_{\alpha}^{\nu}}B^{\beta\lambda} =
  \tensor{\delta}{^{\alpha}_{\beta}}\tensor{A}{_{\alpha}^{\nu}}B^{\beta\lambda}
  = \tensor{A}{_{\alpha}^{\nu}}B^{\alpha\lambda}.
\end{equation}
Non è invece possibile abbassare o alzare solo uno dei due indici muti lasciando
l'altro nella propria posizione, altrimenti si contravviene alla convenzione di
Einstein per la somma descritta a pagina~\pageref{cha:notazioni}.

Si può effettuare la contrazione degli indici anche nel prodotto di due
quadritensori: il prodotto di un quadritensore di tipo $(p,q)$ e di uno di tipo
$(r,s)$ è un quadritensore di tipo $(p+r,q+s)$, ma se si contraggono $n$ indici
si ottiene un quadritensore di tipo $(p+r-n,q+s-n)$.  Per esempio, il
quadritensore $\tensor{A}{^{\alpha}_{\beta}^{\gamma\delta}}$ è di tipo $(3,1)$,
$\tensor{B}{_{\epsilon\zeta}}$ è di tipo $(0,2)$, il prodotto
$\tensor{A}{^{\alpha}_{\beta}^{\gamma\delta}} \tensor{B}{_{\epsilon\zeta}}$ è un
quadritensore di tipo $(3,3)$, ma il prodotto
$\tensor{A}{^{\alpha}_{\beta}^{\gamma\delta}} \tensor{B}{_{\alpha\delta}}$ con
due indici contratti è un quadritensore di tipo $(1,1)$.

\subsubsection{Prodotto scalare di due quadrivettori}
\label{sec:prodotto-scalare-quadrivettori-minkowski}

Il \emph{prodotto scalare} fra due quadrivettori $V$ e $U$ è definito come
\begin{equation}
  \begin{split}
    V\cdot U &= V^{\alpha}U_{\alpha} =
    \tensor{\eta}{_{\alpha\beta}}V^{\alpha}U^{\beta} = V^{0}U_{0} + V^{1}U_{1} +
    V^{2}U_{2} + V^{3}U_{3} \\
    &= -V^{0}U^{0} + V^{1}U^{1} + V^{2}U^{2} + V^{3}U^{3} = -V^{0}U^{0} +
    \bm{V}\cdot\bm{U}.
  \end{split}
\end{equation}

Il prodotto scalare di due quadrivettori è un caso particolare di contrazione
degli indici nel prodotto fra due quadritensori.  Dunque il loro prodotto è un
quadritensore di rango $0$, cioè uno scalare e quindi una quantità che non varia
per effetto di una trasformazione di Lorentz.  Infatti
\begin{equation}
  V^{\alpha}U_{\alpha} \to V'^{\alpha}U'_{\alpha} =
  \tensor{\Lambda}{^{\alpha}_{\beta}} V^{\beta}
  \tensor{\Lambda}{_{\alpha}^{\gamma}} U_{\gamma} =
  \tensor{\delta}{_{\beta}^{\gamma}} V^{\beta} U_{\gamma} = V^{\gamma}U_{\gamma}
  = V^{\alpha} U_{\alpha}.
\end{equation}

Si può definire anche il \index{modulo!di un quadrivettore}
\emph{modulo quadro di un quadrivettore} $V$ come il prodotto scalare di $V$ per
se stesso, vale a dire
\begin{equation}
  \begin{split}
    V\cdot V &= V^{\alpha}V_{\alpha} =
    \tensor{\eta}{_{\alpha\beta}}V^{\alpha}V^{\beta} = V^{0}V_{0} + V^{1}V_{1} +
    V^{2}V_{2} + V^{3}V_{3} \\
    &= -(V^{0})^{2} + (V^{1})^{2} + (V^{2})^{2} + (V^{3})^{2} = -(V^{0})^{2} +
    \bm{V}^{2}.
  \end{split}
\end{equation}
Anche il modulo quadro di un quadrivettore è uno scalare di Lorentz.

\subsubsection{Derivazione}
\label{sec:derivazione-minkowski}

La derivazione di un quadritensore di rango $k$ rispetto alle coordinate
controvarianti $x^{\alpha}$ o covarianti
$x_{\alpha} = (x_{0},x_{i}) = (-x^{0},x^{i})$ è un quadritensore di rango $k+1$.
In dettaglio, la derivata
\begin{equation}
  \parder{}{x^{\sigma}}
  \tensor*{T}{^{\alpha_{1}\dots\alpha_{p}}_{\beta_{1}\dots\beta_{q}}}
\end{equation}
è un quadritensore del tipo
$\tensor*{S}{^{\alpha_{1}\dots\alpha_{p}}_{\beta_{1}\dots\beta_{q}\sigma}}$ e la
derivata
\begin{equation}
  \parder{}{x_{\sigma}}
  \tensor*{T}{^{\alpha_{1}\dots\alpha_{p}}_{\beta_{1}\dots\beta_{q}}}
\end{equation}
è un quadritensore del tipo
$\tensor*{S}{^{\alpha_{1}\dots\alpha_{p}\sigma}_{\beta_{1}\dots\beta_{q}}}$.
Consideriamo il caso particolare di uno scalare $\phi$.  Dalla
trasformazione~\eqref{eq:trasformazione-inversa} abbiamo
\begin{equation}
  \parder{x^{\beta}}{x'^{\alpha}} = \tensor{\Lambda}{_{\alpha}^{\beta}}
\end{equation}
quindi
\begin{equation}
  \parder{\phi}{x^{\alpha}} \to \parder{\phi}{x'^{\alpha}}
  = \parder{\phi}{x^{\beta}} \parder{x^{\beta}}{x'^{\alpha}} =
  \tensor{\Lambda}{_{\alpha}^{\beta}} \parder{\phi}{x^{\beta}}.
\end{equation}
Abbiamo così mostrato che la derivata di uno scalare (quadritensore di rango 0)
rispetto a una coordinata controvariante è un quadrivettore (quadritensore di
rango 1) covariante
\begin{equation}
  \parder{\phi}{x^{\alpha}} = U_{\alpha}.
\end{equation}

Definiamo l'operatore
\index{quadrigradiente!covariante}\emph{quadrigradiente covariante} come
\begin{equation}
  \partial_{\alpha} = \parder{}{x^{\alpha}} = \bigg( \parder{}{t}, \nabla\bigg)
\end{equation}
e l'operatore
\index{quadrigradiente!controvariante}\emph{quadrigradiente controvariante} come
\begin{equation}
  \partial^{\alpha} = \parder{}{x_{\alpha}} =
  \tensor{\eta}{^{\alpha\beta}} \partial_{\beta} = \bigg( -\parder{}{t},
  \nabla\bigg).
\end{equation}
Infine introduciamo l'operatore \index{d'Alambertiano}\emph{d'Alambertiano}
definito da
\begin{equation}
  \dalamb = \partial^{\alpha}\partial_{\alpha} = - \parder{}{*2t} + \nabla^{2}.
\end{equation}
Poiché il d'Alambertiano è il modulo quadro del quadrigradiente, esso è un
operatore differenziale scalare.

\subsection{Quadritensori invarianti}
\label{sec:tensori-invarianti-minkowski}

Ci sono dei particolari quadritensori che assumono la stessa forma in qualsiasi
sistema di riferimento inerziale, così come fanno gli scalari.

\subsubsection{Tensore metrico di Minkowski}
\label{sec:tensore-metrico-minkowski}

Il primo quadritensore che analizzeremo è il tensore metrico di Minkowski.
Dalla relazione fondamentale del tensore
metrico~\eqref{eq:rel-fond-tensore-metrico} abbiamo che
$\tensor{\eta}{_{\alpha\beta}}$ è effettivamente di un quadritensore covariante
di rango $2$ che non cambia per effetto di una trasformazione di Lorentz
\begin{equation}
  \tensor{\eta}{_{\alpha\beta}} \to \tensor{{\eta'}}{_{\alpha\beta}} =
  \tensor{\Lambda}{_{\alpha}^{\gamma}} \tensor{\Lambda}{_{\beta}^{\delta}}
  \tensor{\eta}{_{\gamma\delta}} = \tensor{\eta}{_{\alpha\beta}}.
\end{equation}
In maniera analoga si può dimostrare che $\tensor{\eta}{^{\alpha\beta}}$ è
effettivamente un quadritensore controvariante di rango $2$ e che rimane
immutato sotto una trasformazione di Lorentz.

\subsubsection{Simbolo di Levi-Civita}
\label{sec:simbolo-levi-civita}

Il \index{simbolo!di Levi-Civita}\emph{simbolo di Levi-Civita}
$\epsilon^{\alpha\beta\gamma\delta}$ è un quadritensore definito da
\begin{equation}
  \epsilon^{\alpha\beta\gamma\delta} =
  \begin{cases}
    +1 & \text{se $\alpha\beta\gamma\delta$ è una permutazione pari di $0123$,}
    \\
    -1 & \text{se $\alpha\beta\gamma\delta$ è una permutazione dispari di
      $0123$,} \\
    0 & \text{se due o più indici sono uguali.}
  \end{cases}
\end{equation}
Si può far vedere che il tensore così definito non varia cambiando sistema di
riferimento inerziale e che è \emph{completamente antisimmetrico}, cioè cambia
di segno per ogni permutazione di due indici.

Poiché sussiste la relazione
\begin{equation}
  \epsilon^{\alpha\beta\gamma\delta} = (\det\Lambda)
  \tensor{\Lambda}{^{\alpha}_{\mu}} \tensor{\Lambda}{^{\beta}_{\nu}}
  \tensor{\Lambda}{^{\gamma}_{\rho}} \tensor{\Lambda}{^{\delta}_{\sigma}}
  \epsilon^{\mu\nu\rho\sigma}
\end{equation}
il simbolo di Levi-Civita è uno pseudotesore di rango $4$ e peso $1$.

Riportiamo inoltre qui per convenienza, senza dimostrazione, alcune utili
identità soddisfatte dal simbolo di Levi-Civita
\begin{subequations}
  \begin{align}
    \epsilon^{\mu\nu\rho\sigma}\epsilon_{\mu\nu\rho\sigma} &= -24, \\
    \epsilon^{\mu\nu\rho\sigma}\epsilon_{\lambda\tau\rho\sigma} &=
    -2(\tensor{\delta}{^{\mu}_{\lambda}} \tensor{\delta}{^{\nu}_{\tau}} -
    \tensor{\delta}{^{\mu}_{\tau}} \tensor{\delta}{^{\nu}_{\lambda}}), \\
    \epsilon^{\mu\nu\rho\sigma}\epsilon_{\lambda\nu\rho\sigma} &= -6
    \tensor{\delta}{^{\mu}_{\lambda}}.
  \end{align}
\end{subequations}

\subsubsection{Tensore nullo}
\label{sec:tensore-nullo}

Dati due arbitrari quadritensore dello stesso tipo
$\tensor*{A}{^{\alpha_{1}\dots\alpha_{p}}_{\beta_{1}\dots\beta_{q}}}$ e
$\tensor*{B}{^{\alpha_{1}\dots\alpha_{p}}_{\beta_{1}\dots\beta_{q}}}$, se in un
certo sistema di riferimento inerziale risulta
\begin{equation}
  \label{eq:A-uguale-B}
  \tensor*{A}{^{\alpha_{1}\dots\alpha_{p}}_{\beta_{1}\dots\beta_{q}}} =
  \tensor*{B}{^{\alpha_{1}\dots\alpha_{p}}_{\beta_{1}\dots\beta_{q}}}
\end{equation}
è possibile definire un quadritensore
$\tensor*{Z}{^{\alpha_{1}\dots\alpha_{p}}_{\beta_{1}\dots\beta_{q}}}$, detto
\index{tensore!nullo}\emph{tensore nullo}, invariante per trasformazioni di
Lorentz come la differenza fra i due quadritensori:
\begin{equation}
  \tensor*{Z}{^{\alpha_{1}\dots\alpha_{p}}_{\beta_{1}\dots\beta_{q}}} =
  \tensor*{A}{^{\alpha_{1}\dots\alpha_{p}}_{\beta_{1}\dots\beta_{q}}} -
  \tensor*{B}{^{\alpha_{1}\dots\alpha_{p}}_{\beta_{1}\dots\beta_{q}}}.
\end{equation}
Dalla~\eqref{eq:A-uguale-B} ricaviamo che tutte le componenti di
$\tensor*{Z}{^{\alpha_{1}\dots\alpha_{p}}_{\beta_{1}\dots\beta_{q}}}$ sono
nulle, quindi anche il quadritensore
$\tensor*{{Z'}}{^{\alpha_{1}\dots\alpha_{p}}_{\beta_{1}\dots\beta_{q}}} =
\tensor*{{A'}}{^{\alpha_{1}\dots\alpha_{p}}_{\beta_{1}\dots\beta_{q}}} -
\tensor*{{B'}}{^{\alpha_{1}\dots\alpha_{p}}_{\beta_{1}\dots\beta_{q}}}$
ottenuto dalla trasformazione di Lorentz
\begin{equation}
  \tensor*{{Z'}}{^{\alpha_{1}\dots\alpha_{p}}_{\beta_{1}\dots\beta_{q}}} =
  \tensor{\Lambda}{^{\alpha_{1}}_{\mu_{1}}} \cdots
  \tensor{\Lambda}{^{\alpha_{p}}_{\mu_{p}}}
  \tensor{\Lambda}{_{\beta_{1}}^{\nu_{1}}} \cdots
  \tensor{\Lambda}{_{\beta_{q}}^{\nu_{q}}}
  \tensor*{Z}{^{\mu_{1}\dots\mu_{p}}_{\nu_{1}\dots\nu_{q}}}
\end{equation}
avrà tutte le componenti nulle, così che nel nuovo sistema di riferimento
inerziale risulta anche
\begin{equation}
  \tensor*{{A'}}{^{\alpha_{1}\dots\alpha_{p}}_{\beta_{1}\dots\beta_{q}}} =
  \tensor*{{B'}}{^{\alpha_{1}\dots\alpha_{p}}_{\beta_{1}\dots\beta_{q}}}.
\end{equation}
Questa proprietà del tensore nullo è molto importante poiché ci dice che
\emph{se due quadritensori, dello stesso tipo, sono uguali in un sistema di
  riferimento inerziale, saranno uguali in qualsiasi sistema di riferimento
  inerziale ottenuto dal primo mediante una trasformazione di Lorentz}.
Da questa affermazione deduciamo inoltre che\label{invarianza-lorentz}
\emph{qualsiasi legge fisica espressa come un'uguaglianza fra quadritensori è
  automaticamente invariante per trasformazioni di Lorentz}.

\section{Dinamica particellare}
\label{sec:dinamica-particellare}

Data una particella il cui moto è descritto dalla coordinata controvariante
$x^{\alpha}$, definiamo la sua \index{quadrivelocità}\emph{quadrivelocità} come
la derivata di $x^{\alpha}$ rispetto al tempo proprio $\tau$:
\begin{equation}
  V^{\alpha} = (V^{0}, V^{i}) =\toder{x^{\alpha}}{\tau}
\end{equation}
e poiché risulta
\begin{equation}
  \dd \tau = \sqrt{\dd t^{2} - \dd \bm{x}^{2}} = \dd t \sqrt{1 -
    \bigg(\toder{\bm{x}}{t}\bigg)^{2}} = \dd t\sqrt{1 - v^{2}} = \frac{\dd
    t}{\gamma_{v}}
\end{equation}
abbiamo che la quadrivelocità è uguale
a\footnote{Abbiamo qui indicato con due lettere diverse la velocità classica
  $\bm{v} = \ltoder{\bm{x}}{t}$ e il quadrivettore velocità
  $V^{\alpha} = \ltoder{x^{\alpha}}{\tau}$ perché le componenti spaziali $V^{i}$
  della quadrivelocità differiscono dalle corrispondenti componenti del vettore
  velocità $v_{i}$ per il fattore lorentziano $\gamma_{v}$.  Da qui in poi sarà
  compito del lettore riuscire a distinguere fra le componenti del vettore
  velocità classica e le componenti spaziali del quadrivettore velocità.}
\begin{equation}
  \begin{split}
    V^{\alpha} &= \toder{x^{\alpha}}{t}\toder{t}{\tau} = \toder{x^{\alpha}}{t}
    \gamma_{v} = \gamma_{v} \bigg(\toder{x^{0}}{t}, \toder{\bm{x}}{t}\bigg) =
    \gamma_{v}(1,\bm{v}) \\
    &= (\gamma_{v}, \gamma_{v}\bm{v}).
  \end{split}
\end{equation}
Notiamo che per $\norm{\bm{v}} \ll 1$ il fattore lorentziano vale $1$ e in
questo caso $V^{\alpha} \approx (1, \bm{v})$.  Il
\index{quadrimomento}\emph{quadrimomento} della particella è definito come il
prodotto fra la massa a riposo $m$ della particella e la quadrivelocità
\begin{equation}
  p^{\alpha} = (p^{0},p^{i}) = m V^{\alpha} = m (V^{0}, V^{i}) = (m\gamma_{v},
  m\gamma_{v}\bm{v}).
\end{equation}
Ricordando che l'energia totale $E$ di una particella è definita come
$m\gamma_{v}$ troviamo che la prima componente del quadrivettore momento è
proprio l'energia totale, mentre le componenti $p^{i}$ sono le componenti del
momento relativistico $\bm{p} = m\gamma_{v}\bm{v}$
\begin{equation}
  p^{\alpha} = (E, \bm{p}).
\end{equation}
Vediamo come cambia il quadrimomento per effetto di un boost $\bm{V}$
\begin{equation}
  p^{\alpha} \to p'^{\alpha} = (p'^{0}, p'^{i}) =
  \tensor{\Lambda}{^{\alpha}_{\beta}} p^{\beta} =
  (\tensor{\Lambda}{^{0}_{\beta}}p^{\beta},
  \tensor{\Lambda}{^{i}_{\beta}}p^{\beta}).
\end{equation}
La prima componente ci dà l'energia $E'$ nel nuovo sistema di riferimento
\begin{equation}
  \begin{split}
    E' &= p'^{0} = \tensor{\Lambda}{^{0}_{\beta}}p^{\beta} =
    \tensor{\Lambda}{^{0}_{0}}p^{0} + \tensor{\Lambda}{^{0}_{i}}p^{i} \\
    &= \gamma_{V} E + \gamma_{V} V_{i}p^{i} = \gamma_{V} E + \gamma_{V}
    \bm{V}\cdot\bm{p}.
  \end{split}
\end{equation}
Le tre componenti spaziali sono le componenti $(\bm{p}')_{i}$ del vettore
momento nel nuovo sistema di riferimento
\begin{equation}
  \begin{split}
    (\bm{p}')_{i} &= (p')^{i} = \tensor{\Lambda}{^{i}_{\beta}}p^{\beta} =
    \tensor{\Lambda}{^{i}_{0}}p^{0} + \tensor{\Lambda}{^{i}_{j}}p^{j} \\
    &= \gamma_{V} V_{i}E + \bigg( \delta_{ij} +
    V_{i}V_{j}\frac{\gamma_{V}-1}{V^{2}}\bigg)p^{j} \\
    &= \gamma_{V} V_{i}E + p_{i} +
    V_{i}V_{j}p^{j}\frac{\gamma_{V}-1}{V^{2}} \\
    &= \gamma_{V} V_{i}E + p_{i} + V_{i} \bm{V} \cdot \bm{p}
    \frac{\gamma_{V}-1}{V^{2}},
  \end{split}
\end{equation}
cioè
\begin{equation}
  \bm{p}' = \bm{p} + \bigg(\gamma_{V} E + \bm{V} \cdot \bm{p}
  \frac{\gamma_{V}-1}{V^{2}}\bigg)\bm{V}.
\end{equation}
% NOTA: i risultati delle trasformazioni di momento ed energia sono differenti
% dalle formule [4.103] e [4.104], pagina 132, del Barone.  Weinberg definisce
% la velocità di spostamento fra i due sistemi di riferimento al contrario
% rispetto a quasi tutti gli altri autori (compreso il Barone).
Calcoliamo il modulo quadro del quadrimomento
\begin{equation}
  p^{\alpha}p_{\alpha} = \tensor{\eta}{_{\alpha\beta}}p^{\alpha}p^{\beta} =
  -E^{2} + \bm{p}^{2} = -m^{2}
\end{equation}
avendo usato la relazione $E^{2} = m^{2} + \bm{p}^{2}$.  Questo fa vedere che la
massa a riposo di una particella è uno scalare di Lorentz, quindi un invariante
relativistico.  Il quadrimomento covariante è dato da
\begin{equation}
  p_{\alpha} = \tensor{\eta}{_{\alpha\beta}}p^{\beta} = (-E, \bm{p}).
\end{equation}

La \index{quadriforza}\emph{quadriforza} agente sulla particella è definita come
la derivata rispetto al tempo proprio del quadrimomento
\begin{equation}
  f^{\alpha} = (f^{0},f^{i}) =\toder{p^{\alpha}}{\tau} =
  m\toder{v^{\alpha}}{\tau} = mw^{\alpha}
\end{equation}
in cui
\begin{equation}
  w^{\alpha} = (w^{0},w^{i}) =\toder{v^{\alpha}}{\tau}
\end{equation}
è la \index{quadriaccelerazione}\emph{quadriaccelerazione}.  In un sistema di
riferimento $\mathcal{S}$ in cui la particella è a riposo $\dd \bm{x} = \bm{0}$,
quindi $\dd \tau = \dd t$ e $f^{\alpha} = F^{\alpha}$, in cui $F^{i}$ sono le
componenti cartesiane della forza non relativistica $\bm{F}$ e $F^{0} = 0$:
\begin{equation}
  f^{\alpha} = (f^{0}, f^{i}) = (0, \bm{F}).
\end{equation}
Per calcolare la forza a cui è soggetta la particella in un altro sistema di
riferimento inerziale $\mathcal{S}'$ che vede la particella muoversi con
velocità $\bm{V} = (V_{1}, V_{2}, V_{3})$ dobbiamo effettuare un boost $\bm{V}$
\begin{equation}
  f'^{\alpha} = (f'^{0}, f'^{i}) = \tensor{\Lambda}{^{\alpha}_{\beta}}f^{\beta}.
\end{equation}
Abbiamo
\begin{subequations}
  \begin{align}
    \begin{split}
      f'^{0} &= \tensor{\Lambda}{^{0}_{\alpha}}f^{\alpha} =
      \tensor{\Lambda}{^{0}_{0}}f^{0} + \tensor{\Lambda}{^{0}_{i}}f^{i} =
      \gamma_{V} V_{i}f^{i} = \gamma_{V} \bm{V} \cdot \bm{f} \\
      &= \gamma_{V} \bm{V} \cdot \bm{F}
    \end{split} \\
    \begin{split}
      f'^{i} &= \tensor{\Lambda}{^{i}_{\alpha}}f^{\alpha} =
      \tensor{\Lambda}{^{i}_{0}}f^{0} + \tensor{\Lambda}{^{i}_{j}}f^{j}
      = \bigg( \delta_{ij} + V_{i}V_{j}\frac{\gamma_{V} - 1}{V^{2}} \bigg)f^{j}\\
      &= f^{i} + V_{i}V_{j}f^{j}\frac{\gamma_{V} - 1}{V^{2}} = f^{i} + V_{i}
      \bm{V} \cdot \bm{f} \frac{\gamma_{V} - 1}{V^{2}},
    \end{split}
  \end{align}
\end{subequations}
cioè
\begin{equation}
  \bm{f}' = \bm{F} + \bm{V}(\bm{V}\cdot\bm{F})\frac{\gamma_{V} - 1}{V^{2}}.
\end{equation}
% Questa formula è diversa dalla [4.126] a pagina 135 del Barone perché lui
% parte da un sistema di riferimento in cui la particella si muove con velocità
% `v', noi invece siamo partiti da un sistema in cui la particella è a riposo.

\section{Formulazione covariante dell'elettrodinamica}
\label{sec:elettrodinamica-covariante}

In questo paragrafo vogliamo far vedere come il formalismo della relatività
speciale permetta di riscrivere le leggi dell'elettrodinamica, e in particolare
le equazioni di Maxwell, in modo manifestamente covariante (o invariante in
forma) rispetto alle trasformazioni di Lorentz.

\subsection{Densità di corrente e di carica}
\label{sec:densita-corrente-carica}

Consideriamo, per semplicità, una distribuzione di particelle in cui l'$n$-esima
particella ha carica $e_{n}$ e occupa, in un certo sistema di riferimento
tridimensionale, la posizione $\bm{x}_{n}$.  La \index{densità!di carica}densità
di carica di questa distribuzione di particelle al tempo $t$ è data da
\begin{equation}
  \rho(\bm{x},t) = \sum_{n} e_{n}\delta^{3}(\bm{x} - \bm{x}_{n}(t)),
\end{equation}
mentre la \index{densità!di corrente}densità di corrente è
\begin{equation}
  \bm{J}(\bm{x},t) = \sum_{n} e_{n}\delta^{3}(\bm{x} - \bm{x}_{n}(t))
  \toder{\bm{x}_{n}(t)}{t}.
\end{equation}
Abbiamo qui utilizzato la \index{delta!di Dirac tridimensionale}
\emph{delta di Dirac tridimensionale} $\delta^{3}(\bm{x} - \bm{x}')$, cioè
quella funzione che per ogni funzione $f$ liscia di tre variabili reali ha la
proprietà
\begin{equation}
  \int\limits_{\R^{3}} f(\bm{x})\delta^{3}(\bm{x} - \bm{x}') \dd^{3} \bm{x} =
  f(\bm{x}').
\end{equation}
Possiamo mettere insieme le due quantità $\rho$ e $\bm{J}$ in un quadrivettore
$J^{\alpha}$ che chiameremo \index{quadricorrente}\emph{quadricorrente}
\begin{equation}
  J^{\alpha}(x) = (\rho, \bm{J}) = \sum_{n} e_{n}\delta^{3}(\bm{x} -
  \bm{x}_{n}(t)) \toder{x_{n}^{\alpha}}{t}.
\end{equation}
Introduciamo adesso la \index{delta!di Dirac quadrimensionale}
\emph{delta di Dirac quadrimensionale}
$\delta^{4}(x - x') = \delta(t - t')\delta^{3}(\bm{x} - \bm{x}')$, vale a dire
la funzione che per ogni funzione $f$ liscia con dominio in $\M$ soddisfa la
proprietà
\begin{equation}
  \int\limits_{\M} f(x)\delta^{4}(x - y)\dd^{4} x = f(y).
\end{equation}
Tutte le delta di Dirac sono scalari di Lorentz in quanto non cambiano la natura
tensoriale di $f$.  Moltiplicando $J^{\alpha}(x)$ per la delta di Dirac
$\delta(t - t')$ e integrando sul tempo ricaviamo la seguente espressione
\begin{equation}
  \label{eq:quadricorrente}
  \begin{split}
    J^{\alpha}(x) &= \sum_{n} e_{n}\delta^{3}(\bm{x} - \bm{x}_{n}(t))
    \toder{x_{n}^{\alpha}}{t} \\
    &= \int_{-\infty}^{+\infty}\dd t' \delta(t - t') \sum_{n}
    e_{n}\delta^{3}(\bm{x} - \bm{x}_{n}(t')) \toder{x_{n}^{\alpha}}{t'} \\
    &= \int_{-\infty}^{+\infty} \dd \tau \sum_{n} e_{n} \delta^{4}(x -
    x_{n}(\tau)) \toder{x_{n}^{\alpha}}{\tau}.
  \end{split}
\end{equation}
Nell'ultimo passaggio abbiamo effettuato il cambiamento della variabile di
integrazione da $t'$ all'invariante $\tau$.  Da questa espressione è evidente
che $J^{\alpha}$ è un quadrivettore: $\dd \tau$ è uno scalare, così come la
delta di Dirac $\delta^{4}(x - x_{n}(\tau))$, mentre $\ltoder{x^{\alpha}}{\tau}$
è la quadrivelocità.

In elettrodinamica classica si dimostra
l'\index{equazione!di continuità}\emph{equazione di continuità}
\begin{equation}
  \nabla\cdot\bm{J} + \parder{\rho}{t} = 0
\end{equation}
che, usando il formalismo relativistico, può essere riscritta in termini della
quadricorrente come
\begin{equation}
  \partial_{\alpha}J^{\alpha} = 0.
\end{equation}
Applicando la conseguenza del teorema di Gauss nello spazio di Minkowski
discussa nell'appendice~\ref{cha:teorema-gauss} alla quadricorrente abbiamo che
la quantità
\begin{equation}
  Q = \int J^{0}\dd V = \int \rho \dd V
\end{equation}
è costante rispetto al tempo se l'integrale è esteso a una regione abbastanza
grande dello spazio in modo da racchiudere tutte le cariche del sistema e che
sul suo bordo la quadricorrente sia nulla.  Poiché $\rho$ è la densità di
carica, la quantità $Q$ rappresenta la carica totale del sistema, pertanto
dall'equazione di continuità discende, grazie al teorema di Gauss, la
conservazione della carica.

\subsection{Le equazioni di Maxwell}
\label{sec:equazioni-maxwell}

Ricordiamo le \index{equazioni!di Maxwell}\emph{equazioni di Maxwell}
\begin{subequations}
  \begin{align}
    \nabla \cdot \bm{E} &= \rho, \label{eq:maxwell-divE} \\
    \nabla \times \bm{B} &= \bm{J} + \parder{\bm{E}}{t}, \label{eq:maxwell-rotB}
    \\
    \nabla \cdot \bm{B} &= 0, \label{eq:maxwell-divB} \\
    \nabla \times \bm{E} &= -\parder{\bm{B}}{t}, \label{eq:maxwell-rotE}
  \end{align}
\end{subequations}
in cui i campi elettrico $\bm{E}$ e magnetico $\bm{B}$ sono legati ai potenziali
scalare $\phi$ e vettore $\bm{A}$ dalle relazioni
\begin{subequations}
  \label{eq:campi-potenziali}
  \begin{align}
    \bm{E} &= -\nabla\phi - \parder{\bm{A}}{t}, \\
    \bm{B} &= \nabla \times \bm{A}.
  \end{align}
\end{subequations}
Possiamo mettere insieme i due potenziali per formare un quadrivettore $A^{\mu}$
(controvariante) chiamato \index{quadripotenziale elettromagnetico}
\emph{quadripotenziale elettromagnetico}
\begin{equation}
  A^{\mu} = (A^{0}, A^{i}) = (\phi, \bm{A}).
\end{equation}
Inoltre il quadripotenziale covariante è dato, come al solito, da
\begin{equation}
  A_{\mu} = \eta_{\mu\nu}A^{\nu} = (-A^{0}, A^{i}) = (-\phi, \bm{A}).
\end{equation}
Introduciamo il \index{tensore!del campo elettromagnetico}
\emph{tensore del campo elettromagnetico} $F^{\mu\nu}$ definito dalla relazione
\begin{equation}
  F_{\alpha\beta} = \partial_{\alpha}A_{\beta} - \partial_{\beta}A_{\alpha}.
\end{equation}
Si può verificare che $F_{\alpha\beta}$ è un quadritensore di rango $2$
completamente covariante, inoltre esso è antisimmetrico, infatti
\begin{equation}
  F_{\beta\alpha} = \partial_{\beta}A_{\alpha} - \partial_{\alpha}A_{\beta} =
  -F_{\alpha\beta}.
\end{equation}
Di conseguenza le componenti diagonali $F_{\alpha\alpha}$ sono nulle e le
componenti indipendenti di questo quadritensore sono sei.  Verifichiamo che
queste componenti indipendenti corrispondono proprio alle sei componenti
cartesiane dei campi $\bm{E}$ e $\bm{B}$.  Dalle~\eqref{eq:campi-potenziali}
abbiamo
\begin{subequations}
  \begin{align}
    F_{0i} &= \partial_{0}A_{i} - \partial_{i}A_{0} = \parder{A_{i}}{t}
    + \parder{\phi}{x^{i}} = -E_{i}, \\
    F_{12} &= \partial_{1}A_{2} - \partial_{2}A_{1} = \parder{A_{y}}{x}
    - \parder{A_{x}}{y} = B_{z}, \\
    F_{13} &= \partial_{1}A_{3} - \partial_{3}A_{1} = \parder{A_{z}}{x}
    - \parder{A_{x}}{z} = -B_{y}, \\
    F_{23} &= \partial_{2}A_{3} - \partial_{3}A_{2} = \parder{A_{z}}{y}
    - \parder{A_{y}}{z} = B_{x}.
  \end{align}
\end{subequations}
Le ultime tre componenti possono essere sintetizzate in
\begin{equation}
  F_{ij} = \partial_{i}A_{j} - \partial_{j}A_{i} = \epsilon_{ijk}B^{k}.
\end{equation}
In definitiva l'espressione matriciale di $F_{\alpha\beta}$ è
\begin{equation}
  \label{eq:tensore-elettromagnetico}
  F_{\alpha\beta} =
  \begin{pmatrix}
    0     & -E_{x} & -E_{y} & -E_{z} \\
    E_{x} & 0      & B_{z}  & -B_{y} \\
    E_{y} & -B_{z} & 0      & B_{x}  \\
    E_{z} & B_{y}  & -B_{x} & 0
  \end{pmatrix}.
\end{equation}
Il tensore elettromagnetico completamente controvariante è
$F^{\alpha\beta} = \eta^{\alpha\gamma}\eta^{\beta\delta}F_{\gamma\delta}$ e
risulta quindi $F^{0i} = -F_{0i}$, $F^{i0} = -F_{i0}$ e $F^{ij} = F_{ij}$.
Usando le componenti del tensore elettromagnetico si può vedere come le
componenti dei campi $\bm{E}$ e $\bm{B}$ si modificano per effetto di una
trasformazione di Lorentz.

A partire dal tensore del campo elettromagnetico è possibile costruire le due
quantità invarianti (essendo contrazioni complete di quadritensori)
\begin{subequations}
  \begin{align}
    F^{\alpha\beta}F_{\alpha\beta} &= -2(\bm{E}^{2} - \bm{B}^{2}) = 2(\bm{B}^{2}
    - \bm{E}^{2}), \\
    \epsilon_{\alpha\beta\gamma\delta}F^{\alpha\beta}F^{\gamma\delta} &= -2
    \bm{E} \cdot \bm{B}.
  \end{align}
\end{subequations}
Mentre $F^{\alpha\beta}F_{\alpha\beta}$ è uno scalare di Lorentz, la quantità
$\epsilon_{\alpha\beta\gamma\delta}F^{\alpha\beta}F^{\gamma\delta}$ è uno
pseudoscalare perché è il prodotto scalare del vettore polare $\bm{E}$ e del
vettore assiale $\bm{B}$.

Grazie al tensore del campo elettromagnetico $F^{^{\alpha\beta}}$ e alla
quadricorrente $J^{\alpha}$ possiamo scrivere le quattro
\index{equazioni!di Maxwell}equazioni di Maxwell in forma covariante
\begin{subequations}
  \begin{align}
    \partial_{\alpha}F^{\alpha\beta} &= -J^{\beta}, \label{eq:maxwell-cov1} \\
    \epsilon^{\alpha\beta\gamma\delta} \partial_{\beta} F_{\gamma\delta} &=
    0. \label{eq:maxwell-cov2}
  \end{align}
\end{subequations}
La prima di queste equazioni equivale per $\beta=0$ alla~\eqref{eq:maxwell-divE}
e per $\beta = i$ alle tre componenti della~\eqref{eq:maxwell-rotB}, la seconda
delle equazioni precedenti equivale per $\alpha = 0$
alla~\eqref{eq:maxwell-divB} e per $\beta = i$ alle tre componenti
della~\eqref{eq:maxwell-rotE}.  Il fatto che abbiamo scritto le
\index{equazioni!di Maxwell}equazioni di Maxwell sotto forma di uguaglianza fra
quadrivettori ci dice immediatamente che queste sono invarianti per
trasformazioni di Lorentz, per quanto spiegato a
pagina~\pageref{invarianza-lorentz}.  L'equazione
omogenea~\eqref{eq:maxwell-cov2} può essere espressa anche come
\begin{equation}
  \label{eq:maxwell-cov22}
  \partial_{\alpha}F_{\beta\gamma} + \partial_{\beta}F_{\gamma\alpha}
  + \partial_{\gamma}F_{\alpha\beta} = 0.
\end{equation}

Il tensore $F_{\alpha\beta}$ del campo elettromagnetico è invariante per
trasformazioni di gauge del quadripotenziale del tipo
$A_{\alpha}(x) \to A'_{\alpha}(x) = A_{\alpha}(x) + \partial_{\alpha}\chi(x)$,
in cui $\chi(x)$ è una funzione scalare delle coordinate.  Infatti
\begin{equation}
  \begin{split}
    F_{\alpha\beta} \to F'_{\alpha\beta} &= \partial_{\alpha}A'_{\beta}
    - \partial_{\beta}A'_{\alpha} = \partial_{\alpha}(A_{\beta}
    + \partial_{\beta}\chi) - \partial_{\beta}(A_{\alpha}
    + \partial_{\alpha}\chi) \\
    &= \partial_{\alpha}A_{\beta} - \partial_{\beta}A_{\alpha} =
    F_{\alpha\beta}.
  \end{split}
\end{equation}
Pertanto i campi elettrico e magnetico non cambiano per effetto di una tale
trasformazione, come noto dall'elettrodinamica classica.

La forza elettromagnetica agente su una particella di carica $q$ in un sistema
di riferimento in cui essa è a riposo è
\begin{equation}
  \label{eq:quadriforza-lorentz}
  f^{\alpha} = q\eta_{\beta\gamma}F^{\alpha\beta} \toder{x^{\gamma}}{\tau} = q
  \tensor{F}{^{\alpha}_{\gamma}} \toder{x^{\gamma}}{\tau}.
\end{equation}
Per $\alpha = 0$ si ottiene
\begin{equation}
  \toder{E}{t} = \bm{F}\cdot\bm{v} = q\bm{E}\cdot\bm{v},
\end{equation}
mentre per $\alpha = i$ si ricava la forza di Lorentz
\begin{equation}
  \label{eq:forza-lorentz}
  \bm{F} = q\bm{E} + q\bm{v}\times\bm{B}.
\end{equation}

L'\index{equazioni!di Maxwell}equazione di Maxwell non omogenea in forma
covariante~\eqref{eq:maxwell-cov1} può anche essere scritta come
$\partial^{\alpha}F_{\alpha\beta} = -J_{\beta}$ ed esplicitando il tensore del
campo elettromagnetico abbiamo
\begin{equation}
  \partial^{\alpha}(\partial_{\alpha}A_{\beta} - \partial_{\beta}A_{\alpha})
  = \partial_{\alpha}\partial^{\alpha}A_{\beta}
  - \partial_{\beta}\partial^{\alpha}A_{\alpha} = -J_{\beta}.
\end{equation}
Adottando la \index{gauge!di Lorenz}gauge di Lorenz\footnote{Questa condizione
  di gauge è stata introdotta nel 1867 dal fisico danese Ludvig Lorenz (1829 --
  1891) e non dal fisico olandese Hendrik Lorentz (1853 -- 1928) al quale viene
  spesso erroneamente attribuita.  Si veda per esempio l'articolo di
  J.~D.~Jackson e L.~B.~Okun
  arXiv:\href{https://arxiv.org/abs/hep-ph/0012061}{\texttt{hep-ph/0012061}}.}
$\partial^{\alpha}A_{\alpha} = 0$ otteniamo l'\index{equazione!di
  d'Alambert}equazione di d'Alambert non omogenea
\begin{equation}
  \partial^{\alpha}\partial_{\alpha}A_{\beta} = \dalamb A_{\beta} = \bigg(
  -\parder{}{*{2}{t}} + \nabla^{2} \bigg) A_{\beta} = -J_{\beta}.
\end{equation}
Possiamo effettuare una trasformazione di gauge
$A_{\alpha} \to A'_{\alpha} = A_{\alpha} + \partial_{\alpha}\chi$ compatibile
con la condizione di Lorenz a patto che la funzione scalare $\chi$ che genera la
trasformazione abbia d'Alambertiano nullo $\dalamb \chi = 0$, infatti
\begin{equation}
  \partial^{\alpha} A'_{\alpha} = \partial^{\alpha} A_{\alpha}
  + \partial^{\alpha}\partial_{\alpha} \chi = 0
\end{equation}
solo se
$\dalamb \chi = \partial^{\alpha}\partial_{\alpha} \chi = -\partial^{\alpha}
A_{\alpha} = 0$.

\subsection{Onde elettromagnetiche nel vuoto}
\label{sec:onde-elettromagnetiche-vuoto}

In assenza di sorgenti, cioè nel vuoto, la quadricorrente è nulla quindi si ha
\begin{equation}
  \label{eq:maxwell-vuoto-lorenz}
  \dalamb A_{\beta} = 0.
\end{equation}
La soluzione generale di questa equazione è una sovrapposizione lineare di onde
piane del tipo
\begin{equation}
  A_{\alpha} = \Re \{e_{\alpha} \e^{\uimm k^{\lambda} x_{\lambda}}\},
\end{equation}
in cui $k^{\lambda} = (\omega, \bm{k})$ è il
\index{quadrivettore!d'onda}\emph{quadrivettore d'onda} e $e_{\alpha}$ il
quadrivettore, costante, ampiezza del quadripotenziale.  Se ci limitiamo a
eseguire solo operazioni lineari su $A_{\alpha}$, possiamo omettere, per
semplicità, il simbolo $\Re$.  Sostituendo l'espressione onda piana di
$A_{\alpha}$ nell'equazione di D'alambert omogenea abbiamo
\begin{equation}
  0 = \partial^{\beta}\partial_{\beta} A_{\alpha} = \partial^{\beta}\partial_{\beta}
  e_{\alpha} \e^{\uimm k^{\lambda} x_{\lambda}} = (\uimm k_{\beta})(\uimm
  k^{\beta}) e_{\alpha} \e^{\uimm k^{\lambda} x_{\lambda}} = -(-\omega^{2} +
  \bm{k}^{2}) e_{\alpha} \e^{\uimm k^{\lambda} x_{\lambda}}
\end{equation}
che implica
\begin{equation}
  \omega = \norm{\bm{k}} \iff 2\pi\nu = \frac{2\pi}{\lambda} \iff \lambda\nu=1,
\end{equation}
cioè le onde si muovono alla velocità della luce nel vuoto (che abbiamo posto
uguale a $1$), come ci si aspetta.  Di conseguenza il modulo del quadrivettore
d'onda è nullo $k^{\alpha}k_{\alpha} = -\omega^{2} + \bm{k}^{2} = 0$.  D'altra
parte la gauge di Lorenz comporta
\begin{equation}
  0 = \partial^{\alpha} A_{\alpha} = \uimm k^{\alpha} e_{\alpha} \e^{\uimm
    k^{\lambda}x_{\lambda}},
\end{equation}
cioè l'ortogonalità fra il quadrivettore d'onda, che indica la sua direzione di
propagazione, e l'ampiezza delle onde
\begin{equation}
  k^{\alpha} e_{\alpha} = 0.
\end{equation}
In questo caso si dice che le onde sono \emph{trasversali}.  Questa relazione ci
permette di determinare una delle componenti di $e_{\alpha}$ in funzione delle
altre.  Per esempio, se scegliamo la direzione dell'asse $z$ lungo la direzione
di propagazione dell'onda allora $k^{\lambda} = (k, 0, 0, k)$ e risulta
\begin{equation}
  0 = k^{\alpha} e_{\alpha} = k^{0}e_{0} + k^{3}e_{3} = k(e_{0} + e_{3}) \iff
  e_{0} = -e_{3}.
\end{equation}
Inoltre abbiamo visto che nella condizione di Lorenz è possibile effettuare una
trasformazione di gauge del tipo
$A_{\alpha} \to A'_{\alpha} = A_{\alpha} + \partial_{\alpha}\chi$ con $\chi$
funzione scalare con d'Alambertiano nullo.  Allora anche $\chi$ ha come
soluzione la sovrapposizione di onde piane, scalari, del
tipo\footnote{Si noti che \textcite[258]{weinberg:gravitation} pone
  $\chi = \Re \{\uimm \epsilon\e^{\uimm k^{\lambda}x_{\lambda}}\}$, con
  $\epsilon = -\uimm \tilde{\chi}$.}
\begin{equation}
  \chi = \Re \{\tilde{\chi} \e^{\uimm k^{\lambda}x_{\lambda}}\}
\end{equation}
e, dal momento che (omettendo nuovamente il simbolo $\Re$)
$\partial_{\alpha} \chi = \uimm k_{\alpha}
\tilde{\chi}\e^{\uimm k^{\lambda}x_{\lambda}}$,
la trasformazione di gauge del quadripotenziale è
\begin{equation}
  A_{\alpha} \to A'_{\alpha} = A_{\alpha} + \partial_{\alpha}\chi = e_{\alpha}
  \e^{\uimm k^{\lambda}x_{\lambda}} + \uimm k_{\alpha}  \tilde{\chi}\e^{\uimm
    k^{\lambda}x_{\lambda}} = (e_{\alpha} + \uimm k_{\alpha}
  \tilde{\chi})\e^{\uimm k^{\lambda}x_{\lambda}} = e'_{\alpha}\e^{\uimm
    k^{\lambda}x_{\lambda}}.
\end{equation}
Dopo aver eseguito questa trasformazione di gauge, la libertà sul valore di
$\tilde{\chi}$ ci permette di annullare un'altra componente di $e'_{\alpha}$,
infatti scegliendo $\tilde{\chi} = \uimm e_{0}/k$ abbiamo
\begin{subequations}
  \begin{align}
    e'_{0} &= e_{0} + \uimm k_{0} \tilde{\chi} = 0, \\
    e'_{3} &= -e'_{0} = 0, \\
    e'_{1} &= e_{1} + \uimm k_{1} \tilde{\chi} = e_{1}, \\
    e'_{2} &= e_{2} + \uimm k_{2} \tilde{\chi} = e_{2}.
  \end{align}
\end{subequations}
Dunque la condizione di ortogonalità fra $k_{\alpha}$ e $e_{\alpha}$, derivante
dalla gauge di Lorenz, ha comportato la diminuzione del numero di componenti
indipendenti di $e_{\alpha}$ da quattro a tre e la libertà sulla scelta della
trasformazione di gauge ha ridotto questo numero a due.  Nel nostro caso
ciascuna delle onde piane si propaga lungo l'asse $z$ e le componendi
indipendenti, quelle che hanno reale significato fisico e rappresentano le
polarizzazioni fisiche dell'onda, sono $e_{1}$ ed $e_{2}$.

\subsection{Onde elettromagnetiche in presenza di cariche in moto}
\label{sec:onde-elettromagnetiche-cariche}

\emph{Nota: in questo paragrafo esplicitiamo la velocità $c$ della luce nel
  vuoto e adottiamo il sistema di unità di Gauss.}

In presenza di cariche in moto, l'equazione da risolvere nella gauge di Lorenz è
(usando il sistema di unità di Gauss)
\begin{equation}
  \label{eq:maxwell-cariche}
  \dalamb A_{\beta} = -\frac{4\pi}{c}J_{\beta},
\end{equation}
in cui il secondo membro è diverso da zero.  Questa equazione equivale, al
variare di $\beta$, alle quattro seguenti equazioni differenziali lineari non
omogenee
\begin{subequations}
  \begin{align}
    \label{eq:maxwell-cariche-i}
    \nabla^{2} \bm{A} - \frac{1}{c^{2}} \parder{\bm{A}}{*{2}{t}} &=
    -\frac{4\pi}{c} \bm{J}, \qquad \text{per $\beta = 1, 2, 3$}, \\
    \nabla^{2} \phi - \frac{1}{c^{2}} \parder{\phi}{*{2}{t}} &= -4\pi \rho,
    \qquad \text{per $\beta = 0$}. \label{eq:maxwell-cariche-0}
  \end{align}
\end{subequations}
La soluzione di ciascuna di queste equazioni è data dalla somma della soluzione
generale dell'equazione omogenea associata e di un integrale particolare
dell'equazione completa.
% TODO: magari fare una figura che spieghi meglio la situazione descritta di
% seguito.
Per determinare l'integrale particolare, suddividiamo l'intero spazio in
volumetti infinitesimi $\dd V$ e calcoliamo il campo elettrico generato da una
particella, di carica $\dd q(t)$ (generalmente dipendente dal tempo), in moto
nel volumetto.  Grazie alla linearità delle equazioni, il potenziale effettivo
sarà dato dalla somma dei contributi al potenziale provenienti da tutti i
volumetti che costituiscono lo spazio.

Cominciamo col risolvere la~\eqref{eq:maxwell-cariche-0}.  Per ogni volumetto
infinitesimo, poniamo l'origine del sistema di riferimento nel suo centro in
modo che l'equazione~\eqref{eq:maxwell-cariche-0} possa essere scritta come
\begin{equation}
  \nabla^{2} \phi - - \frac{1}{c^{2}} \parder{\phi}{*{2}{t}} = -4\pi \rho =
  -4\pi \dd q(t) \delta(\bm{R}),
\end{equation}
in cui $\rho = \dd q(t) \delta(\bm{R})$ è la densità di carica e $\bm{R}$ la
distanza dall'origine del sistema di riferimento.  Poiché la delta di Dirac
$\delta (\bm{R})$ è nulla ovunque, eccetto che nell'origine del sistema, abbiamo
\begin{equation}
  \nabla^{2} \phi - \frac{1}{c^{2}} \parder{\phi}{*{2}{t}} = 0.
\end{equation}
Grazie alla nostra scelta del sistema di riferimento, $\phi$ è sfericamente
simmetrico, quindi l'operatore d'Alambertiano assume la forma
$\dalamb = (1/R^{2})\lparder{}{R}(R^{2}\lparder{}{R})$, da cui
\begin{equation}
  \frac{1}{R^{2}}\parder{}{R}\bigg(R^{2} \parder{\phi}{R}\bigg) -
  \frac{1}{c^{2}} \parder{\phi}{*{2}{t}} = 0.
\end{equation}
Ipotizziamo per il potenziale $\phi$ una soluzione del tipo $\phi = \chi/R$, con
$\chi = \chi(R,t)$.  Inserendo questo nell'equazione precedente otteniamo la
seguente equazione per $\chi$
\begin{equation}
  \parder{\chi}{*{2}{R}} - \frac{1}{c^{2}} \parder{\chi}{*{2}{t}} = 0
\end{equation}
che è una delle forme dell'equazione di d'Alambert delle onde, quindi una
soluzione di $\chi$ è data dalla somma di due funzioni $f_{1}$ e $f_{2}$
dipendenti, rispettivamente, dalle variabili $\xi = t-R/c$ e $\eta = t+R/c$
\begin{equation}
  \chi(R,T) = f_{1} + f_{2} = f_{1}(t - R/c) + f_{2}(t + R/c).
\end{equation}
Vogliamo determinare solo un integrale particolare
della~\eqref{eq:maxwell-cariche-0}, quindi possiamo fare in modo che una delle
due funzioni $f_{1}$ e $f_{2}$ si annulli.  È conveniente porre $f_{2} = 0$,
quindi abbiamo
\begin{equation}
  \phi = \frac{\chi}{R} = \frac{\chi(t - R/c)}{R}.
\end{equation}
La funzione $\chi$ non è ancora del tutto definita.  In particolare non abbiamo
posto alcuna condizione sul suo comportamento nell'origine del sistema di
riferimento.  Notiamo che per $R \to 0$, il potenziale diverge, allora le sue
derivate spaziali crescono più rapidamente rispetto alle derivate temporali,
dunque possiamo trascurare $\lparder[2]{\phi}{c^{2}t}$ rispetto a
$\nabla^{2} \phi$ ottenendo \index{equazione!di Poisson!per il campo elettrico}
l'\emph{equazione di Poisson}
\begin{equation}
  \nabla^{2} \phi = -4\pi\rho
\end{equation}
dalla quale si ricava la legge di Coulomb.  Abbiamo così trovato che vicino
all'origine il potenziale $\phi$ deve essere quello della legge di Coulomb, per
cui deve risultare $\chi = \dd q$, cioè
\begin{equation}
  \phi = \frac{\dd q(t - R/c)}{R}.
\end{equation}
Il potenziale generato da tutti i volumetti $\dd V$ si ottiene ponendo
$\dd q = \rho\dd V$ e integrando su tutto lo spazio.  Il potenziale nel punto
$\bm{r}$ al tempo $t$ è
\begin{equation}
  \label{eq:potenz-rit-phi}
  \phi(\bm{r}, t) = \int \frac{\rho(\bm{r}', t - R/c)}{R} \dd V'.
\end{equation}
Qui $\bm{R} = \bm{r} - \bm{r}'$ indica la distanza fra il punto $\bm{r}$ in cui
si calcola il potenziale e il punto $\bm{r}'$ in cui si trova l'elemento
infinitesimo di volume $\dd V'$.  La densità di carica $\rho$ deve essere
calcolata all'istante $t - R/c$ poiché l'informazione non viaggia a velocità
infinita ma con velocità $c$ e $R/c$ è il tempo che essa impiega per andare dal
punto $\bm{r}'$ in cui si trova la sorgente del campo al punto $\bm{r}$.  Per
questo motivo la soluzione da noi determinata, nella gauge di Lorenz, è detta
\index{potenziale!ritardato}\emph{potenziale ritardato}.  Allo stesso modo si
può determinare per la~\eqref{eq:maxwell-cariche-i} una soluzione del tipo
potenziale ritardato
\begin{equation}
  \label{eq:potenz-rit-A}
  \bm{A} = \frac{1}{c} \int \frac{\bm{J}(\bm{r}',t - R/c )}{R} \dd V'.
\end{equation}
A queste soluzioni particolari vanno sommate le soluzioni delle equazioni
omogenee, determinate in modo da soddisfare le condizioni del sistema.

\subsubsection{Radiazione di dipolo}
\label{sec:radiazione-dipolo}

Supponiamo adesso di voler calcolare l'intensità dell'energia irradiata dal
sistema di cariche a una distanza molto più grande della dimensione $a$ del
sistema stesso.  Fissiamo l'origine del sistema di riferimento all'interno della
distribuzione di cariche.  Indichiamo ancora con $\bm{r}$ la posizione del punto
in cui calcoliamo il campo elettromagnetico, $\bm{r}'$ la posizione in cui si
trova l'elemento infinitesimo di carica $\dd q = \rho\dd V$ e con
$\bm{R} = \bm{r} - \bm{r}'$ il vettore che va dall'elemento di carica al punto
di osservazione.  Porsi a grandi distanze dal sistema significa che
$r \gg r' \sim a$, quindi si ha
\begin{equation}
  R = \norm{\bm{r} - \bm{r}'} \approx r - \versor{n}\cdot\bm{r}',
\end{equation}
dove $\versor{n} = \bm{r}/\norm{\bm{r}}$.  Sostituiamo questa espressione di $R$
nelle formule~\eqref{eq:potenz-rit-phi} e \eqref{eq:potenz-rit-A} dei potenziali
ritardati
\begin{subequations}
  \label{eq:potenz-em-distanti}
  \begin{align}
    \phi (\bm{r}, t) &= \frac{1}{r} \int \rho(\bm{r}', t - (r/c -
    \versor{n}\cdot\bm{r}'/c)) \dd V', \\
    \bm{A}(\bm{r}, t) &= \frac{1}{rc} \int \bm{J}(\bm{r}', t - (r/c -
    \versor{n}\cdot\bm{r}'/c)) \dd V'.
  \end{align}
\end{subequations}
Nel denominatore delle funzioni integrande abbiamo potuto trascurare
$\versor{n}\cdot\bm{r}'$ rispetto a $r$, che è costante rispetto alla variabile
di integrazione.  Non è in generale possibile effettuare la stessa operazione
nell'argomento delle densità $\rho$ e $\bm{J}$, poiché queste potrebbero variare
in maniera significativa nel tempo $\versor{n}\cdot\bm{r}'/c$.  Tuttavia, sotto
opportune ipotesi che ora determineremo, è possibile trascurare
$\versor{n}\cdot\bm{r}'/c$ nell'argomento delle densità.  Indichiamo con
$T = a/v$ l'ordine di grandezza del tempo nel quale la distribuzione di carica
varia sensibilmente, con $v$ ordine di grandezza delle velocità delle cariche.
Se queste si muovono molto più lentamente rispetto alla luce nel vuoto,
$v \ll c$, allora le densità $\rho$ e $\bm{J}$ variano in maniera sensibile in
un tempo $T$ molto più grande rispetto a $a/c \sim \versor{n}\cdot\bm{r}'/c$,
quindi in questo caso possiamo scrivere
\begin{subequations}
  \begin{align}
    \phi (\bm{r}, t) &= \frac{1}{r} \int \rho(\bm{r}', t - r/c) \dd V', \\
    \bm{A}(\bm{r}, t) &= \frac{1}{rc} \int \bm{J}(\bm{r}', t - r/c) \dd V'.
  \end{align}
\end{subequations}
La condizione $T \gg a/c$ equivale anche a dire che le dimensioni del sistema
sono molto più piccole della lunghezza d'onda della radiazione emessa
$a \ll cT = \lambda$.  Se oltre a trovarci a grandi distanze dalla sorgente
risulta anche che $R$ sia molto più grande della lunghezza $\lambda$ delle onde
emesse dal sistema, queste possono essere considerate, in piccole regioni dello
spazio, come onde piane.  Una regione in cui è valida la condizione
$R \gg \lambda \gg a$ è detta \index{zona!delle onde}\emph{zona delle onde}.
Il potenziale vettore $\bm{A}$ può anche essere scritto come
\begin{equation}
  \bm{A} = \frac{1}{rc} \int \bm{J}|_{t-r/c} \dd V'
\end{equation}
e non dipende, così, dalle variabili di integrazione.  Ponendo
$\bm{J} = \rho \bm{v}$ abbiamo
\begin{equation}
  \bm{A} = \frac{1}{rc} \int (\rho\bm{v})|_{t-r/c} \dd V' = \frac{1}{rc}
  \sum_{i} q_{i}\bm{v}_{i} = \frac{1}{rc} \dot{\bm{d}},
\end{equation}
con $\bm{v}_{i}$ velocità della particella $i$-esima calcolata al tempo
$t-r/c$ e
\begin{equation}
  \bm{d} = \sum_{k} q_{k} \bm{r}_{k}
\end{equation}
\index{momento!di dipolo di carica}\emph{momento di dipolo di carica} del
sistema, sempre calcolato al tempo $t-r/c$.

Per un'onda piana risulta
$\bm{E} = \bm{B} \times \versor{n} \iff \bm{B} = -\bm{E}\times \versor{n}$ (cioè
$\bm{E}$, $\bm{B}$ e $\versor{n}$ costituiscono una terna di vettori ortogonali
e $\bm{E}$ e $\bm{B}$ hanno lo stesso modulo) e
\begin{subequations}
  \begin{align}
    % Nota: il Landau dice (nota 2 di pagina 228) che l'equazione seguente non
    % è corretta e bisogna invece usare E = (1/c) (\dot{\bm{A}} \times
    % \versor{n}) \times \versor{n}.  Spiegazione della (43.7) di pagina 151:
    % nel calcolo di \bm{H} usa la formula di derivazione composta per \bm{A},
    % dato che dipende solo da t' = t -x/c, quindi c'è il prodotto vettoriale
    % fra il gradiente di t' (che a quanto pare è uguale a -\versor{n}/c) e
    % \bm{A}', cioè la derivata rispetto a t' di \bm{A}.
    \bm{E} &= -\frac{1}{c} \parder{\bm{A}}{t} = -\frac{1}{c} \dot{\bm{A}} =
    -\frac{1}{c^{2}} \frac{\ddot{\bm{d}}}{r}, \\
    \bm{B} &= -\bm{E}\times \versor{n} = \frac{1}{c} \dot{\bm{A}} \times
    \versor{n} = \frac{1}{c^{2}} \frac{\ddot{\bm{d}}}{r} \times \versor{n},
  \end{align}
\end{subequations}
in cui il punto rappresenta la derivata rispetto al tempo.  Inoltre ricordiamo
che il modulo \index{vettore!di Poynting}\emph{vettore di Poynting}
\begin{equation}
  \bm{S} = \frac{c}{4\pi} \bm{E} \times \bm{B} = \frac{c}{4\pi} B^{2} \versor{n}
  = \frac{1}{4\pi c^{3}r^{2}}(\ddot{\bm{d}}\times\versor{n})^{2} \versor{n}
\end{equation}
rappresenta la potenza che per unità di superficie viene emessa attraverso una
superficie perpendicolare alla direzione di propagazione della radiazione.
L'intensità $\dd I$ emessa nell'elemento di angolo solido $\dd\Omega$ alla
distanza $r$ può essere calcolata determinando la potenza emessa attraverso
l'elemento $r^{2}\dd\Omega$ di una superficie sferica di raggio $r$ con centro
nell'origine del sistema di riferimento.  Allora $\dd I$ è dato dal prodotto fra
il modulo del vettore di Poynting e l'elemento $r^{2}\dd\Omega$, cioè
\begin{equation}
  \dd I = \norm{\bm{S}} r^{2} \dd\Omega = \frac{1}{4\pi
    c^{3}r^{2}}(\ddot{\bm{d}}\times\versor{n})^{2} r^{2} \dd\Omega =
  \frac{1}{4\pi c^{3}} (\ddot{\bm{d}})^{2} \sin^{2}\theta \dd\Omega,
\end{equation}
in cui $\theta$ è l'angolo compreso fra i vettori $\ddot{\bm{d}}$ e
$\versor{n}$.  L'intensità totale $I$ emessa dalla radiazione è data
dall'integrale su tutto l'angolo solido, quindi dobbiamo sostituire
$\dd\Omega = 2\pi \sin\theta \dd\theta$ e integrare su $\theta$ fra $0$ e $\pi$
\begin{equation}
  \label{eq:energia-dipolo}
  I = \int \dd I = \frac{2\pi}{4\pi c^{3}} (\ddot{\bm{d}})^{2} \int_{0}^{\pi}
  \sin^{3}\theta \dd\theta = \frac{2}{3c^{2}}(\ddot{\bm{d}})^{2}.
\end{equation}
Questa è dunque l'energia emessa, per unità di tempo, dal sistema
nell'approssimazione di dipolo appena sviluppata.  Poiché
$\ddot{\bm{d}} = \sum_{k} q_{k} \dot{\bm{v}}_{k}$, le cariche emettono
radiazione se sono accelerate.  Non si ha emissione di radiazione di dipolo
neanche se le cariche costituiscono un sistema isolato e hanno tutte lo stesso
rapporto fra carica $q_{k}$ e massa $m_{k}$ perché
\begin{equation}
  \bm{d} = \sum_{k} \frac{q_{k}}{m_{k}} m_{k} \bm{r}_{k} = \text{costante}
  \sum_{k} m_{k} \bm{r}_{k} = \text{costante} \cdot M \bm{R}_{\textup{cm}},
\end{equation}
in cui $M = \sum_{k} m_{k}$ è la massa totale delle particelle e
$\bm{R}_{\textup{cm}}$ è il raggio vettore del centro di massa del sistema, e
dal momento che siamo nell'approssimazione non relativistica $v_{k} \ll c$ in un
sistema isolato si ha $\ddot{\bm{R}}_{\textup{cm}} = \bm{0}$.

I calcoli svolti per giungere alla formula~\eqref{eq:energia-dipolo}
dell'energia emessa da un sistema di cariche è stata ottenuta sviluppando il
potenziale vettore in serie di potenze di $a/\lambda \ll 1$ e arrestandosi al
primo ordine.  È possibile continuare lo sviluppo di $\bm{A}$ in serie di
potenze di $a/\lambda$ includendo i termini successivi dello sviluppo (quindi
senza trascurare $\versor{n}\times\bm{r}'/c$ nell'argomento delle densità $\rho$
e $\bm{J}$) ottenendo\footnote{Vedi \textcite[249-251]{landau:campi}.} il
seguente risultato per l'energia emessa dal sistema di cariche nell'unità di
tempo
\begin{equation}
  \label{eq:potenza-onde-em}
  I = \underbrace{\frac{2}{3c^{3}} (\ddot{\bm{d}})^{2}}_{\text{dipolo di carica}} +
  \underbrace{\frac{2}{3c^{3}} (\ddot{\bm{m}})^{2}}_{\text{dipolo magnetico}} +
  \underbrace{\frac{1}{180c^{5}}
    \dddot{Q}_{ij}\dddot{Q}^{ij}}_{\text{quadrupolo di carica}} + \cdots
\end{equation}
con $\bm{m} = \sum_{k} q_{k} \bm{r}_{k} \times \bm{v}_{k}/(2c)$
\index{momento!di dipolo magnetico}\emph{momento di dipolo magnetico} e
$Q_{ij} = \sum_{k} q_{k}(3x_{i}x_{j} - \delta_{ij}r_{k}^{2})$
\index{momento!di quadrupolo di carica}\emph{momento di quadrupolo di carica}.
Sebbene i termini di dipolo magnetico e quadrupolo di carica siano generalmente
trascurabili rispetto al termine di dipolo di carica, essi diventano i
principali nello sviluppo in multipoli dei potenziali nei casi in cui è assente
la radiazione di dipolo di carica, per esempio perché risulta
$\ddot{\bm{d}} = \bm{0}$.  Osserviamo infine che per un sistema di cariche
isolate in cui il rapporto $q_{k}/m_{k}$ è uguale per tutte le particelle è
nullo il termine di radiazione di dipolo magnetico, oltre a quello di dipolo di
carica, infatti ($m_{k} \bm{v}_{k} = \bm{p}_{k}$ poiché $v_{k} \ll c$)
\begin{equation}
  \bm{m} = \sum_{k} \frac{q_{k}}{2m_{k}c} m_{k} \bm{r}_{k} \times \bm{v}_{k}
  \propto \sum_{k} \bm{r}_{k} \times \bm{p}_{k} = \bm{L},
\end{equation}
con $\bm{L}$ momento angolare totale del sistema, e per un sistema non
relativistico isolato $\dot{\bm{L}} = \bm{0} \implies \dot{\bm{m}} = \bm{0}$.

\section{Equazioni di Eulero-Lagrange di un campo e tensore energia-impulso}
\label{sec:tensore-energia-impulso}

% \emph{Nota: in questo paragrafo, dove non diversamente specificato, adotteremo
%   la convenzione dei segni $(-++)$ seguita
%   da~\textcites{barone:relativita,landau:campi}.}

Vogliamo determinare le equazioni del moto di un generico sistema usando il
formalismo lagrangiano. Come applicazione del metodo considereremo il caso del
campo elettromagnetico.

Consideriamo un sistema isolato il cui integrale d'azione abbia la forma
\begin{equation}
  \label{eq:azione-campo}
  S = \int\limits_{\Omega} \Lambda (q, \lparder{q}{x^{\alpha}}) \dd^{4} x
  = \iint \Lambda (q, \lparder{q}{x^{\alpha}}) \dd t\dd^{3} \bm{x},
\end{equation}
in cui $\Lambda = \Lambda(q, \lparder{q}{x^{\alpha}})$ è una funzione della
coordinata generalizzata $q$, delle sue derivate $\lparder{q}{x^{\alpha}}$ e non
dipende espressamente dalle coordinate $x^{\alpha}$ in quanto il sistema è
isolato.  Poiché l'integrale $\int \Lambda\dd^{3} \bm{x}$ è la lagrangiana del
sistema, $\Lambda$ è chiamata \index{densità!di lagrangiana}\emph{densità di
  lagrangiana}.  Supporremo che $q$ e le sue derivate si annullino all'infinito.

Secondo il principio variazionale di Hamilton, le equazioni del moto del sistema
si ottengono ricercando la configurazione che rende stazionaria
l'azione~\eqref{eq:azione-campo}.  Consideriamo allora una variazione $q \to q +
\delta q$ della coordinata generalizzata, con la condizione che $\delta q$ si
annulli sul bordo $\partial\Omega$ del volume quadrimensionale su cui viene
calcolato l'integrale d'azione.  Le coordinate $x^{\alpha}$ non vengono variate.
Dobbiamo imporre che l'azione sia stazionaria rispetto alla variazione $\delta
q$, cioè che sia $\delta S = 0$.  Per brevità adottiamo la notazione
$q_{,\alpha} = \lparder{q}{x^{\alpha}}$.  In questo modo abbiamo
\begin{equation}
  \begin{split}
    0 &= \delta S = \int \bigg( \parder{\Lambda}{q}\delta q
    + \parder{\Lambda}{q_{,\alpha}}\delta q_{,\alpha} \bigg)\dd^{4} x \\
    &= \int \bigg( \parder{\Lambda}{q}\delta q + \parder{}{x^{\alpha}}\bigg(
    \parder{\Lambda}{q_{,\alpha}}\delta q \bigg) - \delta q\parder{}{x^{\alpha}}
    \bigg( \parder{\Lambda}{q_{,\alpha}} \bigg)\bigg)\dd^{4} x.
  \end{split}
\end{equation}
Per il teorema di Gauss nello spazio di Minkowski (vedi
l'appendice~\ref{cha:teorema-gauss}) il secondo termine nell'integrale diventa
\begin{equation}
  \int\limits_{\Omega}  \partial_{\alpha}
  \bigg( \parder{\Lambda}{q_{,\alpha}}\delta q \bigg) \dd^{4} x =
  \int\limits_{\partial\Omega} \parder{\Lambda}{q_{,\alpha}}\delta q \dd^{3}
  \bm{x} = 0
\end{equation}
perché le variazioni $\delta q = 0$ sul bordo $\partial\Omega$.  Allora
\begin{equation}
  0 = \delta S = \int \delta q\bigg( \parder{\Lambda}{q} - \parder{}{x^{\alpha}}
  \bigg( \parder{\Lambda}{q_{,\alpha}} \bigg)\bigg)\dd^{4} x.
\end{equation}
Si deve avere $\delta S = 0$ per una variazione arbitraria $\delta q$, pertanto
si deve annullare il termine fra parentesi e così giungiamo all'equazione del
moto (o \index{equazione!di Eulero-Lagrange}\emph{equazione di Eulero-Lagrange})
del campo
\begin{equation}
  \label{eq:eulero-lagrange-campi}
  \parder{\Lambda}{q} - \parder{}{x^{\alpha}}\parder{\Lambda}{q_{,\alpha}} = 0.
\end{equation}

% Nota: sembra che nessuno sia in grado di fare il calcolo del quadrigradiente
% del tensore energia impulso senza fare confusione fra derivazione parziale e
% totale.
Il sistema in esame è chiuso se la densità di lagrangiana $\Lambda$ non dipende
esplicitamente dalle coordinate $x^{\alpha}$, analogamente al caso meccanico in
cui la lagrangiana è indipendente dal tempo per un sistema isolato.  Per un
sistema chiuso possiamo determinare una formula analoga alla legge di
conservazione dell'energia in meccanica.  Sfruttando le equazioni di
Eulero-Lagrange~\eqref{eq:eulero-lagrange-campi} abbiamo
\begin{equation}
  \begin{split}
    \toder{\Lambda}{x^{\alpha}} &= \parder{\Lambda}{q}\toder{q}{x^{\alpha}}
    + \parder{\Lambda}{q_{,\beta}}\toder{q_{,\beta}}{x^{\alpha}}
    + \parder{\Lambda}{x^{\alpha}} =
    \toder{}{x^{\beta}}\bigg( \parder{\Lambda}{q_{,\beta}} \bigg)q_{,\alpha}
    + \parder{\Lambda}{q_{,\beta}}\toder{q_{,\beta}}{x^{\alpha}}
    + \parder{\Lambda}{x^{\alpha}} \\
    &= \toder{}{x^{\beta}}\bigg( \parder{\Lambda}{q_{,\beta}} \bigg) q_{,\alpha}
    + \parder{\Lambda}{q_{,\beta}} \toder{q_{,\alpha}}{x^{\beta}}
    + \parder{\Lambda}{x^{\alpha}} \\
    &= \toder{}{x^{\beta}}\bigg(q_{,\alpha} \parder{\Lambda}{q_{,\beta}}\bigg)
    + \parder{\Lambda}{x^{\alpha}}.
  \end{split}
\end{equation}
Poiché sussiste inoltre la relazione
\begin{equation}
  \toder{\Lambda}{x^{\alpha}} =
  \tensor{\delta}{^{\beta}_{\alpha}} \toder{\Lambda}{x^{\beta}} =
  \tensor{\eta}{^{\beta}_{\alpha}} \toder{\Lambda}{x^{\beta}},
\end{equation}
in cui abbiamo ricordato la definizione~\eqref{eq:delta-kronecker}
della\index{delta!di Kronecker quadrimensionale} delta di Kronecker, risulta
\begin{equation}
  \tensor{\eta}{^{\beta}_{\alpha}} \toder{\Lambda}{x^{\beta}}
  = \toder{}{x^{\beta}} \bigg( q_{,\alpha} \parder{\Lambda}{q_{,\beta}} \bigg) +
  \parder{\Lambda}{x^{\alpha}},
\end{equation}
da cui
\begin{equation}
  \toder{}{x^{\beta}} \bigg( q_{,\alpha}\toder{\Lambda}{q_{,\beta}} -
  \tensor{\eta}{^{\beta}_{\alpha}}\Lambda \bigg) =
  -\parder{\Lambda}{x^{\alpha}}.
\end{equation}
Se il sistema è chiuso, $\lparder{\Lambda}{x^{\alpha}}$ è nullo.  Definiamo il
\index{tensore!energia-impulso}\emph{tensore energia-impulso}
$\tensor{T}{^{\beta}_{\alpha}}$ come
\begin{equation}
  \label{eq:tensore-energia-impulso}
  \tensor{T}{^{\beta}_{\alpha}} = q_{,\alpha}\parder{\Lambda}{q_{,\beta}} -
  \tensor{\eta}{^{\beta}_{\alpha}}\Lambda,
\end{equation}
quindi l'equazione precedente si può scrivere, per un sistema chiuso, come
\begin{equation}
  \label{eq:div-tens-en-imp}
  \partial_{\beta} \tensor{T}{^{\beta}_{\alpha}} = 0.
\end{equation}
Per il corollario del teorema di Gauss possiamo allora definire il quadrivettore
\begin{equation}
  P^{\alpha} = \int T^{0\alpha}\dd V,
\end{equation}
in cui l'integrale è esteso a tutto lo spazio, e $P^{\alpha}$ è costante
rispetto al tempo.

Osserviamo che
\begin{equation}
  T^{00} = \parder{q}{x^{0}} \parder{\Lambda}{q_{,0}} - \Lambda =
  \dot{q} \parder{\Lambda}{\dot{q}} - \Lambda
\end{equation}
è la densità di energia associata alla densità di lagrangiana $\Lambda$, infatti
in meccanica l'energia $E$ di un sistema con lagrangiana $L$ è data da
\begin{equation}
  E = \dot{q}\parder{L}{\dot{q}} - L.
\end{equation}
Dunque
\begin{equation}
  P^{0} = \int T^{00}\dd V = \int \bigg(\dot{q}\parder{\Lambda}{\dot{q}} -
  \Lambda\bigg) \dd V
\end{equation}
rappresenta l'energia del sistema.  Invece
\begin{equation}
  T^{0i} = \parder{q}{x_{i}} \parder{\Lambda}{q_{,0}}
\end{equation}
è la densità di momento, pertanto
\begin{equation}
  P^{i} = \int T^{0i}\dd V
\end{equation}
è la componente $i$ della quantità di moto.  Abbiamo così fatto vedere che il
quadrivettore $P^{\alpha}$ qui definito è proprio il quadrimomento e
l'equazione~\eqref{eq:div-tens-en-imp} rappresenta la legge di conservazione del
quadrimomento per un sistema chiuso.  Si possono dare interpretazioni fisiche
anche delle altre componenti del tensore
energia-impulso.\footnote{Vedi~\textcite[417-418]{barone:relativita};
  \textcite[137-138]{misner:gravitation}.}
$T^{i0}$ è la densità di flusso di energia nella direzione $x^{i}$, $T^{ij}$ è
la densità di flusso della componente $j$ del momento nella direzione $i$.
Equivalentemente, $T^{ij}$ può essere interpretata come la componente $i$ della
forza per unità di area su una superficie perpendicolare all'asse $j$.  La parte
``spaziale'' del tensore energia impulso, $T^{ij}$, è chiamata anche
\emph{tensore degli
  sforzi},\footnote{Ricordiamo
  che in meccanica classica si definisce il tensore degli sforzi $T$ come quel
  tensore tridimensionale di rango $2$ tale che la forza $\dd \bm{F}$ agente su
  un elemento di superficie $\dd\bm{A}$ è esprimibile come
  $\dd\bm{F} = T\dd\bm{A}$.} % Vedi Goldstein, Poole & Safko, capitolo 13
le componenti $T^{ij}$, con $i \neq j$ costituiscono gli sforzi di taglio, le
componenti diagonali $T^{ii}$ (senza somma su $i$) sono i cosiddetti sforzi
normali, cioè la pressione agente nella direzione $i$.  Riassumiamo in forma
matriciale il tensore energia impulso
%%% Figura ispirata a questa:
%%% https://it.wikipedia.org/wiki/File:StressEnergyTensor_it.svg
\begin{equation}
  \begin{tikzpicture}[pmatrix]
    \matrix (t)
    {
      T_{00} & T_{01} & T_{02} & T_{03} \\
      T_{10} & T_{11} & T_{12} & T_{13} \\
      T_{20} & T_{21} & T_{22} & T_{23} \\
      T_{30} & T_{31} & T_{32} & T_{33} \\
    } ;

    \begin{scope}[on background layer]
      % evidenzio tensore degli sforzi e densità di flusso di momento
      \draw[color=cyan,fill=purple,opacity=0.3]
      (t-2-2.north west) --
      (t-4-2.south west) --
      (t-4-4.south east) --
      (t-2-4.north east) -- cycle;

      % evidenzio pressione
      \draw[fill,color=green,opacity=0.3]
      (t-2-2.north west) --
      (t-2-2.south west) --
      (t-4-4.south west) --
      (t-4-4.south east) --
      (t-4-4.north east) --
      (t-2-2.north east) -- cycle;

      % evidenzio densità di energia
      \draw[color=cyan,fill=gray,opacity=0.3]
      (t-1-1.north west) --
      (t-1-1.south west) --
      (t-1-1.south east) --
      (t-1-1.north east) -- cycle;

      % evidenzio densità di flusso di energia
      \draw[color=cyan,fill=gray,opacity=0.3]
      (t-2-1.north west) --
      (t-4-1.south west) --
      (t-4-1.south east) --
      (t-2-1.north east) -- cycle;

      % evidenzio densità di momento
      \draw[color=cyan,fill=gray,opacity=0.3]
      (t-1-2.north west) --
      (t-1-2.south west) --
      (t-1-4.south east) --
      (t-1-4.north east) -- cycle;
    \end{scope}

    \begin{scope}[font=\small]
      \draw[color=cyan] ($(t-1-3.north) + (5pt,0)$) -- ++(0,20pt)
      node[align=center,fill=white] {densità di \\ momento};

      \draw[color=cyan] (t-1-1.north) -- ++(0,20pt)
      node[align=center,fill=white] {densità di \\ energia};

      \draw[color=cyan] (t-4-1.south) -- ++(0,-25pt)
      node[align=center,fill=white] {densità di \\ flusso di \\ energia};

      \draw[color=green] (t-4-4.east) -- ++(20pt,0)
      node[align=center,fill=white,label={right:sforzi normali (pressione)}] {};

      \draw[color=purple] ($(t-2-4.east) - (.5pt,0)$) -- ++(20pt,0)
      node[align=center,fill=white,label={right:sforzi di taglio}] {};

      \draw[color=cyan] ($(t-4-3.south) + (5pt,0)$) -- ++(0,-25pt)
      node[align=center,fill=white] {densità di \\ flusso di \\ momento};
    \end{scope}
  \end{tikzpicture}
\end{equation}

Riprendiamo la convenzione dei segni $(+--)$ di~\textcite{weinberg:gravitation}.
Le \index{equazioni!di Maxwell}equazioni di Maxwell che descrivono il campo
elettromagnetico possono essere ricavate applicando il formalismo qui
introdotto.  \index{azione!del campo elettromagnetico}L'azione $S$ di un sistema
composto da un campo elettromagnetico e delle particelle che interagiscono con
esso è formata da tre parti
\begin{equation}
  S = S_{\textup{p}} + S_{\textup{pc}} + S_{\textup{c}}.
\end{equation}
Il termine $S_{\textup{p}}$ dipende solo dalle proprietà delle particelle,
quindi è l'azione delle particelle libere ciascuna di massa $m_{n}$ e vale
$-\sum_{n} m_{n}\int\dd \tau$.  $S_{\textup{pc}}$ è la parte di azione che
descrive l'interazione fra le particelle del sistema e il campo elettromagnetico
e vale $-\sum_{n} q_{n}\int A_{\alpha}\dd x^{\alpha}$ per un sistema di
particelle puntiformi di carica $q_{n}$, $-\int J^{\alpha}A_{\alpha}\dd^{4} x$
se si considera una distribuzione continua di carica.  Infine, $S_{\textup{c}}$
dipende solo dalle proprietà del campo, cioè è l'azione in assenza di cariche, e
vale $-1/4 \int F^{\alpha\beta}F_{\alpha\beta}\dd^{4}
x$,\footnote{Abbiamo
  qui deciso di utilizzare il sistema di unità di Heaviside nel quale il fattore
  davanti all'integrale vale $-1/4$.  Nel sistema di unità di Gauss, usato nel
  paragrafo~\ref{sec:onde-elettromagnetiche-cariche}, il fattore assume il
  valore $-1/(16\pi)$.}  così che l'azione complessiva è
\begin{equation}
  S = -\sum_{n} m_{n}\int \dd\tau - \int J^{\alpha}A_{\alpha} \dd^{4} x
  -\frac{1}{4} \int F^{\alpha\beta}F_{\alpha\beta} \dd^{4} x.
\end{equation}
Per determinare le \index{equazioni!di Maxwell}equazioni di Maxwell in presenza
di sorgenti bisogna considerare solo le parti dell'azione che coinvolgono il
campo, cioè
\begin{equation}
  S_{\textup{pc}} + S_{\textup{c}} =  \int (\Lambda_{\textup{pc}} +
  \Lambda_{\textup{c}}) \dd^{4} x = \int \bigg( -J^{\alpha}A_{\alpha}
  -\frac{1}{4} F^{\alpha\beta}F_{\alpha\beta} \bigg) \dd^{4} x
\end{equation}
e applicare le equazioni di Eulero-Lagrange~\eqref{eq:eulero-lagrange-campi} a
\begin{equation}
  \Lambda = \Lambda_{\textup{pc}} + \Lambda_{\textup{c}} = -J^{\alpha}A_{\alpha}
  -\frac{1}{4} F^{\alpha\beta}F_{\alpha\beta}.
\end{equation}
Le componenti $A_{\alpha}(x)$ del quadripotenziale svolgono il ruolo della
coordinata generalizzata $q(x)$.  Per ottenere le equazioni di Maxwell
\index{equazioni!di Maxwell} in assenza di sorgenti si considera solo $\Lambda =
\Lambda_{\textup{c}}$.

Il tensore energia-impulso $T_{\textup{em}}^{\alpha\beta}$ del campo
elettromagnetico si ottiene dalla definizione~\eqref{eq:tensore-energia-impulso}
usando $\Lambda_{\textup{c}}$ come densità di lagrangiana e le componenti
$A_{\mu}$ del quadrimomento come campi
\begin{equation}
  T_{\textup{em}}^{\alpha\beta}
  = F^{\alpha\gamma}\partial^{\beta}A_{\gamma} -
  \frac{1}{4}\eta^{\alpha\beta}F_{\gamma\delta}F^{\gamma\delta}.
\end{equation}
Si
dimostra\footnote{Vedi \textcite[419]{barone:relativita};
  \textcite[111]{landau:campi}; \textcite[141-421]{misner:gravitation}.}
che, per ragioni puramente fisiche, il tensore energia-impulso deve sempre
essere simmetrico, ma il tensore $T_{\textup{em}}^{\alpha\beta}$ così ottenuto
non lo è.  Mediante un'opportuna procedura di
simmetrizzazione\footnote{Questo metodo è dovuto a Belinfante e Rosenfeld.
  Vedi~\textcite[437]{barone:relativita}; \textcite[114]{landau:campi}.}
si può giungere alla seguente forma simmetrica
\begin{equation}
  T_{\textup{em}}^{\alpha\beta} = \tensor{F}{^{\alpha}_{\gamma}}F^{\beta\gamma}
  - \frac{1}{4}\eta^{\alpha\beta}F_{\gamma\delta}F^{\gamma\delta}.
\end{equation}
Osserviamo che in questo modo risulta
\begin{subequations}
  \begin{align}
    T_{\textup{em}}^{00} &= \frac{1}{2}(\bm{E}^{2} + \bm{B}^{2}), \\
    T_{\textup{em}}^{i0} &= (\bm{E} \times \bm{B})_{i},
  \end{align}
\end{subequations}
che sono, rispettivamente, la densità di energia del campo elettromagnetico e la
componente $i$ del vettore di Poynting, che rappresenta la densità del flusso
energia associata al campo elettromagnetico.  Questi risultati sono coerenti con
quanto detto precedentemente per un campo generico.

Se un sistema non è isolato risulta $\partial_{\beta}T^{\beta\alpha} \neq 0$,
tuttavia considerando anche tutto ciò che interagisce con questo sistema (in
modo da avere un sistema complessivamente chiuso) risulta
$\partial_{\beta}T_{\textup{tot}}^{\beta\alpha} = 0$, in cui
$\partial_{\beta}T_{\textup{tot}}^{\beta\alpha}$ è il tensore energia-impulso
totale.  Ora faremo vedere che per un campo elettromagnetico in presenza di
sorgenti si ha $\partial_{\beta}T_{\textup{em}}^{\beta\alpha} \neq 0$ ma
$\partial_{\beta}(T_{\textup{p}}^{\beta\alpha} + T_{\textup{em}}^{\beta\alpha})
= 0$, in cui $T_{\textup{p}}^{\beta\alpha}$ è il tensore energia-impulso
associato alle particelle che interagiscono con il campo.

Consideriamo un sistema di particelle cariche, ciascuna con quadrimomento
$p_{n}^{\alpha}(t)$.  In analogia alle definizioni di densità di carica e di
corrente date nel paragrafo~\ref{sec:densita-corrente-carica}, definiamo la
densità di momento come
\begin{equation}
  T_{\textup{p}}^{\alpha0}(\bm{x}, t) = \sum_{n} p_{n}^{\alpha}(t)
  \delta^{3}(\bm{x} - \bm{x}_{n}(t))
\end{equation}
e il flusso di $p_{n}^{\alpha}$ come
\begin{equation}
  T_{\textup{p}}^{\alpha i}(\bm{x}, t) = \sum_{n} p_{n}^{\alpha}(t)
  \toder{x_{n}^{i}}{t} \delta^{3}(\bm{x} - \bm{x}_{n}(t)).
\end{equation}
Allora il tensore energia-impulso per le particelle può essere definito come
\begin{equation}
  T_{\textup{p}}^{\alpha\beta}(\bm{x}, t) = \sum_{n} p_{n}^{\alpha}(t)
  \toder{x_{n}^{\beta}}{t} \delta^{3}(\bm{x} - \bm{x}_{n}(t)).
\end{equation}
Osserviamo che
\begin{equation}
  p_{n}^{\beta} = E_{n} \toder{x_{n}^{\beta}}{t} =
  \begin{cases}
    E_{n} & \text{per $\beta = 0$}, \\
    E_{n}(\bm{v}_{n})^{i} = \gamma_{n}m_{n}(\bm{v}_{n})^{i} & \text{per $\beta =
      i$},
  \end{cases}
\end{equation}
quindi $T_{\textup{p}}^{\alpha\beta}$ può essere scritto come
\begin{equation}
  T_{\textup{p}}^{\alpha\beta}(\bm{x}, t) = \sum_{n} \frac{p_{n}^{\alpha}
    p_{n}^{\beta}}{E_{n}} \delta^{3}(\bm{x} - \bm{x}_{n}(t)).
\end{equation}
Da questa relazione è evidente che
$T_{\textup{p}}^{\alpha\beta} = T_{\textup{p}}^{\beta\alpha}$.  Inoltre, in
analogia alla~\eqref{eq:quadricorrente}, il tensore energia-impulso può essere
riscritto nella forma
\begin{equation}
  T_{\textup{p}}^{\alpha\beta} = \sum_{n} \int p_{n}^{\alpha}
  \toder{x_{n}^{\beta}}{\tau} \delta^{4}(x - x_{n}(\tau))\dd \tau,
\end{equation}
nella quale si vede che $T_{\textup{p}}^{\alpha\beta}$ è effettivamente un
quadritensore di rango $2$ controvariante, in quanto $p_{n}^{\alpha}$ e
$\ltoder{x_{n}^{\beta}}{\tau}$ sono quadrivettori controvarianti.  Si può far
vedere che\footnote{Vedi~\textcite[44]{weinberg:gravitation}.}
\begin{equation}
  \partial_{\beta}T_{\textup{p}}^{\alpha\beta} = \sum_{n}
  \delta^{3}(\bm{x} - \bm{x}_{n}(t)) \toder{p_{n}^{\alpha}}{t} = \sum_{n}
  \delta^{3}(\bm{x} - \bm{x}_{n}(t)) \toder{\tau}{t}f_{n}^{\alpha}(t) =
  G^{\alpha}(\bm{x}, t),
\end{equation}
in cui $f_{n}^{\alpha}(t)$ è la quadriforza agente sull'$n$-esima particella.
$G^{\alpha}$ è la densità di forza.  Usando
l'espressione~\eqref{eq:quadriforza-lorentz} della forza $f_{n}^{\alpha}$ agente
sulle particelle di carica $q_{n}$ si trova che la densità di forza vale
\begin{equation}
  \begin{split}
    G^{\alpha} &= \sum_{n} \delta^{3}(\bm{x} - \bm{x}_{n}(t))
    \toder{\tau}{t}q_{n} \tensor{F}{^{\alpha}_{\gamma}}
    \toder{x_{n}^{\gamma}}{\tau} = \tensor{F}{^{\alpha}_{\gamma}} \sum_{n}
    \delta^{3}(\bm{x} - \bm{x}_{n}(t)) q_{n} \toder{x_{n}^{\gamma}}{t} \\
    &= \tensor{F}{^{\alpha}_{\gamma}} J^{\gamma}.
  \end{split}
\end{equation}

Calcoliamo la quadridivergenza del tensore energia-impulso del campo
elettromagnetico
\begin{equation}
  \begin{split}
    \partial_{\beta}T_{\textup{em}}^{\alpha\beta} &=
    \tensor{F}{^{\alpha}}_{\gamma}\partial_{\beta}F^{\beta\gamma} +
    F^{\beta\gamma}\partial_{\beta}\tensor{F}{^{\alpha}_{\gamma}} -
    \frac{1}{4} \partial_{\beta}
    (\eta^{\alpha\beta}F_{\gamma\delta}F^{\gamma\delta}) \\
    &= \tensor{F}{^{\alpha}}_{\gamma}\partial_{\beta}F^{\beta\gamma} +
    F^{\beta\gamma}\partial_{\beta}\tensor{F}{^{\alpha}_{\gamma}} -
    \frac{1}{4} \partial^{\alpha}(F_{\gamma\delta}F^{\gamma\delta}) \\
    &= \tensor{F}{^{\alpha}}_{\gamma}\partial_{\beta}F^{\beta\gamma} +
    F^{\beta\gamma}\partial_{\beta}\tensor{F}{^{\alpha}_{\gamma}} -
    \frac{1}{4}(F^{\gamma\delta}\partial^{\alpha}F_{\gamma\delta} +
    F_{\gamma\delta}\partial^{\alpha}F^{\gamma\delta}) \\
    &= \tensor{F}{^{\alpha}}_{\gamma}\partial_{\beta}F^{\beta\gamma} +
    F^{\beta\gamma}\partial_{\beta}\tensor{F}{^{\alpha}_{\gamma}} -
    \frac{1}{2}F_{\gamma\delta}\partial^{\alpha}F^{\gamma\delta} \\
    &= \tensor{F}{^{\alpha}}_{\gamma}\partial_{\beta}F^{\beta\gamma} +
    F^{\beta\gamma}\partial_{\beta}\tensor{F}{^{\alpha}_{\gamma}} -
    \frac{1}{2}F_{\beta\gamma}\partial^{\alpha}F^{\beta\gamma}.
  \end{split}
\end{equation}
Il secondo termine può essere riscritto come
\begin{equation}
  \begin{split}
    F^{\beta\gamma}\partial_{\beta}\tensor{F}{^{\alpha}_{\gamma}} &=
    \eta^{\beta\mu}\eta^{\gamma\nu}F_{\mu\nu} \partial_{\beta}
    \tensor{F}{^{\alpha}_{\gamma}} = F_{\mu\nu} \partial^{\mu} F^{\alpha\nu} =
    F_{\gamma\beta} \partial^{\gamma} F^{\alpha\beta} \\
    &= -F_{\beta\gamma} \partial^{\gamma} F^{\alpha\beta}.
  \end{split}
\end{equation}
Nell'ultimo passaggio abbiamo sfruttato l'antisimmetria di $F_{\gamma\beta}$.
Inoltre, dall'equazione di Maxwell omogenea~\eqref{eq:maxwell-cov22} scritta
nella forma
$\partial^{\alpha}F^{\beta\gamma} + \partial^{\beta}F^{\gamma\alpha}
+ \partial^{\gamma}F^{\alpha\beta} = 0$ abbiamo
\begin{equation}
  -\partial^{\alpha}F^{\beta\gamma} = \partial^{\beta}F^{\gamma\alpha}
  + \partial^{\gamma}F^{\alpha\beta},
\end{equation}
quindi
\begin{equation}
  \begin{split}
    \partial_{\beta}T_{\textup{em}}^{\alpha\beta} &=
    \tensor{F}{^{\alpha}}_{\gamma}\partial_{\beta}F^{\beta\gamma}
    -F_{\beta\gamma} \partial^{\gamma} F^{\alpha\beta} +
    \frac{1}{2}F_{\beta\gamma}(\partial^{\beta}F^{\gamma\alpha}
    + \partial^{\gamma}F^{\alpha\beta}) \\
    &= \tensor{F}{^{\alpha}}_{\gamma}\partial_{\beta}F^{\beta\gamma} +
    \frac{1}{2}F_{\beta\gamma}(\partial^{\beta}F^{\gamma\alpha}
    - \partial^{\gamma}F^{\alpha\beta}) \\
    &= \tensor{F}{^{\alpha}}_{\gamma}\partial_{\beta}F^{\beta\gamma} -
    \frac{1}{2}F_{\beta\gamma}(\partial^{\beta}F^{\alpha\gamma}
    + \partial^{\gamma}F^{\alpha\beta}).
  \end{split}
\end{equation}
Nell'ultimo passaggio abbiamo nuovamente sfruttato l'antisimmetria del tensore
del campo elettromagnetico.  L'ultimo termine è nullo poiché
\begin{equation}
  \begin{split}
    F_{\beta\gamma}(\partial^{\beta}F^{\alpha\gamma}
    + \partial^{\gamma}F^{\alpha\beta}) &=
    -F_{\gamma\beta}(\partial^{\beta}F^{\alpha\gamma}
    + \partial^{\gamma}F^{\alpha\beta}) \\
    &= -F_{\gamma\beta}(\partial^{\gamma}F^{\alpha\beta}
    + \partial^{\beta}F^{\alpha\gamma}) \\
    &= -F_{\beta\gamma}(\partial^{\beta}F^{\alpha\gamma}
    + \partial^{\gamma}F^{\alpha\beta}).
  \end{split}
\end{equation}
Infine, per la legge di Maxwell~\eqref{eq:maxwell-cov1} risulta
$\partial_{\beta}F^{\beta\gamma} = -J^{\gamma}$, quindi
\begin{equation}
  \partial_{\beta}T_{\textup{em}}^{\alpha\beta} =
  -\tensor{F}{^{\alpha}_{\gamma}}J^{\gamma}
\end{equation}
e la quadridivergenza del tensore energia-impulso
$T_{\textup{tot}}^{\alpha\beta}$ del sistema, isolato, campo elettromagnetico +
particelle interagenti con il campo è
\begin{equation}
  \partial_{\beta}T_{\textup{tot}}^{\alpha\beta}
  = \partial_{\beta}(T_{\textup{p}}^{\alpha\beta} +
  T_{\textup{em}}^{\alpha\beta}) = \tensor{F}{^{\alpha}_{\gamma}}J^{\gamma} -
  \tensor{F}{^{\alpha}_{\gamma}}J^{\gamma} = 0.
\end{equation}

\section{Fluido perfetto relativistico}
\label{sec:fluido-perfetto}

Numerosi corpi macroscopici, compreso, in certe condizioni, l'intero Universo,
possono essere descritti da un \emph{\index{fluido!perfetto}fluido perfetto}
relativistico, cioè un
\emph{fluido costituito da un gran numero di particelle (in modo che possa
  essere considerato praticamente come un corpo continuo) che appare isotropo a
  un osservatore comovente}.
Un sistema di coordinate comoventi con il fluido è un sistema nel quale il
fluido appare punto per punto a riposo.  Un fluido perfetto si può
caratterizzare anche come un fluido privo di viscosità e in cui non ci sia
conduzione di calore.  La viscosità è la presenza di forze parallele alle
interfacce tra le varie porzioni del fluido.

Per ricavare il tensore energia-impulso di un fluido perfetto possiamo sfruttare
l'interpretazione fisica che abbiamo dato per ciascuna delle sue componente.
Iniziamo col considerare il fluido in un sistema comovente indichiamo con
$\tilde{T}^{\alpha\beta}$ il tensore energia-impulso in questo sistema.  Poiché
non c'è conduzione di calore, il flusso di energia è quello associato alle
particelle, che nel sistema comovente è nullo, dunque
$\tilde{T}^{0i} = \tilde{T}^{i0} = 0$.  Ricordiamo che $\tilde{T}^{ij}$ è la
componente $i$ della forza per unità di area su una superficie perpendicolare
all'asse $j$, quindi l'assenza di viscosità in un fluido perfetto comporta che
il tensore degli sforzi $\tilde{T}^{ij}$ è diagonale.  Il fatto che questo sia
vero per ogni sistema comovente, senza alcuna direzione privilegiata, implica in
particolare che $\tilde{T}^{ij}$ è proporzionale alla matrice identità
$\delta^{ij}$ e il fattore di proporzionalità sarà dunque la pressione $p$ del
fluido: $\tilde{T}^{ij} = p \delta^{ij}$.  Infine, $\tilde{T}^{00}$ è la densità
di energia, uguale alla densità di massa $\rho$ avendo posto la velocità della
luce nel vuoto $c = 1$.  In definitiva, il tensore energia-impulso di un fluido
perfetto in un sistema comovente è
\begin{equation}
  \tilde{T}^{\alpha\beta} =
  \begin{pmatrix}
    \rho & 0 & 0 & 0 \\
    0    & p & 0 & 0 \\
    0    & 0 & p & 0 \\
    0    & 0 & 0 & p
  \end{pmatrix}.
\end{equation}

Per ottenere il tensore energia-impulso $T^{\alpha\beta}$ del fluido perfetto in
un qualsiasi altro sistema di riferimento in moto macroscopico con velocità
$\bm{u}$ rispetto a un sistema comovente rispetto al fluido basta effettuare un
boost di $\bm{u}$:
$T^{\alpha\beta} =
\tensor{\Lambda}{^{\alpha}_{\mu}}\tensor{\Lambda}{^{\beta}_{\nu}}\tilde
T^{\mu\nu}$
\begin{subequations}
  \begin{align}
    \begin{split}
      T^{00} &= \tensor{\Lambda}{^{0}_{\mu}} \tensor{\Lambda}{^{0}_{\nu}} \tilde
      T^{\mu\nu} = \tensor{\Lambda}{^{0}_{i}} \tensor{\Lambda}{^{0}_{j}} \tilde
      T^{ij} + (\tensor{\Lambda}{^{0}_{0}})^{2} \tilde{T}^{00} = \gamma^{2} u_{i}
      u_{j} p \delta^{ij} + \gamma^{2} \rho \\
      &= \frac{\rho +pu^{2}}{1-u^{2}},
    \end{split} \\
    \begin{split}
      T^{i0} &= \tensor{\Lambda}{^{i}_{\mu}} \tensor{\Lambda}{^{0}_{\nu}} \tilde
      T^{\mu\nu} = \tensor{\Lambda}{^{i}_{k}} \tensor{\Lambda}{^{0}_{l}} \tilde
      T^{kl} + \tensor{\Lambda}{^{i}_{0}} \tensor{\Lambda}{^{0}_{0}} \tilde
      T^{00} \\
      &= \bigg(\tensor{\delta}{^{i}_{k}} + u^{i}u_{k}\frac{\gamma -
        1}{u^{2}}\bigg) u_{l} \gamma p \delta^{kl} + u^{i} \gamma^{2} \rho \\
      &= \bigg(\tensor{\delta}{^{i}_{k}} + u^{i}u_{k}\frac{\gamma -
        1}{u^{2}}\bigg) u^{k} \gamma p + u^{i} \gamma^{2} \rho = (u^{i} +
      u^{i}(\gamma - 1)) \gamma p + u^{i} \gamma^{2} \rho \\
      &= u^{i}\gamma^{2} p + u^{i} \gamma^{2} \rho = \frac{p + \rho}{1 -
        u^{2}}u^{i},
    \end{split} \\
    \begin{split}
      T^{ij} &= \tensor{\Lambda}{^{i}_{\mu}} \tensor{\Lambda}{^{j}_{\nu}} \tilde
      T^{\mu\nu} = \tensor{\Lambda}{^{i}_{k}} \tensor{\Lambda}{^{j}_{l}} \tilde
      T^{kl} + \tensor{\Lambda}{^{i}_{0}} \tensor{\Lambda}{^{j}_{0}} \tilde
      T^{00} \\
      &= \bigg(\tensor{\delta}{^{i}_{k}} + u^{i}u_{k}\frac{\gamma -
        1}{u^{2}}\bigg) \bigg(\tensor{\delta}{^{j}_{l}} + u^{j}u_{l}\frac{\gamma
        - 1}{u^{2}}\bigg) p \delta^{kl} + u^{i}u^{j} \gamma^{2} \rho \\
      &= \bigg(\tensor{\delta}{^{i}_{k}} + u^{i}u_{k}\frac{\gamma -
        1}{u^{2}}\bigg) \bigg(\delta^{jk} + u^{j}u^{k}\frac{\gamma -
        1}{u^{2}}\bigg) p + u^{i}u^{j} \gamma^{2} \rho \\
      &= \bigg(\delta^{ij} + 2 u^{i}u^{j} \frac{\gamma - 1}{u^{2}} +
      u^{i}u^{j}\frac{(\gamma - 1)^{2}}{u^{2}}\bigg)p + u^{i}u^{j} \gamma^{2}
      \rho \\
      &= \bigg(\delta^{ij} + u^{i}u^{j}\gamma^{2}\bigg)p + u^{i}u^{j} \gamma^{2}
      \rho = p \delta^{ij} + \frac{p + \rho}{1 - u^{2}}u^{i}u^{j}.
    \end{split}
  \end{align}
\end{subequations}
Questi risultati possono essere messi insieme nella seguente espressione
compatta\index{fluido!perfetto}
\begin{equation}
  T^{\alpha\beta} = \eta^{\alpha\beta}p + (p + \rho)U^{\alpha}U^{\beta},
\end{equation}
con $U^{\mu} = (\gamma_{u}, \gamma_{u} \bm{u})$ quadrivelocità associata al
vettore velocità $\bm{u}$.  Si riottiene l'espressione di
$\tilde{T}^{\alpha\beta}$ ponendo
$\bm{u} = \bm{0} \implies U^{\mu} = (1, \bm{0})$.

Nel caso in cui le particelle del fluido si muovono con velocità molto più
piccole di quella della luce nel vuoto (ma la velocità macroscopica $\bm{u}$ del
fluido è arbitraria), il tensore energia-impulso $\tilde{T}^{\alpha\beta}$ assume
un'espressione più semplice.  Innanzitutto nella densità di energia $\rho$
possiamo trascurare tutti i termini piccoli rispetto alla densità di energia di
riposo $\rho_{0}$, con $\rho_{0}$ somma delle masse a riposo delle particelle
contenute nell'unità di volume proprio del fluido.  In questo caso, inoltre,
anche la pressione data dall'energia del moto microscopico delle molecole è
trascurabile rispetto alla densità di energia a riposo $\rho$, quindi possiamo
scrivere
\begin{equation}
  \tilde{T}^{\alpha\beta} = \rho_{0} U^{\alpha} U^{\beta}.
\end{equation}
Se anche il moto macroscopico è lento il fattore lorentziano vale $1$, quindi
$U^{\mu}= (1, \bm{u})$ e abbiamo\index{fluido!perfetto}
\begin{subequations}
  \begin{align}
    \tilde{T}^{00} &= \rho_{0}, \\
    \tilde{T}^{ij} &= \rho_{0} U^{i} U^{j}.
  \end{align}
\end{subequations}


%%% Local Variables:
%%% mode: latex
%%% TeX-master: "../gravitazione"
%%% fill-column: 80
%%% TeX-PDF-mode: t
%%% End:
