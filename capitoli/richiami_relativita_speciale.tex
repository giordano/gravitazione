\cleardoublepage
\chapter{Richiami di relatività speciale}
\label{cha:relativita-speciale}

\section{Trasformazioni di Lorentz}
\label{sec:trasformazioni-lorentz}

Per il \index{principio!di relatività}\emph{principio di relatività} tutte le
leggi fisiche assumono la stessa forma in qualsiasi sistema di riferimento
inerziale.  In particolare, all'interno della teoria della relatività speciale
le leggi fisiche devono essere invarianti per trasformazioni di Lorentz.  In
questo capitolo dedurremo le trasformazioni di Lorentz e richiameremo alcuni
concetto di relatività speciale.

L'ambiente in cui si verificano i fenomeni fisici descritti dalla relatività
speciale è uno spazio quadrimensionale piatto, composto da una coordinata
temporale e tre spaziali, chiamato \index{spazio!di
  Minkowski}\emph{spazio
  di Minkowski}
e che indicheremo con il simbolo $\M$.  Un elemento dello spazio di Minkowski è
chiamato \index{evento}\emph{evento}.  Lo spazio di Minkowski si distingue dal
corrispondente spazio euclideo quadrimensionale $\R^{4}$ e anche dallo spazio
ordinario tridimensionale $\R^{3}$ in quanto utilizza una metrica differente da
quelle usualmente adottate in $\R^{3}$ ed $\R^{4}$.

Nello spazio ordinario $\R^{3}$ si definisce la metrica, ovvero la distanza fra
due punti infinitesimamente vicini di coordinate
$\bm{x} = (x,y,z) \equiv (x_{1},x_{2},x_{3})$ e
$\bm{x} + \dd \bm{x} = (x+\dd x,y+\dd y,z+\dd z) \equiv (x_{1}+\dd
x_{1},x_{2}+\dd x_{2},x_{3}+\dd x_{3})$ come
\begin{equation}
  \dd\bm{x}^{2} = (\dd x_{1})^{2} + (\dd x_{2})^{2} + (\dd x_{3})^{2} =
  \sum_{i=1}^{3} \sum_{j=1}^{3} \delta_{ij} \dd x_{i} \dd x_{j},
\end{equation}
dove $\delta_{ij}$ è il tensore metrico dato da
\begin{equation}
  \delta_{ij} = \diag (+1,+1,+1) =
  \begin{pmatrix}
    +1 & 0  & 0  \\
    0  & +1 & 0  \\
    0  & 0  & +1
  \end{pmatrix}.
\end{equation}
Analogamente, nello spazio $\R^{4}$ si utilizza il tensore metrico rappresentato
da $\diag (+1,+1,+1,+1)$.  Una generica trasformazione lineare omogenea delle
coordinate $(x,y,z) \equiv (x_{1},x_{2},x_{3})$ che lasci invariate la metrica,
cioè che soddisfi la condizione $\dd\bm{x}^{2} = \dd\bm{x}'^{2}$, può essere
scritta
come
\begin{equation}
  x'_{i} = \sum_{j=1}^{3} R_{ij}x_{j}, \qquad i,j = 1,2,3,
\end{equation}
con $R_{ij}$ costanti reali, e la matrice di componenti $R_{ij}$ è una matrice
di rotazione.

Nello spaziotempo di Minkowski definiamo le
\index{coordinata!controvariante}\emph{coordinate controvarianti} di un evento
che avviene al tempo $t$ nel punto dello spazio tridimensionale $\bm{x}$ come
\begin{equation}
  \label{eq:coord-controvariante}
  x^{\alpha} = (x^{0},x^{i}) = (t, \bm{x}),
\end{equation}
con $\alpha$ che varia fra $0$, $1$, $2$ e $3$ e $i$ fra $1$, $2$ e $3$.  La
\index{metrica!nello spazio di Minkowski}\emph{metrica} è data da
\begin{equation}
  \dd \tau^{2} = \dd t^{2} - \dd \bm{x}^{2} = -\tensor{\eta}{_{\alpha\beta}}\dd
  x^{\alpha} \dd x^{\beta},
\end{equation}
in cui $\tensor{\eta}{_{\alpha\beta}}$ è il
\index{tensore!metrico covariante di
  Minkowski}\emph{tensore metrico covariante di Minkowski}
\begin{equation}
  \label{eq:tensore-metrico-controvariante}
  \tensor{\eta}{_{\alpha\beta}} =
  \begin{pmatrix}
   \eta_{00} & \eta_{01} & \eta_{02} & \eta_{03} \\
   \eta_{10} & \eta_{11} & \eta_{12} & \eta_{13} \\
   \eta_{20} & \eta_{21} & \eta_{22} & \eta_{23} \\
   \eta_{30} & \eta_{31} & \eta_{32} & \eta_{33}
  \end{pmatrix}
  =
  \begin{pmatrix}
    -1 & 0  & 0  & 0 \\
    0  & +1 & 0  & 0 \\
    0  & 0  & +1 & 0 \\
    0  & 0  & 0  & +1
  \end{pmatrix}.
\end{equation}
Questo tensore definisce una metrica pseudo euclidea con segnatura
$(-+++)$,\footnote{In questo testo, se non diversamente specificato, abbiamo
  adottato questa segnatura, ma non è l'unica possibile nello spazio di
  Minkowski.  Un'altra segnatura largamente usata è $(+---)$.}
invece in $\R^{4}$ abbiamo visto che si utilizza la segnatura $(++++)$.  Una
\index{trasformazione!di Lorentz}\emph{trasformazione di Lorentz} non omogenea è
una trasformazione lineare nello spazio di Minkowski da un sistema di
riferimento inerziale di coordinate $x^{\alpha}$ a un altro sistema di
riferimento inerziale di coordinate
\begin{equation}
  \label{eq:trasformazione-lineare}
  x'^{\alpha} = \tensor{\Lambda}{^{\alpha}_{\beta}}x^{\beta} + a^{\alpha},
\end{equation}
con $a^{\alpha}$ costanti, che mantenga invariata la distanza fra due eventi,
cioè la metrica $\dd\tau^{2} = -\eta_{\alpha\beta}\dd x^{\alpha}\dd x^{\beta}$
in $\M$.  Richiediamo che la trasformazione $x^{\alpha} \to x'^{\alpha}$ sia
lineare per rispettare la condizione di omogeneità dello spazio (la lunghezza di
una sbarra non deve dipendere dalla posizione degli estremi), del tempo (la
lunghezza della sbarra non deve dipendere dall'istante di tempo in cui la
misuriamo) e di isotropia dello spazio (la lunghezza della sbarra non deve
dipendere neanche dalla direzione in cui la guardiamo).  Inoltre, in questo modo
un moto rettilineo uniforme in un sistema di riferimento inerziale risulta è
ancora un moto rettilineo uniforme anche nell'altro sistema di riferimento
inerziale.  La richiesta di invarianza della metrica, invece, permette di
soddisfare il principio di relatività.  I differenziali delle coordinate nel
nuovo sistema si ricavano dalla~\eqref{eq:trasformazione-lineare}
\begin{equation}
  \label{eq:trasformazione-differenziali}
  \dd x'^{\alpha} = \tensor{\Lambda}{^{\alpha}_{\beta}}\dd x^{\beta}.
\end{equation}
Dall'equazione precedente abbiamo inoltre
\begin{equation}
  \tensor{\Lambda}{^{\alpha}_{\beta}} = \parder{x'^{\alpha}}{x^{\beta}}.
\end{equation}

Affinché una trasformazione lineare~\eqref{eq:trasformazione-lineare} delle
coordinate sia di Lorentz, cioè affinché conservi la metrica, deve valere
\begin{equation}
  \begin{split}
    \dd \tau'^{2} &= -\tensor{\eta}{_{\alpha\beta}}\dd x'^{\alpha}\dd x'^{\beta}
    = -\tensor{\eta}{_{\alpha\beta}} \tensor{\Lambda}{^{\alpha}_{\gamma}} \dd
    x^{\gamma} \tensor{\Lambda}{^{\beta}_{\delta}} \dd x^{\delta} \\
    &= \dd \tau^{2} = -\tensor{\eta}{_{\gamma\delta}} \dd x^{\gamma} \dd x^{\delta},
  \end{split}
\end{equation}
da cui, semplificando nel terzo e nel quinto membro
$\dd x^{\gamma}\dd x^{\delta}$, otteniamo la relazione fondamentale del tensore
metrico
\begin{equation}
  \label{eq:rel-fond-tensore-metrico}
  \tensor{\eta}{_{\alpha\beta}} \tensor{\Lambda}{^{\alpha}_{\gamma}}
  \tensor{\Lambda}{^{\beta}_{\delta}} = \tensor{\eta}{_{\gamma\delta}}.
\end{equation}
Questa relazione impone $10$ condizioni indipendenti in quanto il tensore
metrico $\tensor{\eta}{_{\gamma\delta}}$ è simmetrico quindi solo $10$ delle sue
$16$ componenti sono indipendenti.

Indicando con $\Lambda$ la matrice $4 \times 4$ i cui elementi sono le
$\tensor{\Lambda}{^{\alpha}_{\beta}}$
\begin{equation}
  \Lambda =
  \begin{pmatrix}
    \tensor{\Lambda}{^{0}_{0}} & \tensor{\Lambda}{^{0}_{1}} &
    \tensor{\Lambda}{^{0}_{2}} & \tensor{\Lambda}{^{0}_{3}} \\
    \tensor{\Lambda}{^{1}_{0}} & \tensor{\Lambda}{^{1}_{1}} &
    \tensor{\Lambda}{^{1}_{2}} & \tensor{\Lambda}{^{1}_{3}} \\
    \tensor{\Lambda}{^{2}_{0}} & \tensor{\Lambda}{^{2}_{1}} &
    \tensor{\Lambda}{^{2}_{2}} & \tensor{\Lambda}{^{2}_{3}} \\
    \tensor{\Lambda}{^{3}_{0}} & \tensor{\Lambda}{^{3}_{1}} &
    \tensor{\Lambda}{^{3}_{2}} & \tensor{\Lambda}{^{3}_{3}}
  \end{pmatrix},
\end{equation}
l'equazione matriciale equivalente alla~\eqref{eq:rel-fond-tensore-metrico} è
\begin{equation}
  \Lambda^{\textup{T}}\eta\Lambda = \eta
\end{equation}
dalla quale abbiamo
\begin{equation}
  \det\eta = \det(\Lambda^{\textup{T}}\eta\Lambda) = \det\eta
  \det(\Lambda^{\textup{T}}\Lambda) = \det\eta (\det\Lambda)^{2}
\end{equation}
e quindi deve valere la condizione
\begin{equation}
  \det\Lambda = \pm 1.
\end{equation}
Le trasformazioni di Lorentz con $\det\Lambda = +1$ sono chiamate
\emph{proprie}, quelle con $\det\Lambda = -1$ sono chiamate \emph{improprie}.

\begin{figure}
  \centering
  \tdplotsetmaincoords{70}{100}
  \begin{tikzpicture}[tdplot_main_coords,font=\footnotesize,scale=3]
    \node[shape=coordinate] (O1) at (0,0,0) [label=above right:$O$] {};
    \node[shape=coordinate] (O2) at (-1,0.5,0.3) [label=above right:$O'$] {};
    \draw[->] (O1) -- ++(1,0,0) node[anchor=north east] {$x$}; % asse x
    \draw[->] (O1) -- ++(0,1,0) node[anchor=north west] {$y$}; % asse y
    \draw[->] (O1) -- ++(0,0,1) node[anchor=south] (Z1) {$z$}; % asse z
    \node at (Z1) [label=below right:$\mathcal{S}$] {};% sistema di riferimento
    \draw[->] (O2) -- ++(1,0,0) node[anchor=north east] {$x'$}; % asse x'
    \draw[->] (O2) -- ++(0,1,0) node[anchor=north west] {$y'$}; % asse y'
    \draw[->] (O2) -- ++(0,0,1) node[anchor=south] (Z2) {$z'$}; % asse z'
    \node at (Z2) [label=below right:$\mathcal{S}'$] {};% sistema di riferimento
    \draw[->,thick] (O1) -- node[above] {$\bm{v}$} ++(0.3,-0.4,0.3); % velocità
  \end{tikzpicture}
  \caption[Sistemi di riferimento inerziali in moto l'uno rispetto
  all'altro]{Sistemi di riferimento inerziali, con gli assi corrispondenti
    paralleli, in moto l'uno rispetto all'altro}
  \label{fig:trasformazioni-lorentz}
\end{figure}
Esempi di trasformazioni di Lorentz proprie sono le rotazioni spaziali, che
lasciano invariato il tempo, e le \emph{trasformazioni di velocità}, o
\index{boost}\emph{boost}, vale a dire le trasformazioni tra due sistemi di
riferimento inerziali con gli assi cartesiani corrispondenti paralleli e in moto
l'uno rispetto all'altro con velocità costante.
%%%%% TODO: controllare la validità di questa affermazione:
La composizione di due trasformazioni di Lorentz è ancora una
trasformazione di Lorentz.
%%%%% Fine affermazione da controllare.
Vediamo come si trasformano le coordinate $x^{\alpha}$ di un evento in un
sistema di riferimento per effetto di un boost $\bm{v}$.  Consideriamo una
particella a riposo in un sistema di riferimento $\mathcal{S}$ che si muove con
velocità $\bm{v} = (v_{1},v_{2},v_{3})$ rispetto a un altro sistema di
riferimento inerziale $\mathcal{S}'$, come nella
figura~\ref{fig:trasformazioni-lorentz}.\footnote{Si noti che il sistema di
  riferimento $\mathcal{S}$ vede il sistema $\mathcal{S}'$ muoversi con velocità
  $-\bm{v}$.  Numerosi autori considerano il sistema di riferimento
  $\mathcal{S}'$ in moto rispetto al sistema $\mathcal{S}$ con velocità
  $\bm{v}$, questo comporterà una differenza di segno in numerose equazioni
  rispetto a quelle ricavate da questi autori.}
Poiché nel sistema $\mathcal{S}$ per la particella risulta $\dd\bm{x} = \bm{0}$,
si ricava
\begin{subequations}
  \begin{align}
    \dd x'^{i} &= \tensor{\Lambda}{^{i}_{\alpha}}\dd x^{\alpha} =
    \tensor{\Lambda}{^{i}_{0}}\dd x^{0} + \tensor{\Lambda}{^{i}_{j}}\dd x^{j} =
    \tensor{\Lambda}{^{i}_{0}}\dd t \\
    \dd x'^{0} &= \tensor{\Lambda}{^{0}_{\alpha}}\dd x^{\alpha} =
    \tensor{\Lambda}{^{0}_{0}}\dd x^{0} + \tensor{\Lambda}{^{0}_{j}}\dd x^{j} =
    \tensor{\Lambda}{^{0}_{0}}\dd t.
  \end{align}
\end{subequations}
Dividendo membro a membro abbiamo
\begin{equation}
  \toder{x'^{i}}{x'^{0}} = \toder{x'^{i}}{t'} = v_{i} =
  \frac{\tensor{\Lambda}{^{i}_{0}}}{\tensor{\Lambda}{^{0}_{0}}},
\end{equation}
in quanto $\ltoder{x'^{i}}{t'}$ è la componente $v_{i}$ della velocità della
particella vista dal sistema $\mathcal{S}'$.  Allora
\begin{equation}
  \tensor{\Lambda}{^{i}_{0}} = v_{i}\tensor{\Lambda}{^{0}_{0}}.
\end{equation}
Ponendo $\gamma = \delta = 0$ nella~\eqref{eq:rel-fond-tensore-metrico} risulta
\begin{equation}
  \begin{split}
    \tensor{\eta}{_{00}} &= -1 = \tensor{\eta}{_{\alpha\beta}}
    \tensor{\Lambda}{^{\alpha}_{0}} \tensor{\Lambda}{^{\beta}_{0}} =
    \tensor{\eta}{_{\alpha\beta}}(\tensor{\Lambda}{^{0}_{0}})^{2} +
    \tensor{\eta}{_{ij}} \tensor{\Lambda}{^{i}_{0}} \tensor{\Lambda}{^{j}_{0}}
    \\
    &= -(\tensor{\Lambda}{^{0}_{0}})^{2} + \delta_{ij}
    \tensor{\Lambda}{^{i}_{0}} \tensor{\Lambda}{^{j}_{0}} =
    -(\tensor{\Lambda}{^{0}_{0}})^{2}+\sum_{i}(\tensor{\Lambda}{^{i}_{0}})^{2},
  \end{split}
\end{equation}
quindi
\begin{equation}
  (\tensor{\Lambda}{^{0}_{0}})^{2} = 1 +
  \sum_{i}(\tensor{\Lambda}{^{i}_{0}})^{2} = 1 +
  \sum_{i}(v_{i}\tensor{\Lambda}{^{0}_{0}})^{2} \geq 1,
\end{equation}
cioè deve aversi
\begin{equation}
  \tensor{\Lambda}{^{0}_{0}} \geq 1 \text{ oppure } \tensor{\Lambda}{^{0}_{0}}
  \leq -1.
\end{equation}
Le trasformazioni con $\tensor{\Lambda}{^{0}_{0}} \geq +1$ sono chiamate
\emph{ortocrone}, quelle con $\tensor{\Lambda}{^{0}_{0}} \leq -1$
\emph{anticrone}.  Le trasformazioni di Lorentz proprie e ortocrone sono dette
\emph{trasformazioni di Lorentz ristrette} o \emph{speciali}.  Considerando una
trasformazione ristretta inoltre abbiamo
\begin{equation}
  \tensor{\Lambda}{^{0}_{0}} = \frac{1}{\sqrt{1-\sum_{i}(v_{i})^{2}}} =
  \frac{1}{\sqrt{1-v^{2}}} = \gamma(v) \equiv \gamma,
\end{equation}
con $v^{2} = \sum_{i}(v_{i})^{2}$ modulo quadro del vettore $\bm{v}$.  Quindi
risulta anche
\begin{equation}
  \tensor{\Lambda}{^{i}_{0}} = v_{i}\tensor{\Lambda}{^{0}_{0}} = v_{i}\gamma.
\end{equation}
Le altre componenti $\tensor{\Lambda}{^{\alpha}_{\beta}}$ non sono univocamente
determinate.  Una scelta conveniente che permette di soddisfare
la relazione fondamentale~\eqref{eq:rel-fond-tensore-metrico} è
\begin{subequations}
  \begin{align}
    \tensor{\Lambda}{^{0}_{i}} &= \tensor{\Lambda}{^{i}_{0}} = v_{i}\gamma, \\
    \tensor{\Lambda}{^{i}_{j}} &= \delta_{ij} +
    v_{i}v_{j}\frac{\gamma-1}{v^{2}}.
  \end{align}
\end{subequations}
così che la matrice $\Lambda$ può essere scritta, per un \index{boost}boost
$\bm{v}$, come
\begin{equation}
  \Lambda =
  \begin{pmatrix}
    \gamma & \gamma v_{1} & \gamma v_{2} & \gamma _{3} \\[2.0ex]
    \gamma v_{1} & 1 + (v_{1})^{2}\dfrac{\gamma-1}{v^{2}} &
    v_{1}v_{2}\dfrac{\gamma-1}{v^{2}} & v_{1} v_{3}\dfrac{\gamma-1}{v^{2}}
    \\[2.0ex]
    \gamma v_{2} & v_{2}v_{1}\dfrac{\gamma-1}{v^{2}} & 1 +
    (v_{2})^{2}\dfrac{\gamma-1}{v^{2}} & v_{2}v_{3}\dfrac{\gamma-1}{v^{2}}
    \\[2.0ex]
    \gamma v_{3} & v_{3}v_{1}\dfrac{\gamma-1}{v^{2}} &
    v_{2}v_{3}\dfrac{\gamma-1}{v^{2}} & 1 + (v_{3})^{2}\dfrac{\gamma-1}{v^{2}}
  \end{pmatrix}.
\end{equation}


% TODO: scrivere qualcosa!
\section{\completare{Conseguenze delle trasformazioni di Lorentz}}
\label{sec:conseguenze-lorentz}

\subsection{\completare{Contrazione delle lunghezze}}
\label{sec:contrazione-lunghezze}

% TODO: ricordarsi di definire il tempo proprio (servirà più avanti)
\subsection{\completare{Dilatazione degli intervalli temporali}}
\label{sec:dilatazione-tempo}

\subsection{\completare{Effetto Doppler}}
\label{sec:effetto-doppler}

\section{Calcolo tensoriale nello spazio di Minkowski}
\label{sec:calcolo-tensoriale}

Il calcolo tensoriale è lo strumento matematico alla base della teoria della
relatività.  In questo paragrafo daremo dei cenni relativi al calcolo tensoriale
nello spazio di Minkowski.  Si noti che alcuni dei risultati che presenteremo
qui sono validi solo nello spazio piatto della relatività speciale e non in
quello curvo della relatività generale.

Un \index{quadrivettore!controvariante}\emph{quadrivettore controvariante}
$V^{\alpha} = (V^{0}, V^{1}, V^{2}, V^{3})$ è un insieme di quattro quantità
che, per effetto di una trasformazione di Lorentz ristretta omogenea
$x^{\alpha} \to x'^{\alpha} = \tensor{\Lambda}{^{\alpha}_{\beta}}x^{\beta}$, si
trasformano come le componenti di $\dd
x^{\alpha}$~\eqref{eq:trasformazione-differenziali}, vale a dire
\begin{equation}
  V^{\alpha} \to V'^{\alpha} = \tensor{\Lambda}{^{\alpha}_{\beta}}V^{\beta}.
\end{equation}
Un quadrivettore controvariante può anche essere rappresentato come
\begin{equation}
  V^{\alpha} = (V^{0}, V^{i}) = (V^{0}, \bm{V}),
\end{equation}
dove $\bm{V} = (V^{1}, V^{2}, V^{3})$.

Un \index{quadrivettore!covariante}\emph{quadrivettore covariante}
$U_{\alpha} = (U_{0}, U_{1}, U_{2}, U_{3})$ è un insieme di quattro quantità
che, sotto una trasformazione di Lorentz ristretta, si trasformano come
\begin{equation}
  U_{\alpha} \to U'_{\alpha} = \tensor{\Lambda}{_{\alpha}^{\beta}}U_{\beta}
\end{equation}
dove
\begin{equation}
  \tensor{\Lambda}{_{\alpha}^{\beta}} = \tensor{\eta}{_{\alpha\gamma}}
  \tensor{\eta}{^{\beta\delta}} \tensor{\Lambda}{^{\gamma}_{\delta}}.
\end{equation}
La matrice $\tensor{\eta}{^{\alpha\beta}}$ è il
\index{tensore!metrico controvariante di
  Minkowski}\emph{tensore
  metrico controvariante di Minkowski}
ed è numericamente uguale a $\tensor{\eta}{_{\alpha\beta}}$, vale a dire
\begin{equation}
  \tensor{\eta}{^{\alpha\beta}} = \tensor{\eta}{_{\alpha\beta}}.
\end{equation}
Osserviamo che
\begin{equation}
  \tensor{\eta}{^{\alpha\gamma}} \tensor{\eta}{_{\beta\gamma}} =
  \tensor{\delta}{^{\alpha}_{\beta}} = \tensor{\delta}{_{\alpha}^{\beta}} =
  \begin{cases}
    1 & \text{se } \alpha = \beta, \\
    0 & \text{se } \alpha \neq \beta,
  \end{cases}
\end{equation}
in cui $\tensor{\delta}{^{\alpha}_{\beta}} = \tensor{\delta}{_{\alpha}^{\beta}}$
è la \index{delta!di Kronecker
  quadrimensionale}\emph{delta
  di Kronecker quadrimensionale}.
Usando la relazione fondamentale delle trasformazioni di
Lorentz~\eqref{eq:rel-fond-tensore-metrico} abbiamo
\begin{equation}
  \tensor{\Lambda}{_{\alpha}^{\gamma}} \tensor{\Lambda}{^{\alpha}_{\beta}} =
  \tensor{\eta}{_{\alpha\delta}} \tensor{\eta}{^{\gamma\epsilon}}
  \tensor{\Lambda}{^{\delta}_{\epsilon}} \tensor{\Lambda}{^{\alpha}_{\beta}} =
  \tensor{\eta}{_{\epsilon\beta}} \tensor{\eta}{^{\gamma\epsilon}} =
  \tensor{\delta}{^{\gamma}_{\beta}}.
\end{equation}
Dunque $\tensor{\Lambda}{_{\alpha}^{\beta}}$ è la matrice inversa di
$\tensor{\Lambda}{^{\alpha}_{\beta}}$.

Vogliamo ora determinare la trasformazione inversa $x'^{\alpha} \to x^{\alpha}$.
Dalla trasformazione~\eqref{eq:trasformazione-lineare} risulta che
\begin{equation}
  x'^{\alpha} \tensor{\Lambda}{_{\alpha}^{\gamma}} =
  \tensor{\Lambda}{^{\alpha}_{\beta}} \tensor{\Lambda}{_{\alpha}^{\gamma}}
  x^{\beta} + \tensor{\Lambda}{_{\alpha}^{\gamma}} a^{\alpha} =
  \tensor{\delta}{^{\gamma}_{\beta}} x^{\beta} + a'^{\gamma} = x^{\gamma}
  +a'^{\gamma}
\end{equation}
da cui ricaviamo la trasformazione, rinominando gli indici,
\begin{equation}
  \label{eq:trasformazione-inversa}
  x'^{\alpha} \to x^{\alpha} = \tensor{\Lambda}{_{\beta}^{\alpha}}x'^{\beta} -
  a'^{\alpha}.
\end{equation}
Osserviamo inoltre che risulta
\begin{equation}
  \tensor{\Lambda}{_{\beta}^{\alpha}} = \parder{x^{\alpha}}{x'^{\beta}}.
\end{equation}

Possiamo passare da un quadrivettore controvariante
$V^{\alpha} = (V^{0}, V^{i})$ al corrispondente quadrivettore covariante
$V_{\alpha} = (V_{0}, V_{i})$ moltiplicandolo per il tensore di Minkowski
covariante
\begin{equation}
  \label{eq:abbassamento-indice}
  \begin{split}
    V^{\alpha} \to V_{\alpha} &= \tensor{\eta}{_{\alpha\beta}}V^{\beta} =
    \tensor{\eta}{_{\alpha\beta}} (V^{0}, V^{i}) =
    (\tensor{\eta}{_{0\beta}}V^{0}, \tensor{\eta}{_{i\beta}}V^{i}) \\
    &= (\tensor{\eta}{_{00}}V^{0}, \delta_{i\beta}V^{\beta}) = (-V^{0}, V^{i}).
  \end{split}
\end{equation}
Pertanto, mediante questa operazione, chiamata a volte di
\emph{abbassamento degli indici}, la componente temporale di un quadrivettore
cambia segno mentre quelle spaziali rimangono immutate.  Analogamente si può
effettuare l'\emph{innalzamento degli indici} di un quadrivettore covariante
moltiplicandolo per il tensore di Minkowski controvariante
\begin{equation}
  \label{eq:innalzamento-indice}
  U_{\alpha} \to U^{\alpha} = \tensor{\eta}{^{\alpha\beta}}U_{\beta}.
\end{equation}
Verifichiamo che il quadrivettore $V_{\alpha}$ definito
nella~\eqref{eq:abbassamento-indice} si trasforma effettivamente come un
quadrivettore covariante
\begin{equation}
  V_{\alpha} \to V'_{\alpha} = \tensor{\eta}{_{\alpha\beta}}V'^{\beta} =
  \tensor{\eta}{_{\alpha\beta}} \tensor{\Lambda}{^{\beta}_{\gamma}} V^{\gamma} =
  \tensor{\eta}{_{\alpha\beta}} \tensor{\Lambda}{^{\beta}_{\gamma}}
  \tensor{\eta}{^{\gamma\delta}}V_{\delta} =
  \tensor{\Lambda}{_{\alpha}^{\delta}} V_{\delta}.
\end{equation}
Allo stesso modo si può far vedere che il quadrivettore $U^{\alpha}$ definito
nella~\eqref{eq:innalzamento-indice} si trasforma come un quadrivettore
controvariante.

Possiamo generalizzare le definizioni date finora introducendo il concetto di
quadritensore.  Un \index{quadritensore}\emph{quadritensore}
$\tensor*{A}{^{\alpha_{1}\dots\alpha_{p}}_{\beta_{1}\dots\beta_{q}}}$ (con
$p + q = k$) di rango $k$ è un insieme di $4^{k}$ quantità che sotto una
trasformazione di Lorentz ristretta omogenea
$x^{\alpha} \to x'^{\alpha} = \tensor{\Lambda}{^{\alpha}_{\beta}}x^{\beta}$ si
trasformano come
\begin{equation}
  \tensor*{A}{^{\alpha_{1}\dots\alpha_{p}}_{\beta_{1}\dots\beta_{q}}} \to
  \tensor*{{A'}}{^{\alpha_{1}\dots\alpha_{p}}_{\beta_{1}\dots\beta_{q}}} =
  \tensor{\Lambda}{^{\alpha_{1}}_{\mu_{1}}} \cdots
  \tensor{\Lambda}{^{\alpha_{p}}_{\mu_{p}}}
  \tensor{\Lambda}{_{\beta_{1}}^{\nu_{1}}} \cdots
  \tensor{\Lambda}{_{\beta_{q}}^{\nu_{q}}}
  \tensor*{A}{^{\mu_{1}\dots\mu_{p}}_{\nu_{1}\dots\nu_{q}}}
\end{equation}
Il quadritensore
$\tensor*{A}{^{\alpha_{1}\dots\alpha_{p}}_{\beta_{1}\dots\beta_{q}}}$ è un
quadritensore misto $p$ volte controvariante e $q$ volte covariante, oppure di
tipo $(p,q)$.  Un quadritensore di rango $k$ completamente controvariante è del
tipo $\tensor{A}{^{\alpha_{1}\dots\alpha_{k}}}$ e si trasforma come
\begin{equation}
  \tensor{A}{^{\alpha_{1}\dots\alpha_{k}}} \to
  \tensor{{A'}}{^{\alpha_{1}\dots\alpha_{k}}} =
  \tensor{\Lambda}{^{\alpha_{1}}_{\mu_{1}}} \cdots
  \tensor{\Lambda}{^{\alpha_{k}}_{\mu_{k}}} \tensor{A}{^{\mu_{1}\dots\mu_{k}}},
\end{equation}
similmente un quadritensore di rango $k$ completamente covariante è del tipo
$\tensor{A}{_{\alpha_{1}\dots\alpha_{k}}}$ e si trasforma come
\begin{equation}
  \tensor{A}{_{\alpha_{1}\dots\alpha_{k}}} \to
  \tensor{{A'}}{_{\alpha_{1}\dots\alpha_{k}}} =
  \tensor{\Lambda}{_{\alpha_{1}}^{\mu_{1}}} \cdots
  \tensor{\Lambda}{_{\alpha_{k}}^{\mu_{k}}} \tensor{A}{_{\mu_{1}\dots\mu_{k}}}.
\end{equation}
I quadritensori di rango $1$ sono i \index{quadrivettore}\emph{quadrivettori}, i
quadritensori di rango $0$, che quindi non hanno indici, sono chiamati
\index{scalare!di Lorentz}\emph{scalari di Lorentz} (o brevemente
\emph{scalari}) e sono invarianti sotto le trasformazioni di Lorentz.

\subsection{Operazioni con i quadritensori}
\label{sec:operazioni-tensori}

\subsubsection{Combinazione lineare}
\label{sec:combinazione-lineare}

La combinazione lineare
\begin{equation}
  \tensor*{T}{^{\alpha_{1}\dots\alpha_{p}}_{\beta_{1}\dots\beta_{q}}} = a
  \tensor*{R}{^{\alpha_{1}\dots\alpha_{p}}_{\beta_{1}\dots\beta_{q}}} + b
  \tensor*{S}{^{\alpha_{1}\dots\alpha_{p}}_{\beta_{1}\dots\beta_{q}}},
\end{equation}
con $a$ e $b$ scalari, di due quadritensori dello stesso tipo è un quadritensore
ancora dello stesso tipo.  Per esempio, verifichiamo che
\begin{equation}
  \tensor{T}{^{\alpha}_{\beta}} = a \tensor{R}{^{\alpha}_{\beta}} + b
  \tensor{S}{^{\alpha}_{\beta}}
\end{equation}
è un quadritensore di rango $2$ una volta controvariante e una volta covariante
come lo sono $\tensor{R}{^{\alpha}_{\beta}}$ e
$\tensor{S}{^{\alpha}_{\beta}}$. Abbiamo
\begin{equation}
  \begin{split}
    \tensor{T}{^{\alpha}_{\beta}} \to \tensor{{T'}}{^{\alpha}_{\beta}} &= a
    \tensor{{R'}}{^{\alpha}_{\beta}} + b \tensor{{S'}}{^{\alpha}_{\beta}} = a
    \tensor{\Lambda}{^{\alpha}_{\gamma}} \tensor{\Lambda}{_{\beta}^{\delta}}
    \tensor{R}{^{\gamma}_{\delta}} + b \tensor{\Lambda}{^{\alpha}_{\gamma}}
    \tensor{\Lambda}{_{\beta}^{\delta}} \tensor{S}{^{\gamma}_{\delta}} \\
    &= \tensor{\Lambda}{^{\alpha}_{\gamma}}
    \tensor{\Lambda}{_{\beta}^{\delta}}(a \tensor{R}{^{\gamma}_{\delta}} + b
    \tensor{S}{^{\gamma}_{\delta}}) = \tensor{\Lambda}{^{\alpha}_{\gamma}}
    \tensor{\Lambda}{_{\beta}^{\delta}} \tensor{T}{^{\gamma}_{\delta}}.
  \end{split}
\end{equation}

\subsubsection{Prodotto diretto}
\label{sec:prodotto-diretto}

Il prodotto
$\tensor*{A}{^{\alpha_{1}\dots\alpha_{p}}_{\beta_{1}\dots\beta_{q}}}
\tensor*{B}{^{\alpha_{1}\dots\alpha_{r}}_{\beta_{1}\dots\beta_{s}}}$
di un quadritensore di tipo $(p,q)$ e di uno di tipo $(r,s)$ è un quadritensore
$\tensor*{T}{^{\alpha_{1}\dots\alpha_{p+r}}_{\beta_{1}\dots\beta_{q+s}}}$ di
tipo $(p+r,q+s)$.  Consideriamo per esempio
\begin{equation}
  \tensor{T}{^{\alpha}_{\beta}^{\delta}} = \tensor{A}{^{\alpha}_{\beta}}
  \tensor{B}{^{\gamma}}.
\end{equation}
Abbiamo
\begin{equation}
  \tensor{T}{^{\alpha}_{\beta}^{\delta}} \to
  \tensor{{T'}}{^{\alpha}_{\beta}^{\delta}} = \tensor{{A'}}{^{\alpha}_{\beta}}
  \tensor{{B'}}{^{\gamma}} = \tensor{\Lambda}{^{\alpha}_{\delta}}
  \tensor{\Lambda}{_{\beta}^{\epsilon}} \tensor{A}{^{\delta}_{\epsilon}}
  \tensor{\Lambda}{^{\gamma}_{\zeta}} \tensor{B}{^{\zeta}} =
  \tensor{\Lambda}{^{\alpha}_{\delta}} \tensor{\Lambda}{_{\beta}^{\epsilon}}
  \tensor{\Lambda}{^{\gamma}_{\zeta}} \tensor{T}{^{\delta}_{\epsilon}^{\zeta}}.
\end{equation}

\subsubsection{Contrazione degli indici}
\label{sec:contrazione-indice}

Se in un quadritensore di tipo $(p,q)$ $n$ indici controvarianti sono uguali a
$n$ indici covarianti e si somma su tali indici si ottiene un tensore di tipo
$(p-n,q-n)$.  Usando la convenzione di Einstein sulla somma si omette il simbolo
di sommatoria.  Questa operazione viene detta di
\emph{contrazione degli indici}.  Per esempio, il tensore
$\tensor{T}{^{\alpha}_{\beta}^{\gamma\delta}}$ è un tensore di tipo $(3,1)$, ma
se si contraggono l'indice covariante $\beta$ con il terzo indice controvariante
$\delta$ si ottiene il tensore
$\tensor{T}{^{\alpha}_{\beta}^{\gamma\beta}} =
\tensor{T}{^{\alpha}_{0}^{\gamma 0}} + \tensor{T}{^{\alpha}_{1}^{\gamma 1}} +
\tensor{T}{^{\alpha}_{2}^{\gamma 2}} + \tensor{T}{^{\alpha}_{3}^{\gamma 3}}$
che è completamente controvariante di rango $2$.  Infatti
\begin{equation}
  \begin{split}
    \tensor{T}{^{\alpha}_{\beta}^{\gamma\beta}} \to
    \tensor{{T'}}{^{\alpha}_{\beta}^{\gamma\beta}} &=
    \tensor{\Lambda}{^{\alpha}_{\omega}} \tensor{\Lambda}{_{\beta}^{\mu}}
    \tensor{\Lambda}{^{\gamma}_{\nu}} \tensor{\Lambda}{^{\beta}_{\zeta}}
    \tensor{T}{^{\omega}_{\mu}^{\nu\zeta}} =
    \tensor{\Lambda}{^{\alpha}_{\omega}} \tensor{\Lambda}{^{\gamma}_{\nu}}
    \tensor{\delta}{^{\mu}_{\zeta}} \tensor{T}{^{\omega}_{\mu}^{\nu\zeta}} \\
    &= \tensor{\Lambda}{^{\alpha}_{\omega}} \tensor{\Lambda}{^{\gamma}_{\nu}}
    \tensor{T}{^{\omega}_{\mu}^{\nu\mu}}.
  \end{split}
\end{equation}

Si può effettuare la contrazione degli indici anche nel prodotto di due
quadritensori: il prodotto di un quadritensore di tipo $(p,q)$ e di uno di tipo
$(r,s)$ è un quadritensore di tipo $(p+r,q+s)$, ma se si contraggono $n$ indici
si ottiene un quadritensore di tipo $(p+r-n,q+s-n)$.  Per esempio, il
quadritensore $\tensor{A}{^{\alpha}_{\beta}^{\gamma\delta}}$ è di tipo $(3,1)$,
$\tensor{B}{_{\epsilon\zeta}}$ è di tipo $(0,2)$, il prodotto
$\tensor{A}{^{\alpha}_{\beta}^{\gamma\delta}} \tensor{B}{_{\epsilon\zeta}}$ è un
quadritensore di tipo $(3,3)$, ma il prodotto
$\tensor{A}{^{\alpha}_{\beta}^{\gamma\delta}} \tensor{B}{_{\alpha\delta}}$ con
due indici contratti è un quadritensore di tipo $(1,1)$.

\subsubsection{Prodotto scalare di due quadrivettori}
\label{sec:prodotto-scalare-quadrivettori}

Il \emph{prodotto scalare} fra due quadrivettori $V$ e $U$ è definito come
\begin{equation}
  \begin{split}
    V\cdot U &= V^{\alpha}U_{\alpha} =
    \tensor{\eta}{_{\alpha\beta}}V^{\alpha}U^{\beta} = V^{0}U_{0} + V^{1}U_{1} +
    V^{2}U_{2} + V^{3}U_{3} \\
    &= -V^{0}U^{0} + V^{1}U^{1} + V^{2}U^{2} + V^{3}U^{3} = -V^{0}U^{0} +
    \bm{V}\cdot\bm{U}.
  \end{split}
\end{equation}

Il prodotto scalare di due quadrivettori è un caso particolare di contrazione
degli indici nel prodotto fra due quadritensori.  Dunque il loro prodotto è un
quadritensore di rango $0$, cioè uno scalare e quindi una quantità che non varia
per effetto di una trasformazione di Lorentz.  Infatti
\begin{equation}
  V^{\alpha}U_{\alpha} \to V'^{\alpha}U'_{\alpha} =
  \tensor{\Lambda}{^{\alpha}_{\beta}} V^{\beta}
  \tensor{\Lambda}{_{\alpha}^{\gamma}} U_{\gamma} =
  \tensor{\delta}{_{\beta}^{\gamma}} V^{\beta} U_{\gamma} = V^{\gamma}U_{\gamma}
  = V^{\alpha} U_{\alpha}.
\end{equation}

Si può definire anche il \index{modulo!di un
  quadrivettore}\emph{modulo
  quadro di un quadrivettore}
$V$ come il prodotto scalare di $V$ per se stesso, vale a dire
\begin{equation}
  \begin{split}
    V\cdot V &= V^{\alpha}V_{\alpha} =
    \tensor{\eta}{_{\alpha\beta}}V^{\alpha}V^{\beta} = V^{0}V_{0} + V^{1}V_{1} +
    V^{2}V_{2} + V^{3}V_{3} \\
    &= -(V^{0})^{2} + (V^{1})^{2} + (V^{2})^{2} + (V^{3})^{2} = -(V^{0})^{2} +
    \bm{V}^{2}.
  \end{split}
\end{equation}
Anche il modulo quadro di un quadrivettore è uno scalare di Lorentz.

\subsubsection{Derivazione}
\label{sec:derivazione}

La derivazione di un quadritensore di rango $k$ rispetto alle coordinate
controvarianti $x^{\alpha}$ o covarianti
$x_{\alpha} = (x_{0},x_{1}) = (-x^{0},x^{i})$ è un quadritensore di rango $k+1$.
In dettaglio, la derivata
\begin{equation}
  \parder{}{x^{\sigma}}
  \tensor*{T}{^{\alpha_{1}\dots\alpha_{p}}_{\beta_{1}\dots\beta_{q}}}
\end{equation}
è un quadritensore del tipo
$\tensor*{S}{^{\alpha_{1}\dots\alpha_{p}}_{\beta_{1}\dots\beta_{q}\sigma}}$ e la
derivata
\begin{equation}
  \parder{}{x_{\sigma}}
  \tensor*{T}{^{\alpha_{1}\dots\alpha_{p}}_{\beta_{1}\dots\beta_{q}}}
\end{equation}
è un quadritensore del tipo
$\tensor*{S}{^{\alpha_{1}\dots\alpha_{p}\sigma}_{\beta_{1}\dots\beta_{q}}}$.
Consideriamo il caso particolare di uno scalare $\phi$.  Dalla
trasformazione~\eqref{eq:trasformazione-inversa} abbiamo
\begin{equation}
  \parder{x^{\beta}}{x'^{\alpha}} = \tensor{\Lambda}{_{\alpha}^{\beta}}
\end{equation}
quindi
\begin{equation}
  \parder{\phi}{x^{\alpha}} \to \parder{\phi}{x'^{\alpha}}
  = \parder{\phi}{x^{\beta}} \parder{x^{\beta}}{x'^{\alpha}} =
  \tensor{\Lambda}{_{\alpha}^{\beta}} \parder{\phi}{x^{\beta}}.
\end{equation}
Abbiamo così mostrato che la derivata di uno scalare (quadritensore di rango 0)
rispetto a una coordinata controvariante è un quadrivettore (quadritensore di
rango 1) covariante
\begin{equation}
  \parder{\phi}{x^{\alpha}} = U_{\alpha}.
\end{equation}

Definiamo l'operatore
\index{quadrigradiente!covariante}\emph{quadrigradiente covariante} come
\begin{equation}
  \partial_{\alpha} = \parder{}{x^{\alpha}} = \Big( \parder{}{t}, \nabla\Big)
\end{equation}
e l'operatore
\index{quadrigradiente!controvariante}\emph{quadrigradiente controvariante} come
\begin{equation}
  \partial^{\alpha} = \parder{}{x_{\alpha}} =
  \tensor{\eta}{^{\alpha\beta}} \partial_{\beta} = \Big( -\parder{}{t},
  \nabla\Big).
\end{equation}
Infine introduciamo l'operatore \index{d'Alambertiano}\emph{d'Alambertiano}
definito da
\begin{equation}
  \Box^{2} = \partial^{\alpha}\partial_{\alpha} = - \parder{}{*2t} + \nabla^{2}.
\end{equation}
Poiché il d'Alambertiano è il modulo quadro del quadrigradiente, esso è un
operatore differenziale scalare.

\subsection{Quadritensori invarianti}
\label{sec:tensori-invarianti}

Ci sono dei particolari quadritensori che assumono la stessa forma in qualsiasi
sistema di riferimento inerziale, così come fanno gli scalari.

\subsubsection{Tensore metrico di Minkowski}
\label{sec:tensore-metrico}

Il primo quadritensore che analizzeremo è il tensore metrico di Minkowski.
Dalla relazione fondamentale del tensore
metrico~\eqref{eq:rel-fond-tensore-metrico} abbiamo che
$\tensor{\eta}{_{\alpha\beta}}$ è effettivamente di un quadritensore covariante
di rango $2$ che non cambia per effetto di una trasformazione di Lorentz
\begin{equation}
  \tensor{\eta}{_{\alpha\beta}} \to \tensor{{\eta'}}{_{\alpha\beta}} =
  \tensor{\Lambda}{_{\alpha}^{\gamma}} \tensor{\Lambda}{_{\beta}^{\delta}}
  \tensor{\eta}{_{\gamma\delta}} = \tensor{\eta}{_{\alpha\beta}}.
\end{equation}
In maniera analoga si può dimostrare che $\tensor{\eta}{^{\alpha\beta}}$ è
effettivamente un quadritensore controvariante di rango $2$ e che rimane
immutato sotto una trasformazione di Lorentz.

\subsubsection{Tensore di Levi-Civita}
\label{sec:tensore-levi-civita}

Il \index{tensore!di Levi-Civita}\emph{tensore di Levi-Civita}
$\epsilon^{\alpha\beta\gamma\delta}$ è un quadritensore definito da
\begin{equation}
  \epsilon^{\alpha\beta\gamma\delta} =
  \begin{cases}
    +1 & \text{se $\alpha\beta\gamma\delta$ è una permutazione pari di $0123$,}
    \\
    -1 & \text{se $\alpha\beta\gamma\delta$ è una permutazione dispari di
      $0123$,} \\
    0 & \text{altrimenti.}
  \end{cases}
\end{equation}
Si può far vedere che il tensore così definito non varia cambiando sistema di
riferimento inerziale.

\subsubsection{Tensore nullo}
\label{sec:tensore-nullo}

Dati due arbitrari quadritensore dello stesso tipo
$\tensor*{A}{^{\alpha_{1}\dots\alpha_{p}}_{\beta_{1}\dots\beta_{q}}}$ e
$\tensor*{B}{^{\alpha_{1}\dots\alpha_{p}}_{\beta_{1}\dots\beta_{q}}}$, se in un
certo sistema di riferimento inerziale risulta
\begin{equation}
  \label{eq:A-uguale-B}
  \tensor*{A}{^{\alpha_{1}\dots\alpha_{p}}_{\beta_{1}\dots\beta_{q}}} =
  \tensor*{B}{^{\alpha_{1}\dots\alpha_{p}}_{\beta_{1}\dots\beta_{q}}}
\end{equation}
è possibile definire un quadritensore
$\tensor*{Z}{^{\alpha_{1}\dots\alpha_{p}}_{\beta_{1}\dots\beta_{q}}}$, detto
\index{tensore!nullo}\emph{tensore nullo}, invariante per trasformazioni di
Lorentz come la differenza fra i due quadritensori:
\begin{equation}
  \tensor*{Z}{^{\alpha_{1}\dots\alpha_{p}}_{\beta_{1}\dots\beta_{q}}} =
  \tensor*{A}{^{\alpha_{1}\dots\alpha_{p}}_{\beta_{1}\dots\beta_{q}}} -
  \tensor*{B}{^{\alpha_{1}\dots\alpha_{p}}_{\beta_{1}\dots\beta_{q}}}.
\end{equation}
Dalla~\eqref{eq:A-uguale-B} ricaviamo che tutte le componenti di
$\tensor*{Z}{^{\alpha_{1}\dots\alpha_{p}}_{\beta_{1}\dots\beta_{q}}}$ sono
nulle, quindi anche il quadritensore
$\tensor*{{Z'}}{^{\alpha_{1}\dots\alpha_{p}}_{\beta_{1}\dots\beta_{q}}} =
\tensor*{{A'}}{^{\alpha_{1}\dots\alpha_{p}}_{\beta_{1}\dots\beta_{q}}} -
\tensor*{{B'}}{^{\alpha_{1}\dots\alpha_{p}}_{\beta_{1}\dots\beta_{q}}}$
ottenuto dalla trasformazione di Lorentz
\begin{equation}
  \tensor*{{Z'}}{^{\alpha_{1}\dots\alpha_{p}}_{\beta_{1}\dots\beta_{q}}} =
  \tensor{\Lambda}{^{\alpha_{1}}_{\mu_{1}}} \cdots
  \tensor{\Lambda}{^{\alpha_{p}}_{\mu_{p}}}
  \tensor{\Lambda}{_{\beta_{1}}^{\nu_{1}}} \cdots
  \tensor{\Lambda}{_{\beta_{q}}^{\nu_{q}}}
  \tensor*{Z}{^{\mu_{1}\dots\mu_{p}}_{\nu_{1}\dots\nu_{q}}}
\end{equation}
avrà tutte le componenti nulle, così che nel nuovo sistema di riferimento
inerziale risulta anche
\begin{equation}
  \tensor*{{A'}}{^{\alpha_{1}\dots\alpha_{p}}_{\beta_{1}\dots\beta_{q}}} =
  \tensor*{{B'}}{^{\alpha_{1}\dots\alpha_{p}}_{\beta_{1}\dots\beta_{q}}}.
\end{equation}
Questa proprietà del tensore nullo è molto importante poiché ci dice che
\emph{se due quadritensori, dello stesso tipo, sono uguali in un sistema di
  riferimento inerziale, saranno uguali in qualsiasi sistema di riferimento
  inerziale ottenuto dal primo mediante una trasformazione di Lorentz}.
Da questa affermazione deduciamo inoltre che\label{invarianza-lorentz}
\emph{qualsiasi legge fisica espressa come un'uguaglianza fra quadritensori è
  automaticamente invariante per trasformazioni di Lorentz}.

\section{Dinamica particellare}
\label{sec:dinamica-particellare}

Data una particella il cui moto è descritto dalla coordinata controvariante
$x^{\alpha}$, definiamo la sua \index{quadrivelocità}\emph{quadrivelocità} come
la derivata di $x^{\alpha}$ rispetto al tempo proprio $\tau$:
\begin{equation}
  v^{\alpha} = (v^{0}, v^{i}) =\toder{x^{\alpha}}{\tau}
\end{equation}
e poiché risulta
\begin{equation}
  \dd \tau = \sqrt{\dd t^{2} - \dd \bm{x}^{2}} = \dd t \sqrt{1 -
    \Big(\toder{\bm{x}}{t}\Big)^{2}} = \dd t\sqrt{1 - v^{2}} = \frac{\dd
    t}{\gamma(v)}
\end{equation}
abbiamo che la quadrivelocità è uguale a
\begin{equation}
  \begin{split}
    v^{\alpha} &= \toder{x^{\alpha}}{t}\toder{t}{\tau} = \toder{x^{\alpha}}{t}
    \gamma(v) = \gamma(v) \Big(\toder{x^{0}}{t}, \toder{\bm{x}}{t}\Big) =
    \gamma(v)(1,\bm{v}) \\
    &= (\gamma(v), \gamma(v)\bm{v}).
  \end{split}
\end{equation}
Il \index{quadrimomento}\emph{quadrimomento} della particella è definito come il
prodotto fra la massa a riposo $m$ della particella e la quadrivelocità
\begin{equation}
  p^{\alpha} = (p^{0},p^{i}) = m v^{\alpha} = m (v^{0}, v^{i}) = (m\gamma(v),
  m\gamma(v)\bm{v}).
\end{equation}
Ricordando che l'energia totale $E$ di una particella è definita come $m\gamma(v)$
troviamo che la prima componente del quadrivettore momento è proprio l'energia
totale, mentre le componenti $p^{i}$ sono le componenti del momento
$\bm{p} = m\gamma(v)\bm{v}$
\begin{equation}
  p^{\alpha} = (E, \bm{p}).
\end{equation}
Vediamo come cambia il quadrimomento per effetto di un boost $\bm{V}$
\begin{equation}
  p^{\alpha} \to p'^{\alpha} = (p'^{0}, p'^{i}) =
  \tensor{\Lambda}{^{\alpha}_{\beta}} p^{\beta} =
  (\tensor{\Lambda}{^{0}_{\beta}}p^{\beta},
  \tensor{\Lambda}{^{i}_{\beta}}p^{\beta}).
\end{equation}
La prima componente ci dà l'energia $E'$ nel nuovo sistema di riferimento
\begin{equation}
  \begin{split}
    E' &= p'^{0} = \tensor{\Lambda}{^{0}_{\beta}}p^{\beta} =
    \tensor{\Lambda}{^{0}_{0}}p^{0} + \tensor{\Lambda}{^{0}_{i}}p^{i} \\
    &= \gamma(V) E + \gamma(V) V_{i}p^{i} = \gamma(V) E + \gamma(V)
    \bm{V}\cdot\bm{p}.
  \end{split}
\end{equation}
Le tre componenti spaziali sono le componenti $(\bm{p}')_{i}$ del vettore
momento nel nuovo sistema di riferimento
\begin{equation}
  \begin{split}
    (\bm{p}')_{i} &= (p')^{i} = \tensor{\Lambda}{^{i}_{\beta}}p^{\beta} =
    \tensor{\Lambda}{^{i}_{0}}p^{0} + \tensor{\Lambda}{^{j}_{j}}p^{j} \\
    &= \gamma(V) V_{i}E + \Big( \delta_{ij} +
    V_{i}V_{j}\frac{\gamma(V)-1}{V^{2}}\Big)p^{j} \\
    &= \gamma(V) V_{i}E + p_{i} +
    V_{i}V_{j}p^{j}\frac{\gamma(V)-1}{V^{2}} \\
    &= \gamma(V) V_{i}E + p_{i} + V_{i} \bm{V} \cdot \bm{p}
    \frac{\gamma(V)-1}{V^{2}},
  \end{split}
\end{equation}
cioè
\begin{equation}
  \bm{p}' = \bm{p} + \Big(\gamma(V) E + \bm{V} \cdot \bm{p}
  \frac{\gamma(V)-1}{V^{2}}\Big)\bm{V}.
\end{equation}
% NOTA: i risultati delle trasformazioni di momento ed energia sono differenti
% dalle formule [4.103] e [4.104], pagina 132, del Barone.  Weinberg definisce
% la velocità di spostamento fra i due sistemi di riferimento al contrario
% rispetto a quasi tutti gli altri autori (compreso il Barone).
Calcoliamo il modulo quadro del quadrimomento
\begin{equation}
  p^{\alpha}p_{\alpha} = \tensor{\eta}{_{\alpha\beta}}p^{\alpha}p^{\beta} =
  -E^{2} + \bm{p}^{2} = -m^{2}
\end{equation}
avendo usato la relazione $E^{2} = m^{2} + \bm{p}^{2}$.  Questo fa vedere che la
massa a riposo di una particella è uno scalare di Lorentz, quindi un invariante
relativistico.  Il quadrimomento covariante è dato da
\begin{equation}
  p_{\alpha} = \tensor{\eta}{_{\alpha\beta}}p^{\beta} = (-E, \bm{p}).
\end{equation}

La \index{quadriforza}\emph{quadriforza} agente sulla particella è definita come
la derivata rispetto al tempo proprio del quadrimomento
\begin{equation}
  f^{\alpha} = (f^{0},f^{i}) =\toder{p^{\alpha}}{\tau} =
  m\toder{v^{\alpha}}{\tau} = mw^{\alpha}
\end{equation}
in cui
\begin{equation}
  w^{\alpha} = (w^{0},w^{i}) =\toder{v^{\alpha}}{\tau}
\end{equation}
è la \index{quadriaccelerazione}\emph{quadriaccelerazione}.  In un sistema di
riferimento $\mathcal{S}$ in cui la particella è a riposo $\dd \bm{x} = \bm{0}$,
quindi $\dd \tau = \dd t$ e $f^{\alpha} = F^{\alpha}$, in cui $F^{i}$ sono le
componenti cartesiane della forza non relativistica $\bm{F}$ e $F^{0} = 0$:
\begin{equation}
  f^{\alpha} = (f^{0}, f^{i}) = (0, \bm{F}).
\end{equation}
Per calcolare la forza a cui è soggetta la particella in un altro sistema di
riferimento inerziale $\mathcal{S}'$ che vede la particella muoversi con
velocità $\bm{V} = (V_{1}, V_{2}, V_{3})$ dobbiamo effettuare un boost $\bm{V}$
\begin{equation}
  f'^{\alpha} = (f'^{0}, f'^{i}) = \tensor{\Lambda}{^{\alpha}_{\beta}}f^{\beta}.
\end{equation}
Abbiamo
\begin{subequations}
  \begin{align}
    \begin{split}
      f'^{0} &= \tensor{\Lambda}{^{0}_{\alpha}}f^{\alpha} =
      \tensor{\Lambda}{^{0}_{0}}f^{0} + \tensor{\Lambda}{^{0}_{i}}f^{i} =
      \gamma(V) V_{i}f^{i} = \gamma(V) \bm{V} \cdot \bm{f} \\
      &= \gamma(V) \bm{V} \cdot \bm{F}
    \end{split} \\
    \begin{split}
      f'^{i} &= \tensor{\Lambda}{^{i}_{\alpha}}f^{\alpha} =
      \tensor{\Lambda}{^{i}_{0}}f^{0} + \tensor{\Lambda}{^{i}_{j}}f^{j}
      = \Big( \delta_{ij} + V_{i}V_{j}\frac{\gamma(V) - 1}{V^{2}} \Big)f^{j} \\
      &= f^{i} + V_{i}V_{j}f^{j}\frac{\gamma(V) - 1}{V^{2}} = f^{i} + V_{i}
      \bm{V} \cdot \bm{f} \frac{\gamma(V) - 1}{V^{2}},
    \end{split}
  \end{align}
\end{subequations}
cioè
\begin{equation}
  \bm{f}' = \bm{F} + \bm{V}(\bm{V}\cdot\bm{F})\frac{\gamma(V) - 1}{V^{2}}.
\end{equation}
% Questa formula è diversa dalla [4.126] a pagina 135 del Barone perché lui
% parte da un sistema di riferimento in cui la particella si muove con velocità
% `v', noi invece siamo partiti da un sistema in cui la particella è a riposo.

\section{Formulazione covariante dell'elettrodinamica}
\label{sec:elettrodinamica-covariante}

In questo paragrafo vogliamo far vedere come il formalismo della relatività
speciale permetta di riscrivere le leggi dell'elettrodinamica, e in particolare
le equazioni di Maxwell, in modo manifestamente covariante (o invariante in
forma) rispetto alle trasformazioni di Lorentz.

\subsection{Densità di corrente e di caria}
\label{sec:densita-corrente-carica}

Consideriamo, per semplicità, una distribuzione di particelle in cui l'$n$-esima
particella ha carica $e_{n}$ e occupa, in un certo sistema di riferimento
tridimensionale, la posizione $\bm{x}_{n}$.  La densità di carica di questa
distribuzione di particelle al tempo $t$ è data da
\begin{equation}
  \rho(\bm{x},t) = \sum_{n} e_{n}\delta^{3}(\bm{x} - \bm{x}_{n}(t)),
\end{equation}
mentre la densità di corrente è
\begin{equation}
  \bm{J}(\bm{x},t) = \sum_{n} e_{n}\delta^{3}(\bm{x} - \bm{x}_{n}(t))
  \toder{\bm{x}_{n}(t)}{t}.
\end{equation}
Abbiamo qui utilizzato la \index{delta!di Dirac
  tridimensionale}\emph{delta
  di Dirac tridimensionale}
$\delta^{3}(\bm{x} - \bm{x}')$, cioè quella distribuzione che per ogni funzione
$f$ liscia di tre variabili reali ha la proprietà
\begin{equation}
  \int\limits_{\R^{3}} f(\bm{x})\delta^{3}(\bm{x} - \bm{x}') \dd^{3} x =
  f(\bm{x}').
\end{equation}
Possiamo mettere insieme le due quantità $\rho$ e $\bm{J}$ in un quadrivettore
$J^{\alpha}$ che chiameremo \index{quadricorrente}\emph{quadricorrente}
\begin{equation}
  J^{\alpha}(x) = (\rho, \bm{J}) = \sum_{n} e_{n}\delta^{3}(\bm{x} -
  \bm{x}_{n}(t)) \toder{x^{\alpha}}{t}.
\end{equation}
Introduciamo adesso la \index{delta!di Dirac
  quadrimensionale}\emph{delta
  di Dirac quadrimensionale}
$\delta^{4}(x - x') = \delta(t - t')\delta^{3}(\bm{x} - \bm{x}')$, vale a dire
la distribuzione che per ogni funzione $f$ liscia con dominio in $\M$ soddisfa
la proprietà
\begin{equation}
  \int\limits_{\M} f(x)\delta^{4}(x - y)\dd^{4} x = f(y).
\end{equation}
Moltiplicando $J^{\alpha}(x)$ per la delta di Dirac $\delta(t - t')$ e
integrando sul tempo ricaviamo la seguente espressione
\begin{equation}
  \begin{split}
    J^{\alpha}(x) &= \sum_{n} e_{n}\delta^{3}(\bm{x} - \bm{x}_{n}(t))
    \toder{x^{\alpha}}{t} \\
    &= \int_{-\infty}^{+\infty}\dd t' \delta(t - t') \sum_{n}
    e_{n}\delta^{3}(\bm{x} - \bm{x}_{n}(t')) \toder{x^{\alpha}}{t'} \\
    &= \int_{-\infty}^{+\infty} \dd \tau \sum_{n} e_{n} \delta^{4}(x -
    x_{n}(\tau)) \toder{x^{\alpha}}{\tau}.
  \end{split}
\end{equation}
Nell'ultimo passaggio abbiamo effettuato il cambiamento della variabile di
integrazione da $t'$ all'invariante $\tau$.  Da questa espressione è evidente
che $J^{\alpha}$ è un quadrivettore: $\dd \tau$ è uno scalare, così come la
delta di Dirac $\delta^{4}(x - x_{n}(\tau))$, mentre $\ltoder{x^{\alpha}}{\tau}$
è la quadrivelocità.

In elettrodinamica classica si dimostra l'equazione di continuità
\begin{equation}
  \nabla\cdot\bm{J} + \parder{\rho}{t} = 0
\end{equation}
che, usando il formalismo relativistico, può essere riscritta in termini della
quadricorrente come
\begin{equation}
  \partial_{\alpha}J^{\alpha} = 0.
\end{equation}
Applicando la conseguenza del teorema di Gauss nello spazio di Minkowski
discussa nell'appendice~\ref{cha:teorema-gauss} alla quadricorrente abbiamo che
la quantità
\begin{equation}
  Q = \int J^{0}\dd V = \int \rho \dd V
\end{equation}
è costante rispetto al tempo se l'integrale è esteso a una regione abbastanza
grande dello spazio in modo da racchiudere tutte le cariche del sistema e che
sul suo bordo la quadricorrente sia nulla.  Poiché $\rho$ è la densità di
carica, la quantità $Q$ rappresenta la carica totale del sistema, pertanto
dall'equazione di continuità discende, grazie al teorema di Gauss, la
conservazione della carica.

\subsection{Le equazioni di Maxwell}
\label{sec:equazioni-maxwell}

Ricordiamo le equazioni di Maxwell
\begin{subequations}
  \begin{align}
    \nabla \cdot \bm{E} &= \rho, \label{eq:maxwell-divE} \\
    \nabla \times \bm{B} &= \bm{J} + \parder{\bm{E}}{t}, \label{eq:maxwell-rotB}
    \\
    \nabla \cdot \bm{B} &= 0, \label{eq:maxwell-divB} \\
    \nabla \times \bm{E} &= -\parder{\bm{B}}{t}, \label{eq:maxwell-rotE}
  \end{align}
\end{subequations}
in cui i campi elettrico $\bm{E}$ e magnetico $\bm{B}$ sono legati ai potenziali
scalare $\phi$ e vettore $\bm{A}$ dalle relazioni
\begin{subequations}
  \label{eq:campi-potenziali}
  \begin{align}
    \bm{E} &= -\nabla\phi - \parder{\bm{A}}{t}, \\
    \bm{B} &= \nabla \times \bm{A}.
  \end{align}
\end{subequations}
Possiamo mettere insieme i due potenziali per formare un quadrivettore $A^{\mu}$
(controvariante) chiamato
\index{quadripotenziale
  elettromagnetico}\emph{quadripotenziale elettromagnetico}
\begin{equation}
  A^{\mu} = (A^{0}, A^{i}) = (\phi, \bm{A}).
\end{equation}
Inoltre il quadripotenziale covariante è dato, come al solito, da
\begin{equation}
  A_{\mu} = \eta_{\mu\nu}A^{\nu} = (-A^{0}, A^{i}) = (-\phi, \bm{A}).
\end{equation}
Introduciamo
il \index{tensore!del campo
  elettromagnetico}\emph{tensore
  del campo elettromagnetico} $F^{\mu\nu}$ definito dalla relazione
\begin{equation}
  F_{\alpha\beta} = \partial_{\alpha}A_{\beta} - \partial_{\beta}A_{\alpha}.
\end{equation}
Si può verificare che $F_{\alpha\beta}$ è un quadritensore di rango $2$
completamente covariante, inoltre esso è antisimmetrico, infatti
\begin{equation}
  F_{\beta\alpha} = \partial_{\beta}A_{\alpha} - \partial_{\alpha}A_{\beta} =
  -F_{\alpha\beta}.
\end{equation}
Di conseguenza le componenti diagonali $F_{\alpha\alpha}$ (senza sommare) sono
nulle e le componenti indipendenti di questo quadritensore sono sei.
Verifichiamo che queste componenti indipendenti corrispondono proprio alle sei
componenti cartesiane dei campi $\bm{E}$ e $\bm{B}$.
Dalle~\eqref{eq:campi-potenziali} abbiamo
\begin{subequations}
  \begin{align}
    F_{0i} &= \partial_{0}A_{i} - \partial_{i}A_{0} = \parder{A_{i}}{t}
    + \parder{\phi}{x^{i}} = -E_{i}, \\
    F_{12} &= \partial_{1}A_{2} - \partial_{2}A_{1} = \parder{A_{y}}{x}
    - \parder{A_{x}}{y} = B_{z}, \\
    F_{13} &= \partial_{1}A_{3} - \partial_{3}A_{1} = \parder{A_{z}}{x}
    - \parder{A_{x}}{z} = -B_{y}, \\
    F_{23} &= \partial_{2}A_{3} - \partial_{3}A_{2} = \parder{A_{z}}{y}
    - \parder{A_{y}}{z} = B_{x}.
  \end{align}
\end{subequations}
L'espressione matriciale di $F_{\alpha\beta}$ è
\begin{equation}
  F_{\alpha\beta} =
  \begin{pmatrix}
    0     & -E_{x} & -E_{y} & -E_{z} \\
    E_{x} & 0      & B_{z}  & -B_{y} \\
    E_{y} & -B_{z} & 0      & B_{x}  \\
    E_{z} & B_{y}  & -B_{x} & 0
  \end{pmatrix}.
\end{equation}
Il tensore elettromagnetico completamente controvariante è dato da
$F^{\alpha\beta} = \eta^{\alpha\gamma}\eta^{\beta\delta}F_{\gamma\delta}$ e
risulta quindi $F^{0i} = -F_{0i}$, $F^{i0} = -F_{i0}$ e $F^{ij} = F_{ij}$.
Usando le componenti del tensore elettromagnetico si può vedere come le
componenti dei campi $\bm{E}$ e $\bm{B}$ si modificano per effetto di una
trasformazione di Lorentz.

A partire dal tensore del campo elettromagnetico è possibile costruire le due
quantità invarianti (essendo contrazioni complete di quadritensori)
\begin{subequations}
  \begin{align}
    F^{\alpha\beta}F_{\alpha\beta} &= -2(\bm{E}^{2} - \bm{B}^{2}) = 2(\bm{B}^{2}
    - \bm{E}^{2}), \\
    \epsilon_{\alpha\beta\gamma\delta}F^{\alpha\beta}F^{\gamma\delta} &= -2
    \bm{E} \cdot \bm{B}.
  \end{align}
\end{subequations}
Mentre $F^{\alpha\beta}F_{\alpha\beta}$ è uno scalare di Lorentz, la quantità
$\epsilon_{\alpha\beta\gamma\delta}F^{\alpha\beta}F^{\gamma\delta}$ è uno
pseudoscalare perché è il prodotto vettoriale del vettore polare $\bm{E}$ e del
vettore assiale $\bm{B}$.

Grazie al tensore del campo elettromagnetico $F^{^{\alpha\beta}}$ e alla
quadricorrente $J^{\alpha}$ possiamo scrivere le quattro equazioni di Maxwell
in forma covariante
\begin{subequations}
  \begin{align}
    \partial_{\alpha}F^{\alpha\beta} &= -J^{\beta}, \label{eq:maxwell-cov1} \\
    \epsilon^{\alpha\beta\gamma\delta} \partial_{\beta} F_{\gamma\delta} &=
    0. \label{eq:maxwell-cov2}
  \end{align}
\end{subequations}
La prima di queste equazioni equivale per $\beta=0$ alla~\eqref{eq:maxwell-divE}
e per $\beta = i$ alle tre componenti della~\eqref{eq:maxwell-rotB}, la seconda
delle equazioni precedenti equivale per $\alpha = 0$
alla~\eqref{eq:maxwell-divB} e per $\beta = i$ alle tre componenti
della~\eqref{eq:maxwell-rotE}.  Il fatto che abbiamo scritto le equazioni di
Maxwell sotto forma di uguaglianza fra quadrivettori ci dice immediatamente che
queste sono invarianti per trasformazioni di Lorentz, per quanto spiegato a
pagina~\pageref{invarianza-lorentz}.  L'equazione
omogenea~\eqref{eq:maxwell-cov2} può essere espressa anche come
\begin{equation}
  \partial_{\alpha}F_{\beta\gamma} + \partial_{\beta}F_{\gamma\alpha}
  + \partial_{\gamma}F_{\alpha\beta} = 0.
\end{equation}

Il tensore $F_{\alpha\beta}$ del campo elettromagnetico è invariante per
trasformazioni di gauge del quadripotenziale del tipo
$A_{\alpha}(x) \to A'_{\alpha}(x) = A_{\alpha}(x) + \partial_{\alpha}\chi(x)$,
in cui $\chi(x)$ è una funzione scalare delle coordinate.  Infatti
\begin{equation}
  \begin{split}
    F_{\alpha\beta} \to F'_{\alpha\beta} &= \partial_{\alpha}A'_{\beta}
    - \partial_{\beta}A'_{\alpha} = \partial_{\alpha}(A_{\beta}
    + \partial_{\beta}\chi) - \partial_{\beta}(A_{\alpha}
    + \partial_{\alpha}\chi) \\
    &= \partial_{\alpha}A_{\beta} - \partial_{\beta}A_{\alpha} =
    F_{\alpha\beta}.
  \end{split}
\end{equation}
Pertanto i campi elettrico e magnetico non cambiano per effetto di una tale
trasformazione, come noto dall'elettrodinamica classica.

L'equazione di Maxwell non omogenea in forma covariante~\eqref{eq:maxwell-cov1}
può anche essere scritta come $\partial^{\alpha}F_{\alpha\beta} = -J_{\beta}$ ed
esplicitando il tensore del campo elettromagnetico abbiamo
\begin{equation}
  \partial^{\alpha}(\partial_{\alpha}A_{\beta} - \partial_{\beta}A_{\alpha})
  = \partial_{\alpha}\partial^{\alpha}A_{\beta}
  - \partial_{\beta}\partial^{\alpha}A_{\alpha} = -J_{\beta}.
\end{equation}
Adottando la \index{gauge!di Lorenz}gauge di Lorenz
$\partial^{\alpha}A_{\alpha} = 0$ otteniamo
l'\index{equazione!di d'Alambert}equazione di d'Alambert non omogenea
\begin{equation}
  \partial^{\alpha}\partial_{\alpha}A_{\beta} = \Box^{2}A_{\beta} = \Big(
  -\parder{}{*{2}{t}} + \nabla^{2} \Big) A_{\beta} = -J_{\beta}.
\end{equation}

A titolo di informazione diciamo che le equazioni di Maxwell che descrivono il
campo elettromagnetico possono essere ricavate applicando il principio
variazionale all'azione di un sistema composto da un campo elettromagnetico e
delle particelle che interagiscono con esso.  L'azione $S$ di questo sistema è
formata da tre parti
\begin{equation}
  S = S_{\textup{p}} + S_{\textup{pc}} + S_{\textup{c}}.
\end{equation}
Il termine $S_{\textup{p}}$ dipende solo dalle proprietà delle particelle,
quindi è l'azione delle particelle libere ciascuna di massa $m_{n}$ e vale
$-\sum_{n} m_{n}\int\dd \tau$.  $S_{\textup{pc}}$ è la parte di azione che
descrive l'interazione fra le particelle del sistema e il campo elettromagnetico
e vale $-\sum_{n} q_{n}\int A_{\alpha}\dd x^{\alpha}$ per un sistema di
particelle puntiformi di carica $q_{n}$, $-\int J^{\alpha}A_{\alpha}\dd^{4} x$
se si considera una distribuzione continua di carica.  Infine, $S_{\textup{c}}$
dipende solo dalle proprietà del campo, cioè è l'azione in assenza di cariche, e
vale $-1/(16\pi) \int F^{\alpha\beta}F_{\alpha\beta}\dd^{4} x$, così che
l'azione complessiva è
\begin{equation}
  S = -\sum_{n} m_{n}\int \dd\tau - \int J^{\alpha}A_{\alpha} \dd^{4} x
  -\frac{1}{16\pi} \int F^{\alpha\beta}F_{\alpha\beta} \dd^{4} x.
\end{equation}
Per determinare le equazioni di Maxwell bisogna imporre $\delta S = 0$ trattando
le componenti $A_{\alpha}$ del quadripotenziale come coordinate generalizzate.

\subsection{Il tensore energia-impulso}
\label{sec:tensore-energia-impulso}



%%% Local Variables:
%%% mode: latex
%%% TeX-master: "../astrofisica-teorica"
%%% fill-column: 80
%%% End:
