\chapter{Richiami di relatività speciale}
\label{cha:relativita-speciale}

\section{Trasformazioni di Lorentz}
\label{sec:trasformazioni-lorentz}

Per il \index{principio!di relatività}\emph{principio di relatività} tutte le
leggi fisiche assumono la stessa forma in qualsiasi sistema di riferimento
inerziale.  In particolare, all'interno della teoria della relatività speciale
le leggi fisiche devono essere invarianti per trasformazioni di Lorentz.  In
questo capitolo dedurremo le trasformazioni di Lorentz e richiameremo alcuni
concetto di relatività speciale.

L'ambiente in cui si verificano i fenomeni fisici descritti dalla relatività
speciale è uno spazio quadrimensionale piatto, composto da una coordinata
temporale e tre spaziali, chiamato \index{spazio!di
  Minkowski}\emph{spazio
  di Minkowski}
e che indicheremo con il simbolo $\numberset{M}$.  Un elemento dello spazio di
Minkowski è chiamato \index{evento}\emph{evento}.  Lo spazio di Minkowski si
distingue dal corrispondente spazio euclideo quadrimensionale $\R^{4}$ e anche
dallo spazio ordinario tridimensionale $\R^{3}$ in quanto utilizza una metrica
differente da quelle usualmente adottate in $\R^{3}$ ed $\R^{4}$.

Nello spazio ordinario $\R^{3}$ si definisce la metrica, ovvero la distanza fra
due punti infinitesimamente vicini di coordinate
$\bm{x} = (x,y,z) \equiv (x_{1},x_{2},x_{3})$ e
$\bm{x} + \dd \bm{x} = (x+\dd x,y+\dd y,z+\dd z) \equiv (x_{1}+\dd
x_{1},x_{2}+\dd x_{2},x_{3}+\dd x_{3})$ come
\begin{equation}
  \dd\bm{x}^{2} = (\dd x_{1})^{2} + (\dd x_{2})^{2} + (\dd x_{3})^{2} =
  \sum_{i=1}^{3} \sum_{j=1}^{3} \delta_{ij} \dd x_{i} \dd x_{j},
\end{equation}
dove $\delta_{ij}$ è il tensore metrico dato da
\begin{equation}
  \delta_{ij} = \diag (+1,+1,+1) =
  \begin{pmatrix}
    +1 & 0  & 0  \\
    0  & +1 & 0  \\
    0  & 0  & +1
  \end{pmatrix}.
\end{equation}
Analogamente, nello spazio $\R^{4}$ si utilizza il tensore metrico rappresentato
da $\diag (+1,+1,+1,+1)$.  Una generica trasformazione lineare omogenea delle
coordinate $(x,y,z) \equiv (x_{1},x_{2},x_{3})$ che lasci invariate la metrica,
cioè che soddisfi la condizione $\dd\bm{x}^{2} = \dd\bm{x}'^{2}$, può essere
scritta
come
\begin{equation}
  x'_{i} = \sum_{j=1}^{3} R_{ij}x_{j}, \qquad i,j = 1,2,3,
\end{equation}
con $R_{ij}$ costanti reali, e la matrice di componenti $R_{ij}$ è una matrice
di rotazione.

Nello spaziotempo di Minkowski definiamo le
\index{coordinata controvariante}\emph{coordinate controvarianti} di un evento
che avviene al tempo $t$ nel punto dello spazio tridimensionale $\bm{x}$ come
\begin{equation}
  \label{eq:coord-controvariante}
  x^{\alpha} = (x^{0},x^{i}) = (t, \bm{x}),
\end{equation}
con $\alpha$ che varia fra $0$, $1$, $2$ e $3$ e $i$ fra $1$, $2$ e $3$.  La
\index{metrica!nello spazio di Minkowski}\emph{metrica} è data da
\begin{equation}
  \dd \tau^{2} = \dd t^{2} - \dd \bm{x}^{2} = -\tensor{\eta}{_{\alpha\beta}}\dd
  x^{\alpha} \dd x^{\beta},
\end{equation}
in cui $\tensor{\eta}{_{\alpha\beta}}$ è il
\index{tensore!metrico covariante di
  Minkowski}\emph{tensore metrico covariante di Minkowski}
\begin{equation}
  \label{eq:tensore-metrico-controvariante}
  \tensor{\eta}{_{\alpha\beta}} =
  \begin{pmatrix}
   \eta_{00} & \eta_{01} & \eta_{02} & \eta_{03} \\
   \eta_{10} & \eta_{11} & \eta_{12} & \eta_{13} \\
   \eta_{20} & \eta_{21} & \eta_{22} & \eta_{23} \\
   \eta_{30} & \eta_{31} & \eta_{32} & \eta_{33}
  \end{pmatrix}
  =
  \begin{pmatrix}
    -1 & 0  & 0  & 0 \\
    0  & +1 & 0  & 0 \\
    0  & 0  & +1 & 0 \\
    0  & 0  & 0  & +1
  \end{pmatrix}.
\end{equation}
Questo tensore definisce una metrica pseudo euclidea con segnatura $(-+++)$,
invece in $\R^{4}$ abbiamo visto che si utilizza la segnatura $(++++)$.  Una
\index{trasformazione!di Lorentz}\emph{trasformazione di Lorentz} non omogenea è
una trasformazione lineare nello spazio di Minkowski da un sistema di
riferimento di coordinate $x^{\alpha}$ a un altro sistema di riferimento di
coordinate
\begin{equation}
  \label{eq:trasformazione-lineare}
  x'^{\alpha} = \tensor{\Lambda}{^{\alpha}_{\beta}}x^{\beta} + a^{\alpha},
\end{equation}
con $a^{\alpha}$ costanti, che mantenga invariata la distanza fra due eventi,
cioè la metrica $\dd\tau^{2} = -\eta_{\alpha\beta}\dd x^{\alpha}\dd x^{\beta}$
in $\numberset{M}$.  Richiediamo che la trasformazione
$x^{\alpha} \to x'^{\alpha}$ sia lineare per rispettare la condizione di
omogeneità dello spazio (la lunghezza di una sbarra non deve dipendere dalla
posizione degli estremi), del tempo (la lunghezza della sbarra non deve
dipendere dall'istante di tempo in cui la misuriamo) e di isotropia dello spazio
(la lunghezza della sbarra non deve dipendere neanche dalla direzione in cui la
guardiamo).  Inoltre, in questo modo un moto rettilineo uniforme in un sistema
di riferimento inerziale risulta è ancora un moto rettilineo uniforme anche
nell'altro sistema di riferimento inerziale.  La richiesta di invarianza della
metrica, invece, permette di soddisfare il principio di relatività.  I
differenziali delle coordinate nel nuovo sistema si ricavano
dalla~\eqref{eq:trasformazione-lineare}
\begin{equation}
  \label{eq:trasformazione-differenziali}
  \dd x'^{\alpha} = \tensor{\Lambda}{^{\alpha}_{\beta}}x^{\beta}.
\end{equation}
Affinché una trasformazione lineare~\eqref{eq:trasformazione-lineare} delle
coordinate sia di Lorentz, cioè affinché conservi la metrica, deve valere
\begin{equation}
  \begin{split}
    \dd \tau'^{2} &= -\tensor{\eta}{_{\alpha\beta}}\dd x'^{\alpha}\dd x'^{\beta}
    = -\tensor{\eta}{_{\alpha\beta}} \tensor{\Lambda}{^{\alpha}_{\gamma}} \dd
    x^{\gamma} \tensor{\Lambda}{^{\beta}_{\delta}} \dd x^{\delta} \\
    &= \dd \tau^{2} = -\tensor{\eta}{_{\gamma\delta}} \dd x^{\gamma} \dd x^{\delta},
  \end{split}
\end{equation}
da cui, semplificando nel terzo e nel quinto membro
$\dd x^{\gamma}\dd x^{\delta}$, otteniamo la relazione fondamentale del tensore
metrico
\begin{equation}
  \label{eq:rel-fond-tensore-metrico}
  \tensor{\eta}{_{\alpha\beta}} \tensor{\Lambda}{^{\alpha}_{\gamma}}
  \tensor{\Lambda}{^{\beta}_{\delta}} = \tensor{\eta}{_{\gamma\delta}}.
\end{equation}
Questa relazione impone $10$ condizioni indipendenti in quanto il tensore
metrico $\tensor{\eta}{_{\gamma\delta}}$ è simmetrico quindi solo $10$ delle sue
$16$ componenti sono indipendenti.

Indicando con $\Lambda$ la matrice $4 \times 4$ i cui elementi sono le
$\tensor{\Lambda}{^{\alpha}_{\beta}}$
\begin{equation}
  \Lambda =
  \begin{pmatrix}
    \tensor{\Lambda}{^{0}_{0}} & \tensor{\Lambda}{^{0}_{1}} &
    \tensor{\Lambda}{^{0}_{2}} & \tensor{\Lambda}{^{0}_{3}} \\
    \tensor{\Lambda}{^{1}_{0}} & \tensor{\Lambda}{^{1}_{1}} &
    \tensor{\Lambda}{^{1}_{2}} & \tensor{\Lambda}{^{1}_{3}} \\
    \tensor{\Lambda}{^{2}_{0}} & \tensor{\Lambda}{^{2}_{1}} &
    \tensor{\Lambda}{^{2}_{2}} & \tensor{\Lambda}{^{2}_{3}} \\
    \tensor{\Lambda}{^{3}_{0}} & \tensor{\Lambda}{^{3}_{1}} &
    \tensor{\Lambda}{^{3}_{2}} & \tensor{\Lambda}{^{3}_{3}}
  \end{pmatrix},
\end{equation}
l'equazione matriciale equivalente alla~\eqref{eq:rel-fond-tensore-metrico} è
\begin{equation}
  \Lambda^{\textup{T}}\eta\Lambda = \eta
\end{equation}
dalla quale abbiamo
\begin{equation}
  \det\eta = \det(\Lambda^{\textup{T}}\eta\Lambda) = \det\eta
  \det(\Lambda^{\textup{T}}\Lambda) = \det\eta (\det\Lambda)^{2}
\end{equation}
e quindi deve valere la condizione
\begin{equation}
  \det\Lambda = \pm 1.
\end{equation}
Le trasformazioni di Lorentz con $\det\Lambda = +1$ sono chiamate
\emph{proprie}, quelle con $\det\Lambda = -1$ sono chiamate \emph{improprie}.

\begin{figure}
  \centering
  \tdplotsetmaincoords{70}{100}
  \begin{tikzpicture}[tdplot_main_coords,font=\footnotesize,scale=3]
    \node[shape=coordinate] (O1) at (0,0,0) [label=above right:$O$] {};
    \node[shape=coordinate] (O2) at (-1,0.5,0.3) [label=above right:$O'$] {};
    \draw[->] (O1) -- ++(1,0,0) node[anchor=north east] {$x$}; % asse x
    \draw[->] (O1) -- ++(0,1,0) node[anchor=north west] {$y$}; % asse y
    \draw[->] (O1) -- ++(0,0,1) node[anchor=south] (Z1) {$z$}; % asse z
    \node at (Z1) [label=below right:$\mathcal{S}$] {};% sistema di riferimento
    \draw[->] (O2) -- ++(1,0,0) node[anchor=north east] {$x'$}; % asse x'
    \draw[->] (O2) -- ++(0,1,0) node[anchor=north west] {$y'$}; % asse y'
    \draw[->] (O2) -- ++(0,0,1) node[anchor=south] (Z2) {$z'$}; % asse z'
    \node at (Z2) [label=below right:$\mathcal{S}'$] {};% sistema di riferimento
    \draw[->,thick] (O1) -- node[above] {$\bm{v}$} ++(0.3,-0.4,0.3); % velocità
  \end{tikzpicture}
  \caption[Sistemi di riferimento inerziali in moto l'uno rispetto
  all'altro]{Sistemi di riferimento inerziali, con gli assi corrispondenti
    paralleli, in moto l'uno rispetto all'altro}
  \label{fig:trasformazioni-lorentz}
\end{figure}
Esempi di trasformazioni di Lorentz proprie sono le rotazioni spaziali, che
lasciano invariato il tempo, e le \emph{trasformazioni di velocità}, o
\index{boost}\emph{boost}, vale a dire le trasformazioni tra due sistemi di
riferimento inerziali con gli assi cartesiani corrispondenti paralleli e in moto
l'uno rispetto all'altro con velocità costante.
%%%%% TODO: controllare la validità di questa affermazione:
La composizione di due trasformazioni di Lorentz è ancora una
trasformazione di Lorentz.
%%%%% Fine affermazione da controllare.
Vediamo come si trasformano le coordinate $x^{\alpha}$ di un evento in un
sistema di riferimento per effetto di un boost $\bm{v}$.  Consideriamo una
particella a riposo in un sistema di riferimento $\mathcal{S}$ che si muove con
velocità $\bm{v} = (v_{1},v_{2},v_{3})$ rispetto a un altro sistema di
riferimento inerziale $\mathcal{S}'$, come nella
figura~\ref{fig:trasformazioni-lorentz}.\footnote{Si noti che il sistema di
  riferimento $\mathcal{S}$ vede il sistema $\mathcal{S}'$ muoversi con velocità
  $-\bm{v}$.  Numerosi autori considerano il sistema di riferimento
  $\mathcal{S}'$ in moto rispetto al sistema $\mathcal{S}$ con velocità
  $\bm{v}$, questo comporterà una differenza di segno in numerose equazioni
  rispetto a quelle ricavate da questi autori.}
Poiché nel sistema $\mathcal{S}$ per la particella risulta $\dd\bm{x} = \bm{0}$,
si ricava
\begin{subequations}
  \begin{align}
    \dd x'^{i} &= \tensor{\Lambda}{^{i}_{\alpha}}\dd x^{\alpha} =
    \tensor{\Lambda}{^{i}_{0}}\dd x^{0} + \tensor{\Lambda}{^{i}_{j}}\dd x^{j} =
    \tensor{\Lambda}{^{i}_{0}}\dd t \\
    \dd x'^{0} &= \tensor{\Lambda}{^{0}_{\alpha}}\dd x^{\alpha} =
    \tensor{\Lambda}{^{0}_{0}}\dd x^{0} + \tensor{\Lambda}{^{0}_{j}}\dd x^{j} =
    \tensor{\Lambda}{^{0}_{0}}\dd t.
  \end{align}
\end{subequations}
Dividendo membro a membro abbiamo
\begin{equation}
  \toder{x'^{i}}{x'^{0}} = \toder{x'^{i}}{t'} = v_{i} =
  \frac{\tensor{\Lambda}{^{i}_{0}}}{\tensor{\Lambda}{^{0}_{0}}},
\end{equation}
in quanto $\ltoder{x'^{i}}{t'}$ è la componente $v_{i}$ della velocità della
particella vista dal sistema $\mathcal{S}'$.  Allora
\begin{equation}
  \tensor{\Lambda}{^{i}_{0}} = v_{i}\tensor{\Lambda}{^{0}_{0}}.
\end{equation}
Ponendo $\gamma = \delta = 0$ nella~\eqref{eq:rel-fond-tensore-metrico} risulta
\begin{equation}
  \begin{split}
    \tensor{\eta}{_{00}} &= -1 = \tensor{\eta}{_{\alpha\beta}}
    \tensor{\Lambda}{^{\alpha}_{0}} \tensor{\Lambda}{^{\beta}_{0}} =
    \tensor{\eta}{_{\alpha\beta}}(\tensor{\Lambda}{^{0}_{0}})^{2} +
    \tensor{\eta}{_{ij}} \tensor{\Lambda}{^{i}_{0}} \tensor{\Lambda}{^{j}_{0}}
    \\
    &= -(\tensor{\Lambda}{^{0}_{0}})^{2} + \delta_{ij}
    \tensor{\Lambda}{^{i}_{0}} \tensor{\Lambda}{^{j}_{0}} =
    -(\tensor{\Lambda}{^{0}_{0}})^{2}+\sum_{i}(\tensor{\Lambda}{^{i}_{0}})^{2},
  \end{split}
\end{equation}
quindi
\begin{equation}
  (\tensor{\Lambda}{^{0}_{0}})^{2} = 1 +
  \sum_{i}(\tensor{\Lambda}{^{i}_{0}})^{2} = 1 +
  \sum_{i}(v_{i}\tensor{\Lambda}{^{0}_{0}})^{2} \geq 1,
\end{equation}
cioè deve aversi
\begin{equation}
  \tensor{\Lambda}{^{0}_{0}} \geq 1 \text{ oppure } \tensor{\Lambda}{^{0}_{0}}
  \leq -1.
\end{equation}
Le trasformazioni con $\tensor{\Lambda}{^{0}_{0}} \geq +1$ sono chiamate
\emph{ortocrone}, quelle con $\tensor{\Lambda}{^{0}_{0}} \leq -1$
\emph{anticrone}.  Le trasformazioni di Lorentz proprie e ortocrone sono dette
\emph{trasformazioni di Lorentz ristrette} o \emph{speciali}.  Considerando una
trasformazione ristretta inoltre abbiamo
\begin{equation}
  \tensor{\Lambda}{^{0}_{0}} = \frac{1}{\sqrt{1-\sum_{i}(v_{i})^{2}}} =
  \frac{1}{\sqrt{1-v^{2}}} = \gamma(v) \equiv \gamma,
\end{equation}
con $v^{2} = \sum_{i}(v_{i})^{2}$ modulo quadro del vettore $\bm{v}$.  Quindi
risulta anche
\begin{equation}
  \tensor{\Lambda}{^{i}_{0}} = v_{i}\tensor{\Lambda}{^{0}_{0}} = v_{i}\gamma.
\end{equation}
Le altre componenti $\tensor{\Lambda}{^{\alpha}_{\beta}}$ non sono univocamente
determinate.  Una scelta conveniente che permette di soddisfare
la relazione fondamentale~\eqref{eq:rel-fond-tensore-metrico} è
\begin{subequations}
  \begin{align}
    \tensor{\Lambda}{^{0}_{i}} &= \tensor{\Lambda}{^{i}_{0}} = v_{i}\gamma, \\
    \tensor{\Lambda}{^{i}_{j}} &= \delta_{ij} +
    v_{i}v_{j}\frac{\gamma-1}{v^{2}}.
  \end{align}
\end{subequations}
così che la matrice $\Lambda$ può essere scritta, per un \index{boost}boost
$\bm{v}$, come
\begin{equation}
  \Lambda =
  \begin{pmatrix}
    \gamma & \gamma v_{1} & \gamma v_{2} & \gamma _{3} \\[2.0ex]
    \gamma v_{1} & 1 + (v_{1})^{2}\dfrac{\gamma-1}{v^{2}} &
    v_{1}v_{2}\dfrac{\gamma-1}{v^{2}} & v_{1} v_{3}\dfrac{\gamma-1}{v^{2}}
    \\[2.0ex]
    \gamma v_{2} & v_{2}v_{1}\dfrac{\gamma-1}{v^{2}} & 1 +
    (v_{2})^{2}\dfrac{\gamma-1}{v^{2}} & v_{2}v_{3}\dfrac{\gamma-1}{v^{2}}
    \\[2.0ex]
    \gamma v_{3} & v_{3}v_{1}\dfrac{\gamma-1}{v^{2}} &
    v_{2}v_{3}\dfrac{\gamma-1}{v^{2}} & 1 + (v_{3})^{2}\dfrac{\gamma-1}{v^{2}}
  \end{pmatrix}.
\end{equation}


% TODO: scrivere qualcosa!
\section{Conseguenze delle trasformazioni di Lorentz}
\label{sec:conseguenze-lorentz}

\subsection{Contrazione delle lunghezze}
\label{sec:contrazione-lunghezze}

% TODO: ricordarsi di definire il tempo proprio (servirà più avanti)
\subsection{Dilatazione degli intervalli temporali}
\label{sec:dilatazione-tempo}

\subsection{Effetto Doppler}
\label{sec:effetto-doppler}

\section{Calcolo tensoriale nello spazio di Minkowski}
\label{sec:calcolo-tensoriale}

Il calcolo tensoriale è lo strumento matematico alla base della teoria della
relatività.  In questo paragrafo daremo dei cenni relativi al calcolo tensoriale
nello spazio di Minkowski.  Si noti che alcuni dei risultati che presenteremo
qui sono validi solo nello spazio piatto della relatività speciale e non in
quello curvo della relatività generale.

Un \index{quadrivettore!controvariante}\emph{quadrivettore controvariante}
$V^{\alpha} = (V^{0}, V^{1}, V^{2}, V^{3})$ è un insieme di quattro quantità
che, mediante una trasformazione di Lorentz ristretta omogenea
$x^{\alpha} \to x'^{\alpha} = \tensor{\Lambda}{^{\alpha}_{\beta}}x^{\beta}$, si
trasformano come le componenti di $\dd x^{\alpha}$, vale a dire
\begin{equation}
  V^{\alpha} \to V'^{\alpha} = \tensor{\Lambda}{^{\alpha}_{\beta}}V^{\beta}.
\end{equation}
Un quadrivettore controvariante può anche essere rappresentato come
\begin{equation}
  V^{\alpha} = (V^{0}, V^{i}) = (V^{0}, \bm{V}),
\end{equation}
dove $\bm{V} = (V^{1}, V^{2}, V^{3})$.

Un \index{quadrivettore!covariante}\emph{quadrivettore covariante}
$U_{\alpha} = (U_{0}, U_{1}, U_{2}, U_{3})$ è un insieme di quattro quantità
che, sotto una trasformazione di Lorentz ristretta, si trasformano come
\begin{equation}
  U_{\alpha} \to U'_{\alpha} = \tensor{\Lambda}{_{\alpha}^{\beta}}U_{\beta}
\end{equation}
dove
\begin{equation}
  \tensor{\Lambda}{_{\alpha}^{\beta}} = \tensor{\eta}{_{\alpha\gamma}}
  \tensor{\eta}{^{\beta\delta}} \tensor{\Lambda}{^{\gamma}_{\delta}}.
\end{equation}
La matrice $\tensor{\eta}{^{\alpha\beta}}$ è il
\index{tensore!metrico controvariante di
  Minkowski}\emph{tensore
  metrico controvariante di Minkowski}
ed è numericamente uguale a $\tensor{\eta}{_{\alpha\beta}}$, vale a dire
\begin{equation}
  \tensor{\eta}{^{\alpha\beta}} = \tensor{\eta}{_{\alpha\beta}}.
\end{equation}
Osserviamo che
\begin{equation}
  \tensor{\eta}{^{\alpha\gamma}} \tensor{\eta}{_{\beta\gamma}} =
  \tensor{\delta}{^{\alpha}_{\beta}} = \tensor{\delta}{_{\alpha}^{\beta}} =
  \begin{cases}
    1 & \text{se } \alpha = \beta, \\
    0 & \text{se } \alpha \neq \beta,
  \end{cases}
\end{equation}
in cui $\tensor{\delta}{^{\alpha}_{\beta}} = \tensor{\delta}{_{\alpha}^{\beta}}$
è la \index{delta!di Kronecker
  quadrimensionale}\emph{delta
  di Kronecker quadrimensionale}.
Usando la relazione fondamentale delle trasformazioni di
Lorentz~\eqref{eq:rel-fond-tensore-metrico} abbiamo
\begin{equation}
  \tensor{\Lambda}{_{\alpha}^{\gamma}} \tensor{\Lambda}{^{\alpha}_{\beta}} =
  \tensor{\eta}{_{\alpha\delta}} \tensor{\eta}{^{\gamma\epsilon}}
  \tensor{\Lambda}{^{\delta}_{\epsilon}} \tensor{\Lambda}{^{\alpha}_{\beta}} =
  \tensor{\eta}{_{\epsilon\beta}} \tensor{\eta}{^{\gamma\epsilon}} =
  \tensor{\delta}{^{\gamma}_{\beta}}.
\end{equation}
Dunque $\tensor{\Lambda}{_{\alpha}^{\beta}}$ è la matrice inversa di
$\tensor{\Lambda}{^{\alpha}_{\beta}}$.

Vogliamo ora determinare la trasformazione inversa $x'^{\alpha} \to x^{\alpha}$.
Dalla trasformazione~\eqref{eq:trasformazione-lineare} risulta che
\begin{equation}
  x'^{\alpha} \tensor{\Lambda}{_{\alpha}^{\gamma}} =
  \tensor{\Lambda}{^{\alpha}_{\beta}} \tensor{\Lambda}{_{\alpha}^{\gamma}}
  x^{\beta} + \tensor{\Lambda}{_{\alpha}^{\gamma}} a^{\alpha} =
  \tensor{\delta}{^{\gamma}_{\beta}} x^{\beta} + a'^{\gamma} = x^{\gamma}
  +a'^{\gamma}
\end{equation}
da cui ricaviamo la trasformazione, rinominando gli indici,
\begin{equation}
  \label{eq:trasformazione-inversa}
  x'^{\alpha} \to x^{\alpha} = \tensor{\Lambda}{_{\beta}^{\alpha}}x'^{\beta} -
  a'^{\alpha}.
\end{equation}

Possiamo passare da un quadrivettore controvariante
$V^{\alpha} = (V^{0}, V^{i})$ al corrispondente quadrivettore covariante
$V_{\alpha} = (V_{0}, V_{i})$ moltiplicandolo per il tensore metrico covariante
\begin{equation}
  \label{eq:abbassamento-indice}
  \begin{split}
    V^{\alpha} \to V_{\alpha} &= \tensor{\eta}{_{\alpha\beta}}V^{\beta} =
    \tensor{\eta}{_{\alpha\beta}} (V^{0}, V^{i}) =
    (\tensor{\eta}{_{0\beta}}V^{0}, \tensor{\eta}{_{i\beta}}V^{i}) \\
    &= (\tensor{\eta}{_{00}}V^{0}, \delta_{i\beta}V^{\beta}) = (-V^{0}, V^{i}).
  \end{split}
\end{equation}
Pertanto, mediante questa operazione, chiamata a volte di
\emph{abbassamento degli indici}, la componente temporale di un quadrivettore
cambia segno mentre quelle spaziali rimangono immutate.  Analogamente si può
effettuare l'\emph{innalzamento degli indici} di un quadrivettore covariante
moltiplicandolo per il tensore metrico controvariante
\begin{equation}
  \label{eq:innalzamento-indice}
  U_{\alpha} \to U^{\alpha} = \tensor{\eta}{^{\alpha\beta}}U_{\beta}.
\end{equation}
Verifichiamo che il quadrivettore $V_{\alpha}$ definito
nella~\eqref{eq:abbassamento-indice} si trasforma effettivamente come un
quadrivettore covariante
\begin{equation}
  V_{\alpha} \to V'_{\alpha} = \tensor{\eta}{_{\alpha\beta}}V'^{\beta} =
  \tensor{\eta}{_{\alpha\beta}} \tensor{\Lambda}{^{\beta}_{\gamma}} V^{\gamma} =
  \tensor{\eta}{_{\alpha\beta}} \tensor{\Lambda}{^{\beta}_{\gamma}}
  \tensor{\eta}{^{\gamma\delta}}V_{\delta} =
  \tensor{\Lambda}{_{\alpha}^{\delta}} V_{\delta}.
\end{equation}
Allo stesso modo si può far vedere che il quadrivettore $U^{\alpha}$ definito
nella~\eqref{eq:innalzamento-indice} si trasforma come un quadrivettore
controvariante.

Possiamo generalizzare le definizioni date finora introducendo il concetto di
quadritensore.  Un \index{quadritensore}\emph{quadritensore}
$\tensor*{A}{^{\alpha_{1}\dots\alpha_{p}}_{\beta_{1}\dots\beta_{q}}}$ (con
$p + q = k$) di rango $k$ è un insieme di $4^{k}$ quantità che sotto una
trasformazione di Lorentz ristretta omogenea
$x^{\alpha} \to x'^{\alpha} = \tensor{\Lambda}{^{\alpha}_{\beta}}x^{\beta}$ si
trasformano come
\begin{equation}
  \tensor*{A}{^{\alpha_{1}\dots\alpha_{p}}_{\beta_{1}\dots\beta_{q}}} \to
  \tensor*{{A'}}{^{\alpha_{1}\dots\alpha_{p}}_{\beta_{1}\dots\beta_{q}}} =
  \tensor{\Lambda}{^{\alpha_{1}}_{\mu_{1}}} \cdots
  \tensor{\Lambda}{^{\alpha_{p}}_{\mu_{p}}}
  \tensor{\Lambda}{_{\beta_{1}}^{\nu_{1}}} \cdots
  \tensor{\Lambda}{_{\beta_{q}}^{\nu_{q}}}
  \tensor*{A}{^{\mu_{1}\dots\mu_{p}}_{\nu_{1}\dots\nu_{q}}}
\end{equation}
Il quadritensore
$\tensor*{A}{^{\alpha_{1}\dots\alpha_{p}}_{\beta_{1}\dots\beta_{q}}}$ è un
quadritensore misto $p$ volte controvariante e $q$ volte covariante, oppure di
tipo $(p,q)$.  Un quadritensore di rango $k$ completamente controvariante è del
tipo $\tensor{A}{^{\alpha_{1}\dots\alpha_{k}}}$ e si trasforma come
\begin{equation}
  \tensor{A}{^{\alpha_{1}\dots\alpha_{k}}} \to
  \tensor{{A'}}{^{\alpha_{1}\dots\alpha_{k}}} =
  \tensor{\Lambda}{^{\alpha_{1}}_{\mu_{1}}} \cdots
  \tensor{\Lambda}{^{\alpha_{k}}_{\mu_{k}}} \tensor{A}{^{\mu_{1}\dots\mu_{k}}},
\end{equation}
similmente un quadritensore di rango $k$ completamente covariante è del tipo
$\tensor{A}{_{\alpha_{1}\dots\alpha_{k}}}$ e si trasforma come
\begin{equation}
  \tensor{A}{_{\alpha_{1}\dots\alpha_{k}}} \to
  \tensor{{A'}}{_{\alpha_{1}\dots\alpha_{k}}} =
  \tensor{\Lambda}{_{\alpha_{1}}^{\mu_{1}}} \cdots
  \tensor{\Lambda}{_{\alpha_{k}}^{\mu_{k}}} \tensor{A}{_{\mu_{1}\dots\mu_{k}}}.
\end{equation}
I quadritensori di rango $1$ sono i \index{quadrivettore}\emph{quadrivettori}, i
quadritensori di rango $0$, che quindi non hanno indici, sono chiamati
\index{scalare!di Lorentz}\emph{scalari di Lorentz} (o brevemente
\emph{scalari}) e sono invarianti sotto le trasformazioni di Lorentz.

\subsection{Operazioni con i quadritensori}
\label{sec:operazioni-tensori}

\subsubsection{Combinazione lineare}
\label{sec:combinazione-lineare}

La combinazione lineare
\begin{equation}
  \tensor*{T}{^{\alpha_{1}\dots\alpha_{p}}_{\beta_{1}\dots\beta_{q}}} = a
  \tensor*{R}{^{\alpha_{1}\dots\alpha_{p}}_{\beta_{1}\dots\beta_{q}}} + b
  \tensor*{S}{^{\alpha_{1}\dots\alpha_{p}}_{\beta_{1}\dots\beta_{q}}},
\end{equation}
con $a$ e $b$ scalari, di due quadritensori dello stesso tipo è un quadritensore
ancora dello stesso tipo.  Per esempio, verifichiamo che
\begin{equation}
  \tensor{T}{^{\alpha}_{\beta}} = a \tensor{R}{^{\alpha}_{\beta}} + b
  \tensor{S}{^{\alpha}_{\beta}}
\end{equation}
è un quadritensore di rango $2$ una volta controvariante e una volta covariante
come lo sono $\tensor{R}{^{\alpha}_{\beta}}$ e
$\tensor{S}{^{\alpha}_{\beta}}$. Abbiamo
\begin{equation}
  \begin{split}
    \tensor{T}{^{\alpha}_{\beta}} \to \tensor{{T'}}{^{\alpha}_{\beta}} &= a
    \tensor{{R'}}{^{\alpha}_{\beta}} + b \tensor{{S'}}{^{\alpha}_{\beta}} = a
    \tensor{\Lambda}{^{\alpha}_{\gamma}} \tensor{\Lambda}{_{\beta}^{\delta}}
    \tensor{R}{^{\gamma}_{\delta}} + b \tensor{\Lambda}{^{\alpha}_{\gamma}}
    \tensor{\Lambda}{_{\beta}^{\delta}} \tensor{S}{^{\gamma}_{\delta}} \\
    &= \tensor{\Lambda}{^{\alpha}_{\gamma}}
    \tensor{\Lambda}{_{\beta}^{\delta}}(a \tensor{R}{^{\gamma}_{\delta}} + b
    \tensor{S}{^{\gamma}_{\delta}}) = \tensor{\Lambda}{^{\alpha}_{\gamma}}
    \tensor{\Lambda}{_{\beta}^{\delta}} \tensor{T}{^{\gamma}_{\delta}}.
  \end{split}
\end{equation}

\subsubsection{Prodotto diretto}
\label{sec:prodotto-diretto}

Il prodotto
$\tensor*{A}{^{\alpha_{1}\dots\alpha_{p}}_{\beta_{1}\dots\beta_{q}}}
\tensor*{B}{^{\alpha_{1}\dots\alpha_{r}}_{\beta_{1}\dots\beta_{s}}}$
di un quadritensore di tipo $(p,q)$ e di uno di tipo $(r,s)$ è un quadritensore
$\tensor*{T}{^{\alpha_{1}\dots\alpha_{p+r}}_{\beta_{1}\dots\beta_{q+s}}}$ di
tipo $(p+r,q+s)$.  Consideriamo per esempio
\begin{equation}
  \tensor{T}{^{\alpha}_{\beta}^{\delta}} = \tensor{A}{^{\alpha}_{\beta}}
  \tensor{B}{^{\gamma}}.
\end{equation}
Abbiamo
\begin{equation}
  \tensor{T}{^{\alpha}_{\beta}^{\delta}} \to
  \tensor{{T'}}{^{\alpha}_{\beta}^{\delta}} = \tensor{{A'}}{^{\alpha}_{\beta}}
  \tensor{{B'}}{^{\gamma}} = \tensor{\Lambda}{^{\alpha}_{\delta}}
  \tensor{\Lambda}{_{\beta}^{\epsilon}} \tensor{A}{^{\delta}_{\epsilon}}
  \tensor{\Lambda}{^{\gamma}_{\zeta}} \tensor{B}{^{\zeta}} =
  \tensor{\Lambda}{^{\alpha}_{\delta}} \tensor{\Lambda}{_{\beta}^{\epsilon}}
  \tensor{\Lambda}{^{\gamma}_{\zeta}} \tensor{T}{^{\delta}_{\epsilon}^{\zeta}}.
\end{equation}

\subsubsection{Contrazione degli indici}
\label{sec:contrazione-indice}

Se in un quadritensore di tipo $(p,q)$ $n$ indici controvarianti sono uguali a
$n$ indici covarianti e si somma su tali indici si ottiene un tensore di tipo
$(p-n,q-n)$.  Usando la convenzione di Einstein sulla somma si omette il simbolo
di sommatoria.  Questa operazione viene detta di
\emph{contrazione degli indici}.  Per esempio, il tensore
$\tensor{T}{^{\alpha}_{\beta}^{\gamma\delta}}$ è un tensore di tipo $(3,1)$, ma
se si contraggono l'indice covariante $\beta$ con il terzo indice controvariante
$\delta$ si ottiene il tensore
$\tensor{T}{^{\alpha}_{\beta}^{\gamma\beta}} =
\tensor{T}{^{\alpha}_{0}^{\gamma 0}} + \tensor{T}{^{\alpha}_{1}^{\gamma 1}} +
\tensor{T}{^{\alpha}_{2}^{\gamma 2}} + \tensor{T}{^{\alpha}_{3}^{\gamma 3}}$
che è completamente controvariante di rango $2$.  Infatti
\begin{equation}
  \begin{split}
    \tensor{T}{^{\alpha}_{\beta}^{\gamma\beta}} \to
    \tensor{{T'}}{^{\alpha}_{\beta}^{\gamma\beta}} &=
    \tensor{\Lambda}{^{\alpha}_{\omega}} \tensor{\Lambda}{_{\beta}^{\mu}}
    \tensor{\Lambda}{^{\gamma}_{\nu}} \tensor{\Lambda}{^{\beta}_{\zeta}}
    \tensor{T}{^{\omega}_{\mu}^{\nu\zeta}} =
    \tensor{\Lambda}{^{\alpha}_{\omega}} \tensor{\Lambda}{^{\gamma}_{\nu}}
    \tensor{\delta}{^{\mu}_{\zeta}} \tensor{T}{^{\omega}_{\mu}^{\nu\zeta}} \\
    &= \tensor{\Lambda}{^{\alpha}_{\omega}} \tensor{\Lambda}{^{\gamma}_{\nu}}
    \tensor{T}{^{\omega}_{\mu}^{\nu\mu}}.
  \end{split}
\end{equation}

Si può effettuare la contrazione degli indici anche nel prodotto di due
quadritensori: il prodotto di un quadritensore di tipo $(p,q)$ e di uno di tipo
$(r,s)$ è un quadritensore di tipo $(p+r,q+s)$, ma se si contraggono $n$ indici
si ottiene un quadritensore di tipo $(p+r-n,q+s-n)$.  Per esempio, il
quadritensore $\tensor{A}{^{\alpha}_{\beta}^{\gamma\delta}}$ è di tipo $(3,1)$,
$\tensor{B}{_{\epsilon\zeta}}$ è di tipo $(0,2)$, il prodotto
$\tensor{A}{^{\alpha}_{\beta}^{\gamma\delta}} \tensor{B}{_{\epsilon\zeta}}$ è un
quadritensore di tipo $(3,3)$, ma il prodotto
$\tensor{A}{^{\alpha}_{\beta}^{\gamma\delta}} \tensor{B}{_{\alpha\delta}}$ con
due indici contratti è un quadritensore di tipo $(1,1)$.

\subsubsection{Prodotto scalare di due quadrivettori}
\label{sec:prodotto-scalare-quadrivettori}

Il \emph{prodotto scalare} fra due quadrivettori $V$ e $U$ è definito come
\begin{equation}
  \begin{split}
    V\cdot U &= V^{\alpha}U_{\alpha} =
    \tensor{\eta}{_{\alpha\beta}}V^{\alpha}U^{\beta} = V^{0}U_{0} + V^{1}U_{1} +
    V^{2}U_{2} + V^{3}U_{3} \\
    &= -V^{0}U^{0} + V^{1}U^{1} + V^{2}U^{2} + V^{3}U^{3} = -V^{0}U^{0} +
    \bm{V}\cdot\bm{U}.
  \end{split}
\end{equation}

Il prodotto scalare di due quadrivettori è un caso particolare di contrazione
degli indici nel prodotto fra due quadritensori.  Dunque il loro prodotto è un
quadritensore di rango $0$, cioè uno scalare e quindi una quantità che non varia
per effetto di una trasformazione di Lorentz.  Infatti
\begin{equation}
  V^{\alpha}U_{\alpha} \to V'^{\alpha}U'_{\alpha} =
  \tensor{\Lambda}{^{\alpha}_{\beta}} V^{\beta}
  \tensor{\Lambda}{_{\alpha}^{\gamma}} U_{\gamma} =
  \tensor{\delta}{_{\beta}^{\gamma}} V^{\beta} U_{\gamma} = V^{\gamma}U_{\gamma}
  = V^{\alpha} U_{\alpha}.
\end{equation}

Si può definire anche il \index{modulo!di un
  quadrivettore}\emph{modulo
  quadro di un quadrivettore}
$V$ come il prodotto scalare di $V$ per se stesso, vale a dire
\begin{equation}
  \begin{split}
    V\cdot V &= V^{\alpha}V_{\alpha} =
    \tensor{\eta}{_{\alpha\beta}}V^{\alpha}V^{\beta} = V^{0}V_{0} + V^{1}V_{1} +
    V^{2}V_{2} + V^{3}V_{3} \\
    &= -(V^{0})^{2} + (V^{1})^{2} + (V^{2})^{2} + (V^{3})^{2} = -(V^{0})^{2} +
    \bm{V}^{2}.
  \end{split}
\end{equation}
Anche il modulo quadro di un quadrivettore è uno scalare di Lorentz.

\subsubsection{Derivazione}
\label{sec:derivazione}

La derivazione di un quadritensore di rango $k$ rispetto alle coordinate
controvarianti $x^{\alpha}$ o covarianti
$x_{\alpha} = (x_{0},x_{1}) = (-x^{0},x^{i})$ è un quadritensore di rango $k+1$.
In dettaglio, la derivata
\begin{equation}
  \parder{}{x^{\sigma}}
  \tensor*{T}{^{\alpha_{1}\dots\alpha_{p}}_{\beta_{1}\dots\beta_{q}}}
\end{equation}
è un quadritensore del tipo
$\tensor*{S}{^{\alpha_{1}\dots\alpha_{p}}_{\beta_{1}\dots\beta_{q}\sigma}}$ e la
derivata
\begin{equation}
  \parder{}{x_{\sigma}}
  \tensor*{T}{^{\alpha_{1}\dots\alpha_{p}}_{\beta_{1}\dots\beta_{q}}}
\end{equation}
è un quadritensore del tipo
$\tensor*{S}{^{\alpha_{1}\dots\alpha_{p}\sigma}_{\beta_{1}\dots\beta_{q}}}$.
Consideriamo il caso particolare di uno scalare $\phi$.  Dalla
trasformazione~\eqref{eq:trasformazione-inversa} abbiamo
\begin{equation}
  \parder{x^{\beta}}{x'^{\alpha}} = \tensor{\Lambda}{_{\alpha}^{\beta}}
\end{equation}
quindi
\begin{equation}
  \parder{\phi}{x^{\alpha}} \to \parder{\phi}{x'^{\alpha}}
  = \parder{\phi}{x^{\beta}} \parder{x^{\beta}}{x'^{\alpha}} =
  \tensor{\Lambda}{_{\alpha}^{\beta}} \parder{\phi}{x^{\beta}}.
\end{equation}
Abbiamo così mostrato che la derivata di uno scalare (quadritensore di rango 0)
rispetto a una coordinata controvariante è un quadrivettore (quadritensore di
rango 1) covariante
\begin{equation}
  \parder{\phi}{x^{\alpha}} = U_{\alpha}.
\end{equation}

Definiamo l'operatore
\index{quadrigradiente!covariante}\emph{quadrigradiente covariante} come
\begin{equation}
  \partial_{\alpha} = \parder{}{x^{\alpha}} = \Big( \parder{}{t}, \nabla\Big)
\end{equation}
e l'operatore
\index{quadrigradiente!controvariante}\emph{quadrigradiente controvariante} come
\begin{equation}
  \partial^{\alpha} = \parder{}{x_{\alpha}} =
  \tensor{\eta}{^{\alpha\beta}} \partial_{\beta} = \Big( -\parder{}{t},
  \nabla\Big).
\end{equation}
Infine introduciamo l'operatore \index{d'Alambertiano}\emph{d'Alambertiano}
definito da
\begin{equation}
  \Box^{2} = \partial^{\alpha}\partial_{\alpha} = - \parder{}{*2t} + \nabla^{2}.
\end{equation}
Poiché il d'Alambertiano è il modulo quadro del quadrigradiente, esso è un
operatore differenziale scalare.

\subsection{Quadritensori invarianti}
\label{sec:tensori-invarianti}

Ci sono dei particolari quadritensori che assumono la stessa forma in qualsiasi
sistema di riferimento inerziale, così come fanno gli scalari.

\subsubsection{Tensore metrico di Minkowski}
\label{sec:tensore-metrico}

Il primo quadritensore che analizzeremo è il tensore metrico di Minkowski.
Dalla relazione fondamentale del tensore
metrico~\eqref{eq:rel-fond-tensore-metrico} abbiamo che
$\tensor{\eta}{_{\alpha\beta}}$ è effettivamente di un quadritensore covariante
di rango $2$ che non cambia per effetto di una trasformazione di Lorentz
\begin{equation}
  \tensor{\eta}{_{\alpha\beta}} \to \tensor{{\eta'}}{_{\alpha\beta}} =
  \tensor{\Lambda}{_{\alpha}^{\gamma}} \tensor{\Lambda}{_{\beta}^{\delta}}
  \tensor{\eta}{_{\gamma\delta}} = \tensor{\eta}{_{\alpha\beta}}.
\end{equation}
In maniera analoga si può dimostrare che $\tensor{\eta}{^{\alpha\beta}}$ è
effettivamente un quadritensore controvariante di rango $2$ e che rimane
immutato sotto una trasformazione di Lorentz.

\section{Dinamica particellare}
\label{sec:dinamica-particellare}

Data una particella il cui moto è descritto dalla coordinata controvariante
$x^{\alpha}$, definiamo la sua \index{quadrivelocità}\emph{quadrivelocità} come
la derivata di $x^{\alpha}$ rispetto al tempo proprio $\tau$:
\begin{equation}
  v^{\alpha} = (v^{0}, v^{i}) =\toder{x^{\alpha}}{\tau}
\end{equation}
e poiché risulta
\begin{equation}
  \dd \tau = \sqrt{\dd t^{2} - \dd \bm{x}^{2}} = \dd t \sqrt{1 -
    \Big(\toder{\bm{x}}{t}\Big)^{2}} = \dd t\sqrt{1 - v^{2}} = \frac{\dd
    t}{\gamma(v)}
\end{equation}
abbiamo che la quadrivelocità è uguale a
\begin{equation}
  \begin{split}
    v^{\alpha} &= \toder{x^{\alpha}}{t}\toder{t}{\tau} = \toder{x^{\alpha}}{t}
    \gamma(v) = \gamma(v) \Big(\toder{x^{0}}{t}, \toder{\bm{x}}{t}\Big) =
    \gamma(v)(1,\bm{v}) \\
    &= (\gamma(v), \gamma(v)\bm{v}).
  \end{split}
\end{equation}
Il \index{quadrimomento}\emph{quadrimomento} della particella è definito come il
prodotto fra la massa a riposo $m$ della particella e la quadrivelocità
\begin{equation}
  p^{\alpha} = (p^{0},p^{i}) = m v^{\alpha} = m (v^{0}, v^{i}) = (m\gamma(v),
  m\gamma(v)\bm{v}).
\end{equation}
Ricordando che l'energia totale $E$ di una particella è definita come $m\gamma(v)$
troviamo che la prima componente del quadrivettore momento è proprio l'energia
totale, mentre le componenti $p^{i}$ sono le componenti del momento
$\bm{p} = m\gamma(v)\bm{v}$
\begin{equation}
  p^{\alpha} = (E, \bm{p}).
\end{equation}
Vediamo come cambia il quadrimomento per effetto di un boost $\bm{V}$
\begin{equation}
  p^{\alpha} \to p'^{\alpha} = (p'^{0}, p'^{i}) =
  \tensor{\Lambda}{^{\alpha}_{\beta}} p^{\beta} =
  (\tensor{\Lambda}{^{0}_{\beta}}p^{\beta},
  \tensor{\Lambda}{^{i}_{\beta}}p^{\beta}).
\end{equation}
La prima componente ci dà l'energia $E'$ nel nuovo sistema di riferimento
\begin{equation}
  \begin{split}
    E' &= p'^{0} = \tensor{\Lambda}{^{0}_{\beta}}p^{\beta} =
    \tensor{\Lambda}{^{0}_{0}}p^{0} + \tensor{\Lambda}{^{0}_{i}}p^{i} \\
    &= \gamma(V) E + \gamma(V) V_{i}p^{i} = \gamma(V) E + \gamma(V)
    \bm{V}\cdot\bm{p}.
  \end{split}
\end{equation}
Le tre componenti spaziali sono le componenti $(\bm{p}')_{i}$ del vettore
momento nel nuovo sistema di riferimento
\begin{equation}
  \begin{split}
    (\bm{p}')_{i} &= (p')^{i} = \tensor{\Lambda}{^{i}_{\beta}}p^{\beta} =
    \tensor{\Lambda}{^{i}_{0}}p^{0} + \tensor{\Lambda}{^{j}_{j}}p^{j} \\
    &= \gamma(V) V_{i}E + \Big( \delta_{ij} +
    V_{i}V_{j}\frac{\gamma(V)-1}{V^{2}}\Big)p^{j} \\
    &= \gamma(V) V_{i}E + p_{i} +
    V_{i}V_{j}p^{j}\frac{\gamma(V)-1}{V^{2}} \\
    &= \gamma(V) V_{i}E + p_{i} + V_{i} \bm{V} \cdot \bm{p}
    \frac{\gamma(V)-1}{V^{2}},
  \end{split}
\end{equation}
cioè
\begin{equation}
  \bm{p}' = \bm{p} + \Big(\gamma(V) E + \bm{V} \cdot \bm{p}
  \frac{\gamma(V)-1}{V^{2}}\Big)\bm{V}.
\end{equation}
% NOTA: i risultati delle trasformazioni di momento ed energia sono differenti
% dalle formule [4.103] e [4.104], pagina 132, del Barone.  Weinberg definisce
% la velocità di spostamento fra i due sistemi di riferimento al contrario
% rispetto a quasi tutti gli altri autori (compreso il Barone).
Calcoliamo il modulo quadro del quadrimomento
\begin{equation}
  p^{\alpha}p_{\alpha} = \tensor{\eta}{_{\alpha\beta}}p^{\alpha}p^{\beta} =
  -E^{2} + \bm{p}^{2} = -m^{2}
\end{equation}
avendo usato la relazione $E^{2} = m^{2} + \bm{p}^{2}$.  Questo fa vedere che la
massa a riposo di una particella è uno scalare di Lorentz, quindi un invariante
relativistico.  Il quadrimomento covariante è dato da
\begin{equation}
  p_{\alpha} = \tensor{\eta}{_{\alpha\beta}}p^{\beta} = (-E, \bm{p}).
\end{equation}

La \index{quadriforza}\emph{quadriforza} agente sulla particella è definita come
la derivata rispetto al tempo proprio del quadrimomento
\begin{equation}
  f^{\alpha} = (f^{0},f^{i}) =\toder{p^{\alpha}}{\tau} =
  m\toder{v^{\alpha}}{\tau} = mw^{\alpha}
\end{equation}
in cui
\begin{equation}
  w^{\alpha} = (w^{0},w^{i}) =\toder{v^{\alpha}}{\tau}
\end{equation}
è la \index{quadriaccelerazione}\emph{quadriaccelerazione}.  Consideriamo un
sistema di riferimento $\mathcal{S}$ in cui la particella si muove con velocità
$\bm{v}$.  Calcoliamo la quadriforza a cui è soggetta la particella in un altro
sistema di riferimento inerziale $\mathcal{S}'$ che vede muoversi il sistema
$\mathcal{S}$ con velocità costante $\bm{V} = (V_{1}, V_{2}, V_{3})$:
\begin{equation}
  \begin{split}
    f'^{\alpha} &= \toder{p'^{\alpha}}{\tau'} = \toder{p'^{\alpha}}{t'}\gamma(V)
    = \gamma(V)\Big( \toder{p'^{0}}{t'}, \toder{p'^{i}}{t'}\Big) \\
    &= \gamma(V) \Big( \toder{(\gamma(V)(E + \bm{V}\cdot\bm{p}))}{t'},
    \toder{(\gamma(V)V_{i}E + p_{i} +
      V_{i}\bm{V}\cdot\bm{p}(\gamma(V)-1)/V^{2})}{t'} \Big).
  \end{split}
\end{equation}
Osserviamo che
$t' = x'^{0} = \tensor{\Lambda}{^{0}_{\alpha}}x^{\alpha} =
\tensor{\Lambda}{^{0}_{0}}x^{0} + \tensor{\Lambda}{^{0}_{i}}x^{i} =
\gamma(V)x^{0} + \gamma(V)V_{i}x^{i} = \gamma(V)(t+\bm{V}\cdot\bm{x})$, quindi
\begin{equation}
  \toder{}{t'} = \toder{t}{t'}\toder{}{t} =
  \Big(\toder{t'}{t}\Big)^{-1}\toder{}{t} =
  (\gamma(V)(1+\bm{V}\cdot\bm{v}))^{-1}\toder{}{t},
\end{equation}
allora
\begin{equation}
  \begin{split}
    f'^{0} &= \gamma(V) \toder{(\gamma(V)(E + \bm{V}\cdot\bm{p}))}{t'} =
    \frac{\gamma^{2}(V)}{\gamma(V)(1 + \bm{V}\cdot\bm{v})}
    \toder{(E + \bm{V}\cdot\bm{p})}{t} \\
    &= \frac{\gamma(V)}{1 + \bm{V}\cdot\bm{v}} (\bm{F}\cdot\bm{v} +
    \bm{F}\cdot\bm{V}) = \frac{\gamma(V)}{1 + \bm{V}\cdot\bm{v}}
    \bm{F}\cdot(\bm{v} + \bm{V}),
  \end{split}
\end{equation}
avendo ricordato che $\ltoder{E}{t} = \bm{F}\cdot\bm{v}$.  Per le tre componenti
spaziali $f'^{i}$ della quadriforza, che rappresentano le componenti della forza
$\bm{F}'$ nel sistema di riferimento $\mathcal{S}'$ abbiamo
\begin{equation}
  \begin{split}
    f'^{i} &= \gamma(V)\toder{(\gamma(V)V_{i}E + p_{i} +
      V_{i}\bm{V}\cdot\bm{p}(\gamma(V)-1)/V^{2})}{t'} \\
    &= \frac{\gamma(V)}{\gamma(V)(1 + \bm{V}\cdot\bm{v})}
    \toder{(\gamma(V)V_{i}E + p_{i} +
      V_{i}\bm{V}\cdot\bm{p}(\gamma(V)-1)/V^{2})}{t} \\
    &= \frac{1}{1 + \bm{V}\cdot\bm{v}} \Big( \gamma(V)V_{i}\bm{F}\cdot\bm{v} +
    F_{i} + V_{i}\bm{V}\cdot\bm{F}\frac{\gamma(V)-1}{V^{2}} \Big),
  \end{split}
\end{equation}
cioè
\begin{equation}
  \bm{f}'= \bm{F}' = \frac{1}{1 + \bm{V}\cdot\bm{v}} \Big(
  \bm{F} + \Big( \gamma(V)\bm{F}\cdot\bm{v} +
  \bm{V}\cdot\bm{F}\frac{\gamma(V)-1}{V^{2}} \Big)\bm{V} \Big).
\end{equation}
% TODO: c'è un fattore gamma(V) al denominatore di differenza rispetto alla
% formula [4.126], pagina 135, del Barone.  Scoprire il perché della differenza!

%%% Local Variables:
%%% mode: latex
%%% TeX-master: "../astrofisica-teorica"
%%% fill-column: 80
%%% End:
