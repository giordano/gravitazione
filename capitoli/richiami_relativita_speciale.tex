\chapter{Richiami di relatività speciale}
\label{cha:relativita-speciale}

Coordinata controvariante
\begin{equation}
  \label{eq:coord-controvariante}
  x^{\alpha} = (x^{0},x^{i}) = (t, \bm{x}),
\end{equation}
con $\alpha$ che varia fra $0$, $1$, $2$ e $3$ e $i$ fra $1$, $2$ e $3$.

Nello spazio ordinario si defisce
\begin{equation}
  \dd l^{2}   = \dd x^{2} + \dd y^{2} + \dd z^{2} = \sum_{i} (\dd x^{i})^{2} =
  \delta_{ij}\dd x^{i} \dd x^{j},
\end{equation}
con
\begin{equation}
  \delta_{ij} =
  \begin{pmatrix}
    1 & 0 & 0 \\
    0 & 1 & 0 \\
    0 & 0 & 1
  \end{pmatrix}.
\end{equation}
Il trasformato della coordinata controvariante~\eqref{eq:coord-controvariante} è
\begin{equation}
  x'^{i} = \tensor{R}{^{i}_{j}}x^{j} + a^{i}
\end{equation}
e la matrice $R$ è univocamente definita imponendo la condizione che $\dd l^{2}$
sia invariante cambiando sistema di riferimento.  Si scopre che $R$ è la matrice
di rotazione.

Nello spazio di Minkowski definiamo
\begin{equation}
  \dd \tau^{2} = \dd t^{2} - \dd \bm{x}^{2} = -\tensor{\eta}{_{\alpha\beta}}\dd
  x^{\alpha} \dd x^{\beta},
\end{equation}
in cui $\tensor{\eta}{_{\alpha\beta}}$ è il \emph{tensore metrico di Minkowski}
\begin{equation}
  \tensor{\eta}{_{\alpha\beta}} =
  \begin{pmatrix}
    -1 & 0 & 0 & 0 \\
    0 & 1 & 0 & 0 \\
    0 & 0 & 1 & 0 \\
    0 & 0 & 0 & 1
  \end{pmatrix}.
\end{equation}
Questo tensore definisce una metrica pseudo euclidea con segnatura $(-+++)$.
Imponendo che $\dd \tau$ sia invariante per trasformazioni di Lorentz possiamo
passare da un sistema di riferimento inerziale $O$ a un altro sistema di
riferimento inerziale $O'$ in moto rispetto a $O$.  Consideriamo due eventi
$E_{1}$ ed $E_{2}$ relativi a un fotone
\begin{gather}
  E_{1}: \quad (0,\bm{0}), \\
  E_{2}: \quad (\dd t, \dd \bm{x})
\end{gather}
\begin{equation}
  \dd \tau^{2} = \dd t^{2}\Big( 1 - \toder{\bm{x}^{2}}{t^{2}}\Big) = \dd t^{2}
  (1 - 1^{2}) = 0
\end{equation}
perché per un fotone si ha
\begin{equation}
  \bm{v}^{2} = \Big(\toder{\bm{x}}{t}\Big)^{2} = c^{2} = 1.
\end{equation}


\begin{equation}
  \dd \tau'^{2} = -\tensor{\eta}{_{\alpha\beta}}\dd x'^{\alpha}\dd x'^{\beta}
\end{equation}
Richiediamo che la trasformazione $x^{\alpha} \to x'^{\alpha}$ sia lineare
per rispettare la condizione di omogeneità dello spazio (la lunghezza di una
sbarra non deve dipendere dalla posizione degli estremi), del tempo (la
lunghezza della sbarra non deve dipendere dall'istante di tempo in cui la
misuriamo) e di isotropia dello spazio (la lunghezza della sbarra non deve
dipendere neanche dalla direzione in cui la guardiamo).  La più generica
trasformazione lineare è
\begin{equation}
  \label{eq:trasformazione-lineare}
  x'^{\alpha} = \tensor{\Lambda}{^{\alpha}_{\beta}}x^{\beta} + a^{\alpha}.
\end{equation}
Imponiamo che l'intervallo $\dd \tau^{2}$ non vari passando da un sistema di
riferimento all'altro, cioè che $\dd \tau^{2} = \dd \tau'^{2}$, così abbiamo
\begin{equation}
  \dd \tau'^{2} = -\tensor{\eta}{_{\alpha\beta}}
  \tensor{\Lambda}{^{\alpha}_{\gamma}} \dd x^{\gamma}
  \tensor{\Lambda}{^{\beta}_{\delta}} \dd x^{\delta} = \dd \tau^{2} =
  -\tensor{\eta}{_{\alpha\beta}} \dd x^{\alpha} \dd x^{\beta},
\end{equation}
da cui, semplificando negli ultimi due membri $\dd x^{\gamma}\dd x^{\delta}$,
otteniamo la relazione fondamentale del tensore metrico
\begin{equation}
  \label{eq:rel-fond-tensore-metrico}
  \tensor{\eta}{_{\alpha\beta}} \tensor{\Lambda}{^{\alpha}_{\gamma}}
  \tensor{\Lambda}{^{\beta}_{\delta}} = \tensor{\eta}{_{\gamma\delta}}.
\end{equation}
Questa è la condizione che deve essere soddisfatta da una
trasformazione~\eqref{eq:trasformazione-lineare} affinché sia una trasformazione
di Lorentz.  L'equazione~\eqref{eq:rel-fond-tensore-metrico} impone $10$
condizioni indipendenti in quanto il tensore metrico
$\tensor{\eta}{_{\gamma\delta}}$ è simmetrico quindi solo $10$ delle sue $16$
componenti sono indipendenti.

L'equazione matriciale equivalente alla~\eqref{eq:rel-fond-tensore-metrico} è
\begin{equation}
  \Lambda^{\textup{T}}\eta\Lambda = \eta
\end{equation}
dalla quale abbiamo
\begin{equation}
  \det\eta = \det(\Lambda^{\textup{T}}\eta\Lambda) = \det\eta
  \det(\Lambda^{\textup{T}}\Lambda) = \det\eta (\det\Lambda)^{2}
\end{equation}
e quindi deve valere la condizione
\begin{equation}
  \det\Lambda = \pm 1.
\end{equation}
Le trasformazioni di Lorentz con $\det\Lambda = +1$ sono chiamate
\emph{proprie}, quelle con $\det\Lambda = -1$ sono chiamate \emph{improprie}.

Consideriamo una particella a riposo nell'origine del sistema di riferimento $O$
(quindi $\dd \bm{x} = \bm{0}$) che si muove con velocità
$\bm{v} = (v^{1},v^{2},v^{3})$ rispetto a un altro sistema di riferimento
inerziale $O'$
% TODO: probabilmente questi calcoli andranno spezzati in più ambienti
% `equation' spiegando che passaggi vengono fatti in mezzo
\begin{gather}
  \dd x'^{\alpha} = \tensor{\Lambda}{^{\alpha}_{\beta}} \dd x^{\beta} =
  \tensor{\Lambda}{^{\alpha}_{0}} \dd x^{0} + \tensor{\Lambda}{^{\alpha}_{i}}\dd
  x^{i} \\
  \dd x'^{i} = \tensor{\Lambda}{^{i}_{0}}\dd t \\
  \dd x'^{0} = \tensor{\Lambda}{^{0}_{0}}\dd t.
\end{gather}
Dividendo membro a membro abbiamo
\begin{equation}
  \toder{x'^{i}}{x'^{0}} = \toder{x'^{i}}{t'} = v_{i} =
  \frac{\tensor{\Lambda}{^{i}_{0}}}{\tensor{\Lambda}{^{0}_{0}}},
\end{equation}
da cui
\begin{equation}
  \tensor{\Lambda}{^{i}_{0}} = v_{i}\tensor{\Lambda}{^{0}_{0}}.
\end{equation}
Ponendo $\gamma = \delta = 0$ nella~\eqref{eq:rel-fond-tensore-metrico} risulta
\begin{equation}
  \begin{split}
    \tensor{\eta}{_{00}} &= -1 = \tensor{\eta}{_{\alpha\beta}}
    \tensor{\Lambda}{^{\alpha}_{0}} \tensor{\Lambda}{^{\beta}_{0}} =
    \tensor{\eta}{_{\alpha\beta}}(\tensor{\Lambda}{^{0}_{0}})^{2} +
    \tensor{\eta}{_{ij}} \tensor{\Lambda}{^{i}_{0}} \tensor{\Lambda}{^{j}_{0}}
    \\
    &= -(\tensor{\Lambda}{^{0}_{0}})^{2} + \delta_{ij}
    \tensor{\Lambda}{^{i}_{0}} \tensor{\Lambda}{^{j}_{0}} =
    -(\tensor{\Lambda}{^{0}_{0}})^{2} + \sum_{i}(\tensor{\Lambda}{^{i}_{0}})^{2},
  \end{split}
\end{equation}
quindi
\begin{equation}
  (\tensor{\Lambda}{^{0}_{0}})^{2} = 1 +
  \sum_{i}(\tensor{\Lambda}{^{i}_{0}})^{2} = 1 +
  \sum_{i}(v_{i}\tensor{\Lambda}{^{0}_{0}})^{2} \geq 1,
\end{equation}
cioè deve risultare
\begin{equation}
  \tensor{\Lambda}{^{0}_{0}} \geq 1 \text{ oppure } \tensor{\Lambda}{^{0}_{0}}
  \leq -1.
\end{equation}
Inoltre abbiamo
\begin{equation}
  \tensor{\Lambda}{^{0}_{0}} = \frac{1}{\sqrt{1-\sum_{i}(v_{i})^{2}}} =
  \frac{1}{\sqrt{1-v^{2}}} = \gamma.
\end{equation}
Quindi risulta anche
\begin{equation}
  \tensor{\Lambda}{^{i}_{0}} = v_{i}\tensor{\Lambda}{^{0}_{0}} = v_{i}\gamma.
\end{equation}
Le altre componenti $\tensor{\Lambda}{^{\alpha}_{\beta}}$ non sono univocamente
determinate.  Una scelta conveniente che permette di soddisfare
la~\eqref{eq:trasformazione-lineare} è
\begin{align}
  \tensor{\Lambda}{^{0}_{i}} &= \tensor{\Lambda}{^{i}_{0}} = v_{i}\gamma, \\
  \tensor{\Lambda}{^{i}_{j}} &= \delta_{ij} + v_{i}v_{j}\frac{\gamma-1}{v^{2}}.
\end{align}


%%% Local Variables:
%%% mode: latex
%%% TeX-master: "../astrofisica-teorica"
%%% fill-column: 80
%%% End:
