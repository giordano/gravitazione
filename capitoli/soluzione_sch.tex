\cleardoublepage
\chapter{Soluzione di Schwarzschild}
\label{cha:soluzione-sch}

\section{Campo gravitazionale statico e a simmetria sferica}
\label{sec:campo-statico-sferico}

Vogliamo determinare la forma più generale della metrica che descrive un campo
gravitazionale a simmetria sferica e statico.  In questo caso $\dd\tau^{2} =
-g_{\mu\nu}\dd x^{\mu}\dd x^{\nu}$ non deve dipendere dalla coordinata temporale
$x^{0}$ e deve risultare simmetrico rispetto all'operazione di inversione
temporale.\footnote{Un campo solo indipendente dalla coordinata temporale si
  dice \emph{stazionario}, la condizione di staticità è quindi più forte.}
Queste condizioni impongono che il campo gravitazionale dipenda dalle coordinate
spaziali solo attraverso gli invarianti per rotazione che possono essere
costruiti a partire da $\bm{x}$ e $\dd\bm{x}$, cioè $\bm{x} \cdot \dd\bm{x}$,
$\bm{x} \cdot \bm{x}$, $\dd\bm{x} \cdot \dd\bm{x}$.  La forma più generale della
metrica con queste proprietà è
\begin{equation}
  \dd\tau^{2} = F(r) \dd t^{2} - 2E(r) \bm{x} \cdot \dd\bm{x} \dd t -
  D(r)(\bm{x} \cdot \dd\bm{x})^{2} - C(r)\dd\bm{x}^{2},
\end{equation}
in cui $F(r)$, $E(r)$, $D(r)$ e $C(r)$ sono funzioni della distanza dall'origine
$r = \norm{\bm{x}}$.  Data la simmetria del sistema che vogliamo analizzare
conviene passare in coordinate sferiche
\begin{subequations}
  \begin{align}
    x^{1} &= x = r \sin\theta \cos\phi, \\
    x^{2} &= y = r \sin\theta \sin\phi, \\
    x^{3} &= z = r \cos\theta
  \end{align}
\end{subequations}
così che la metrica diventa
\begin{equation}
  \dd\tau^{2} = F(r) \dd t^{2} - 2E(r) r\dd r\dd t - r^{2}D(r)\dd r^{2} -
  C(r)(\dd r^{2} + r^{2}\dd\theta^{2} + r^{2}\sin^{2}\theta \dd\phi^{2}).
\end{equation}

La libertà che si ha, in relatività generale, nella scelta del sistema di
coordinate ci consente di semplificare questa espressione.  Se ridefiniamo il
tempo ponendo
\begin{equation}
  t \to t' = t - r\frac{E(r)}{F(r)}
\end{equation}
il termine $g_{rt}=0$ e la metrica diventa
\begin{equation}
  \dd \tau^{2} = F(r)\dd t'^{2} - G(r)\dd r^{2} -C(r)(\dd r^{2} +
  r^{2}\dd\theta^{2} + r^{2}\sin^{2}\theta \dd\phi^{2})
\end{equation}
con
\begin{equation}
  G(r) = r^{2} \bigg(D(r) + \frac{E^{2}(r)}{F(r)} \bigg).
\end{equation}

Effettuando inoltre la trasformazione 
\begin{equation}
  r \to r' = \sqrt{C(r)}r
\end{equation}
otteniamo la \index{metrica!forma standard della}\emph{forma standard} della
metrica (rinominiamo le variabili $r'$ e $t'$ in $r$ e $t$)
\begin{equation}
  \label{eq:metrica-standard}
  \dd\tau^{2} = B(r)\dd t^{2} - A(r)\dd r^{2} - r^{2}(\dd\theta^{2} +
  \sin^{2}\theta \dd\phi^{2}),
\end{equation}
in cui
\begin{subequations}
  \begin{align}
    B(r) &= F(r), \\
    A(r) &= \bigg(1 + \frac{G(r)}{C(r)} \bigg) \bigg( 1 + \frac{r}{2C(r)}
           \toder{C(r)}{r} \bigg)^{-2}.
  \end{align}
\end{subequations}
Il tensore metrico è rappresentato dalla seguente matrice diagonale
\begin{equation}
  g_{\mu\nu} =
  \begin{pmatrix}
    g_{tt}       & g_{tr}       & g_{t\theta}      & g_{t\phi}      \\
    g_{rt}       & g_{rr}       & g_{r\theta}      & g_{r\phi}      \\
    g_{\theta t} & g_{\theta r} & g_{\theta\theta} & g_{\theta\phi} \\
    g_{\phi t}   & g_{\phi r}   & g_{\phi\theta}   & g_{\phi\phi}
  \end{pmatrix}
  =
  \begin{pmatrix}
    -B(r) & 0    & 0     & 0 \\
    0     & A(r) & 0     & 0 \\
    0     & 0    & r^{2} & 0 \\
    0     & 0    & 0     & r^{2}\sin^{2}\theta
  \end{pmatrix}.
\end{equation}

Poiché $g_{\mu\nu}$ è diagonale, gli elementi del tensore metrico controvariante
sono semplicemente $g^{\mu\nu} = 1/g_{\mu\nu}$, quindi
$g^{\mu\nu} = \diag(-B^{-1}(r), A^{-1}(r), r^{-2}, r^{-2}\sin^{-2}\theta)$.
Osserviamo che $g = -\det(g_{\mu\nu}) = r^{4}A(r)B(r)\sin^{2}\theta$, quindi
l'elemento di volume invariante è
\begin{equation}
  \sqrt{g} \dd^{4} x = r^{2}\sqrt{A(r) B(r)} \sin\theta \dd^{4} x.
\end{equation}

Gli elementi della connessione affine possono essere calcolate con la
formula~\eqref{eq:connessione-metrica}.  Si trova che le uniche componenti non
nulle sono (qui l'apice indica la derivazione rispetto a $r$)
\begin{subequations}
  \begin{align}
    \tensor{\Gamma}{^{r}_{rr}} &= \frac{A'(r)}{2A(r)}, &
    \tensor{\Gamma}{^{r}_{\theta\theta}} &= -\frac{r}{A(r)}, &
    \tensor{\Gamma}{^{r}_{\phi\phi}} &= -\frac{r \sin^{2}\theta}{A(r)}, \\
    \tensor{\Gamma}{^{r}_{tt}} &= \frac{B'(r)}{2A(r)}, &
    \tensor{\Gamma}{^{\theta}_{r\theta}} &=
    \tensor{\Gamma}{^{\theta}_{\theta r}} = \frac{1}{r}, &
    \tensor{\Gamma}{^{\theta}_{\phi\phi}} &= -\sin\theta \cos\theta, \\
    \tensor{\Gamma}{^{\phi}_{\phi r}} &= \tensor{\Gamma}{^{\phi}_{r\phi}} =
    \frac{1}{r}, & \tensor{\Gamma}{^{\phi}_{\phi\theta}} &=
    \tensor{\Gamma}{^{\phi}_{\theta\phi}} = \cot\theta, &
    \tensor{\Gamma}{^{t}_{tr}} &= \tensor{\Gamma}{^{t}_{rt}} =
    \frac{B'(r)}{2B(r)}.
  \end{align}
\end{subequations}

Dopo di ciò è possibile calcolare il tensore di Ricci usando la sua
definizione~\eqref{eq:tens-ricci}.  Grazie all'invarianza per rotazioni e
l'indipendenza dal tempo della metrica, anche il tensore di Ricci è diagonale
con componenti non nulle
\begin{subequations}
  \label{eq:ricci-sferico}
  \begin{align}
    R_{tt} &= -\frac{B''(r)}{2A(r)} + \frac{1}{4} \frac{B'(r)}{A(r)}
    \bigg(\frac{A'(r)}{A(r)} + \frac{B'(r)}{B(r)}\bigg) -
    \frac{1}{r}\frac{B'(r)}{A(r)}, \\
    R_{rr} &= \frac{B''(r)}{2B(r)} - \frac{1}{4} \frac{B'(r)}{B(r)}
    \bigg(\frac{A'(r)}{A(r)} + \frac{B'(r)}{B(r)}\bigg) -
    \frac{1}{r}\frac{A'(r)}{A(r)}, \\
    R_{\theta\theta} &= -1 + \frac{r}{2A(r)}\bigg(-\frac{A'(r)}{A(r)} +
    \frac{B'(r)}{B(r)}\bigg) + \frac{1}{A(r)}, \\
    R_{\phi\phi} &= R_{\theta\theta} \sin^{2}\theta.
  \end{align}
\end{subequations}
Osserviamo che risulta
\begin{equation}
  \label{eq:bar}
  \frac{R_{tt}}{B(r)} + \frac{R_{rr}}{A(r)} =
  -\frac{1}{rA(r)}\bigg(\frac{A'(r)}{A(r)} + \frac{B'(r)}{B(r)}\bigg).
\end{equation}

Concludiamo dicendo che si può
dimostrare\footnote{Vedi~\textcite[390-393]{landau:campi};
  \textcite[843-844]{misner:gravitation};
  \textcite[337-340]{ohanian:gravitazione};
  \textcite[166-171]{straumann:general};
  \textcite[335-337]{weinberg:gravitation}.}  il \emph{teorema di Birkhoff}
\index{teorema!di Birkhoff} il quale afferma che \emph{un campo gravitazionale a
  simmetria sferica in assenza di materia deve essere necessariamente
  statico}. Quindi l'ipotesi di staticità del campo è superflua in quanto
conseguenza della sua simmetria sferica.

\subsection{Soluzione di Schwarzschild}
\label{sec:metrica-schwarzschild}

La soluzione di Schwarzschild delle equazioni di Einstein descrive il campo
gravitazionale all'esterno di un corpo sferico non rotante di massa $M$,
elettricamente neutro e in assenza di materia, cioè con tensore energia-impulso
$T_{\mu\nu} = 0$.  Applicheremo i risultati che otterremo al campo
gravitazionale generato dal Sole.

Abbiamo visto che in assenza di materia le equazioni di Einstein sono 
$R_{\mu\nu} = 0$.  Di conseguenza, ponendo uguale a $0$ il secondo membro
della~\eqref{eq:bar} abbiamo
\begin{equation}
  \frac{A'(r)}{A(r)} = -\frac{B'(r)}{B(r)}
\end{equation}
da cui
\begin{equation}
  \ln A(r) = - \ln B(r) + \text{costante} \implies A(r)B(r) = \text{costante}.
\label{468}
\end{equation}
Per determinare il valore della costante notiamo che a distanza infinita dalla
sorgente, la metrica deve tendere a quella di Minkowski, che in coordinate
sferiche si esprime come $\dd\tau^{2} = \dd t^{2} -\dd r^{2} -
r^{2}(\dd\theta^{2} + \sin^{2}\theta \dd\phi^{2})$. Confrontando con la metrica
nella forma standard~\eqref{eq:metrica-standard} troviamo che deve risultare
$A(r) B(r) \xrightarrow{r \to +\infty} 1$ e pertanto la costante
nell'equazione~\eqref{468} deve essere pari a 1.  Quindi troviamo
\begin{equation}
  A(r) = \frac{1}{B(r)}.
\end{equation}
Sostituendo questo risultato nelle equazioni~\eqref{eq:ricci-sferico} abbiamo
\begin{subequations}
  \begin{align}
    R_{\theta\theta} &= -1 + rB'(r) + B(r), \\
    R_{rr} &= \frac{B''(r)}{2B(r)} + \frac{B'(r)}{rB(r)} =
    \frac{R'_{\theta\theta}(r)}{2rB(r)}.
  \end{align}
\end{subequations}
Imponendo che anche la componente $R_{\theta\theta}$ si annulli, da cui deriva
inoltre che anche $R_{rr}$ e $R_{\phi\phi}$ sono nulle, si ricava
\begin{equation}
  \toder{}{r}[rB(r)] = 1 \implies rB(r) = r + \text{costante} \implies B(r) = 1
  + \frac{\text{costante}}{r}.
\end{equation}
Per determinare il valore di questa nuova costante richiediamo che a grandi
distanze dal corpo che genera il campo (regione in cui il campo diventa debole)
sia valida l'approssimazione vista nel paragrafo~\ref{sec:limite-newtoniano}
$-B(r) \equiv g_{00} =  -1 + r_{\textup{S}}/r$, con $r_{\textup{S}}=2GM/c^2$ 
raggio di Schwarzschild del corpo di massa $M$.  Si ha quindi 
$\text{costante} = r_{\textup{S}}$. Abbiamo in definitiva la
\index{metrica!di Schwarzschild}\emph{metrica di Schwarzschild}, per campi a
simmetria sferica e indipendenti dal tempo data da
\begin{equation}
  \label{eq:metrica-schwarzschild}
  \dd\tau^{2} = \bigg(1 - \frac{r_{\textup{S}}}{r} \bigg) \dd t^{2} - \bigg(1 -
  \frac{r_{\textup{S}}}{r}\bigg)^{-1}\dd r^{2} - r^{2}\dd\theta^{2} -
  r^{2}\sin^{2}\theta \dd\phi^{2}.
\end{equation}

Osserviamo che questa metrica ha una singolarità per $r = r_{\textup{S}}$ e
un'altra per $r = 0$.  Nel caso specifico del Sole, e di numerosi altri corpi,
la prima singolarità non costituisce un problema poiché la metrica vale
all'esterno del corpo e il raggio di questa stella è molto più grande del suo
raggio di Schwarzschild.  Si
dimostra\footnote{Vedi~\textcite[511]{barone:relativita};
  \textcite[403-404]{landau:campi}; \textcite[374-378]{ohanian:gravitazione};
  \textcite[207-208]{weinberg:gravitation}.}  inoltre che la singolarità per $r
= r_{\textup{S}}$ è eliminabile adottando un opportuno sistema di coordinate,
quindi essa non è, in realtà, una singolarità fisica.

Le coordinate $t$, $r$, $\theta$ e $\phi$ utilizzate nella metrica di
Schwarzschild~\eqref{eq:metrica-schwarzschild} non sono banalmente le usuali
coordinate sferiche di uno spazio-tempo piatto.  Mentre si può far vedere che le
coordinate angolari $\theta$ e $\phi$ coincidono con quelle
sferiche,\footnote{Vedi~\textcite[333]{ohanian:gravitazione}.} lo stesso non
vale per le coordinate $t$ e $r$.  Lungo un meridiano, l'elemento di lunghezza è
dato da $\dd l = r\dd\theta$ e, come in uno spazio piatto, la lunghezza di una
circonferenza di ``raggio'' $r$ vale
\begin{equation}
  \int_{0}^{l} \dd l' = \int_{0}^{2\pi}r\dd\theta = 2\pi r.
\end{equation}
Ma $r$ non è anche la distanza radiale perché, invece, l'intervallo spaziale,
che è ottenuto dalla metrica ponendo $\dd t = \dd\theta = \dd\phi = 0$, è dato
da $\dd l = \dd r/\sqrt{1-r_{\textup{S}}^{}/r}$.  Possiamo dunque interpretare
la coordinata $r$ come la lunghezza della circonferenza divisa $2\pi$.  Il fatto
che il rapporto fra la lunghezza della circonferenza e la distanza dal suo
centro è diverso da $2\pi$ evidenzia che lo spazio considerato non è piatto.
Per interpretare la coordinata $t$, vediamo quando l'intervallo di tempo proprio
$\dd\tau$ è uguale all'intervallo della coordinata temporale $\dd t$, cioè per
quale osservatore risulta che $\sqrt{g_{00}}$ sia uguale a 1
\begin{equation}
  1 = \sqrt{g_{00}} = \sqrt{1 - \frac{r_{\textup{S}}}{r}} \implies r \to \infty.
\end{equation}
Quindi la coordinata temporale $t$ può essere interpretata come la coordinata
temporale di un osservatore all'infinito, $r \to \infty$.  L'intervallo di tempo
trascorso tra due eventi che accadono nello stesso posto (quindi con $\dd r =
\dd\theta = \dd\phi = 0$) è invece ottenuto integrando $\dd\tau = \dd t
\sqrt{1-r_ {\textup{S}}}$.

%%% Local Variables:
%%% mode: latex
%%% TeX-master: "../gravitazione"
%%% fill-column: 80
%%% End:
