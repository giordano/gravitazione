\chapter{Onde gravitazionali}
\label{cha:onde-grav}

Nel paragrafo~\ref{sec:campo-statico-sferico} abbiamo risolto le equazioni di
Einstein nel vuoto in un campo gravitazionale a simmetria sferica.  In questo
capitolo ci occuperemo della risoluzione delle equazioni di Einstein
nell'approssimazione di campo debole, quindi ci serviremo di alcuni risultati
trovati nei paragrafi~\ref{sec:limite-newtoniano} e
\ref{sec:limite-newtoniano-einstein}.  Troveremo che il campo gravitazionale può
emettere radiazione analogamente a quanto succede per il campo elettromagnetico,
come abbiamo visto nei paragrafi~\ref{sec:onde-elettromagnetiche-vuoto} e
\ref{sec:onde-elettromagnetiche-cariche}.

Sebbene questo risultato sarà ottenuto nell'approssimazione di campo debole, è
un fatto del tutto generale, ma la trattazione non approssimata presenta
notevoli difficoltà matematiche legate alla non linearità delle equazioni di
Einstein.

\section{Onde gravitazionali nel vuoto}
\label{sec:onde-grav-vuoto}

In questo paragrafo risolviamo, in particolare, le equazioni di Einstein nel
vuoto, cioè con $T^{\mu\nu} = 0$.  Ricordiamo che nell'approssimazione di campo
debole il tensore metrico covariante è dato da
\begin{equation}
  g_{\mu\nu} = \eta_{\mu\nu} + h_{\mu\nu},
\end{equation}
con $\abs{h_{\mu\nu}} \ll 1$, mentre il tensore metrico controvariante è
\begin{equation}
  g^{\mu\nu} = \eta^{\mu\nu} - h^{\mu\nu},
\end{equation}
con $h^{\mu\nu} = \eta^{\mu\lambda}\eta^{\mu\sigma} h^{\lambda\sigma}$.
L'innalzamento e l'abbassamento degli indici di $h_{\mu\nu}$ si esegue con il
tensore metrico non perturbato $\eta_{\mu\nu}$.
Dalla~\eqref{eq:christoffel-approx} abbiamo che la connessione affine
approssimata al primo ordine in $h_{\mu\nu}$ è
\begin{equation}
  \tensor{\Gamma}{^{\lambda}_{\mu\nu}} \approx \frac{1}{2}\eta^{\sigma\lambda}
  (\partial_{\mu}h_{\nu\sigma} + \partial_{\nu}h_{\mu\sigma}
  - \partial_{\sigma}h_{\mu\nu})
\end{equation}
e quindi il tensore di Ricci, trascurando i termini superiori al primo ordine in
$h_{\mu\nu}$, è dato da
\begin{equation}
  \begin{split}
    R_{\mu\nu} &= \partial_{\nu}\tensor{\Gamma}{^{\lambda}_{\mu\lambda}}
    - \partial_{\lambda} \tensor{\Gamma}{^{\lambda}_{\mu\nu}} +
    \overbrace{\tensor{\Gamma}{^{\eta}_{\mu\lambda}}
      \tensor{\Gamma}{^{\lambda}_{\eta\nu}} -
      \tensor{\Gamma}{^{\eta}_{\mu\nu}}
      \tensor{\Gamma}{^{\lambda}_{\eta\lambda}}}^{\mathcal{O}(h^{2})}
    \approx \partial_{\nu}\tensor{\Gamma}{^{\lambda}_{\mu\lambda}}
    - \partial_{\lambda} \tensor{\Gamma}{^{\lambda}_{\mu\nu}} \\
    &= \frac{1}{2}\partial_{\nu} (\eta^{\lambda\sigma}(\partial_{\mu}
    h_{\sigma\lambda} + \partial_{\lambda}h_{\mu\sigma}
    - \partial_{\sigma}h_{\mu\lambda})) -
    \frac{1}{2}\partial_{\lambda}(\eta^{\lambda\sigma}(\partial_{\mu}h_{\sigma\nu}
    + \partial_{\nu}h_{\mu\sigma} - \partial_{\sigma}h_{\mu\nu})) \\
    &= \frac{1}{2} \eta^{\lambda\sigma}(\partial_{\mu}\partial_{\nu}
    h_{\sigma\lambda} + \partial_{\lambda}\partial_{\nu} h_{\mu\sigma}
    - \partial_{\sigma}\partial_{\nu} h_{\mu\lambda}
    - \partial_{\mu}\partial_{\lambda} h_{\sigma\nu}
    - \partial_{\nu}\partial_{\lambda} h_{\mu\sigma}
    + \partial_{\sigma}\partial_{\lambda} h_{\mu\nu}) \\
    &= \frac{1}{2}(\partial_{\mu}\partial_{\nu}
    \tensor{h}{^{\lambda}_{\lambda}} - \partial^{\lambda}\partial_{\nu}
    h_{\mu\lambda} - \partial^{\lambda}\partial_{\mu} h_{\nu\lambda}
    + \partial_{\lambda}\partial^{\lambda} h_{\mu\nu}) \\
    &= \frac{1}{2} (\dalamb h_{\mu\nu} + \partial_{\mu}\partial_{\nu} h
    - \partial^{\lambda}\partial_{\nu} h_{\mu\lambda}
    - \partial_{\mu}\partial^{\lambda} h_{\nu\lambda}),
  \end{split}
\end{equation}
in cui $h = \tensor{h}{^{\lambda}_{\lambda}} = \eta^{\mu\lambda} h_{\mu\lambda}$
è la traccia della perturbazione.  Nel vuoto allora abbiamo
\begin{equation}
  R_{\mu\nu} = \frac{1}{2} (\dalamb h_{\mu\nu} + \partial_{\mu}\partial_{\nu} h
  - \partial^{\lambda}\partial_{\nu} h_{\mu\lambda}
  - \partial_{\mu}\partial^{\lambda} h_{\nu\lambda}) = 0.
\end{equation}

A questo punto vogliamo manipolare l'espressione ottenuta per il tensore di
Ricci in modo che le equazioni di Einstein nel vuoto si linearizzino, diventando
analoghe alle equazioni di Maxwell nel vuoto $\dalamb A_{\beta} = 0$.  Come nel
caso elettromagnetico, è possibile fare ciò scegliendo una gauge opportuna.  In
questo caso conviene scegliere la \index{gauge!armonica}\emph{gauge armonica}
\begin{equation}
  \label{eq:gauge-armonica}
  \partial_{\mu} \tensor{h}{^{\mu}_{\nu}} =
  \frac{1}{2}\partial_{\nu}\tensor{h}{^{\mu}_{\mu}}.
\end{equation}
Infatti usando questa gauge gli ultimi tre termini del tensore di Ricci si
annullano
\begin{equation}
  \begin{split}
    \partial_{\mu}\partial_{\nu} h - \partial^{\lambda}\partial_{\nu}
    h_{\mu\lambda} - \partial_{\mu}\partial^{\lambda} h_{\nu\lambda}
    &= \partial_{\mu}(2 \partial_{\lambda}\tensor{h}{^{\lambda}_{\nu}})
    - \partial_{\nu}(\partial_{\lambda}\tensor{h}{^{\lambda}_{\mu}})
    - \partial_{\lambda}\partial_{\mu} \tensor{h}{^{\lambda}_{\nu}} \\
    &= \partial_{\mu}\partial_{\lambda} \tensor{h}{^{\lambda}_{\nu}}
    - \partial_{\nu}(\partial_{\lambda} \tensor{h}{^{\lambda}_{\mu}}) \\
    &= \partial_{\mu}\bigg(\frac{1}{2}\partial_{\nu}
    \tensor{h}{^{\lambda}_{\lambda}}\bigg)
    - \partial_{\nu}\bigg(\frac{1}{2}\partial_{\mu}
    \tensor{h}{^{\lambda}_{\lambda}}\bigg) = 0
  \end{split}
\end{equation}
e quindi $R_{\mu\nu} = \dalamb h_{\mu\nu}/2$.  Le equazioni di Einstein nel
vuoto si possono, in questo modo, scrivere nella seguente forma lineare
\begin{equation}
  \label{eq:einstein-lineare}
  \dalamb h_{\mu\nu} = 0.
\end{equation}
Osserviamo che la condizione di gauge armonica~\eqref{eq:gauge-armonica} si può
anche porre nella forma
$\Gamma^{\lambda} = \eta^{\mu\nu} = \tensor{\Gamma}{^{\lambda}_{\mu\nu}} = 0$ in
quanto
\begin{equation}
  \begin{split}
    0 &=\eta^{\mu\nu} \tensor{\Gamma}{^{\lambda}_{\mu\nu}} = \frac{1}{2}
    \eta^{\lambda\sigma} (\partial^{\mu}h_{\mu\sigma} + \partial^{\mu}
    h_{\mu\sigma} - \partial_{\sigma}\tensor{h}{^{\mu}_{\mu}}) \\
    &= \frac{1}{2} (\partial^{\mu} \tensor{h}{_{\mu}^{\lambda}} + \partial^{\mu}
    \tensor{h}{_{\mu}^{\lambda}} - \partial^{\lambda} \tensor{h}{^{\mu}_{\mu}}),
  \end{split}
\end{equation}
da cui
\begin{equation}
  \partial^{\lambda} \tensor{h}{^{\mu}_{\mu}} = 2 \partial^{\mu}
  \tensor{h}{_{\mu}^{\lambda}}
\end{equation}
che equivale alla~\eqref{eq:gauge-armonica}.

Si può sempre scegliere un sistema di riferimento in cui sia valida la
condizione armonica.  Sia $x^{\mu}$ un sistema di coordinate in cui questa
condizione non è valida, allora possiamo effettuare la trasformazione
infinitesima $x^{\mu} \to x'^{\mu} = x^{\mu} + \epsilon^{\mu}(x)$, con
$\abs{\epsilon^{\mu}(x)} \ll 1$, richiedendo che la metrica si possa ancora
sviluppare come $g'_{\mu\nu} = \eta_{\mu\nu} + h'_{\mu\nu}$, in modo da poter
ancora ottenere la forma linearizzata~\eqref{eq:einstein-lineare} delle
equazioni di Einstein.  Possiamo determinare un'espressione esplicita per
$h'_{\mu\nu}$ imponendo che il tensore metrico controvariante nel nuovo sistema
$g'^{\mu\nu} = \eta^{\mu\nu} - h'^{\mu\nu}$ si trasformi come un tensore
controvariante di rango $2$.  Trascurando termini del secondo ordine in
$\epsilon$ o dell'ordine di $h\epsilon$ abbiamo
\begin{equation}
  \begin{split}
    g'^{\mu\nu} &=
    g^{\lambda\rho} \parder{x'^{\mu}}{x^{\lambda}} \parder{x'^{\nu}}{x^{\rho}} =
    (\eta^{\lambda\rho} + h^{\lambda\rho}) (\tensor{\delta}{^{\mu}_{\lambda}}
    + \partial_{\lambda}\epsilon^{\mu}) (\tensor{\delta}{^{\nu}_{\rho}}
    + \partial_{\rho} \epsilon^{\nu}) \\
    &= (\eta^{\mu\rho} - h^{\mu\rho} +
    \eta^{\lambda\rho}\partial_{\lambda}\epsilon^{\mu} -
    \underbrace{h^{\lambda\rho} \partial_{\lambda}
      \epsilon^{\mu}}_{\mathcal{O}(h\epsilon)})
    (\tensor{\delta}{^{\nu}_{\rho}} + \partial_{\rho} \epsilon^{\nu}) \\
    &\approx \eta^{\mu\nu} - h^{\mu\nu} + \eta^{\lambda\nu}\partial_{\lambda}
    \epsilon^{\mu} + \eta^{\mu\rho}\partial_{\rho} \epsilon^{\nu} -
    \underbrace{h^{\mu\rho}\partial_{\rho}\epsilon^{\nu}}_{\mathcal{O}(h\epsilon)}
    + \underbrace{\partial^{\rho} \epsilon^{\mu} \partial_{\rho}
      \epsilon^{\nu}}_{\mathcal{O}(\epsilon^{2})} \\
    & \approx \eta^{\mu\nu} - h^{\mu\nu} + \partial^{\nu} \epsilon^{\mu}
    + \partial^{\mu} \epsilon^{\nu}.
  \end{split}
\end{equation}
Dunque
\begin{equation}
  h'^{\mu\nu} = h^{\mu\nu} - \partial^{\nu} \epsilon^{\mu} + \partial^{\mu}
  \epsilon^{\nu}.
\end{equation}
In questo nuovo sistema di riferimento calcoliamo ambo i membri
della~\eqref{eq:gauge-armonica}
\begin{equation}
  \begin{split}
    &\partial_{\mu} \tensor{{h'}}{^{\mu}_{\nu}} = \frac{1}{2}\partial_{\nu}
    \tensor{{h'}}{^{\mu}_{\mu}} \iff \\
    &\partial_{\mu} \tensor{h}{^{\mu}_{\nu}}
    - \partial_{\mu} \partial_{\nu} \epsilon^{\nu}
    - \partial_{\mu}\partial^{\mu} \epsilon_{\nu}
    = -\frac{1}{2} \partial_{\nu} \partial_{\mu} \epsilon^{\mu} -
    \frac{1}{2}\partial_{\nu} \partial_{\mu} \epsilon^{\mu} +
    \frac{1}{2} \partial_{\nu} \tensor{h}{^{\mu}_{\mu}} \iff \\
    &\partial_{\mu} \tensor{h}{^{\mu}_{\nu}}
    - \partial_{\mu}\partial^{\mu} \epsilon_{\nu}
    = \frac{1}{2} \partial_{\nu} \tensor{h}{^{\mu}_{\mu}}.
  \end{split}
\end{equation}
Quindi se scegliamo il generatore $\epsilon_{\nu}$ della trasformazione
infinitesima delle coordinate tale che
\begin{equation}
  \dalamb \epsilon_{\nu} = \partial_{\mu} \tensor{h}{^{\mu}_{\nu}} -
  \frac{1}{2} \partial_{\nu} \tensor{h}{^{\mu}_{\mu}}
\end{equation}
allora nel nuovo sistema di riferimento la condizione
armonica~\eqref{eq:gauge-armonica} sarà soddisfatta.  Se la condizione era già
valida nel sistema di partenza, bisognerà allora porre
$\dalamb \epsilon_{\nu} = 0$.  Questa libertà di scelta ricorda la libertà che
si ha nella gauge di Lorenz per il potenziale elettromagnetico.

Abbiamo visto che con la gauge armonica~\eqref{eq:gauge-armonica} l'equazione di
Einstein nel vuoto può essere scritta nella forma
lineare~\eqref{eq:einstein-lineare}, che è del tutto analoga alle equazioni di
Maxwell~\eqref{eq:maxwell-vuoto-lorenz} nel vuoto con la gauge di Lorenz, quindi
anche per $h_{\mu\nu}$ la soluzione generale sarà una sovrapposizione lineare di
onde piane del tipo
\begin{equation}
  h_{\mu\nu} = \Re\{e_{\mu\nu} \e^{\uimm k^{\lambda}x_{\lambda}} \}.
\end{equation}
Queste sono le \index{onda!gravitazionale}\emph{onde gravitazionali}.
Dalla~\eqref{eq:einstein-lineare} abbiamo (omettendo il simbolo $\Re$ per quanto
detto nel paragrafo~\ref{sec:onde-elettromagnetiche-vuoto})
\begin{equation}
  0 = \dalamb h_{\mu\nu} = \uimm k^{\lambda} \uimm k_{\lambda} e_{\mu\nu}
  \e^{\uimm k^{\lambda}x_{\lambda}} \implies k^{\lambda}k_{\lambda} = 0,
\end{equation}
cioè le onde si muovono, come quelle elettromagnetiche, alla velocità della luce
nel vuoto.  Inoltre dalla gauge armonica~\eqref{eq:gauge-armonica} abbiamo
\begin{equation}
  \uimm k_{\lambda} \tensor{e}{^{\lambda}_{\nu}} \e^{\uimm
    k^{\lambda}x_{\lambda}} = \frac{\uimm}{2} k_{\nu}
  \tensor{e}{^{\lambda}_{\lambda}} \e^{\uimm k^{\lambda}x_{\lambda}} \iff
  k_{\lambda} \tensor{e}{^{\lambda}_{\nu}} = \frac{1}{2} k_{\nu}
  \tensor{e}{^{\lambda}_{\lambda}}.
\end{equation}
Dal momento che $h_{\mu\nu}$ è simmetrico nello scambio degli indici, lo dovrà
essere anche $e_{\mu\nu}$, chiamato \index{tensore!di polarizzazione}
\emph{tensore di polarizzazione}, il quale quindi avrà indipendenti al più dieci
delle sue sedici componenti.  Al variare di $\nu$ nell'equazione precedente ci
sono quattro ulteriori relazioni che legano le componenti del tensore di
polarizzazione, quindi il numero di componenti indipendenti non può superare
sei.

Scegliamo l'asse $z$ lungo la direzione di propagazione dell'onda, di modo
che il vettore d'onda abbia la forma
$k^{\mu} = (k^{0}, k^{1}, k^{2}, k^{3}) = (k, 0, 0, k)$.  Ponendo nell'equazione
precedente $\nu = 1$ abbiamo
\begin{equation}
  k_{\mu}\tensor{e}{^{\mu}_{1}} = \frac{1}{2} k_{1}\tensor{e}{^{\mu}_{\mu}} = 0.
\end{equation}
Esplicitiamo il primo membro
\begin{equation}
  k_{0} \tensor{e}{^{0}_{1}} + k_{3} \tensor{e}{^{3}_{1}} = 0 \implies
  \tensor{e}{^{0}_{1}} = -\tensor{e}{^{3}_{1}} \implies e_{01} = -e_{31}.
\end{equation}
Analogamente per $\nu = 2$ si trova
\begin{equation}
  k_{\mu} \tensor{e}{^{\mu}_{2}} = \frac{1}{2} k_{2} \tensor{e}{^{\mu}_{\mu}} =
  0 \implies e_{02} = -e_{32}.
\end{equation}
Per $\nu = 0$ risulta
\begin{equation}
  \begin{split}
    &k_{\mu} \tensor{e}{^{\mu}_{0}} = \frac{1}{2} k_{0}\tensor{e}{^{\mu}_{\mu}}
    = \frac{1}{2}\eta_{00}k^{0}\tensor{e}{^{\mu}_{\mu}} =
    -\frac{1}{2}k^{0}\tensor{e}{^{\mu}_{\mu}} \implies \\
    &k^{0} e_{00} + k^{3}e_{30} = -\frac{1}{2}k^{0}(\tensor{e}{^{1}_{1}} +
    \tensor{e}{^{2}_{2}} + \tensor{e}{^{3}_{3}} - e_{00})
  \end{split}
\end{equation}
e poiché $\tensor{e}{^{i}_{i}} = \eta^{ij}e_{ij} = \delta^{ij}e_{ij} = e_{ii}$
ricaviamo
\begin{equation}
  e_{00} + e_{30} = -\frac{1}{2}(e_{11} + e_{22} + e_{33} - e_{00}).
\end{equation}
In maniera simile, per $\nu = 3$ si trova
\begin{equation}
  e_{03} + e_{33} = \frac{1}{2}(e_{11} + e_{22} + e_{33} - e_{00}).
\end{equation}
Sommando membro a membro le ultime due equazioni e sfruttando la simmetria del
tensore di polarizzazione abbiamo
\begin{equation}
  e_{03} = -\frac{1}{2}(e_{33} + e_{00})
\end{equation}
e inserendo questo risultato nell'equazione trovata per $\nu = 3$ otteniamo
\begin{equation}
  e_{22} = -e_{11}.
\end{equation}
Riepilogando, fino a questo punto abbiamo trovato per un'onda gravitazionale che
si propaga lungo l'asse $z$ si ha
\begin{subequations}
  \begin{align}
    e_{01} &= -e_{31}, \\
    e_{02} &= -e_{32}, \\
    e_{03} &= -\frac{1}{2}(e_{33} + e_{00}), \\
    e_{22} &= -e_{11}.
  \end{align}
\end{subequations}
Abbiamo dedotto tutto ciò nell'ipotesi che sia valida la condizione
armonica~\eqref{eq:gauge-armonica}.  Effettuiamo una trasformazione infinitesima
delle coordinate $x^{\mu} \to x'^{\mu} = x^{\mu} + \epsilon^{\mu}(x)$.  Affinché
sia ancora valida l'equazione di Einstein linearizzata e la gauge armonica
abbiamo visto che deve valere $\dalamb \epsilon^{\mu} = 0$.  Poiché anche
$\epsilon^{\mu}$ soddisfa l'equazione delle onde, la sua soluzione sarà data
dalla sovrapposizione lineare di onde del tipo
\begin{equation}
  \epsilon^{\mu} = \Re\{\tilde\epsilon^{\mu}\e^{\uimm k^{\lambda}x_{\lambda}}\}.
\end{equation}
La perturbazione della metrica in questo nuovo sistema è
\begin{equation}
  h'_{\mu\nu} = \Re\{e'_{\mu\nu} \e^{\uimm k^{\lambda}x_{\lambda}}\} =
  h_{\mu\nu} - \partial_{\nu}\epsilon_{\mu} - \partial_{\mu}\epsilon_{\nu}
\end{equation}
e poiché
$\partial_{\nu}\epsilon_{\mu} = \uimm k_{\nu}\tilde\epsilon_{\mu}
\e^{\uimm k^{\lambda}x_{\lambda}}$ si ha
\begin{equation}
  e'_{\mu\nu} = e_{\mu\nu} - \uimm k_{\nu}\tilde\epsilon_{\mu} - \uimm
  k_{\mu}\tilde\epsilon_{\nu}.
\end{equation}
Così risulta
\begin{subequations}
  \begin{align}
    e'_{11} &= e_{11} - \uimm k_{1} \tilde\epsilon_{1} - \uimm
    k_{1}\tilde\epsilon_{1} = e_{11}, \\
    e'_{12} &= e_{12} - \uimm k_{2}\tilde\epsilon_{1} - \uimm
    k_{1}\tilde\epsilon_{2} = e_{12}, \\
    e'_{13} &= e_{13} - \uimm k_{3}\tilde\epsilon_{1} - \uimm
    k_{1}\tilde\epsilon_{3} = e_{13} - \uimm k_{3}\tilde\epsilon_{1} = 0
  \end{align}
\end{subequations}
avendo posto $\tilde\epsilon_{1} = -\uimm e_{13}/k_{3}$.
% TODO: controllare i seguenti calcoli
Scegliendo in maniera opportuna gli altri $\epsilon_{\mu}$ si possono annullare
anche $e'_{13} = e'_{23} = e'_{33} = e'_{00} = 0$ e anche
$e_{01} = -e_{31} = e_{02} = -e_{32} = e_{03} = 0$, quindi rimangono diversi da
zero solo le componenti $e_{22} = -e_{11}$ e $e_{12} = e_{21}$ per un'onda che
si propaga lungo l'asse $z$.  Abbiamo trovato che la simmetria del tensore di
polarizzazione riduce da sedici a dieci il numero di sue componenti
indipendenti, la condizione armonica diminuisce questo numero a sei e la libertà
di scelta sulla trasformazione di gauge fissa il numero di componenti
indipendenti a due, in maniera analoga a quello che succede per le onde
elettromagnetico nel vuoto.  Sono quindi due gli stati di polarizzazione fisica
delle onde gravitazionali.  Questa scelta di coordinate determina la
\index{gauge!trasversa a traccia nulla}\emph{gauge trasversa a traccia nulla}:
``a traccia nulla'' perché
$\tensor{e}{^{\mu}_{\mu}} = \tensor{e}{^{0}_{0}} + \tensor{e}{^{1}_{1}} +
\tensor{e}{^{2}_{2}} + \tensor{e}{^{3}_{3}} = \tensor{e}{^{1}_{1}} +
\tensor{e}{^{2}_{2}} = 0$,
``trasversa'' perché le componenti non nulle del tensore di polarizzazione sono
sulle direzioni perpendicolari alla direzione di propagazione dell'onda (per
un'onda gravitazionale che si muovo lungo l'asse $z$ sono non nulle le
componenti associate agli assi $x$ e $y$).  Poiché i tensori di polarizzazione
rappresentano l'ampiezza delle onde da cui è costituito $h_{\mu\nu}$, anche
$h_{\mu\nu}$ sarà trasverso e con traccia $h = \tensor{h}{^{\mu}_{\mu}}$ nulla.

% TODO: completare dopo aver rivisto i calcoli, al momento non mi convincono
% molto.  Vedi appunti dei giorni 24-25/05/2012
\section{\completare{Tecniche di rivelazione delle onde gravitazionali}}
\label{sec:rivelazione-onde}

Consideriamo due particelle $A$ e $B$, descritte rispettivamente dalle linee
d'universo $x^{\mu}_{A} = x^{\mu}(\tau)$ e
$x^{\mu}_{B} = x^{\mu}_{A}(\tau) + \delta x^{\mu}(\tau)$, cioè il vettore
$\delta x^{\mu}(\tau)$ rappresenta la separazione fra $A$ e $B$.  Dall'equazione
di deviazione geodetica % TODO: mettere riferimento
sappiamo che
\begin{equation}
  \curder[2]{\delta x^{\mu}} = \tensor{R}{^{\lambda}_{\nu\mu\rho}} \delta
  x^{\mu} \toder{x^{\nu}}{\tau} \toder{x^{\rho}}{\tau}
\end{equation}

\section{Onde gravitazionali in presenza di masse in moto}
\label{sec:onde-grav-masse}



%%% Local Variables:
%%% mode: latex
%%% TeX-master: "../astrofisica-teorica"
%%% End:
