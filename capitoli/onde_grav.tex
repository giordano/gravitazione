\chapter{Onde gravitazionali}
\label{cha:onde-grav}

Nel paragrafo~\ref{sec:campo-statico-sferico} abbiamo risolto le equazioni di
Einstein nel vuoto in un campo a simmetria sferica.  In questo capitolo ci
occuperemo della risoluzione delle equazioni di Einstein nel vuoto e
nell'approssimazione di campo debole, quindi ci serviremo di alcuni risultati
trovati nei paragrafi~\ref{sec:limite-newtoniano} e
\ref{sec:limite-newtoniano-einstein}.  Troveremo che il campo gravitazionale può
emettere radiazione, analogamente a quanto succede per il campo
elettromagnetico, come abbiamo visto nel
paragrafo~\ref{sec:onde-elettromagnetiche}.  Sebbene questo risultato sarà
ottenuto nell'approssimazione di campo debole, è un fatto del tutto generale, ma
la trattazione non approssimata presenta notevoli difficoltà matematiche legate
alla non linearità delle equazioni di Einstein.

\section{Approssimazione di campo debole}
\label{sec:approx-campo-debole-onde}

Ricordiamo che nell'approssimazione di campo debole il tensore metrico
covariante è dato da
\begin{equation}
  g_{\mu\nu} = \eta_{\mu\nu} + h_{\mu\nu},
\end{equation}
con $\abs{h_{\mu\nu}} \ll 1$, mentre il tensore metrico controvariante è
\begin{equation}
  g^{\mu\nu} = \eta^{\mu\nu} - h^{\mu\nu},
\end{equation}
con $h^{\mu\nu} = \eta^{\mu\lambda}\eta^{\mu\sigma} h^{\lambda\sigma}$.
L'innalzamento e l'abbassamento degli indici di $h_{\mu\nu}$ si esegue con il
tensore metrico non perturbato $\eta_{\mu\nu}$.
Dalla~\eqref{eq:christoffel-approx} abbiamo che la connessione affine
approssimata al primo ordine in $h_{\mu\nu}$ è
\begin{equation}
  \tensor{\Gamma}{^{\lambda}_{\mu\nu}} \approx \frac{1}{2}\eta^{\sigma\lambda}
  (\partial_{\mu}h_{\nu\sigma} + \partial_{\nu}h_{\mu\sigma}
  - \partial_{\sigma}h_{\mu\nu})
\end{equation}
e quindi il tensore di Ricci, trascurando i termini superiori al primo ordine in
$h_{\mu\nu}$, è dato da
\begin{equation}
  \begin{split}
    R_{\mu\nu} &= \partial_{\nu}\tensor{\Gamma}{^{\lambda}_{\mu\lambda}}
    - \partial_{\lambda} \tensor{\Gamma}{^{\lambda}_{\mu\nu}} +
    \overbrace{\tensor{\Gamma}{^{\eta}_{\mu\lambda}}
      \tensor{\Gamma}{^{\lambda}_{\eta\nu}} -
      \tensor{\Gamma}{^{\eta}_{\mu\nu}}
      \tensor{\Gamma}{^{\lambda}_{\eta\lambda}}}^{\mathcal{O}(h^{2})}
    \approx \partial_{\nu}\tensor{\Gamma}{^{\lambda}_{\mu\lambda}}
    - \partial_{\lambda} \tensor{\Gamma}{^{\lambda}_{\mu\nu}} \\
    &= \frac{1}{2}\partial_{\nu} (\eta^{\lambda\sigma}(\partial_{\mu}
    h_{\sigma\lambda} + \partial_{\lambda}h_{\mu\sigma}
    - \partial_{\sigma}h_{\mu\lambda})) -
    \frac{1}{2}\partial_{\lambda}(\eta^{\lambda\sigma}(\partial_{\mu}h_{\sigma\nu}
    + \partial_{\nu}h_{\mu\sigma} - \partial_{\sigma}h_{\mu\nu})) \\
    &= \frac{1}{2} \eta^{\lambda\sigma}(\partial_{\mu}\partial_{\nu}
    h_{\sigma\lambda} + \partial_{\lambda}\partial_{\nu} h_{\mu\sigma}
    - \partial_{\sigma}\partial_{\nu} h_{\mu\lambda}
    - \partial_{\mu}\partial_{\lambda} h_{\sigma\nu}
    - \partial_{\nu}\partial_{\lambda} h_{\mu\sigma}
    + \partial_{\sigma}\partial_{\lambda} h_{\mu\nu}) \\
    &= \frac{1}{2}(\partial_{\mu}\partial_{\nu}
    \tensor{h}{^{\lambda}_{\lambda}} - \partial^{\lambda}\partial_{\nu}
    h_{\mu\lambda} - \partial^{\lambda}\partial_{\mu} h_{\nu\lambda}
    + \partial_{\lambda}\partial^{\lambda} h_{\mu\nu}) \\
    &= \frac{1}{2} (\dalamb h_{\mu\nu} + \partial_{\mu}\partial_{\nu} h
    - \partial^{\lambda}\partial_{\nu} h_{\mu\lambda}
    - \partial_{\mu}\partial^{\nu} h_{\nu\lambda}),
  \end{split}
\end{equation}
in cui $h = \tensor{h}{^{\lambda}_{\lambda}}$ è la contrazione della
perturbazione.

%%% Local Variables:
%%% mode: latex
%%% TeX-master: "../astrofisica-teorica"
%%% End:
