\chapter{Onde gravitazionali}
\label{cha:onde-grav}

Nel paragrafo~\ref{sec:campo-statico-sferico} abbiamo risolto le equazioni di
Einstein nel vuoto in un campo gravitazionale a simmetria sferica.  In questo
capitolo ci occuperemo della risoluzione delle equazioni di Einstein
nell'approssimazione di campo debole, quindi ci serviremo di alcuni risultati
trovati nei paragrafi~\ref{sec:limite-newtoniano} e
\ref{sec:limite-newtoniano-einstein}.  Troveremo che il campo gravitazionale può
emettere radiazione analogamente a quanto succede per il campo elettromagnetico,
come abbiamo visto nei paragrafi~\ref{sec:onde-elettromagnetiche-vuoto} e
\ref{sec:onde-elettromagnetiche-cariche}.

Sebbene questo risultato sarà ottenuto nell'approssimazione di campo debole, è
un fatto del tutto generale, ma la trattazione non approssimata presenta
notevoli difficoltà matematiche legate alla non linearità delle equazioni di
Einstein.

\section{Pseudotensore energia-impulso del campo gravitazionale}
\label{sec:energ-impulso-grav}

Ricordiamo che nell'approssimazione di campo
debole il tensore metrico covariante è dato da
\begin{equation}
  g_{\mu\nu} = \eta_{\mu\nu} + h_{\mu\nu},
\end{equation}
con $\abs{h_{\mu\nu}} \ll 1$, mentre il tensore metrico controvariante è
\begin{equation}
  g^{\mu\nu} = \eta^{\mu\nu} - h^{\mu\nu},
\end{equation}
con $h^{\mu\nu} = \eta^{\mu\lambda}\eta^{\mu\sigma} h^{\lambda\sigma}$.
L'innalzamento e l'abbassamento degli indici di $h_{\mu\nu}$ si esegue con il
tensore metrico non perturbato $\eta_{\mu\nu}$.  Per linearizzare le equazioni
di Einsetin dobbiamo sviluppare $G_{\mu\nu}$ in serie di potenze di $h_{\mu\nu}$
e considerare solo i termini del primo ordine.

Innanzitutto osserviamo che possiamo sviluppare il tensore di Ricci in serie di
potenze di $h_{\mu\nu}$ nel seguente modo
\begin{equation}
  R_{\mu\nu} = R_{\mu\nu}^{(0)} + R_{\mu\nu}^{(1)} + R_{\mu\nu}^{(2)} +
  \mathcal{O}(h^{3}) = R_{\mu\nu}^{(1)} + R_{\mu\nu}^{(2)} + \mathcal{O}(h^{3}),
\end{equation}
con $R_{\mu\nu}^{(0)} = 0$ perché nel limite di campo gravitazionale
completamente assente il tensore di Riemann è nullo, così come quello di Ricci.
L'equazione di Einstein esatta, in presenza di materia, può essere scritta in
questo modo
\begin{equation}
  \label{eq:einstein-lineare-esatta}
  \bigg(R_{\mu\nu}^{(1)} - \frac{1}{2}\eta_{\mu\nu}R^{(1)}\bigg) = -8\pi
  G(T_{\mu\nu} + t_{\mu\nu}),
\end{equation}
in cui $R^{(1)} = \eta^{\mu\nu}R_{\mu\nu}^{(1)}$ e $t_{\mu\nu}$ è definito da
\begin{equation}
  8\pi G t_{\mu\nu} = R_{\mu\nu} - \frac{1}{2}\eta_{\mu\nu} R - R_{\mu\nu}^{(1)}
  + \frac{1}{2}\eta_{\mu\nu}R^{(1)} \approx R_{\mu\nu}^{(2)} -
  \frac{1}{2}(g_{\mu\nu}R)^{(2)}.
\end{equation}
Dunque $t_{\mu\nu}$ contiene solo termini superiori al primo ordine in
$h_{\mu\nu}$.  La~\eqref{eq:einstein-lineare-esatta} può essere interpretata
come un'equazione d'onda lineare in $h_{\mu\nu}$ in cui il termine di sorgente
$\tau_{\mu\nu} = T_{\mu\nu} + t_{\mu\nu}$ dipende esplicitamente dal campo
gravitazionale $h_{\mu\nu}$ stesso.  Questo ci suggerisce che $h_{\mu\nu}$ è
generato dal flusso e dalla densità totale dell'energia e del momento e
$t_{\mu\nu}$ rappresenta il ``tensore'' energia-impulso del solo campo
gravitazionale.\footnote{In realtà si dimostra che $t_{\mu\nu}$ si comporta come
  uno pseudotensore.  Vedi~\textcite[367-374]{landau:campi}.}
Allora $\tau_{\mu\nu}$, che differisce da $T_{\mu\nu}$ solo per termini del
secondo ordine in $h_{\mu\nu}$ derivanti da $R_{\mu\nu} - \eta_{\mu\nu}R/2$, può
essere interpretato come il ``tensore'' energia-impulso totale comprendente i
contributi della materia e del campo
gravitazionale.\footnote{Ulteriori prove a sostegno di questa interpretazione
  sono riportate in~\textcite[165-171]{weinberg:gravitation}.}

Trascurare nelle equazioni di Einstein i termini superiori al primo ordine di
infinitesimo significa pertanto trascurare i contributi dell'energia del campo
gravitazionale stesso.

\section{Onde gravitazionali nel vuoto}
\label{sec:onde-grav-vuoto}

In questo paragrafo risolviamo, in particolare, le equazioni di Einstein nel
vuoto, cioè con $\tau^{\mu\nu} = 0$.  Dalla~\eqref{eq:christoffel-approx}
abbiamo che la connessione affine approssimata al primo ordine in $h_{\mu\nu}$ è
\begin{equation}
  \tensor{\Gamma}{^{\lambda}_{\mu\nu}^{(1)}} \approx \frac{1}{2}\eta^{\sigma\lambda}
  (\partial_{\mu}h_{\nu\sigma} + \partial_{\nu}h_{\mu\sigma}
  - \partial_{\sigma}h_{\mu\nu})
\end{equation}
e quindi il tensore di Ricci, trascurando i termini superiori al primo ordine in
$h_{\mu\nu}$, è dato da
\begin{equation}
  \begin{split}
    R_{\mu\nu}^{(1)} &= \partial_{\nu}
    \tensor{\Gamma}{^{\lambda}_{\mu\lambda}^{(1)}}
    - \partial_{\lambda} \tensor{\Gamma}{^{\lambda}_{\mu\nu}^{(1)}} +
    \overbrace{\tensor{\Gamma}{^{\eta}_{\mu\lambda}^{(1)}}
      \tensor{\Gamma}{^{\lambda}_{\eta\nu}^{(1)}} -
      \tensor{\Gamma}{^{\eta}_{\mu\nu}^{(1)}}
      \tensor{\Gamma}{^{\lambda}_{\eta\lambda}^{(1)}}}^{\mathcal{O}(h^{2})} \\
    &\approx \partial_{\nu}\tensor{\Gamma}{^{\lambda}_{\mu\lambda}^{(1)}}
    - \partial_{\lambda} \tensor{\Gamma}{^{\lambda}_{\mu\nu}^{(1)}} \\
    &= \frac{1}{2}\partial_{\nu} (\eta^{\lambda\sigma}(\partial_{\mu}
    h_{\sigma\lambda} + \partial_{\lambda}h_{\mu\sigma}
    - \partial_{\sigma}h_{\mu\lambda})) -
    \frac{1}{2}\partial_{\lambda}(\eta^{\lambda\sigma}(\partial_{\mu}h_{\sigma\nu}
    + \partial_{\nu}h_{\mu\sigma} - \partial_{\sigma}h_{\mu\nu})) \\
    &= \frac{1}{2} \eta^{\lambda\sigma}(\partial_{\mu}\partial_{\nu}
    h_{\sigma\lambda} + \partial_{\lambda}\partial_{\nu} h_{\mu\sigma}
    - \partial_{\sigma}\partial_{\nu} h_{\mu\lambda}
    - \partial_{\mu}\partial_{\lambda} h_{\sigma\nu}
    - \partial_{\nu}\partial_{\lambda} h_{\mu\sigma}
    + \partial_{\sigma}\partial_{\lambda} h_{\mu\nu}) \\
    &= \frac{1}{2}(\partial_{\mu}\partial_{\nu}
    \tensor{h}{^{\lambda}_{\lambda}} - \partial^{\lambda}\partial_{\nu}
    h_{\mu\lambda} - \partial^{\lambda}\partial_{\mu} h_{\nu\lambda}
    + \partial_{\lambda}\partial^{\lambda} h_{\mu\nu}) \\
    &= \frac{1}{2} (\dalamb h_{\mu\nu} + \partial_{\mu}\partial_{\nu} h
    - \partial^{\lambda}\partial_{\nu} h_{\mu\lambda}
    - \partial_{\mu}\partial^{\lambda} h_{\nu\lambda}),
  \end{split}
\end{equation}
in cui $h = \tensor{h}{^{\lambda}_{\lambda}} = \eta^{\mu\lambda} h_{\mu\lambda}$
è la traccia della perturbazione.  Nel vuoto allora abbiamo
\begin{equation}
  R_{\mu\nu}^{(1)} = \frac{1}{2} (\dalamb h_{\mu\nu}
  + \partial_{\mu}\partial_{\nu} h - \partial^{\lambda}\partial_{\nu}
  h_{\mu\lambda} - \partial_{\mu}\partial^{\lambda} h_{\nu\lambda}) = 0.
\end{equation}

A questo punto vogliamo manipolare l'espressione ottenuta per il tensore di
Ricci in modo che le equazioni di Einstein nel vuoto si linearizzino, diventando
analoghe alle equazioni di Maxwell nel vuoto $\dalamb A_{\beta} = 0$.  Come nel
caso elettromagnetico, è possibile fare ciò scegliendo una gauge opportuna.  In
questo caso conviene scegliere la \index{gauge!armonica}\emph{gauge armonica}
\begin{equation}
  \label{eq:gauge-armonica}
  \partial_{\mu} \tensor{h}{^{\mu}_{\nu}} =
  \frac{1}{2}\partial_{\nu}\tensor{h}{^{\mu}_{\mu}}.
\end{equation}
Infatti usando questa gauge gli ultimi tre termini del tensore di Ricci si
annullano
\begin{equation}
  \begin{split}
    \partial_{\mu}\partial_{\nu} h - \partial^{\lambda}\partial_{\nu}
    h_{\mu\lambda} - \partial_{\mu}\partial^{\lambda} h_{\nu\lambda}
    &= \partial_{\mu}(2 \partial_{\lambda}\tensor{h}{^{\lambda}_{\nu}})
    - \partial_{\nu}\partial_{\lambda}\tensor{h}{^{\lambda}_{\mu}}
    - \partial_{\lambda}\partial_{\mu} \tensor{h}{^{\lambda}_{\nu}} \\
    &= \partial_{\mu}\partial_{\lambda} \tensor{h}{^{\lambda}_{\nu}}
    - \partial_{\nu}\partial_{\lambda} \tensor{h}{^{\lambda}_{\mu}} \\
    &= \partial_{\mu}\bigg(\frac{1}{2}\partial_{\nu}
    \tensor{h}{^{\lambda}_{\lambda}}\bigg)
    - \partial_{\nu}\bigg(\frac{1}{2}\partial_{\mu}
    \tensor{h}{^{\lambda}_{\lambda}}\bigg) = 0
  \end{split}
\end{equation}
e quindi $R_{\mu\nu}^{(1)} = \dalamb h_{\mu\nu}/2$.  Le equazioni di Einstein
nel vuoto si possono, in questo modo, scrivere nella seguente forma lineare
\begin{equation}
  \label{eq:einstein-lineare}
  \dalamb h_{\mu\nu} = 0.
\end{equation}
Osserviamo che la condizione di gauge armonica si può anche porre nella forma
$\eta^{\mu\nu}\tensor{\Gamma}{^{\lambda}_{\mu\nu}} = 0$ in quanto
\begin{equation}
  \begin{split}
    0 &=\eta^{\mu\nu} \tensor{\Gamma}{^{\lambda}_{\mu\nu}} = \frac{1}{2}
    \eta^{\lambda\sigma} (\partial^{\mu}h_{\mu\sigma} + \partial^{\mu}
    h_{\mu\sigma} - \partial_{\sigma}\tensor{h}{^{\mu}_{\mu}}) \\
    &= \frac{1}{2} (\partial^{\mu} \tensor{h}{_{\mu}^{\lambda}} + \partial^{\mu}
    \tensor{h}{_{\mu}^{\lambda}} - \partial^{\lambda} \tensor{h}{^{\mu}_{\mu}})
    \iff \partial^{\lambda} \tensor{h}{^{\mu}_{\mu}} = 2 \partial^{\mu}
    \tensor{h}{_{\mu}^{\lambda}}
  \end{split}
\end{equation}
che equivale alla~\eqref{eq:gauge-armonica}.

Si può sempre scegliere un sistema di riferimento in cui sia valida la
condizione armonica.  Sia $x^{\mu}$ un sistema di coordinate in cui questa
condizione non è valida, allora possiamo effettuare la trasformazione
infinitesima $x^{\mu} \to x'^{\mu} = x^{\mu} + \epsilon^{\mu}(x)$, con
$\abs{\epsilon^{\mu}(x)} \ll 1$, richiedendo che la metrica si possa ancora
sviluppare come $g'_{\mu\nu} = \eta_{\mu\nu} + h'_{\mu\nu}$, in modo da poter
ancora ottenere la forma linearizzata~\eqref{eq:einstein-lineare} delle
equazioni di Einstein.  Possiamo determinare un'espressione esplicita per
$h'_{\mu\nu}$ imponendo che il tensore metrico controvariante nel nuovo sistema
$g'^{\mu\nu} = \eta^{\mu\nu} - h'^{\mu\nu}$ si trasformi come un tensore
controvariante di rango $2$.  Trascurando termini del secondo ordine in
$\epsilon$ o dell'ordine di $h\epsilon$ abbiamo
\begin{equation}
  \begin{split}
    g'^{\mu\nu} &=
    g^{\lambda\rho} \parder{x'^{\mu}}{x^{\lambda}} \parder{x'^{\nu}}{x^{\rho}} =
    (\eta^{\lambda\rho} + h^{\lambda\rho}) (\tensor{\delta}{^{\mu}_{\lambda}}
    + \partial_{\lambda}\epsilon^{\mu}) (\tensor{\delta}{^{\nu}_{\rho}}
    + \partial_{\rho} \epsilon^{\nu}) \\
    &= (\eta^{\mu\rho} - h^{\mu\rho} +
    \eta^{\lambda\rho}\partial_{\lambda}\epsilon^{\mu} -
    \underbrace{h^{\lambda\rho} \partial_{\lambda}
      \epsilon^{\mu}}_{\mathcal{O}(h\epsilon)})
    (\tensor{\delta}{^{\nu}_{\rho}} + \partial_{\rho} \epsilon^{\nu}) \\
    &\approx \eta^{\mu\nu} - h^{\mu\nu} + \eta^{\lambda\nu}\partial_{\lambda}
    \epsilon^{\mu} + \eta^{\mu\rho}\partial_{\rho} \epsilon^{\nu} -
    \underbrace{h^{\mu\rho}\partial_{\rho}\epsilon^{\nu}}_{\mathcal{O}(h\epsilon)}
    + \underbrace{\partial^{\rho} \epsilon^{\mu} \partial_{\rho}
      \epsilon^{\nu}}_{\mathcal{O}(\epsilon^{2})} \\
    & \approx \eta^{\mu\nu} - h^{\mu\nu} + \partial^{\nu} \epsilon^{\mu}
    + \partial^{\mu} \epsilon^{\nu}.
  \end{split}
\end{equation}
Dunque
\begin{equation}
  h'^{\mu\nu} = h^{\mu\nu} - \partial^{\nu} \epsilon^{\mu} - \partial^{\mu}
  \epsilon^{\nu}.
\end{equation}
In questo nuovo sistema di riferimento calcoliamo ambo i membri
della~\eqref{eq:gauge-armonica}
\begin{equation}
  \begin{split}
    &\partial_{\mu} \tensor{{h'}}{^{\mu}_{\nu}} = \frac{1}{2}\partial_{\nu}
    \tensor{{h'}}{^{\mu}_{\mu}} \iff \\
    &\partial_{\mu} \tensor{h}{^{\mu}_{\nu}}
    - \partial_{\mu} \partial_{\nu} \epsilon^{\mu}
    - \partial_{\mu}\partial^{\mu} \epsilon_{\nu}
    = -\frac{1}{2} \partial_{\nu} \partial_{\mu} \epsilon^{\mu} -
    \frac{1}{2}\partial_{\nu} \partial_{\mu} \epsilon^{\mu} +
    \frac{1}{2} \partial_{\nu} \tensor{h}{^{\mu}_{\mu}} \iff \\
    &\partial_{\mu} \tensor{h}{^{\mu}_{\nu}}
    - \partial_{\mu}\partial^{\mu} \epsilon_{\nu}
    = \frac{1}{2} \partial_{\nu} \tensor{h}{^{\mu}_{\mu}}.
  \end{split}
\end{equation}
Quindi se scegliamo il generatore $\epsilon_{\nu}$ della trasformazione
infinitesima delle coordinate tale che
\begin{equation}
  \dalamb \epsilon_{\nu} = \partial_{\mu} \tensor{h}{^{\mu}_{\nu}} -
  \frac{1}{2} \partial_{\nu} \tensor{h}{^{\mu}_{\mu}}
\end{equation}
allora nel nuovo sistema di riferimento la condizione
armonica~\eqref{eq:gauge-armonica} sarà soddisfatta.  Se la condizione era già
valida nel sistema di partenza, bisognerà allora porre
$\dalamb \epsilon_{\nu} = 0$.  Questa libertà di scelta ricorda la libertà che
si ha nella gauge di Lorenz per il potenziale elettromagnetico.

Abbiamo visto che con la gauge armonica~\eqref{eq:gauge-armonica} l'equazione di
Einstein nel vuoto può essere scritta nella forma
lineare~\eqref{eq:einstein-lineare}, che è del tutto analoga alle equazioni di
Maxwell~\eqref{eq:maxwell-vuoto-lorenz} nel vuoto con la gauge di Lorenz, quindi
anche per $h_{\mu\nu}$ la soluzione generale sarà una sovrapposizione lineare di
onde piane del tipo
\begin{equation}
  h_{\mu\nu} = \Re\{e_{\mu\nu} \e^{\uimm k^{\lambda}x_{\lambda}} \}.
\end{equation}
Queste sono le \index{onda!gravitazionale}\emph{onde gravitazionali}.
Dalla~\eqref{eq:einstein-lineare} abbiamo (omettendo il simbolo $\Re$ per quanto
detto nel paragrafo~\ref{sec:onde-elettromagnetiche-vuoto})
\begin{equation}
  0 = \dalamb h_{\mu\nu} = \uimm k^{\lambda} \uimm k_{\lambda} e_{\mu\nu}
  \e^{\uimm k^{\lambda}x_{\lambda}} \implies k^{\lambda}k_{\lambda} = 0,
\end{equation}
cioè le onde si muovono, come quelle elettromagnetiche, alla velocità della luce
nel vuoto.  Inoltre dalla gauge armonica~\eqref{eq:gauge-armonica} abbiamo
\begin{equation}
  \uimm k_{\lambda} \tensor{e}{^{\lambda}_{\nu}} \e^{\uimm
    k^{\lambda}x_{\lambda}} = \frac{\uimm}{2} k_{\nu}
  \tensor{e}{^{\lambda}_{\lambda}} \e^{\uimm k^{\lambda}x_{\lambda}} \iff
  k_{\lambda} \tensor{e}{^{\lambda}_{\nu}} = \frac{1}{2} k_{\nu}
  \tensor{e}{^{\lambda}_{\lambda}}.
\end{equation}
Dal momento che $h_{\mu\nu}$ è simmetrico nello scambio degli indici, lo dovrà
essere anche $e_{\mu\nu}$, chiamato \index{tensore!di polarizzazione}
\emph{tensore di polarizzazione}, il quale quindi avrà indipendenti al più dieci
delle sue sedici componenti.  Al variare di $\nu$ nell'equazione precedente ci
sono quattro ulteriori relazioni che legano le componenti del tensore di
polarizzazione, quindi il numero di componenti indipendenti non può superare
sei.

Scegliamo l'asse $z$ lungo la direzione di propagazione dell'onda, di modo che
il vettore d'onda controvariante abbia la forma
$k^{\mu} = (k^{0}, k^{1}, k^{2}, k^{3}) = (k, 0, 0, k)$.  Ponendo nell'equazione
precedente $\nu = 1$ abbiamo
\begin{equation}
  k_{\mu}\tensor{e}{^{\mu}_{1}} = \frac{1}{2} k_{1}\tensor{e}{^{\mu}_{\mu}} = 0.
\end{equation}
Esplicitiamo il primo membro
\begin{equation}
  0 = k_{0} \tensor{e}{^{0}_{1}} + k_{3} \tensor{e}{^{3}_{1}} = -k
  \tensor{e}{^{0}_{1}} + k \tensor{e}{^{3}_{1}} \implies
  \tensor{e}{^{0}_{1}} = \tensor{e}{^{3}_{1}} \implies e_{01} = -e_{31}.
\end{equation}
Analogamente per $\nu = 2$ si trova
\begin{equation}
  k_{\mu} \tensor{e}{^{\mu}_{2}} = \frac{1}{2} k_{2} \tensor{e}{^{\mu}_{\mu}} =
  0 \implies e_{02} = -e_{32}.
\end{equation}
Per $\nu = 0$ risulta
\begin{equation}
  \begin{split}
    &k_{\mu} \tensor{e}{^{\mu}_{0}} = \frac{1}{2} k_{0}\tensor{e}{^{\mu}_{\mu}}
    = -\frac{1}{2}k^{0}\tensor{e}{^{\mu}_{\mu}} \implies \\
    &k^{0} e_{00} + k^{3}e_{30} = -\frac{1}{2}k^{0}(\tensor{e}{^{1}_{1}} +
    \tensor{e}{^{2}_{2}} + \tensor{e}{^{3}_{3}} - e_{00})
  \end{split}
\end{equation}
e poiché per le componenti diagonali si ha (stiamo considerando le singole
componenti del tensore $e$, non la sua traccia, quindi non si somma su $i$)
$\tensor{e}{^{i}_{i}} = \eta^{ij}e_{ij} = \delta^{ij}e_{ij} = e_{ii}$ ricaviamo
\begin{equation}
  e_{00} + e_{30} = -\frac{1}{2}(e_{11} + e_{22} + e_{33} - e_{00}).
\end{equation}
In maniera simile, per $\nu = 3$ si trova
\begin{equation}
  e_{03} + e_{33} = \frac{1}{2}(e_{11} + e_{22} + e_{33} - e_{00}).
\end{equation}
Sommando membro a membro le ultime due equazioni e sfruttando la simmetria del
tensore di polarizzazione abbiamo
\begin{equation}
  e_{03} = -\frac{1}{2}(e_{33} + e_{00})
\end{equation}
e inserendo questo risultato nell'equazione trovata per $\nu = 3$ otteniamo
\begin{equation}
  e_{22} = -e_{11}.
\end{equation}
Riepilogando, fino a questo punto abbiamo trovato che per un'onda gravitazionale
la quale si propaga lungo l'asse $z$ si ha
\begin{subequations}
  \begin{align}
    e_{01} &= -e_{31}, \\
    e_{02} &= -e_{32}, \\
    e_{03} &= -\frac{1}{2}(e_{33} + e_{00}), \\
    e_{22} &= -e_{11}.
  \end{align}
\end{subequations}
Abbiamo dedotto tutto ciò nell'ipotesi che sia valida la condizione
armonica~\eqref{eq:gauge-armonica}.  Effettuiamo una trasformazione infinitesima
delle coordinate $x^{\mu} \to x'^{\mu} = x^{\mu} + \epsilon^{\mu}(x)$.  Affinché
sia ancora valida l'equazione di Einstein linearizzata e la gauge armonica
abbiamo visto che deve valere $\dalamb \epsilon^{\mu} = 0$.  Poiché anche
$\epsilon^{\mu}$ soddisfa l'equazione delle onde, la sua soluzione sarà data
dalla sovrapposizione lineare di onde del tipo
\begin{equation}
  \epsilon^{\mu} = \Re\{\tilde\epsilon^{\mu}\e^{\uimm k^{\lambda}x_{\lambda}}\}.
\end{equation}
La perturbazione della metrica in questo nuovo sistema è
\begin{equation}
  h'_{\mu\nu} = \Re\{e'_{\mu\nu} \e^{\uimm k^{\lambda}x_{\lambda}}\} =
  h_{\mu\nu} - \partial_{\nu}\epsilon_{\mu} - \partial_{\mu}\epsilon_{\nu}
\end{equation}
e poiché
$\partial_{\nu}\epsilon_{\mu} = \uimm k_{\nu}\tilde\epsilon_{\mu}
\e^{\uimm k^{\lambda}x_{\lambda}}$ si ha
\begin{equation}
  e'_{\mu\nu} = e_{\mu\nu} - \uimm k_{\nu}\tilde\epsilon_{\mu} - \uimm
  k_{\mu}\tilde\epsilon_{\nu}.
\end{equation}
Così risulta
\begin{subequations}
  \begin{align}
    e'_{11} &= e_{11} - \uimm k_{1} \tilde\epsilon_{1} - \uimm
    k_{1}\tilde\epsilon_{1} = e_{11}, \\
    e'_{12} &= e_{12} - \uimm k_{2}\tilde\epsilon_{1} - \uimm
    k_{1}\tilde\epsilon_{2} = e_{12}, \\
    e'_{13} &= e_{13} - \uimm k_{3}\tilde\epsilon_{1} - \uimm
    k_{1}\tilde\epsilon_{3} = e_{13} - \uimm k\tilde\epsilon_{1} = 0, \\
    e'_{23} &= e_{23} - \uimm k_{3}\tilde\epsilon_{2} - \uimm
    k_{2}\tilde\epsilon_{3} = e_{23} - \uimm k\tilde\epsilon_{2} = 0, \\
    e'_{33} &= e_{33} - 2\uimm k_{3}\tilde\epsilon_{3} = 0, \\
    e'_{00} &= e_{00} - 2\uimm k_{0}\tilde\epsilon_{0} = 0,
  \end{align}
\end{subequations}
avendo posto $\tilde\epsilon_{1} = -\uimm e_{13}/k$,
$\tilde\epsilon_{2} = -\uimm e_{23}/k$,
$\tilde\epsilon_{3} = -\uimm e_{33}/(2k)$ e
$\tilde\epsilon_{0} = \uimm e_{00}/(2k)$.  Inoltre avevamo visto che
$e'_{01} = -e'_{31} = 0$, $e'_{02} = -e'_{32} = 0$ e
$e'_{03} = -(e'_{33}+e'_{00})/2 = 0$, quindi per un'onda che si propaga lungo
l'asse $z$ rimangono diversi da zero solo le componenti $e_{22} = -e_{11}$ e
$e_{12} = e_{21}$.  Abbiamo trovato che la simmetria del tensore di
polarizzazione riduce da sedici a dieci il numero di sue componenti
indipendenti, la condizione armonica diminuisce questo numero a sei e la libertà
di scelta sulla trasformazione di gauge fissa il numero di componenti
indipendenti a due, in maniera analoga a quello che succede per le onde
elettromagnetico nel vuoto.  Sono quindi due gli stati di polarizzazione fisica
delle onde gravitazionali.  Questa scelta di coordinate determina la
\index{gauge!trasversa a traccia nulla}\emph{gauge trasversa a traccia nulla}
(gauge TT): ``a traccia nulla'' perché
$\tensor{e}{^{\mu}_{\mu}} = \tensor{e}{^{0}_{0}} + \tensor{e}{^{1}_{1}} +
\tensor{e}{^{2}_{2}} + \tensor{e}{^{3}_{3}} = \tensor{e}{^{1}_{1}} +
\tensor{e}{^{2}_{2}} = 0$,
``trasversa'' perché le componenti non nulle del tensore di polarizzazione sono
sulle direzioni perpendicolari alla direzione di propagazione dell'onda (per
un'onda gravitazionale che si muovo lungo l'asse $z$ sono non nulle le
componenti associate agli assi $x$ e $y$).  Poiché i tensori di polarizzazione
rappresentano l'ampiezza delle onde da cui è costituito $h_{\mu\nu}$, anche
$h_{\mu\nu}$ sarà trasverso e con traccia $h = \tensor{h}{^{\mu}_{\mu}}$ nulla.
In particolare, per un'onda che si muove lungo l'asse $z$, $h_{\mu\nu}$ è dato
dalla parte reale di
\begin{equation}
  \begin{pmatrix}
    0 & 0      & 0       & 0 \\
    0 & e_{11} & e_{12}  & 0 \\
    0 & e_{12} & -e_{11} & 0 \\
    0 & 0      & 0       & 0
  \end{pmatrix}\cdot \e^{\uimm k^{\lambda}x_{\lambda}}.
\end{equation}
A volte si scrive $h_{\mu\nu}^{\textup{TT}}$ per specificare che si è scelta la
gauge trasversa a traccia nulla.

In teoria dei campi si fa la seguente classificazione:
\begin{itemize}
\item un \emph{campo scalare} è un campo che rimane invariato per trasformazioni
  arbitrarie delle coordinate.  Un campo scalare ha spin $0$;
\item un \emph{campo vettoriale} è un campo che, per effetto di un cambiamento
  arbitrario delle coordinate, si trasforma come un vettore.  Un campo
  vettoriale ha spin $1$;
\item un \emph{campo tensoriale} è un campo che si trasforma come un tensore di
  rango $2$ sotto un cambiamento di coordinate.  Un campo tensoriale ha spin
  $2$.
\end{itemize}
Possiamo quindi osservare che i fotoni, i quanti del campo elettromagnetico,
sono descritti dal campo vettoriale $A_{\mu}$ e quindi hanno spin $1$ e massa
nulla in quanto il modulo quadro $k^{\mu}k_{\mu}$ del vettore d'onda è nullo,
come visto nel paragrafo~\ref{sec:onde-elettromagnetiche-vuoto}.  Gli ipotetici
quanti del campo gravitazionale, i \index{gravitone}\emph{gravitoni}, sono
descritti dal campo tensoriale $h_{\mu\nu}$, quindi sono particelle con spin $2$
e anch'essi massa nulla perché $k^{\mu}k_{\mu} = 0$.

% TODO: completare, vedi appunti del 25/05/2012
\section{\completare{Tecniche di rivelazione delle onde gravitazionali}}
\label{sec:rivelazione-onde}

Consideriamo due particelle $A$ e $B$, descritte rispettivamente dalle linee
d'universo $x^{\mu}_{A} = x^{\mu}(\tau)$ e
$x^{\mu}_{B} = x^{\mu}(\tau) + \delta x^{\mu}(\tau)$, cioè il vettore
$\delta x^{\mu}(\tau)$ rappresenta la separazione fra $A$ e $B$.  Dall'equazione
di deviazione geodetica % TODO: mettere riferimento
sappiamo che
\begin{equation}
  \curder[2]{\delta x^{\lambda}} = \tensor{R}{^{\lambda}_{\nu\mu\rho}} \delta
  x^{\mu} \toder{x^{\nu}}{\tau} \toder{x^{\rho}}{\tau}.
\end{equation}
Abbiamo poi visto che nell'approssimazione di campo debole le uniche componenti
non nulle del tensore di Riemann sono
$R_{j0k0} = R_{0j0k} = -R_{j00k} = -R_{0jk0}$.  Inoltre nella gauge TT tutte le
componenti $h_{\mu 0} = h_{0\mu}$ sono nulle, quindi
dalla~\eqref{eq:riemann-approx} abbiamo
\begin{equation}
  R_{j0k0}^{(1)} = R_{0j0k}^{(1)} = \frac{1}{2}(\partial_{k}\partial_{j}h_{00}
  - \partial_{k}\partial_{0}h_{j0} - \partial_{0}\partial_{j}h_{0k}
  + \partial_{0}\partial_{0}h_{jk}) = \frac{1}{2}\ddot{h}^{\textup{TT}}_{jk,00}.
\end{equation}
Se i corpi si muovono con velocità piccole, molto minori di quella della luce
nel vuoto, allora $\dd\tau \approx\dd t$ e l'equazione di deviazione geodetica
diventa
\begin{equation}
  \toder[2]{\delta x^{j}}{t} = \frac{1}{2} \parder{}{*{2}{t}}
  \tensor{h}{^{j}_{k}}\delta x^{k}.
\end{equation}
Se le due particelle sono a riposo a una distanza costante $\delta x^{j}(0)$
prima che l'onda arrivi (cioè con $h_{jk} = 0$), la soluzione di questa
equazione è
% Per la risoluzione di questa equazione vedi anche MTW, pagina 951
\begin{equation}
  \label{eq:spostamento-onda-grav}
  \delta x^{j}(t) \approx \delta x^{k}(0) \bigg(\tensor{\delta}{^{j}_{k}} +
  \frac{1}{2}\tensor{h}{^{j}_{k}}(x^{\mu}(t))\bigg).
\end{equation}
Questo risultato ci permette di analizzare l'effetto di un'onda gravitazionale
che investe un sistema di due corpi inizialmente in quiete.

\begin{figure}
  \centering
  \subfloat[][Onda longitudinale al sistema\label{fig:onda-grav1}]{
    \begin{tikzpicture}[font=\small,scale=0.9]
      \fill[black] (3,2.5) node[label={right:$A$}] (A) {} circle (0.08) ++(0,2)
      node[label={right:$B$}] (B) {} circle (0.08);
      \draw[->,decorate,decoration=snake] (3,0.5) -- ++(0,6);% onda gravitaz
      \draw[<->] ($(A) - (0.5,0)$) -- node[above,sloped] {$\delta x^{j}$}
      ($(B) - (0.5,0)$);
      \draw[->] (0,0) -- ++(0,7) node[left] {$z$}; % asse z
      \draw[->] (0,0) -- ++(6,0) node[below] {$y$}; % asse y
    \end{tikzpicture}}
  \qquad
  \subfloat[][Onda trasversale al sistema\label{fig:onda-grav2}]{
    \begin{tikzpicture}[font=\small,scale=0.9]
      \fill[black] (1,3.5) node[label={above:$A$}] (A) {}
      circle (0.08) ++(4,0) node[label={above:$B$}] (B) {} circle (0.08);
      \draw[->,decorate,decoration=snake] (3,0.5) -- ++(0,6);% onda gravitaz
      \draw[<->] ($(A) - (0,0.5)$) -- node[below left] {$\delta x^{j}$}
      ($(B) - (0,0.5)$);
      \draw[->] (0,0) -- ++(0,7) node[left] {$z$}; % asse z
      \draw[->] (0,0) -- ++(6,0) node[below] {$y$}; % asse y
    \end{tikzpicture}}
  \caption[Onda gravitazionale incidente su un sistema di due particelle]{Onda
    gravitazionale (rappresentata da \tikz \draw[->,decorate,decoration=snake]
    (0,0) -- ++(1.5,0);) che investe un sistema costituito da due corpi}
  \label{fig:onda-grav}
\end{figure}
Consideriamo un'onda gravitazionale che si propaga lungo la direzione $z$, per
la quale allora le uniche componenti non nulle di $h_{\mu\nu}$ nella gauge TT
sono
\begin{subequations}
  \begin{align}
    h_{11}^{\textup{TT}} &= - h_{22}^{\textup{TT}}, \\
    h_{12}^{\textup{TT}} &= h_{21}^{\textup{TT}}.
  \end{align}
\end{subequations}
Supponiamo che i due corpi siano allineati nella direzione $z$ a una posizione
iniziale reciproca costante $\delta x^{j}(0) = (0,0,L_{0})$ come nella
figura~\ref{fig:onda-grav1}, allora dalla~\eqref{eq:spostamento-onda-grav}
abbiamo che dopo il passaggio dell'onda
\begin{equation}
  \delta x^{j}(t) = \delta x^{k}(0)\bigg(\tensor{\delta}{^{j}_{k}} + \frac{1}{2}
  \tensor{h}{^{j}_{k}}\bigg) = \delta x^{3}(0)\bigg(\tensor{\delta}{^{j}_{3}} +
  \frac{1}{2}\tensor{h}{^{j}_{3}}\bigg) = \delta x^{3}(0)
  \tensor{\delta}{^{j}_{3}},
\end{equation}
cioè la separazione rimane costante $\delta x^{j}(t) = \delta x^{j}(0)$ lungo
ogni direzione $j$.  Supponiamo ora che i due corpi siano allineati lungo una
direzione perpendicolare a $z$, per esempio $y$, con separazione iniziale
$\delta x^{j}(0) = (0,L_{0},0)$, come nella figura~\ref{fig:onda-grav2}.  Dopo
il passaggio dell'onda, dalla~\eqref{eq:spostamento-onda-grav} risulta
\begin{equation}
  \delta x^{j}(t) = \delta x^{k}(0) \bigg(\tensor{\delta}{^{j}_{k}} +
  \frac{1}{2}\tensor{h}{^{j}_{k}}(x^{\mu}(t))\bigg) = \delta
  x^{2}(0)\bigg(\tensor{\delta}{^{j}_{2}} +
  \frac{1}{2}\tensor{h}{^{j}_{2}}(x^{\mu}(t))\bigg),
\end{equation}
quindi solo le separazioni fra i due corpi lungo $x$ e lungo $y$ cambiano: le
onde gravitazionali sono trasversali anche negli effetti fisici, oltre che nella
loro descrizione matematica.  Questi fatti ci suggeriscono delle possibili
tecniche per la rivelazione diretta delle onde gravitazionali.  Tuttavia, prima
di illustrarle, dobbiamo osservare che il rapporto fra la variazione $\var L$
della distanza fra i corpi, indotta dal passaggio dell'onda, e la distanza
iniziale $L_{0}$ è dell'ordine di
\begin{equation}
  \frac{\var L}{L_{0}} \sim h
\end{equation}
che è generalmente una quantità molto piccola.  Come vedremo, la precisione
richiesta negli esperimenti che si propongono di mettere in evidenza le onde
gravitazionali in questo modo è molto elevata.

\section{Onde gravitazionali in presenza di masse in moto}
\label{sec:onde-grav-masse}

\emph{Nota: in questo paragrafo esplicitiamo la velocità $c$ della luce nel
  vuoto.}

Studiamo adesso un campo gravitazionale debole generato da masse in movimento
con velocità $v \ll c$.  Possiamo ottenere immediatamente delle stime degli
ordini di grandezza delle quantità coinvolte ricordando che nello studio delle
onde elettromagnetiche in presenza di cariche in moto
(paragrafo~\ref{sec:radiazione-dipolo}) abbiamo trovato che i termini principali
nello sviluppo in multipoli dell'energia emessa dal sistema per unità di tempo è
dato dalla formula~\eqref{eq:potenza-onde-em}
\begin{equation}
  I_{\textup{em}} = \underbrace{\frac{2}{3c^{3}}
    (\ddot{\bm{d}}_{\textup{em}})^{2}}_{\text{dipolo di carica}} +
  \underbrace{\frac{2}{3c^{3}} (\ddot{\bm{m}}_{\textup{em}})^{2}}_{\text{dipolo
      magnetico}} +
  \underbrace{\frac{1}{180c^{5}}
    \dddot{Q}_{ij,\textup{em}}\dddot{Q}^{ij,\textup{em}}}_{\text{quadrupolo di
      carica}} + \cdots
\end{equation}
con $\bm{d}_{\textup{em}} = \sum_{k} q_{k} \bm{r}_{k}$ momento di dipolo di
carica,
$\bm{m}_{\textup{em}} = \sum_{k} q_{k} \bm{r}_{k} \times \bm{v}_{k}/(2c)$
momento di dipolo magnetico e
$Q_{ij,\textup{em}} = \sum_{k} q_{k}(3x_{i}x_{j} -\delta_{ij}r_{k}^{2})$ momento
di quadrupolo di carica.  Per passare al caso gravitazionale possiamo effettuare
la sostituzione $\text{carica} \to \text{massa}$, vale a dire $q_{k} \to m_{k}$.
In questo modo
\begin{subequations}
  \begin{align}
    \bm{d}_{\textup{g}} &= \sum_{k} m_{k} \bm{r}_{k} = M \bm{R}_{\textup{cm}},\\
    \begin{split}
      \bm{m}_{\textup{g}} &= \sum_{k} (\text{posizione del corpo $k$}) \times
      (\text{corrente dovuta al corpo $k$}) \\
      &= \sum_{k} \bm{r}_{k} \times (m_{k} \bm{v}_{k}) = \bm{L},
    \end{split} \\
    Q_{ij,\textup{g}} &= \sum_{k} m_{k}(3x_{i}x_{j} -
    \delta_{ij}r_{k}^{2}),
  \end{align}
\end{subequations}
in cui $M = \sum_{k} m_{k}$ è la massa totale delle particelle del sistema,
$\bm{R}_{\textup{cm}} = \sum_{k} m_{k} \bm{r}_{k}/M$ è la posizione del centro
di massa e $\bm{L}$ è il momento angolare totale.  Per un sistema non
relativistico isolato, cioè per nel quale non agiscono forze esterne né è
presente un momento torcente esterno diverso da zero, risulta
\begin{subequations}
  \begin{align}
    \ddot{\bm{d}}_{\textup{g}} &= M \ddot{\bm{R}}_{\textup{cm}} =
    \dot{\bm{P}} = \bm{0}, \\
    \dot{\bm{m}}_{\textup{g}} &= \dot{\bm{L}} = \bm{\tau} = \bm{0},
  \end{align}
\end{subequations}
con $\bm{P}$ quantità di moto totale del sistema e $\bm{\tau}$ momento torcente
totale.  Dunque per un sistema di masse in moto con velocità $v \ll c$ che
emettono onde gravitazionali i due termini di dipoli, i principali nello
sviluppo in multipoli dell'energia emessa, sono nulli e il termine dominante
sarà quello di quadrupolo, del tipo
\begin{equation}
  \frac{1}{45c^{5}}\dddot{Q}_{ij}\dddot{Q}^{ij}.
\end{equation}
Il fattore numerico $45$, invece di $180$ del caso elettromagnetico, è legato al
fatto che il campo gravitazionale ha spin $2$, mentre il campo elettromagnetico
ha spin $1$.\footnote{Vedi \textcite[471]{shapiro:black-holes}.}

Esaminiamo più in dettaglio il problema.  Sappiamo che nell'approssimazione di
campo debole $R_{\mu\nu}^{(1)} = \dalamb h_{\mu\nu}/2$, quindi l'equazione di
Einstein in presenza della materia~\eqref{eq:einstein-lineare-esatta} diventa
\begin{equation}
  \label{eq:einstein-lineare-materia1}
  \dalamb \bigg(h_{\mu\nu} - \frac{1}{2} \eta_{\mu\nu}h \bigg) = -\frac{16\pi
    G}{c^{4}} \tau_{\mu\nu}.
\end{equation}
Introduciamo il tensore $\psi_{\mu\nu}$ definito da
\begin{equation}
  \psi_{\mu\nu} = h_{\mu\nu} - \frac{1}{2} \eta_{\mu\nu} h.
\end{equation}
Osserviamo che nella gauge TT risulta $h = 0$ e quindi
$\psi_{\mu\nu} = h_{\mu\nu}$.  La gauge armonica~\eqref{eq:gauge-armonica} può
essere definita con riferimento a $\psi_{\mu\nu}$ come
\begin{equation}
  \label{eq:gauge-armonica2}
  \partial_{\mu}\psi^{\mu\nu} = 0
\end{equation}
e l'equazione di Einstein linearizzata~\eqref{eq:einstein-lineare-materia1} in
presenza di materia si scrive come
\begin{equation}
  \label{eq:einstein-lineare-materia2}
  \dalamb \psi_{\mu\nu} = -\frac{16\pi G}{c^{4}} \tau_{\mu\nu}.
\end{equation}
Da qui deriva inoltre che se è verificata la condizione
armonica~\eqref{eq:gauge-armonica2} risulta anche
\begin{equation}
  \label{eq:gauge-armonica-tau}
  \partial_{\mu}\tau^{\mu\nu} = 0,
\end{equation}
che nell'approssimazione lineare sostituisce l'equazione più generale
$\tensor{T}{^{\mu\nu}_{;\mu}} = 0$.

Vogliamo determinare il campo gravitazionale a grandi distanze rispetto al
sistema in esame, in particolare nella \index{zona!delle onde}zona delle onde,
cioè in una regione a distanza $R$ molto maggiore rispetto alla lunghezza
$\lambda$ delle onde irradiate dal sistema.  Dunque possiamo seguire gli stessi
metodi utilizzati nel paragrafo~\ref{sec:onde-elettromagnetiche-cariche}
ottenendo le formule per $\psi_{\mu\nu}$ analoghe alle formule dei potenziali
elettromagnetici ritardati.  D'altra parte osserviamo che l'equazione di
Einstein~\eqref{eq:einstein-lineare-materia2} in presenza di masse in moto,
nella gauge armonica, è formalmente uguale all'equazione di
Maxwell~\eqref{eq:maxwell-cariche} in presenza di cariche in moto, nella gauge
di Lorenz, quindi possiamo scrivere direttamente la soluzione
\begin{equation}
  \psi_{\mu\nu} = \frac{4G}{c^{4}}\int \frac{(\tau_{\mu\nu})|_{t - R/c}}{R} \dd
  V',
\end{equation}
in cui si è usata la stessa notazione del
paragrafo~\ref{sec:onde-elettromagnetiche-cariche}.  Poiché inoltre le velocità
con cui si muovono i corpi sono molto più piccole di quelle della luce nel
vuoto, possiamo adottare l'approssimazione di dipolo e scrivere
\begin{equation}
  \label{eq:psi-mu-nu}
  \psi_{\mu\nu} = \frac{4G}{c^{4}r} \int(\tau_{\mu\nu})|_{t-r/c} \dd V'.
\end{equation}
In seguito ometteremo, per brevità, di specificare che le funzione integrande
devono essere valutate all'istante $t-r/c$.
% Nota: nel paragrafo 110 il Landau fa dei calcoli molto strani: abbassa un
% indice muto (cioè cambia la posizione verticale di un solo indice della coppia
% di indici muti), definisce il tensore D_{\alpha\beta} completamente covariante
% uguale a un oggetto che è somma di un tensore completamente controvariante e
% un altro completamente covariante.  Magari a me sfugge qualcosa ma per il
% momento questi calcoli non mi convincono.  Ho seguito invece il Gasperini,
% "Relatività generale e teoria della gravitazione", il quale fa calcoli
% analoghi ma più sensati.
Dalla~\eqref{eq:gauge-armonica-tau}, osservando che $\tau_{\mu\nu}$ è simmetrico
rispetto allo scambio degli indici,
\begin{subequations}
  \begin{align}
    \label{eq:baz1}
    \partial^{k}\tau_{ik} + \partial^{0}\tau_{i0} = 0, \\
    \label{eq:baz2}
    \partial^{k}\tau_{0k} + \partial^{0}\tau_{00} = 0.
  \end{align}
\end{subequations}
Moltiplichiamo la prima di queste equazioni per $x_{j}$ e integriamo su tutto lo spazio
\begin{equation}
  \begin{split}
    \partial^{0} \int \tau_{i0} x_{j} \dd V &= -\int \partial^{k} \tau_{ik} x_{j}
    \dd V = - \int \partial^{k}(\tau_{ik} x_{j}) \dd V + \int \tau_{ik}
    \tensor{\delta}{^{k}_{j}} \dd V = \\
    &= - \int \partial^{k}(\tau_{ik} x_{j}) \dd V + \int \tau_{ij} \dd V.
  \end{split}
\end{equation}
Il primo integrale dell'ultimo membro può essere trasformato, usando il teorema
di Gauss, in un integrale sulla superficie all'infinito di $\tau_{ik}x_{j}$, ma
poiché $\tau_{ik}$ è nullo all'infinito l'integrale è uguale a zero.  Dunque
\begin{equation}
  \int \tau_{ij} \dd V = \partial^{0} \int \tau_{i0} x_{j} \dd V =
  \frac{1}{2} \partial^{0} \int (\tau_{i0} x_{j} + \tau_{j0}x_{i}) \dd V.
\end{equation}
Abbiamo simmetrizzato l'integrale a secondo membro perché $\tau_{ij}$ è
simmetrico.  Svolgendo analoghe considerazioni, moltiplichiamo
la~\eqref{eq:baz2} per $x_{i}x_{j}$ e integriamo su tutto lo spazio
\begin{equation}
  \begin{split}
    \partial^{0} \int \tau_{00} x_{i} x_{j} \dd V &= -\int \partial^{k}
    \tau_{0k} x_{i} x_{j} \dd V \\
    &= -\int \partial^{k}(\tau_{0k} x_{i} x_{j}) \dd V + \int \tau_{0k}
    \tensor{\delta}{^{k}_{i}} x_{j} \dd V + \int \tau_{0k} x_{i}
    \tensor{\delta}{^{k}_{j}} \dd V \\
    &= \int (\tau_{0j}x_{i} + \tau_{0i}x_{j}) \dd V.
  \end{split}
\end{equation}
Mettendo insieme i due risultati ottenuti abbiamo
\begin{equation}
  \begin{split}
    \int \tau_{ij} \dd V &= \frac{1}{2} \partial^{0} \int (\tau_{0j}x_{i} +
    \tau_{0i}x_{j}) \dd V = \frac{1}{2}(\partial^{0})^{2} \int \tau_{00} x_{i}
    x_{j} \dd V \\
    &= \frac{1}{2c^{2}} \parder{}{*{2}{t}} \int \tau_{00} x_{i} x_{j} \dd V.
  \end{split}
\end{equation}
Questa equazione ci dice che tutti gli integrali di volume delle componenti
$\tau_{ij}$ possono essere espressi come integrali della componente $\tau_{00}$
che coincide con la componente $T_{00}$ del tensore energia-impulso del sistema
di masse, quindi $\tau_{00} = \rho c^{2}$, con $\rho$ densità di massa a
riposo.  Sostituiamo nella~\eqref{eq:psi-mu-nu}
\begin{equation}
  \psi_{ij} = \frac{2G}{c^{4}r} \parder{}{*{2}{t}}\int \rho x_{i} x_{j} \dd V.
\end{equation}
Come abbiamo visto nel caso elettromagnetico, nella zona delle onde, cioè per
$R \gg \lambda \gg a$ con $a$ dimensione del sistema, le onde possono essere
trattate come piane in piccole regioni dello spazio.  Adottando inoltre la gauge
TT, nella quale $\psi_{ij} = h_{ij}$, abbiamo
\begin{equation}
  \label{eq:h-ij}
  h_{ij} = \frac{2G}{c^{4}r} \parder{}{*{2}{t}}\int \rho x_{i} x_{j} \dd V =
  \frac{2G}{3c^{4}r} \bigg(\ddot{Q}_{ij} + \delta_{ij}\parder{}{*{2}{t}} \int
  \rho r^{2} \dd V\bigg),
\end{equation}
in cui
\begin{equation}
  Q_{ij} = \int \rho(3x_{i} x_{j} - r^{2}\delta_{ij})
  \dd V
\end{equation}
è il \index{momento!di quadrupolo di massa}
\emph{momento di quadrupolo delle masse} e il punto indica la derivazione
rispetto al tempo.

Si può dimostrare\footnote{Vedi~\textcite[449]{landau:campi}.} che
\begin{equation}
  t_{\mu\nu} = \frac{c^{4}}{32\pi G} \partial_{\mu}
  \tensor{h}{^{\lambda}_{\rho}} \partial_{\nu} \tensor{h}{^{\rho}_{\lambda}}
\end{equation}
Inoltre per un'onda piana che si propaga in una direzione $x^{i}$, il flusso di
energia è diretto lungo lo stesso asse $x^{i}$ ed è determinato da $ct^{0i}$.
In particolare, supponendo che l'onda gravitazionale si propaghi lungo $x^{1}$,
le uniche componenti non nulle di $h_{\mu\nu}$ nella gauge TT sono
$h_{22} = -h_{33}$ e $h_{23} = h_{32}$ e il flusso di energia è dato da
\begin{equation}
  ct^{01} = \frac{c^{3}}{16\pi G}\bigg((\dot{h}_{23})^{2} +
  \frac{1}{4}(\dot{h}_{22} - \dot{h}_{33})^{2}\bigg).
\end{equation}
Inserendo la~\eqref{eq:h-ij} in questa formula abbiamo
\begin{subequations}
  \begin{align}
    h_{23} &= \frac{2G}{3c^{4}r} \ddot{Q}_{23}, \\
    h_{22} - h_{33} &= \frac{2G}{3c^{4}r}(\ddot{Q}_{22} - \ddot{Q}_{33}), \\
    ct^{01} &= \frac{G}{36\pi c^{5}r}\bigg((\dddot{Q}_{23})^{2} +
    \frac{1}{4}(\dddot{Q}_{22} - \dddot{Q}_{33})^{2}\bigg).
  \end{align}
\end{subequations}
Per ottenere l'energia totale $I$ emessa, nell'unità di tempo, dal sistema
bisogna moltiplicare la media di $ct^{01}$ su tutti gli stati di polarizzazione
per $r^{2}\dd\Omega$ e integrare su tutto l'angolo solido.  Sviluppando i
calcoli si ottiene il seguente
risultato\footnote{Vedi~\textcite[460-461]{landau:campi}.}
\begin{equation}
  I = \frac{G}{45c^{5}} \langle \dddot{Q}_{ij} \dddot{Q}^{ij}\rangle.
\end{equation}
Osserviamo che l'energia emessa dal sistema è proporzionale $G/c^{5}$, che è un
fattore estremamente piccolo, quindi l'efficienza di emissione di radiazione da
parte di masse in movimento è, generalmente, molto bassa.

\section{Sorgenti di onde gravitazionali}
\label{sec:sorgenti-onde-grav}

Le onde gravitazionali possono essere generate in diversi modi, fra i quali
sorgenti di laboratorio, sorgenti astronomiche, sistemi binari, urto frontale di
due oggetti, buchi neri supermassivi al centro delle galassie, collasso
gravitazionale di un corpo non sfericamente simmetrico.  In questo paragrafo ne
considereremo alcuni in particolare.

\subsection{Potenza emessa}
\label{sec:potenza-emessa}

Innanzitutto determiniamo l'ordine di grandezza dell'energia irradiata da un
sistema che emette onde gravitazionali.  Consideriamo un sistema costituito da
due corpi di uguale massa $M$ che impiegano un tempo $T = r/v$ per attraversare
l'intero sistema di dimensione $r$ alla velocità $v$.  Approssimiamo ogni
componente del tensore momento di quadrupolo di massa con
\begin{equation}
  \begin{split}
    Q_{ij} &\approx (\text{massa del sistema che si muove}) \cdot
    (\text{dimensione del sistema})^{2} \\
    &= M r^{2}
  \end{split}
\end{equation}
e si può stimare la sua derivata temporale terza dividendo $Q_{ij}$ per $T^{3}$
\begin{equation}
  \dddot{Q}_{ij} \approx \frac{Mr^{2}}{T^{3}} = \frac{Mv^{3}}{r}.
\end{equation}
Poniamo $L_{0} = c^{5}/G = \SI{3.6e59}{\erg\per\second}$, allora
\begin{equation}
  I = \frac{G}{45c^{5}} \dddot{Q}_{ij} \dddot{Q}^{ij} \approx \frac{G}{45c^{5}}
  \frac{G}{c^{10}} \bigg(\frac{M v^{3}}{r}\bigg)^{2} \frac{c^{10}}{G} \sim
  L_{0} \bigg(\frac{r_{\textup{S}}}{r}\bigg)^{2} \bigg(\frac{v}{c}\bigg)^{6}.
\end{equation}
I sistemi astrofisici sono generalmente legati gravitazionalmente, allora per il
teorema del viriale $2K + U = 0$, con $K = Mv^{2}/2 \approx M r^{2}/(2T^{2})$
energia cinetica e $U = -GM/r$ energia potenziale gravitazionale, cioè
\begin{equation}
  \frac{Mr^{2}}{T^{2}} \approx \frac{GM^{2}}{r},
\end{equation}
da cui
\begin{equation}
  T^{2} \approx \frac{r^{3}}{GM} \implies \frac{v^{2}}{c^{2}} =
  \frac{r^{2}}{c^{2}T^{2}} \approx \frac{GM}{rc^{2}} \sim
  \frac{r_{\textup{S}}}{r}.
\end{equation}
In definitiva
\begin{equation}
  I \sim L_{0} \bigg(\frac{r_{\textup{S}}}{r}\bigg)^{5},
\end{equation}
quindi per poter avere un'emissione efficiente di onde gravitazionali bisogna
avere oggetti collassati, cioè corpi per i quali si ha $r_{\textup{S}} \sim r$.

\subsection{Sorgenti di laboratorio}
\label{sec:sorgenti-laboratorio}

Onde gravitazionali possono essere prodotte in laboratorio usando una trave in
rotazione intorno al suo punto medio.  Consideriamo una trave di acciaio con
raggio $r = \SI{1}{\metre}$, lunga $l = \SI{20}{\metre}$, di densità
$\rho = \SI{7.8}{\gram\per\centi\metre\cubed}$, massa $M = \SI{490}{\tonne}$,
carico di rottura $t = \SI{3e9}{\dyne\per\centi\metre\squared}$.  La velocità di
rotazione massima $\omega$ si ottiene bilanciando la forza centrifuga con il
carico di rottura
\begin{equation}
  \omega = \sqrt{\frac{8t}{\rho l^{2}}} = \SI{28}{\radian\per\second}.
\end{equation}
Per questo sistema, la potenza emessa come onde gravitazionali è data
da\footnote{Vedi~\textcite[980]{misner:gravitation}.}
\begin{equation}
  I = \frac{2}{45} M^{2} l^{2} \omega^{6} = \SI{2.2e-22}{\erg\per\second}.
\end{equation}
Dunque in laboratorio non è possibile produrre onde gravitazionali rilevabili
con la strumentazione oggi a nostra disposizione.

\subsection{Sistema binario}
\label{sec:sistema-binario}

Per un sistema binario di corpi in orbita circolare con massa ridotta
$\mu = m_{1}m_{2}/(m_{1} + m_{2})$, si
trova\footnote{Vedi \textcite[461]{landau:campi},
  \textcite[986]{misner:gravitation}, \textcite[476]{shapiro:black-holes}.}
che l'energia $I$ emessa nell'unità di tempo sotto forma di onde gravitazionali
è data da
\begin{equation}
  \label{eq:potenza-onde-sistema-binario}
  -\toder{E}{t} = I = \frac{32}{5} \frac{G}{c^{5}} \mu^{2} \omega^{6} r^{4} =
  \frac{32}{5} \frac{G^{4} m_{1}^{2} m_{2}^{2} (m_{1} + m_{2})}{c^{5}r^{5}}
\end{equation}
con
\begin{equation}
  \omega^{2} = \frac{G(m_{1} + m_{2})}{r^{3}}.
\end{equation}
Poiché i due corpi si attraggono, diminuisce l'energia potenziale del sistema,
che viene emessa sotto forma di onde gravitazionali, e diminuisce anche la
distanza.  Nel caso di orbite circolari $E = -Gm_{1}m_{2}/(2r)$, allora la
variazione della distanza reciproca $r$ fra i due corpi varia, nell'unità di
tempo, come
\begin{equation}
  \toder{r}{t} = -\frac{Gm_{1}m_{2}}{2} \toder{(1/E)}{t} =
  \frac{2r^{2}}{Gm_{1}m_{2}} \toder{E}{t} = -\frac{64}{5}\frac{G^{3}}{c^{5}}
  \frac{m_{1}m_{2}(m_{1} + m_{2})}{r^{3}}.
\end{equation}

Consideriamo il sistema composto dal Sole e dal pianeta Giove.  Poiché la massa
di Giove $M = \SI{2e30}{\gram}$ è molto più piccola della massa del Sole
$M_{\odot} = \SI{2e33}{\gram}$, la massa ridotta $\mu$ del sistema può essere
approssimata con quella di Giove.  Inserendo i dati orbitali del pianeta
$\omega = \SI[per-mode=reciprocal]{2e-8}{\per\second}$,
$r = \SI{8e13}{\centi\metre}$ nella~\eqref{eq:potenza-onde-sistema-binario}
abbiamo che la potenza emessa da Giove sotto forma di onde gravitazionali è
\begin{equation}
  I = \frac{32}{5} \frac{G}{c^{5}} M^{2} \omega^{6} r^{4} =
  \SI{5.3e10}{\erg\per\second}.
\end{equation}
Per confronto, la radiazione gravitazionale emessa dalle collisioni termiche del
plasma solare è\footnote{Vedi~\textcite[266]{weinberg:gravitation}.}
dell'ordine di \SI{e15}{\erg\per\second}.  Sono necessari tempi scala molto più
grandi dell'età del sistema solare per poter osservare degli effetti sull'orbita
di Giove dovuti alla perdita di energia per radiazione gravitazionale.

A circa \SI{7.3}{\kilo\parsec} di distanza da noi esiste un sistema binario
molto legato costituito da due stelle di neutroni, di cui una è la pulsar PSR
B1913+16 (scoperta da Hulse e Taylor nel 1974), in orbita l'una attorno
all'altra con un periodo di circa $8$ ore.  Questo sistema è importante perché
fornisce una forte evidenza sperimentale, seppur indiretta, dell'esistenza di
onde gravitazionali.  Grazie alle regolari emissioni di impulsi elettromagnetici
da parte delle pulsar, è stato possibile misurare i parametri orbitali del
sistema.  Le misure della precessione del perielio del sistema sono in perfetto
accordo con le previsioni della teoria della relatività generale se si tiene
conto dell'energia emessa sotto forma di onde gravitazionali da parte della
pulsar.  Le masse delle due stelle sono di circa $\SI{1.4}{\solarmass}$, la
distanza fra le stelle è circa
$\SI{1.9e11}{\centi\metre} = \SI{1.9e6}{\kilo\metre}$ e la pulsar si muove alla
velocità di \SI{300}{\kilo\metre\per\second}.  Questo sistema binario funge da
laboratorio per la verifica di previsioni della teoria della relatività
riguardanti la dilatazione gravitazionale del tempo, oltre alla precessione del
perielio e le onde gravitazionali.

\subsection{Collasso gravitazionale}
\label{sec:collasso-grav}

Un'altra forte sorgente di onde è il collasso gravitazionale di un corpo.
Poiché un corpo sfericamente simmetrico produrrebbe un campo a simmetria sferica
che, per il teorema di Birkhoff, sarebbe anche statico, affinché si abbia
produzione di onde è necessario che il corpo non sia dotato di simmetria
sferica.

Un esempio di questo fenomeno è quello di una stella che collassa in una pulsar
di massa dell'ordine di $M \sim \SI{1}{\solarmass}$.  Approssimiamo la stella
originaria con un ellissoide di ellitticità
\begin{equation}
  e = \frac{\text{differenza fra i raggi equatoriali}}{\text{raggio equatoriale
      medio}}  = \frac{a-b}{(a+b)/2} = 10^{-4},
\end{equation}
con $a$ e $b$ raggi equatoriali dell'ellissoide dell'ordine di
$R_{0} \sim \SI{700e3}{\kilo\metre}$.  Dopo il collasso la pulsar ha una
dimensione di $R \sim \SI{10}{\kilo\metre}$ e ruota con velocità angolare
$\omega \sim \SI[per-mode=reciprocal]{e4}{\per\second}$.  La potenza emessa
sotto forma di onde gravitazionali
è\footnote{Vedi \textcite[488]{shapiro:black-holes},
  \textcite[272]{weinberg:gravitation}.}
\begin{equation}
  I = \frac{32}{5} \frac{G}{c^{5}} \mathcal{I}^{2} \omega^{6} e^{2} \sim
  \SI{e47}{\erg\per\second}
\end{equation}
con $\mathcal{I} = \text{momento di inerzia} \sim MR^{2}$.  L'energia
rotazionale alla formazione della pulsar è
\begin{equation}
  E = \frac{1}{2} \mathcal{I} \omega^{2} \sim \SI{e53}{\erg}.
\end{equation}
Se la quantità di energia emessa per unità di tempo rimane costante (in
particolare non varia l'ellitticità) allora l'energia rotazionale dovrebbe
esaurirsi in un tempo $\var t = E/I$ dell'ordine dell'anno, cioè dopo questo
periodo di tempo la stella dovrebbe smettere di ruotare.  In realtà la stella di
neutroni circolarizza velocemente, quindi l'ellitticità non è costante ma tende
a $0$ e di conseguenza si annulla anche la potenza emessa dalla pulsar, prima di
esaurire tutta l'energia rotazionale.  Dopo che la stella ha assunto simmetria
sferica continua a emettere radiazione elettromagnetica.

%%% Local Variables:
%%% mode: latex
%%% TeX-master: "../gravitazione"
%%% End:
