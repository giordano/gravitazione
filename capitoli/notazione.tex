\cleardoublepage
\chapter{Convenzioni e notazioni usate}
\label{cha:notazioni}

Nella teoria della relatività generale sono diffuse diverse notazioni e
convenzioni sui segni e i fattori numerici che compaiono in alcune formule.
Nella tabella~\ref{tab:notazione} è riportata la notazione adottata negli
appunti.  Gran parte dei presenti appunti si basano
su~\textcite{weinberg:gravitation}, quindi, se non diversamente specificato,
adotteremo le stesse convenzioni di questo testo.  Riportiamo di seguito in
dettagli le convenzioni scelte.

Con le lettere greche ($\alpha,\beta,\dots,\mu,\nu,\dots$) rappresenteremo
indici che possono assumere i valori $0$, $1$, $2$ e $3$, mentre con le lettere
latine ($a,b,\dots,i,j,\dots$) indici che assumono i valori $1$, $2$, $3$.  Si
noti che \textcite{landau:campo} seguono la convenzione opposta: gli indici
latini assumono i valori $0$, $1$, $2$ e $3$, gli indici greci $1$, $2$ e $3$.

Dove non chiaramente dichiarato il contrario, adottiamo, come è di consuetudine
all'interno della relatività, la \index{convenzione!di Einstein per la somma}
\emph{convenzione di Einstein per la somma}: se in un termine un indice compare
due volte, di cui una volta controvariante (sopra) e una volta covariante
(sotto), esso è chiamato \index{indice!muto}\emph{muto} (o \emph{contratto}), si
sottintende una somma sui possibili valori che l'indice può assumere e si omette
il simbolo di sommatoria.  Gli indici non muti sono chiamati
\index{indice!libero}\emph{liberi}.  In un'identità fra tensori gli indici
liberi devono comparire in ugual numero e nelle stesse posizioni (superiormente
o inferiormente) in ambo i membri.  Sebbene possa succedere che, casualmente, in
un certo sistema di riferimento le componenti controvarianti di un tensore $A$
siano uguali alle componenti covarianti di un altro tensore $B$
\begin{equation}
  \tensor{A}{^{\alpha}_{\beta}^{\gamma}} = B^{\alpha\beta\gamma},
\end{equation}
questo non è valido in un altro sistema di riferimento e l'equazione precedente
non è, quindi, legittima.  Si noti inoltre che questa non è corretta dal punto
di vista tensoriale poiché l'indice libero $\beta$ è in un membro controvariante
e nell'altro covariante.  L'espressione
$A^{\mu\sigma}B_{\mu} = A^{0\sigma}B_{0} + A^{i\sigma}B_{i}$, scritta
utilizzando la convenzione di Einstein, equivale, esplicitando la sommatoria, a
\begin{equation}
  \sum_{\mu=0}^{3} A^{\mu\sigma}B_{\mu} = A^{0\sigma}B_{0} + \sum_{i=1}^{3}
  A^{i\sigma}B_{i} = A^{0\sigma}B_{0} + A^{1\sigma}B_{1} + A^{2\sigma}B_{2} +
  A^{3\sigma}B_{3},
\end{equation}
in cui gli indici $\mu$ e $i$ sono muti e $\sigma$ è libero.  Gli indici muti
sono così denominati poiché non compaiono nell'espressione esplicita.  Gli
indici muti possono essere rinominati (e spesso eseguiremo questa operazione
senza dichiararla esplicitamente) a patto di non utilizzare indici muti già
utilizzati in quel termine oppure indici liberi.  Quindi l'espressione
$\tensor{T}{^{\alpha\beta\gamma}_{\alpha\gamma}}$ può essere anche scritta come
$\tensor{T}{^{\mu\beta\nu}_{\mu\nu}}$ ma \emph{non} come
$\tensor{T}{^{\beta\beta\gamma}_{\beta\gamma}}$ oppure
$\tensor{T}{^{\alpha\beta\alpha}_{\alpha\alpha}}$.

La velocità della luce nel vuoto $c = \SI{299792458}{\metre\per\second}$ è posta
uguale a $1$.  Di conseguenza tutte le velocità sono adimensionali (in
particolare sono espresse in unità di $c$), lunghezze e tempo hanno la stessa
dimensione.  Per ristabilire la coerenza dimensionale delle equazioni bisogna
utilizzare l'adeguato numero di potenze di $c$.

Il tensore metrico di Minkowski $\eta_{\alpha\beta}$ ha, per noi, segnatura
$-+++$.  I testi \textcites{barone:relativita,landau:campo,ohanian:gravitazione}
adottano la segnatura opposta $+---$.

Il tensore di Riemann è definito come
\begin{equation}
    \tensor{R}{^{\lambda}_{\mu\nu\kappa}}
  = \parder{\tensor{\Gamma}{^{\lambda}_{\mu\nu}}}{x^{\kappa}}
  - \parder{\tensor{\Gamma}{^{\lambda}_{\mu\kappa}}}{x^{\nu}} +
  \tensor{\Gamma}{^{\eta}_{\mu\nu}} \tensor{\Gamma}{^{\lambda}_{\eta\kappa}} -
  \tensor{\Gamma}{^{\eta}_{\mu\kappa}} \tensor{\Gamma}{^{\lambda}_{\eta\nu}}.
\end{equation}
I testi \textcites{barone:relativita,landau:campo,ohanian:gravitazione}
definiscono il tensore di Riemann con il segno opposto rispetto a quello qui
stabilito.

Il tensore di Ricci $R_{\mu\kappa}$ è ottenuto contraendo il primo e il terzo
indice del tensore di Riemann $\tensor{R}{^{\lambda}_{\mu\nu\kappa}}$, o
equivalentemente il secondo e il quarto indice.  Al contrario,
\textcite{ohanian:gravitazione} definiscono il tensore di Ricci contraendo primo
e quarto indice $R_{\mu\kappa} = \tensor{R}{^{\lambda}_{\mu\kappa\lambda}}$ e si
ottiene così un tensore opposto rispetto a quello da noi definito.

\begin{longtable}{c
    p{.37\textwidth}
    p{.28\textwidth}}
  % intestazione iniziale
  \caption{Notazione usata in questi appunti}
  \label{tab:notazione} \\
  \toprule
  Simbolo & Significato & Riferimento \\
  \midrule
  \endfirsthead
  % intestazione normale
  \multicolumn{3}{l}{\footnotesize\itshape
    Continua dalla pagina precedente} \\
  \toprule
  Simbolo & Significato & Riferimento \\
  \midrule
  \endhead
  % piede normale
  \midrule \multicolumn{3}{r}{\footnotesize\itshape
    Continua nella prossima pagina} \\
  \endfoot
  % piede finale
  \bottomrule \multicolumn{3}{r}{\footnotesize\itshape
    Si conclude dalla pagina precedente} \\
  \endlastfoot
  % inizio corpo della tabella
  $\bm{A} = (A^{1}, A^{2}, A^{3})$ & Vettore tridimensionale di componenti
  $A^{1}$, $A^2$ e $A^{3}$. & \\
  $A^{\mu} = (A^{0}, \bm{A})$ & Quadrivettore controvariante di componente
  temporale $A^{0}$ e componenti spaziali $\bm{A}$. & Vedi
  paragrafo~\ref{sec:calcolo-tensoriale-minkowski} e
  capitolo~\ref{cha:calcolo-tensoriale}. \\
  $A_{\mu} = (A_{0}, \bm{A})$ & Quadrivettore covariante con componente
  temporale $A_{0}$ e componente spaziale $\bm{A}$. & Vedi
  paragrafo~\ref{sec:calcolo-tensoriale-minkowski} e
  capitolo~\ref{cha:calcolo-tensoriale}. \\
  $\tensor*{T}{^{\alpha_{1}\dots\alpha_{p}}_{\beta_{1}\dots\beta_{q}}}$ &
  Quadritensore. & Vedi paragrafo~\ref{sec:calcolo-tensoriale-minkowski} e
  capitolo~\ref{cha:calcolo-tensoriale}. \\
  $\dd \tau^{2} = \dd t^{2} - \dd \bm{x}^{2}$ & Metrica nello spazio di
  Minkowski. & Vedi paragrafo~\ref{sec:trasformazioni-lorentz}. \\
  $\eta_{\alpha\beta}$ & Tensore metrico di Minkowski. & Vedi
  paragrafo~\ref{sec:trasformazioni-lorentz}. \\
  $\tensor{\delta}{^{\alpha}_{\beta}}$ & Delta di Kronecker quadrimensionale. &
  Vedi paragrafo~\ref{sec:trasformazioni-lorentz}. \\
  $\tensor{\Lambda}{^{\alpha}_{\beta}}$ & Componente della matrice di Lorentz. &
  Vedi paragrafo~\ref{sec:trasformazioni-lorentz}. \\
  $\gamma(v) = \dfrac{1}{\sqrt{1-v^{2}}}$ & Fattore di Lorentz associato al
  boost $v$. & Vedi paragrafo~\ref{sec:trasformazioni-lorentz}. \\
  $\partial_{\mu} A_{\nu} = A_{\nu,\mu} =
  \displaystyle\parder{A_{\nu}}{x^{\mu}}$
  & Derivata ordinaria di $A_{\nu}$ rispetto alla coordinata $x^{\mu}$. & Vedi
  paragrafo~\ref{sec:derivazione-minkowski}. \\
  $\partial^{\mu} A_{\nu} = \tensor{A}{_{\nu}^{,\mu}} =
  \displaystyle\parder{A_{\nu}}{x_{\mu}}$
  & Derivata ordinaria di $A_{\nu}$ rispetto alla coordinata $x_{\mu}$. & Vedi
  paragrafo~\ref{sec:derivazione-minkowski}. \\
  $ \Box^{2} = \partial^{\alpha}\partial_{\alpha}$ & Operatore d'Alambertiano. &
  Vedi paragrafo~\ref{sec:derivazione-minkowski}. \\
  $\epsilon^{\alpha\beta\gamma\delta}$ & Tensore di Levi-Civita & Vedi
  paragrafo~\ref{sec:tensore-levi-civita}. \\
  $\dd^{4} x = \dd x^{0}\dd x^{1}\dd x^{2}\dd x^{3}$ & Elemento infinitesimo di
  volume quadrimensionale. & \\
  $\dd \Sigma_{\mu}$ & Elemento infinitesimo di ipersuperficie
  tridimensionale. & Vedi appendice~\ref{cha:teorema-gauss}. \\
  $\dd^{3} \bm{x} = \dd V$ & Elemento infinitesimo di volume tridimensionale. &
  \\
  $F^{\alpha\beta}$ & Tensore del campo elettromagnetico. & Vedi
  paragrafo~\ref{sec:equazioni-maxwell}. \\
  $S = \displaystyle\int \Lambda \dd^{4} x$ & Azione associata alla densità di
  lagrangiana $\Lambda$. & Vedi paragrafo~\ref{sec:tensore-energia-impulso}. \\
  $\tensor{T}{^{\beta}_{\alpha}}$ & Tensore energia-impulso. & Vedi
  paragrafo~\ref{sec:tensore-energia-impulso}. \\[1.6ex]
  $\tensor{\Gamma}{^{\lambda}_{\mu\nu}}$ & Connessione affine. & Vedi
  paragrafi~\ref{sec:equazione-moto}, \ref{sec:relazione-g-Gamma} e
  \ref{sec:derivazione-covariante}. \\
  $g_{\mu\nu}$ & Tensore metrico nella relatività generale. & Vedi
  paragrafi~\ref{sec:equazione-moto}, \ref{sec:relazione-g-Gamma} e
  \ref{sec:tensore-metrico}. \\
  $r_{\textup{G}} = r_{\textup{S}}$ & Raggio di Schwarzschild. & Vedi
  paragrafo~\ref{sec:limite-newtoniano}. \\[1.6ex]
  $\sqrt{g} = \sqrt{-\det(g_{\mu\nu})}$ & Radice quadrata dell'opposto del
  determinante del tensore metrico. & Vedi
  paragrafo~\ref{sec:tensore-metrico}. \\
  $V_{\mu;\nu} = \displaystyle\covder{V_{\mu}}{x^{\nu}}$ & Derivata covariante
  di $V_{\mu}$ rispetto alla coordinata $x^{\nu}$. & Vedi
  paragrafo~\ref{sec:derivazione-covariante}. \\
  $\cdd V_{\mu}$ & Differenziale covariante di $V_{\mu}$. & Vedi
  paragrafo~\ref{sec:differenziale-covariante}. \\[1.6ex]
  $\displaystyle\curder{V_{\mu}}$ & Derivata di $V_{\mu}$ lungo una curva
  parametrizzata da $\tau$. & Vedi
  paragrafo~\ref{sec:differenziale-covariante}. \\
  $\tensor{R}{^{\lambda}_{\mu\nu\kappa}}$ & Tensore di curvatura di
  Riemann-Christoffel. & Vedi paragrafo~\ref{sec:tensore-riemann}. \\
  $R_{\mu\kappa} = \tensor{R}{^{\lambda}_{\mu\lambda\kappa}}$ & Tensore di
  Ricci. & Vedi paragrafo~\ref{sec:tensore-riemann}. \\[1.6ex]
  $R = \tensor{R}{^{\mu}_{\mu}}$ & Curvatura scalare & Vedi
  paragrafo~\ref{sec:tensore-riemann}. \\[1.6ex]
  $G^{\mu\nu} = R^{\mu\nu} - \dfrac{1}{2} g^{\mu\nu}R$ & Tensore di Einstein &
  Vedi paragrafo~\ref{sec:identita-bianchi}. \\
\end{longtable}


%%% Local Variables:
%%% mode: latex
%%% TeX-master: "../astrofisica-teorica"
%%% fill-column: 80
%%% End:
