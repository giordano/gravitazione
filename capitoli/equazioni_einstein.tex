\cleardoublepage

\chapter{Equazioni del campo gravitazionale}
\label{cha:equazioni-campo-grav}

\section{Derivazione delle Equazioni di Einstein}

È possibile derivare le equazioni di Einstein per il campo gravitazionale usando
considerazioni del tutto generali.  Le equazioni cercate devono essere
(vedi~\textcite[151]{weinberg:gravitation}):
\begin{enumerate}
\item in forma tensoriale;
\item fornire per campi deboli e statici la soluzione Newtoniana;
\item contenere termini quadratici nella derivata prima e/o lineari
  nella derivata seconda della metrica (invarianza in scala).
\end{enumerate}
L'equazione di campo newtoniana è l'equazione di Poisson
\begin{equation}
  \nabla^2 \phi = 4 \pi G \rho,
\end{equation}
dove $\rho$ è la densità di massa che per sistemi non relativistici possiamo
identificare con $T_{00}$.  Con l'ulteriore identificazione (nel caso di campi
deboli e statici) $h_{00} = - 2 \phi$, l'equazione di Poisson diventa
\begin{equation}
  \nabla^2 h_{00} = - 8 \pi G T_{00}.
\end{equation}
È naturale ipotizzare per l'equazione di campo ricercata la forma
\begin{equation}
  G_{\mu \nu} = - 8 \pi G T_{\mu \nu}
\end{equation}
dove $G_{\mu \nu}$ deve
\begin{itemize}
\item[a)] essere un tensore;
\item[b)] soddisfare la 3;
\item[c)] essere simmetrico così come $T_{\mu \nu}$;
\item[d)] avere 4-divergenza nulla poiché il tensore energia impulso soddisfa la
  condizione di conservazione $\tensor{T}{^{\mu \nu}_{; \mu}} =0$;
\item[e)] nel caso di un campo statico e debole $G_{00} \simeq \nabla^2 g_{00}$.
\end{itemize}
Dalle condizioni a) + b) segue che necessariamente
\begin{equation}
  \label{5.4}
  G_{\mu \nu} = C_1 R_{\mu \nu} + C_2 R g_{\mu \nu}
\end{equation}
dove $R_{\mu \nu}$ è il tensore di Ricci ed $R$ la curvatura scalare dello
spazio-tempo.

La condizione c) è immediatamente soddisfatta per la simmetria del tensore di
Ricci; la condizione d) implica la relazione $C_1 \tensor{R}{^{\mu \nu}_{;\mu}}
+ C_2 R_{;\mu} g^{\mu \nu} = 0$.  Utilizzando l'identità di Bianchi contratta,
$\tensor{R}{^{\mu \nu}_{; \mu}} = (1/2) g^{\mu \nu} R_{; \mu}$, la
relazione~\eqref{5.4} diventa $(C_2 + C_1/2) R_{; \nu} =0$, che, senza perdere
di generalità ($R_{; \nu} \ne 0$), implica la relazione $C_2 = -C_1/2$.  Abbiamo
allora
\begin{equation}
  G_{\mu \nu} = C_1 \left(R_{\mu \nu} - \frac{1}{2} g_{\mu \nu} R \right).
\end{equation}
Usando la condizione e) troviamo $C_1=1$.  Infatti, per un sistema non
relativistico $T_{00} \simeq \rho \gg T_{ij} \simeq \rho v^i v^j$.  Analogamente
sarà $G_{00} \gg G_{ij} \simeq 0$, il che implica $C_1 (R_{ij} -(1/2) R) \simeq
0$ e quindi $R_{ij} \simeq (1/2) R$.  Moltiplicando la precedente per
$\eta_{ij}$ (per campi deboli la contrazione di indici avviene moltiplicando per
la metrica non perturbata $\eta_{\mu \nu}$) troviamo $\tensor{R}{^{j}_{j}}
\simeq (3/2) R$.  Allora per la curvatura $R \simeq \eta^{\mu \nu} R_{\mu \nu}
\simeq \tensor{R}{^{j}_{j}} - R_{00}$, abbiamo $R \simeq (3/2)R-R_{00}$, da cui deduciamo la
relazione $R \simeq 2 R_{00}$.

Abbiamo quindi $G_{00} = C_1 (R_{00}-(1/2) g_{00} R) \simeq C_1 (R_{00}-(1/2)
\eta_{00} R) \simeq 2 C_1 R_{00}$.  Ora $R_{00} = \tensor{R}{^{l}_{0l0}} = (1/2)
\nabla^2 g_{00}$ e quindi in $G_{00} \simeq C_1 \nabla^2 g_{00}$, da cui in
conclusione $C_1=1$.  Pertanto le equazioni cercate sono
\begin{equation}
  \label{5.6}
  G_{\mu \nu} = R_{\mu \nu} - \frac{1}{2} g_{\mu \nu} R = -8 \pi G T_{\mu\nu}.
\end{equation}

Una diversa forma delle equazioni di Einstein è ottenuta contraendo nella
precedente gli indici (moltiplicando per $g^{\mu \nu}$).  Si ha $R-(1/2) 4 R =
-8 \pi G \tensor{T}{^{\lambda}_{\lambda}}$ da cui segue $R = 8 \pi G
\tensor{T}{^{\lambda}_{\lambda}}$, e sostituendo nell'equazione~\eqref{5.6}:
\begin{equation}
  R_{\mu \nu}= -8 \pi G \left( T_{\mu \nu} - \frac{1}{2} g_{\mu \nu}
    \tensor{T}{^{\lambda}_{\lambda}} \right).
\end{equation}

Per commentare queste equazioni si suol dire che la materia dice allo
spazio-tempo come curvarsi (informazione contenuta nel tensore di Ricci) e allo
stesso tempo lo spazio-tempo curvo dice alla materia come muoversi (tensore
energia-impulso).  Osserviamo che dalla~\eqref{eq:divergenza-tens-einstein}
deriva che anche il tensore energia-impulso della materia ha quadridivergenza
covariante nulla $\tensor{T}{^{\mu\nu}_{;\mu}} = 0$.

Nel vuoto le equazioni di Einstein assumono la forma
\begin{equation}
  R_{\mu \nu}=0.
\end{equation}
Si noti che questa equazione non significa che uno spazio-tempo vuoto è anche
piatto; per avere quest'ultima condizione deve verificarsi la condizione più
forte $\tensor{R}{^{\lambda}_{\mu\nu\kappa}} = 0$ (il tensore di Riemann ha 20
componenti indipendenti, il tendore di Ricci ne ha 10).

\section{Derivazione formale delle Equazioni di Einstein}
\label{sec:equazioni-einstein}

\emph{Nota: in questo paragrafo, dove non diversamente specificato, indicheremo
  esplicitamente la velocità della luce nel vuoto $c$ e adotteremo la
  convenzione dei segni $(-++)$ seguita
  da~\textcites{barone:relativita,landau:campi}.}

Vogliamo determinare le equazioni del campo gravitazionale a partire dal
principio variazionale di Hamilton.  Assumiamo che l'azione $S$ sia somma di una
componente che descrive solo il campo gravitazionale $S_{\textup{g}}
(g_{\mu\nu}, \lparder{g_{\mu\nu}}{x^{\lambda}})$, dipendente dal tensore metrico
e dalle sue derivate prime, e una componente relativa alla materia che genera e
interagisce con il campo $S_{\textup{m}} (g_{\mu\nu},
\lparder{g_{\mu\nu}}{x^{\lambda}}; q, \lparder{q}{x^{\lambda}})$, dipendente
dalla metrica, dalle sue derivate prime, oltre che dalle coordinate e velocità
generalizzate della materia.  L'azione deve essere uno scalare, in modo che
risulti invariante per trasformazioni arbitrarie delle coordinate.  Inoltre
nella derivazione dlle equazioni di campo si assume assegnato il moto della
materia ($\delta q=0$).

Hilbert propose questa espressione per l'azione del campo
gravitazionale\footnote{Il fattore davanti all'integrale è stato scelto in
  maniera tale che nel limite newtoniano si ottenga l'equazione di Poisson
  $\nabla^{2} \phi = 4\pi G\rho$.}
\begin{equation}
  S_{\textup{g}} = -\frac{c^{3}}{16\pi G} \int R \sqrt{g} \dd^{4} x,
\end{equation}
chiamata \index{azione!di Hilbert-Einstein} \emph{azione di Hilbert-Einstein}.
$G = \SI{6.673 84(80)e-11}{\cubic\metre\per\kilo\metre\per\second\squared}$ è la
\index{costante!di gravitazione universale}
\emph{costante di gravitazione universale}.

L'azione della materia ha invece l'espressione
\begin{equation}
  S_{\textup{m}} = \frac{1}{c} \int \Lambda \sqrt{g} \dd^{4} x,
\end{equation}
in cui $\Lambda = \Lambda(g^{\mu\nu}, \partial_{\lambda}g^{\mu\nu};
q, \partial_{q} g_{\mu \nu})$ è la densità di lagrangiana associata al sistema
che interagisce con il campo gravitazionale.  L'interazione fra il campo e la
materia è inclusa in $S_{\textup{m}}$.

Per determinare le equazioni del campo (fissato il moto delle sorgenti) dobbiamo
richiedere che l'azione sia stazionaria rispetto alle variazioni, a volume
fissato $\Omega$ (tutto lo spazio ordinario tra due istanti di tempo), $\delta
g_{\mu\nu}$ della metrica, con la condizione che dette variazioni si annullino
sul bordo $\delta\Omega$ dell'ipervolume di integrazione.  Dunque dobbiamo
imporre $\delta S = \delta(S_{\textup{g}} + S_{\textup{m}}) = 0$.  Calcoliamo il
contributo di $\delta S_{\textup{g}}$
\begin{equation}
  \label{eq:var-HE}
  \begin{split}
    \delta \int R \sqrt{g} \dd^{4} x &= \delta \int g^{\mu\nu} R_{\mu\nu}
    \sqrt{g} \dd^{4} x \\
    &= \int (\delta g^{\mu\nu} R_{\mu\nu} \sqrt{g} + g^{\mu\nu} R_{\mu\nu}
    \delta \sqrt{g} + g^{\mu\nu} \delta R_{\mu\nu} \sqrt{g}) \dd^{4} x.
  \end{split}
\end{equation}
Osserviamo che
\begin{equation}
  \delta\sqrt{g} = \parder{\sqrt{g}}{g} \parder{g}{g^{\mu\nu}} \delta g^{\mu\nu}
  = \frac{1}{2 \sqrt{g}} (-g g_{\mu\nu} ) \delta g^{\mu\nu} = - \frac{1}{2}
  g_{\mu\nu} \sqrt{g} \delta g^{\mu\nu},
\end{equation}
in cui abbiamo ricordato la~\eqref{eq:g_mu_rho-determinante2}.  Sostituendo
nella~\eqref{eq:var-HE} otteniamo
\begin{equation}
  \delta \int R \sqrt{g} \dd^{4} x = \int \bigg( R_{\mu\nu} - \frac{1}{2}
  g_{\mu\nu} R \bigg)\sqrt{g} \delta g^{\mu\nu} \dd^{4} x + \int g^{\mu\nu}
  \delta R_{\mu\nu} \sqrt{g} \dd^{4} x.
\end{equation}

Mostriamo che il secondo integrale è nullo.  Innanzitutto la
\index{tensore!di Ricci!variazione del}variazione del tensore di Ricci può
essere espressa in termini di variazioni della connessione affine mediante
l'\index{identità!di Palatini}\emph{identità di Palatini}\footnote{Si
  noti che la variazione della connessione affine è un tensore poiché è la
  differenza fra due connessioni affini, la quale è un tensore come visto nel
  paragrafo~\ref{sec:connessione-affine}.}
(vedi l'appendice~\ref{sec:dimostr-palatini})
\begin{equation}
  \delta R_{\mu\nu} = (\delta \tensor{\Gamma}{^{\lambda}_{\mu\nu}})_{;\lambda} -
  (\delta \tensor{\Gamma}{^{\lambda}_{\mu\lambda}})_{;\nu}.
\end{equation}
Poiché $g^{\mu\nu}$ commuta con la derivazione covariante, si può riscrivere
l'integrale in esame come
\begin{equation}
  \begin{split}
    \int g^{\mu\nu} \delta R_{\mu\nu} \sqrt{g} \dd^{4} x &= \int [ ( g^{\mu\nu}
    \delta \tensor{\Gamma}{^{\lambda}_{\mu\nu}} )_{;\lambda} - ( g^{\mu\nu} \delta
    \tensor{\Gamma}{^{\lambda}_{\mu\lambda}})_{;\nu} ] \sqrt{g} \dd^{4} x \\
    &= \int g^{\mu\nu} \delta \tensor{\Gamma}{^{\lambda}_{\mu\nu}} \sqrt{g} \dd
    \Sigma_{\lambda} - \int g^{\mu\nu}
    \delta\tensor{\Gamma}{^{\lambda}_{\mu\lambda}} \sqrt{g}\dd\Sigma_{\nu} = 0
  \end{split}
\end{equation}
in cui abbiamo applicato il teorema di Gauss in forma
covariante~\eqref{eq:gauss-covariante}.  Gli ultimi due integrali sulle
ipersuperfici sono entrambi nulli poiché vanno calcolati sul bordo
$\delta\Omega$ dell'ipervolume su cui si calcola la variazione dell'azione e sul
bordo le variazioni di $g_{\mu\nu} = 0$ e quindi anche $\delta
\tensor{\Gamma}{^{\lambda}_{\mu\nu}} = 0$ e $\delta
\tensor{\Gamma}{^{\lambda}_{\mu\lambda}} = 0$.  Dunque abbiamo trovato che la
variazione dell'azione di Hilbert-Einstein è
\begin{equation}
  \delta S_{\textup{g}} = -\frac{c^{3}}{16\pi G} \int \bigg( R_{\mu\nu} -
  \frac{1}{2} g_{\mu\nu} R \bigg)\sqrt{g} \delta g^{\mu\nu} \dd^{4} x.
\end{equation}

Per calcolare la variazione $\delta S_{\textup{m}}$ dell'azione associata alla
materia ragioniamo in maniera analoga a quanto fatto nel
paragrafo~\ref{sec:tensore-energia-impulso}
\begin{equation}
  \begin{split}
    \delta S_{\textup{m}} &= \frac{1}{c} \delta \int \Lambda(g^{\mu\nu},
    \tensor{g}{^{\mu\nu}_{,\lambda}}) \sqrt{g} \dd^{4} x = \frac{1}{c} \int
    \bigg( \parder{(\sqrt{g}\Lambda)}{g^{\mu\nu}} \delta g^{\mu\nu}
    + \parder{(\sqrt{g}\Lambda)}{\tensor{g}{^{\mu\nu}_{,\lambda}}}
    \delta \tensor{g}{^{\mu\nu}_{,\lambda}} \bigg) \dd^{4} x \\
    &= \frac{1}{c} \int \bigg( \parder{(\sqrt{g}\Lambda)}{g^{\mu\nu}} \delta
    g^{\mu\nu} + \parder{}{x^{\lambda}}
    \bigg( \parder{(\sqrt{g}\Lambda)}{\tensor{g}{^{\mu\nu}_{,\lambda}}} \delta
    g^{\mu\nu}\bigg) - \delta g^{\mu\nu} \parder{}{x^{\lambda}}
    \bigg( \parder{(\sqrt{g}\Lambda)}{\tensor{g}{^{\mu\nu}_{,\lambda}}} \bigg)
    \bigg) \dd^{4} x \\
    &= \frac{1}{c} \int \bigg( \parder{(\sqrt{g}\Lambda)}{g^{\mu\nu}}
    - \parder{}{x^{\lambda}}
    \bigg( \parder{(\sqrt{g}\Lambda)}{\tensor{g}{^{\mu\nu}_{,\lambda}}} \bigg)
    \bigg) \delta g^{\mu\nu} \dd^{4} x.
  \end{split}
\end{equation}
Nell'ultimo passaggio abbiamo sfruttato il fatto che l'integrale del secondo
termine si trasforma, per il teorema di Gauss, in un integrale sul bordo del
volume di integrazione sul quale le variazioni $\delta g^{\mu\nu}$ sono nulle.
Introduciamo il tensore $T_{\mu\nu}$ definito da
\begin{equation}
  \label{5.18}
  - \frac{1}{2} \sqrt{g} T_{\mu\nu} = \parder{(\sqrt{g}\Lambda)}{g^{\mu\nu}}
  - \parder{}{x^{\lambda}}
  \bigg( \parder{(\sqrt{g}\Lambda)}{\tensor{g}{^{\mu\nu}_{,\lambda}}} \bigg)
\end{equation}
% TODO: spiegare da qualche parte che questo è proprio il tensore
% energia-impulso (e mettere \index{tensore!energia-impulso})
così la variazione dell'azione della materia diventa
\begin{equation}
  \label{5.19}
  \delta S_{\textup{m}} = - \frac{1}{2c} \int T_{\mu\nu} \delta g^{\mu\nu}
  \sqrt{g} \dd^{4} x
\end{equation}
e la condizione di stazionarietà dell'azione totale del sistema costituito dal
campo gravitazionale e delle sorgenti è
\begin{equation}
  \delta S = \delta(S_{\textup{g}} + S_{\textup{m}}) = -\int \bigg[
  \frac{c^{3}}{16\pi G} \bigg( R_{\mu\nu} - \frac{1}{2} g_{\mu\nu} R \bigg) -
  \frac{1}{2c} T_{\mu\nu} \bigg] \sqrt{g} \delta g^{\mu\nu} \dd^{4} x = 0,
\end{equation}
e per l'arbitrarietà delle variazioni $\delta g^{\mu\nu}$ ricaviamo
\begin{equation}
  \label{eq:einstein}
  G_{\mu\nu} = R_{\mu\nu} - \frac{1}{2} g_{\mu\nu} R = \frac{8\pi G}{c^{4}}
  T_{\mu\nu}
\end{equation}
che sono le \emph{equazioni del campo gravitazionale} in presenza di sorgenti.
Queste equazioni sono dette \index{equazioni!di Einstein}\emph{equazioni di
  Einstein}.

Si noti che se riprendiamo la convenzione dei segni $(+--)$
di~\textcite{weinberg:gravitation}, compare un segno meno nel membro di destra
dell'equazione di Einstein, come spiegato a
pagina~\pageref{eq:convenzione-segni}.  Contraendo poi gli indici $\mu$ e $\nu$
otteniamo
\begin{equation}
  R = \frac{8\pi G}{c^{4}}\tensor{T}{^{\lambda}_{\lambda}},
\end{equation}
e sostituendo nella~\eqref{eq:einstein} otteniamo una diversa espressione per le
\index{equazioni!di Einstein} equazioni di Einstein
\begin{equation}
  R_{\mu\nu} = -\frac{8\pi G}{c^{4}} \bigg( T_{\mu\nu} -
  \frac{1}{2}g_{\mu\nu}\tensor{T}{^{\lambda}_{\lambda}} \bigg).
\end{equation}
Nel vuoto, cioè in assenza di materia $T_{\mu \nu} =0$, si ha
\begin{equation}
  R_{\mu\nu} = 0.
\end{equation}

\subsection{Limite newtoniano delle equazioni di Einstein}
\label{sec:limite-newtoniano-einstein}

Vogliamo determinare il limite newtoniano (campi gravitazionali deboli e
statici, velocità non relativistiche) delle equazioni di Einstein e per fare
questo lavoriamo in maniera analoga a quanto fatto nel
paragrafo~\ref{sec:limite-newtoniano}, quindi poniamo
$g_{\mu\nu} = \eta_{\mu\nu} + h_{\mu\nu}$.  Ci aspettiamo che il tensore
energia-impulso sia in queste condizioni quello ricavato nel limite di basse
velocità nel paragrafo~\ref{sec:fluido-perfetto} per un corpo macroscopico
continuo (come un \index{fluido!perfetto}fluido perfetto)
\begin{subequations}
  \begin{align}
    T_{ij} &= \rho c^{2} u_{i} u_{j}, \\
    T_{00} &= \rho c^{2},
  \end{align}
\end{subequations}
in cui qui con $\rho$ indichiamo la densità di massa a riposo (e quindi
$\rho c^{2}$ è la densità di energia a riposo, per brevità omettiamo il pedice
$0$) e $u_{\mu}$ la quadrivelocità.  Per l'ipotesi di velocità non
relativistiche si devono trascurare le componenti spaziali della quadrivelocità
rispetto alla componente temporale, cioè $u_{i} = 0$, e quindi $T_{ij} = 0$ da
cui abbiamo inoltre
\begin{equation}
  0 = G_{ij} \approx R_{ij} - \frac{1}{2}\eta_{ij}R.
\end{equation}
Allora l'unica equazione di Einstein rilevante è quella con $G_{00}$.
Contraendo gli indici dell'equazione precedente otteniamo
\begin{equation}
  \eta^{ij}R_{ij} \approx \frac{1}{2} \eta_{ij}\eta^{ij} R = \frac{3}{2} R.
\end{equation}
La curvatura scalare può essere approssimata come
\begin{equation}
  R = g^{\mu\nu}R_{\mu\nu} \approx \eta^{\mu\nu} R_{\mu\nu} = \eta^{00}R_{00} +
  \eta^{ij}R_{ij} = -R_{00} + \frac{3}{2} R
\end{equation}
dunque $R \approx 2 R_{00}$.  Sostituendo nell'equazione di Einstein abbiamo
\begin{equation}
  G_{00} = R_{00} - \frac{1}{2} \eta_{00}R = R_{00} + \frac{1}{2} R = 2 R_{00} =
  -\frac{8\pi G}{c^{4}} T_{00} = -\frac{8\pi G}{c^{4}} \rho c^{2}
\end{equation}
da cui
\begin{equation}
  R_{00} = -\frac{4\pi G}{c^{2}} \rho.
\end{equation}
D'altra parte osserviamo che
\begin{equation}
  R_{00} = g^{\mu\nu}R_{\mu 0\nu 0} \approx \eta^{\mu\nu}R_{\mu 0\nu 0}
\end{equation}
Dalla~\eqref{eq:riemann-covariante}, arrestandoci ai termini del primo ordine in
$h_{\mu\nu}$, abbiamo
\begin{equation}
  \label{eq:riemann-approx}
  \begin{split}
    R_{\lambda\mu\nu\kappa} &\approx \frac{1}{2}
    \bigg( \parder{h_{\lambda\nu}}{x^{\kappa}, x^{\mu}}
    - \parder{h_{\mu\nu}}{x^{\kappa},
      x^{\lambda}}- \parder{h_{\lambda\kappa}}{x^{\nu}, x^{\mu}}
    + \parder{h_{\mu\kappa}}{x^{\nu}, x^{\lambda}} \bigg) \\
    &\quad{} + \eta_{\eta\sigma}
    (\underbrace{\tensor{\Gamma}{^{\eta}_{\nu\lambda}}
      \tensor{\Gamma}{^{\sigma}_{\mu\kappa}} -
      \tensor{\Gamma}{^{\eta}_{\kappa\lambda}}
      \tensor{\Gamma}{^{\sigma}_{\mu\nu}}}_{\mathcal{O}(h^{2})}) \\
    &= \frac{1}{2} \bigg( \parder{h_{\lambda\nu}}{x^{\kappa}, x^{\mu}}
    - \parder{h_{\mu\nu}}{x^{\kappa},
      x^{\lambda}}- \parder{h_{\lambda\kappa}}{x^{\nu}, x^{\mu}}
    + \parder{h_{\mu\kappa}}{x^{\nu}, x^{\lambda}} \bigg) + \mathcal{O}(h^{2})
  \end{split}
\end{equation}
e in particolare, sfruttando l'ipotesi di campi costanti,
\begin{subequations}
  \begin{align}
    R_{0000} &\approx 0, \\
    R_{i0j0} &\approx \frac{1}{2} \parder{h_{00}}{x^{i},x^{j}}
  \end{align}
\end{subequations}
allora
\begin{equation}
  R_{00} = \eta^{00}R_{0000} + \eta^{ij}R_{i0j0} = \delta^{ij}R_{i0j0} \approx
  \frac{1}{2}\nabla^{2} h_{00} = -\nabla^{2}\frac{\phi}{c^{2}}
\end{equation}
avendo ricordato la~\eqref{eq:h00}.  Dunque mettendo insieme le due espressioni
di $R_{00}$ ottenute troviamo
\index{equazione!di Poisson!per il campo gravitazionale}l'equazione di Poisson
per il campo gravitazionale
\begin{equation}
  \nabla^{2} \phi = 4\pi G\rho.
\end{equation}

\section{Isometrie delo spazio-tempo; vettori di Killing}
\label{sec:vettori di Killing}

\emph{Per approfondire questo argomento puoi
  vedere~\textcites[300-306]{ohanian:gravitazione}[176]{weinberg:gravitation}.}

Se il tensore metrico $g_{\mu \nu}$ di una data geometria spazio-temporale è
invariante rispetto ad una trasformazione delle coordinate
$x'^{\mu}=x'^{\mu}(x)$, si dice che questa trasformazione è una isometria dello
spazio-tempo.  Questo significa che $g'_{\mu \nu}(x')$ dipende da $x'^{\mu}$
nello stesso modo in cui $g_{\mu \nu}(x)$ dipende da $x^{\mu}$, cioè\footnote{Ad
  esempio, traslazioni, rotazioni e trasformazioni di Lorentz sono isometrie
  dello spazio-tempo piatto, nel senso che la metrica $\eta_{\mu \nu}$ non
  cambia relativamente a ognuna di queste trasformazioni.}
\begin{subequations}
  \label{W13.1.1}
  \begin{align}
    g'_{\mu \nu} (x') &= g_{\mu \nu}(x') \\
    g'_{\mu \nu} (x) &= g_{\mu \nu}(x).
  \end{align}
\end{subequations}
Queste condizioni sono differenti dalla condizione per un campo scalare, per il
quale $\phi'(x')=\phi(x)$.  La condizione per avere una isometria per il campo
scalare è che $\phi'(x') = \phi(x')$.

Vediamo ora quali condizioni devono essere soddisfatte perché una generica
trasformazione di coordinate $x'^{\mu} = x'^{\mu}(x)$ sia una isometria.
Sappiamo che a un assegnato punto dello spazio-tempo di coordinate $x^{\mu}$ la
metrica si trasforma come un tensore
\begin{equation}
  g_{\mu \nu} (x) = \parder{x'^{\rho}}{x^{\mu}} \parder{x'^{\sigma}}{x^{\nu}}
  g'_{\rho \sigma}(x').
\end{equation}
Se assumiamo valide le condizioni~\eqref{W13.1.1}, possiamo sostituire $g'_{\rho
  \sigma}(x')$ con $g_{\rho \sigma}(x')$ e ottenere la relazione fondamentale
per l'invarianza in forma della metrica
\begin{equation}
  \label{W13.1.2}
  g_{\mu \nu} (x) = \parder{x'^{\rho}}{x^{\mu}} \parder{x'^{\sigma}}{x^{\nu}}
  g_{\rho \sigma}(x')
\end{equation}
In generale, l'equazione~\eqref{W13.1.2} introduce una complicata restrizione
alle funzioni $x'^{\mu}(x)$, che può essere semplificata considerando
trasformazioni di coordinate infinitesime:
\begin{equation}
  x'^{\mu} = x^{\mu} + \epsilon \xi^{\mu}, \qquad \text{con} \quad
  \abs{\epsilon} \ll 1.
\end{equation}
Per una trasformazione infinitesima si ha
\begin{equation}
  \parder{x'{^\rho}}{x^{\mu}} = \delta^{\rho}_{\mu} + \epsilon
  \parder{\xi^{\rho}}{x^{\mu}}
\end{equation}
così che, al primo ordine in $\epsilon$ l'equazione~\eqref{W13.1.2} diventa
\begin{equation}
  g_{\mu \nu} (x) = \biggl(\tensor{\delta}{^{\rho}_{\mu}} +
  \epsilon \parder{\xi^{\rho}}{x^{\mu}}\biggr)
  \biggl(\tensor{\delta}{^{\sigma}_{\nu}} +
  \epsilon \parder{\xi^{\sigma}}{x^{\nu}}\biggr) g_{\rho \sigma}(x')
\end{equation}
da cui segue
\begin{equation}
  g_{\mu \nu} (x) = g_{\mu \nu}(x')
  + \epsilon \parder{\xi^{\rho}}{x^{\mu}} g_{\rho \nu}(x') +
  \epsilon \parder{\xi^{\sigma}}{x^{\nu}} g_{\mu \sigma}(x').
\end{equation}
Possiamo ora espandere $g_{\mu \nu}(x'=x+\epsilon \xi)$ in serie di Taylor
attorno a $x$,
\begin{equation}
  g_{\mu \nu} (x') \simeq
  g_{\mu \nu} (x) + \epsilon \xi^{\rho} \frac {\partial g_{\mu \nu}(x)}{\partial x^{\rho}}
\label{13.36}
\end{equation}
In questo modo, trascurando i termini dell'ordine di $\epsilon^2$, si ottiene
\begin{equation}
  0 = \xi^{\rho} g_{\mu \nu ,\rho} + \tensor{\xi}{^{\rho}_{,\mu}} g_{\rho \nu} +
  \tensor{\xi}{^{\sigma}_{,\nu}} g_{\mu \sigma}
\end{equation}
Questa equazione può essere riscritta in termini delle componenti covarianti del
vettore $\xi_{\nu} = g_{\rho \nu} \xi^{\rho}$ e $\xi_{\mu} = g_{\mu \sigma}
\xi^{\sigma}$.  Si ottiene:
\begin{equation}
  0 = \xi^{\rho} g_{\mu \nu ,\rho}
  +   \xi_{\nu,\mu} + \xi_{\mu,\nu}
  - \xi^{\rho} g_{\rho \nu,\mu} - \xi^{\sigma} g_{\mu \sigma,\nu}
\end{equation}
anche uguale a (utilizzando la definizione di $\tensor{\Gamma}{^{\rho}_{\mu
    \nu}}$)
\begin{equation}
  0 =   \xi_{\nu,\mu} + \xi_{\mu,\nu}
  + \xi^{\rho} (g_{\mu \nu,\rho}
  -             g_{\rho \nu,\mu} - g_{\mu \rho,\nu})
  =   \xi_{\nu,\mu} + \xi_{\mu,\nu} -2 \xi_{\rho} \tensor{\Gamma}{^{\rho}_{\mu
      \nu}}
\end{equation}
e infine
\begin{equation}
  \label{W13.1.5}
  \xi_{\mu;\nu}+\xi_{\nu;\mu} =0.
\end{equation}
Questa equazione differenziale è detta \emph{equazione di Killing}.  Essa è
stata ottenuta per trasformazioni infinitesime, ma si può dimostrare che una
trasformazione finita può essere ottenuta (integrando) a partire da
trasformazioni infinitesime.

Si osservi anche che dall'equazione~\eqref{13.36} si ha in generale che per una
trasformazione infinitesima (vedi~\textcite[355]{landau:campi})
\begin{equation}
  \delta g_{\mu \nu}(x) \equiv g'_{\ \mu \nu}(x) - g_{\mu \nu}(x) =
  -\xi_{\mu;\nu}-\xi_{\nu;\mu}
\label{L94.3}
\end{equation}
Questa relazione permette di scrivere la variazione dell'azione della materia
$\delta S_{\textup{m}}$ in una forma differente, da cui si deduce che il tensore
$T_{\mu \nu}$ definito nell'equazione~\eqref{5.18} soddisfa l'equazione
\begin{equation}
  \tensor{T}{^{\mu}_{\nu ; \mu}} = 0.
\end{equation}
Confrontando questa equazione con la legge di conservazione
$\lparder{\tensor{T}{^{\mu}_{\nu}}}{x^{\mu}}=0$ ottenuta in relatività speciale,
vediamo appunto che $T_{\mu \nu}$ è il tensore energia-impulso.

Cominciamo a osservare che $\delta g^{\mu \nu}(x) = \xi^{\mu;\nu}+\xi^{\nu;\mu}$
poiché $\delta g^{\mu \nu}(x) \delta g_{\rho \nu} =
\tensor{\delta}{^{\mu}_{\rho}}$.  Ora possiamo riscrivere $\delta
S_{\textup{m}}$ nel seguente modo:
\begin{equation}
  \delta S_{\textup{m}} = \frac{1}{2c}\int T_{\mu \nu} (\xi^{\mu ;\nu} +
  \xi^{\nu;\mu}) \sqrt{g} d^4x = \frac{1}{c} \int T_{\mu \nu}  \xi^{\mu;\nu}
  \sqrt{g} \dd^4x
\end{equation}
Trasformiamo ora questa espressione nel seguente modo:
\begin{equation}
  \delta S_{\textup{m}} = \frac{1}{c} \int (\tensor{T}{^{\mu}_{\nu}}  \xi^{\nu})_{;\mu}
  \sqrt{g} \dd^4x - \frac{1}{c} \int \tensor{T}{^{\mu}_{\nu; \mu}} \xi^{\nu}
  \sqrt{g} \dd^4x.
\end{equation}
Il primo integrale è nullo perché attraverso il teorema di Gauss può essere
trasformato in un integrale al bordo $\delta \Omega$ su cui le variazioni
$\delta g_{\mu \nu}=0$ e quindi anche $\xi^{\nu}=0$.  Si trova dunque che
annullando $\delta S_{\textup{m}}$:
\begin{equation}
  \delta S_{\textup{m}} = - \frac{1}{c} \int \tensor{T}{^{\mu}_{\nu; \mu}}
  \xi^{\nu} \sqrt{g} \dd^4x
\end{equation}
e quindi poiché le variazioni $\xi^{\nu}$ della metrica sono arbitrarie,
$\tensor{T}{^{\mu}_{\nu:\mu}}=0$.

È ben noto che secondo il teorema di Noether a ogni simmetria di un sistema
fisico corrisponde una legge di conservazione.  Ad esempio, nello spazio-tempo
piatto la simmetria per traslazioni spaziali implica la conservazione
dell'impulso, la simmetria per traslazioni temporali implica la conservazione
dell'energia e quella per rotazioni la conservazione del momento angolare.
Nello spazio-tempo curvo possiamo formulare le leggi di conservazione a partire
dai vettori di Killing.

Ad esempio, dimostriamo che se $\xi_{\mu}$ è un vettore di Killing per la
metrica, allora se una particella si muove lungo una geogetica, il prodotto
scalare del vettore di Killing per il 4-impulso della particella $P^{\mu}= m
\ltoder{x^{\mu}}{\tau}$ è costante.  Infatti
\begin{equation}
  \label{ORp305,186}
  \toder{}{\tau} (\xi_{\mu} P^{\mu}) = \xi_{\mu} \toder{P^{\mu}}{\tau} + P^{\mu}
  \toder{\xi_{\mu}}{\tau}.
\end{equation}
Usiamo l'equazione della geodetica
\begin{equation}
  \toder{P^{\mu}}{\tau} = -\tensor{\Gamma}{^{\mu}_{\alpha \nu}} P^{\alpha}
  \toder{x^{\nu}}{\tau}
\end{equation}
e la definizione di derivata covariante
\begin{equation}
  \toder{\xi_{\mu}}{\tau} = \parder{\xi_{\mu}}{x^{\nu}} \toder{x^{\nu}}{\tau} =
  (\xi_{\mu;\nu} + \tensor{\Gamma}{^{\alpha}_{\mu \nu}} \xi_{\alpha})
  \toder{x^{\nu}}{\tau}.
\end{equation}
Sostituiamo ora queste due espressioni nell'equazione~\eqref{ORp305,186}.  I due
termini con la connessione affine si elidono e rimane
\begin{equation}
  \label{ORp305,187}
  \toder{}{\tau} (\xi_{\mu} P^{\mu}) = \xi_{\mu;\nu} P^{\mu}
  \toder{x^{\nu}}{\tau} = \frac{1}{2m} (\xi_{\mu;\nu} + \xi_{\nu;\mu}) P^{\mu}
  P^{\nu}
\end{equation}
da cui se $\xi_{\mu}$ soddisfa l'equazione~\eqref{W13.1.5} di Killing,
$\xi_{\mu} P^{\mu}$ si conserva lungo la geodetica.

Il metodo usuale per la costruzione dei vettori di Killing della metrica è
quello di cercare quali sono le coordinate dello spazio-tempo dalle quali la
metrica risulta indipendente.  Ad esempio, se $g_{\mu \nu}$ non dipende dal
tempo un vettore di Killing associato a questa metrica è $\xi^{\mu} = (1,0,0,0)$
e come conseguenza dalla condizione $\xi^{\mu} P_{\mu}= \text{cost}$ segue che
l'energia della particella è costante nel tempo (se la particella si muove sulla
geodetica):
\begin{equation}
  \xi^{0} g_{00} p^0 = - \biggl(1 - \frac{r}{r_{\rm S}}\biggr) \toder{t}{\tau} =
  -E = \text{cost}.
\end{equation}

In modo analogo, se la metrica non dipende dalla coordinata $\phi$, (la metrica
è quindi invariante per rotazioni attorno all'asse $z$) il corrispondente
vettore di Killing è $\xi^{\mu}=(0,0,0,1)$, da cui segue la legge di
conservazione del momento angolare. In particolare poiché per il moto lungo una
geodetica il moto avviene in un piano $\theta=\text{cost}$ (e quindi senza
perdere in generalità $\sin \theta=1$) dalla condizione $\xi^{\mu} P_{\mu}=
\text{cost}$ si ha
\begin{equation}
  \xi^{3} g_{33} p^3 = - r^2 \toder{\phi }{\tau} = - \frac{L}{m} = \text{cost}.
\end{equation}

Un modo equivalente per derivare la conservazione dell'energia e del momento
angolare di una particella in moto lungo una geodetica è quello di scrivere
esplicitamente la lagrangiana della particella
\begin{equation}
  \mathcal{L} = \frac{1}{2} g_{\mu \nu} \dot{x}^{\mu} \dot{x}^{\nu}.
\end{equation}
Per derivare l'espressione precedente uso il principio di generale covarianza:
scrivo la lagrangiana di una particella libera in relatività speciale e
sostituisco la metrica $g_{\mu \nu}$ a $\eta_{\mu \nu}$.  Ad esempio, per la
metrica di Schwarzschild
\begin{equation}
  2\mathcal{L} = \biggl(1-\frac{2m}{r}\biggr)\dot{t}^2 -
  \biggl(1-\frac{2m}{r}\biggr)^{-1}\dot{r}^2 - r^2 (\dot{\theta}^2 +
  \sin^2\theta\dot{\phi}^2)
\end{equation}
dove il punto denota la derivata rispetto a $\tau$ ed $m = GM/c^{2}$.  Poiché le
variabili $t$ e $\phi$ sono cicliche (la metrica non dipende da queste
variabili), si ha che i corrispondenti momenti coniugati sono costanti del
moto. In particolare
\begin{subequations}
  \begin{align}
    -\parder{\mathcal{L}}{\dot{\phi}} &= r^2 \dot{\phi} = L = \text{cost}, \\
    \parder{\mathcal{L}}{\dot{t}} &= \biggl(1-\frac{2m}{r}\biggr) \dot{t} = E =
                                    \text{cost}.
  \end{align}
\end{subequations}

\section{Pseudo tensore energia-impulso del campo gravitazionale}
\label{sec:pseudo-tensore-energia-impulso-campo-grav}

Vedi~\textcite[367-374]{landau:campi}.

%%% Local Variables:
%%% mode: latex
%%% TeX-master: "../gravitazione"
%%% fill-column: 80
%%% End:
