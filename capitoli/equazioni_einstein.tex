\cleardoublepage

\chapter{Equazioni del campo gravitazionale}
\label{cha:equazioni-campo-grav}

\section{Derivazione delle Equazioni di Einstein}

È possibile derivare le equazioni di Einstein per il campo gravitazionale usando
criteri del tutto generali.  Infatti equazioni di campo accettabili devono
essere (vedi~\textcite[151]{weinberg:gravitation}):
\begin{enumerate}
\item in forma tensoriale;
\item fornire per campi deboli e statici la soluzione Newtoniana;
\item contenere termini (con $N=2$) quadratici nella derivata prima e lineari
  nella derivata seconda della metrica.  Tale richiesta nasce dal fatto che si
  vuole che le equazioni cercate siano invarianti in scala.
\end{enumerate}
L'equazione di campo newtoniana è l'equazione di Poisson
\begin{equation}
  \nabla^2 \phi = 4 \pi G \rho,
\end{equation}
dove $\rho$ è la densità di massa che naturalmete possiamo identificare con la
componente $00$ del tensore energia-impulso: $\rho = T_{00}$.  Con l'ulteriore
identificazione (per campi deboli e statici) $h_{00} = - 2 \phi$, possiamo
riscrivere l'equazione di Poisson nella forma
\begin{equation}
  \nabla^2 h_{00} = - 8 \pi G T_{00}.
\end{equation}
È naturale a questo punto ipotizzare per l'equazione cercate la forma
\begin{equation}
  G_{\mu \nu} = - 8 \pi G T_{\mu \nu}
\end{equation}
dove il termine $G_{\mu \nu}$ deve
\begin{itemize}
\item[a)] essere un tensore;
\item[b)] soddisfare la 3;
\item[c)] essere simmetrico così come $T_{\mu \nu}$;
\item[d)] avere 4-divergenza nulla poiché il tensore energia impulso soddisfa la
  condizione di conservazione $\tensor{T}{^{\mu \nu}_{; \mu}} =0$;
\item[e)] per un campo statico e debole prodotto da materia non relativistica
  $G_{00} \simeq \nabla^2 g_{00}$.
\end{itemize}
Dalle condizioni a) + b) segue che necessariamente
\begin{equation}
  G_{\mu \nu} = C_1 R_{\mu \nu} + C_2 R g_{\mu \nu}
\end{equation}
dove $R_{\mu \nu}$ è il tensore di Ricci ed $R$ la curvatura scalare dello
spazio-tempo.

La condizione c) è immediatamente soddisfatta per la simmetria del tensore di
Ricci; la condizione d) implica la relazione $C_1 \tensor{R}{^{\mu \nu}_{;\mu}}
+ C_2 R_{;\mu} g^{\mu \nu} = 0$.  Utilizzando l'identità di Bianchi contratta
($\tensor{R}{^{\mu \nu}_{; \mu}} = (1/2) g^{\mu \nu} R_{; \mu}$), l'ultima
relazione diventa $(C_2 + C_1/2) R_{; \nu} =0$, la quale senza perdere di
generalità implica infine la relazione tra i coefficienti $C_2 = -C_1/2$.
Abbiamo allora
\begin{equation}
  G_{\mu \nu} = C_1 \left(R_{\mu \nu} - \frac{1}{2} g_{\mu \nu} R \right).
\end{equation}
Dalla condizione e) fissiamo $C_1=1$.  Per un sistema non relativistico risulta
$T_{00} \simeq \rho \gg T_{ij} \simeq \rho v^i v^j$. Allora è naturale porre
$G_{00} \gg G_{ij} \simeq 0$.  Questo implica $G_{ij} = C_1 (R_{ij} -(1/2) R)
\simeq 0$.  Quindi $R_{ij} \simeq (1/2) R$.  Moltiplicando la precedente per
$\eta_{ij}=\delta_{ij}$ abbiamo $R^j_j \simeq (3/2) R$.  Allora la curvatura $R
\simeq \eta^{\mu \nu} R_{\mu \nu} \simeq R^j_j - R_{00} \simeq (3/2)R-R_{00}$,
da cui deduciamo la relazione $R \simeq 2 R_{00}$.

Abbiamo quindi $G_{00} = C_1 (R_{00}-(1/2) g_{00} R) \simeq C_1 (R_{00}-(1/2)
eta_{00} R) \simeq 2 C_1 R_{00}$ Ora $R_{00} = \tensor{R}{^{l}_{0l0}} = (1/2)
\nabla^2 g_{00}$ e quindi in conclusione $C_1=1$.  Pertanto le equazioni cercate
sono
\begin{equation}
  G_{\mu \nu} = R_{\mu \nu} - \frac{1}{2} g_{\mu \nu} R = -8 \pi G T_{\mu\nu}.
\end{equation}
Un'espressione alternativa delle equazioni di Einstein può essere ottenuta
contraendo nella precedente gli indici (moltiplicando per $g^{\mu \nu}$).  Si ha
$R-(1/2) 4 R = -8 \pi G \tensor{T}{^{\lambda}_{\lambda}}$ da cui segue $R= 8 \pi
G = -8 \pi G \tensor{T}{^{\lambda}_{\lambda}}$, e sostituendo:
\begin{equation}
  R_{\mu \nu}= -8 \pi G \left( T_{\mu \nu} - \frac{1}{2} g_{\mu \nu}
    \tensor{T}{^{\lambda}_{\lambda}} \right),
\end{equation}
Nel vuoto le equazioni di Einstein sono
\begin{equation}
  R_{\mu \nu}=0.
\end{equation}

\section{Derivazione formale delle Equazioni di Einstein}
\label{sec:equazioni-einstein}

\emph{Nota: in questo paragrafo, dove non diversamente specificato, indicheremo
  esplicitamente la velocità della luce nel vuoto $c$ e adotteremo la
  convenzione dei segni $(-++)$ seguita
  da~\textcites{barone:relativita,landau:campi}.}

Vogliamo ora determinare le equazioni del campo gravitazionale $g_{\mu\nu}$ a
partire dal principio variazionale di Hamilton.  Assumiamo che l'azione $S$ sia
somma di una componente $S_{\textup{g}}$ che descrive solo il campo
gravitazionale e una componente $S_{\textup{m}}$ relativa alla materia che
interagisce con il campo.  L'azione deve essere uno scalare, in modo che risulti
invariante per trasformazioni arbitrarie delle coordinate, e dipenderà dal
tensore metrico, dalle sue derivate prime, dalle coordinate generalizzate $q$ e
dalle velocità generalizzate $\lparder{q}{x^{\lambda}}$, $S = S(g_{\mu\nu},
\lparder{g_{\mu\nu}}{x^{\lambda}}; q, \lparder{q}{x^{\lambda}})$.

Hilbert propose questa espressione per l'azione del campo
gravitazionale\footnote{Il fattore davanti all'integrale è stato scelto in
  maniera tale che nel limite newtoniano si ottenga l'equazione di Poisson
  $\nabla^{2} \phi = 4\pi G\rho$.}
\begin{equation}
  S_{\textup{g}} = -\frac{c^{3}}{16\pi G} \int R \sqrt{g} \dd^{4} x,
\end{equation}
chiamata \index{azione!di Hilbert-Einstein} \emph{azione di Hilbert-Einstein}.
$G = \SI{6.673 84(80)e-11}{\cubic\metre\per\kilo\metre\per\second\squared}$ è la
\index{costante!di gravitazione universale}
\emph{costante di gravitazione universale}.  L'azione della materia ha invece
espressione\footnote{In uno spazio-tempo piatto $g = 1$, quindi
  $S_{\textup{m}} = (1/c)\int \Lambda \dd^{4} x = \iint \Lambda \dd t\dd^{3}
  \bm{x}$, come nella~\eqref{eq:azione-campo}.}
\begin{equation}
  S_{\textup{m}} = \frac{1}{c} \int \Lambda \sqrt{g} \dd^{4} x,
\end{equation}
in cui $\Lambda = \Lambda(g^{\mu\nu}, \partial_{\lambda}g^{\mu\nu})$ è la densità
di lagrangiana associata al sistema che interagisce con il campo
gravitazionale.  L'interazione fra il campo e la materia è inclusa in
$S_{\textup{m}}$.

Per determinare le equazioni del campo gravitazionale dobbiamo richiedere che
l'azione sia stazionaria rispetto alle variazioni, a volume fissato,
$\delta g_{\mu\nu}$ del campo che devono soddisfare inoltre la condizione che si
annullino sul bordo dell'ipervolume di integrazione.  Dunque dobbiamo imporre
$\delta S = \delta(S_{\textup{g}} + S_{\textup{m}}) = 0$.  Calcoliamo il
contributo di $\delta S_{\textup{g}}$
\begin{equation}
  \label{eq:var-HE}
  \begin{split}
    \delta \int R \sqrt{g} \dd^{4} x &= \delta \int g^{\mu\nu} R_{\mu\nu}
    \sqrt{g} \dd^{4} x \\
    &= \int (\delta g^{\mu\nu} R_{\mu\nu} \sqrt{g} + g^{\mu\nu} R_{\mu\nu}
    \delta \sqrt{g} + g^{\mu\nu} \delta R_{\mu\nu} \sqrt{g}) \dd^{4} x.
  \end{split}
\end{equation}
Osserviamo che
\begin{equation}
  \delta\sqrt{g} = \parder{\sqrt{g}}{g} \parder{g}{g^{\mu\nu}} \delta g^{\mu\nu}
  = \frac{1}{2 \sqrt{g}} (-g g_{\mu\nu} ) \delta g^{\mu\nu} = - \frac{1}{2}
  g_{\mu\nu} \sqrt{g} \delta g^{\mu\nu},
\end{equation}
in cui abbiamo ricordato la~\eqref{eq:g_mu_rho-determinante2}.  Sostituendo
nella~\eqref{eq:var-HE} otteniamo
\begin{equation}
  \delta \int R \sqrt{g} \dd^{4} x = \int \bigg( R_{\mu\nu} - \frac{1}{2}
  g_{\mu\nu} R \bigg)\sqrt{g} \delta g^{\mu\nu} \dd^{4} x + \int g^{\mu\nu}
  \delta R_{\mu\nu} \sqrt{g} \dd^{4} x.
\end{equation}

Mostriamo che il secondo integrale è nullo.  Innanzitutto la
\index{tensore!di Ricci!variazione del}variazione del tensore di Ricci può
essere espressa in termini di variazioni della connessione affine mediante
l'\index{identità!di Palatini}\emph{identità di
  Palatini}\footnote{Si
  noti che la variazione della connessione affine è un tensore poiché è la
  differenza fra due connessioni affini, la quale è un tensore come visto nel
  paragrafo~\ref{sec:connessione-affine}.}
(vedi l'appendice~\ref{sec:dimostr-palatini})
\begin{equation}
  \delta R_{\mu\nu} = (\delta \tensor{\Gamma}{^{\lambda}_{\mu\nu}})_{;\lambda} -
  (\delta \tensor{\Gamma}{^{\lambda}_{\mu\lambda}})_{;\nu}.
\end{equation}
Allora, ricordando che $g^{\mu\nu}$ commuta con la derivazione covariante, si
può riscrivere l'integrale in esame come
\begin{equation}
  \begin{split}
    \int g^{\mu\nu} \delta R_{\mu\nu} \sqrt{g} \dd^{4} x &= \int [ ( g^{\mu\nu}
    \delta \tensor{\Gamma}{^{\lambda}_{\mu\nu}} )_{;\lambda} - ( g^{\mu\nu} \delta
    \tensor{\Gamma}{^{\lambda}_{\mu\lambda}})_{;\nu} ] \sqrt{g} \dd^{4} x \\
    &= \int g^{\mu\nu} \delta \tensor{\Gamma}{^{\lambda}_{\mu\nu}} \sqrt{g} \dd
    \Sigma_{\lambda} - \int g^{\mu\nu}
    \delta\tensor{\Gamma}{^{\lambda}_{\mu\lambda}} \sqrt{g}\dd\Sigma_{\nu} = 0
  \end{split}
\end{equation}
in cui abbiamo applicato il teorema di Gauss in forma
covariante~\eqref{eq:gauss-covariante}.  Gli ultimi due integrali sulle
ipersuperfici sono entrambi nulli poiché vanno calcolati sul bordo
dell'ipervolume costante su cui si calcola la variazione dell'azione e sul bordo
sono nulle le variazioni di $g_{\mu\nu}$ e quindi anche della connessione
affine, la quale dipende dal tensore metrico.  Dunque abbiamo trovato che la
variazione dell'azione di Hilbert-Einstein è
\begin{equation}
  \delta S_{\textup{g}} = -\frac{c^{3}}{16\pi G} \int \bigg( R_{\mu\nu} -
  \frac{1}{2} g_{\mu\nu} R \bigg)\sqrt{g} \delta g^{\mu\nu} \dd^{4} x.
\end{equation}

Per calcolare la variazione $\delta S_{\textup{m}}$ dell'azione associata alla
materia ragioniamo in maniera analoga a quanto fatto nel
paragrafo~\ref{sec:tensore-energia-impulso}
\begin{equation}
  \begin{split}
    \delta S_{\textup{m}} &= \frac{1}{c} \delta \int \Lambda(g^{\mu\nu},
    \tensor{g}{^{\mu\nu}_{,\lambda}}) \sqrt{g} \dd^{4} x = \frac{1}{c} \int
    \bigg( \parder{(\sqrt{g}\Lambda)}{g^{\mu\nu}} \delta g^{\mu\nu}
    + \parder{(\sqrt{g}\Lambda)}{\tensor{g}{^{\mu\nu}_{,\lambda}}}
    \delta \tensor{g}{^{\mu\nu}_{,\lambda}} \bigg) \dd^{4} x \\
    &= \frac{1}{c} \int \bigg( \parder{(\sqrt{g}\Lambda)}{g^{\mu\nu}} \delta
    g^{\mu\nu} + \parder{}{x^{\lambda}}
    \bigg( \parder{(\sqrt{g}\Lambda)}{\tensor{g}{^{\mu\nu}_{,\lambda}}} \delta
    g^{\mu\nu}\bigg) - \delta g^{\mu\nu} \parder{}{x^{\lambda}}
    \bigg( \parder{(\sqrt{g}\Lambda)}{\tensor{g}{^{\mu\nu}_{,\lambda}}} \bigg)
    \bigg) \dd^{4} x \\
    &= \frac{1}{c} \int \bigg( \parder{(\sqrt{g}\Lambda)}{g^{\mu\nu}}
    - \parder{}{x^{\lambda}}
    \bigg( \parder{(\sqrt{g}\Lambda)}{\tensor{g}{^{\mu\nu}_{,\lambda}}} \bigg)
    \bigg) \delta g^{\mu\nu} \dd^{4} x.
  \end{split}
\end{equation}
Nell'ultimo passaggio abbiamo sfruttato il fatto che l'integrale del secondo
termine si trasforma, per il teorema di Gauss, in un integrale sul bordo del
volume di integrazione sul quale le variazioni $\delta g^{\mu\nu}$ sono nulle.
Introduciamo il tensore $T_{\mu\nu}$ definito da
\begin{equation}
 - \frac{1}{2} \sqrt{g} T_{\mu\nu} = \parder{(\sqrt{g}\Lambda)}{g^{\mu\nu}}
  - \parder{}{x^{\lambda}}
  \bigg( \parder{(\sqrt{g}\Lambda)}{\tensor{g}{^{\mu\nu}_{,\lambda}}} \bigg)
\end{equation}
% TODO: spiegare da qualche parte che questo è proprio il tensore
% energia-impulso (e mettere \index{tensore!energia-impulso})
così la variazione dell'azione della materia diventa
\begin{equation}
  \delta S_{\textup{m}} = - \frac{1}{2c} \int T_{\mu\nu} \delta g^{\mu\nu}
  \sqrt{g} \dd^{4} x
\end{equation}
e la condizione di stazionarietà dell'azione totale del sistema costituito dal
campo gravitazionale e dalle particelle interagenti è
\begin{equation}
  \delta S = \delta(S_{\textup{g}} + S_{\textup{m}}) = -\int \bigg(
  \frac{c^{3}}{16\pi G} \bigg( R_{\mu\nu} - \frac{1}{2} g_{\mu\nu} R \bigg) -
  \frac{1}{2c} T_{\mu\nu} \bigg) \sqrt{g} \delta g^{\mu\nu} \dd^{4} x = 0.
\end{equation}
Data l'arbitrarietà delle variazioni $\delta g^{\mu\nu}$ ricaviamo
\begin{equation}
  \label{eq:einstein}
  G_{\mu\nu} = R_{\mu\nu} - \frac{1}{2} g_{\mu\nu} R = \frac{8\pi G}{c^{4}}
  T_{\mu\nu}
\end{equation}
che sono le \emph{equazioni del campo gravitazionale} in presenza di sorgenti,
chiamate anche \index{equazioni!di Einstein}\emph{equazioni di Einstein}.

Per commentare queste equazioni si suol dire che la materia dice allo
spazio-tempo come curvarsi (informazione contenuta nel tensore di Ricci) e allo
stesso tempo lo spazio-tempo curvo dice alla materia come muoversi (tensore
energia-impulso).  Osserviamo che dalla~\eqref{eq:divergenza-tens-einstein}
deriva che anche il tensore energia-impulso della materia ha quadridivergenza
covariante nulla $\tensor{T}{^{\mu\nu}_{;\mu}} = 0$.

Riprendiamo la convenzione dei segni $(+--)$ di~\textcite{weinberg:gravitation},
nella quale compare un segno meno nell'ultimo membro dell'equazione di Einstein,
come spiegato a pagina~\pageref{eq:convenzione-segni}.  Moltiplicando gli ultimi
due membri per $g^{\mu\nu}$, in modo da contrarre gli indici $\mu$ e $\nu$,
otteniamo
\begin{equation}
  R - \frac{1}{2}4R = -R = -\frac{8\pi G}{c^{4}}\tensor{T}{^{\lambda}_{\lambda}}
\end{equation}
e sostituendo nella~\eqref{eq:einstein} otteniamo un'espressione alternativa per
le \index{equazioni!di Einstein}equazioni di Einstein
\begin{equation}
  R_{\mu\nu} = -\frac{8\pi G}{c^{4}} \bigg( T_{\mu\nu} -
  \frac{1}{2}g_{\mu\nu}\tensor{T}{^{\lambda}_{\lambda}} \bigg).
\end{equation}
Da qui possiamo osservare che nel vuoto, cioè in assenza di materia, il tensore
energia-impulso è nullo quindi le \index{equazioni!di Einstein}equazioni di
Einstein si riducono a
\begin{equation}
  R_{\mu\nu} = 0.
\end{equation}
Si noti che questa equazione non significa che uno spazio-tempo vuoto è anche
piatto, per avere quest'ultima condizione deve verificarsi la condizione più
forte $\tensor{R}{^{\lambda}_{\mu\nu\kappa}} = 0$.

\subsection{Limite newtoniano delle equazioni di Einstein}
\label{sec:limite-newtoniano-einstein}

Vogliamo determinare il limite newtoniano (campi gravitazionali deboli e
statici, velocità non relativistiche) delle equazioni di Einstein e per fare
questo lavoriamo in maniera analoga a quanto fatto nel
paragrafo~\ref{sec:limite-newtoniano}, quindi poniamo
$g_{\mu\nu} = \eta_{\mu\nu} + h_{\mu\nu}$.  Ci aspettiamo che il tensore
energia-impulso sia in queste condizioni quello ricavato nel limite di basse
velocità nel paragrafo~\ref{sec:fluido-perfetto} per un corpo macroscopico
continuo (come un \index{fluido!perfetto}fluido perfetto)
\begin{subequations}
  \begin{align}
    T_{ij} &= \rho c^{2} u_{i} u_{j}, \\
    T_{00} &= \rho c^{2},
  \end{align}
\end{subequations}
in cui qui con $\rho$ indichiamo la densità di massa a riposo (e quindi
$\rho c^{2}$ è la densità di energia a riposo, per brevità omettiamo il pedice
$0$) e $u_{\mu}$ la quadrivelocità.  Per l'ipotesi di velocità non
relativistiche si devono trascurare le componenti spaziali della quadrivelocità
rispetto alla componente temporale, cioè $u_{i} = 0$, e quindi $T_{ij} = 0$ da
cui abbiamo inoltre
\begin{equation}
  0 = G_{ij} \approx R_{ij} - \frac{1}{2}\eta_{ij}R.
\end{equation}
Allora l'unica equazione di Einstein rilevante è quella con $G_{00}$.
Contraendo gli indici dell'equazione precedente otteniamo
\begin{equation}
  \eta^{ij}R_{ij} \approx \frac{1}{2} \eta_{ij}\eta^{ij} R = \frac{3}{2} R.
\end{equation}
La curvatura scalare può essere approssimata come
\begin{equation}
  R = g^{\mu\nu}R_{\mu\nu} \approx \eta^{\mu\nu} R_{\mu\nu} = \eta^{00}R_{00} +
  \eta^{ij}R_{ij} = -R_{00} + \frac{3}{2} R
\end{equation}
dunque $R \approx 2 R_{00}$.  Sostituendo nell'equazione di Einstein abbiamo
\begin{equation}
  G_{00} = R_{00} - \frac{1}{2} \eta_{00}R = R_{00} + \frac{1}{2} R = 2 R_{00} =
  -\frac{8\pi G}{c^{4}} T_{00} = -\frac{8\pi G}{c^{4}} \rho c^{2}
\end{equation}
da cui
\begin{equation}
  R_{00} = -\frac{4\pi G}{c^{2}} \rho.
\end{equation}
D'altra parte osserviamo che
\begin{equation}
  R_{00} = g^{\mu\nu}R_{\mu 0\nu 0} \approx \eta^{\mu\nu}R_{\mu 0\nu 0}
\end{equation}
Dalla~\eqref{eq:riemann-covariante}, arrestandoci ai termini del primo ordine in
$h_{\mu\nu}$, abbiamo
\begin{equation}
  \label{eq:riemann-approx}
  \begin{split}
    R_{\lambda\mu\nu\kappa} &\approx \frac{1}{2}
    \bigg( \parder{h_{\lambda\nu}}{x^{\kappa}, x^{\mu}}
    - \parder{h_{\mu\nu}}{x^{\kappa},
      x^{\lambda}}- \parder{h_{\lambda\kappa}}{x^{\nu}, x^{\mu}}
    + \parder{h_{\mu\kappa}}{x^{\nu}, x^{\lambda}} \bigg) \\
    &\quad{} + \eta_{\eta\sigma}
    (\underbrace{\tensor{\Gamma}{^{\eta}_{\nu\lambda}}
      \tensor{\Gamma}{^{\sigma}_{\mu\kappa}} -
      \tensor{\Gamma}{^{\eta}_{\kappa\lambda}}
      \tensor{\Gamma}{^{\sigma}_{\mu\nu}}}_{\mathcal{O}(h^{2})}) \\
    &= \frac{1}{2} \bigg( \parder{h_{\lambda\nu}}{x^{\kappa}, x^{\mu}}
    - \parder{h_{\mu\nu}}{x^{\kappa},
      x^{\lambda}}- \parder{h_{\lambda\kappa}}{x^{\nu}, x^{\mu}}
    + \parder{h_{\mu\kappa}}{x^{\nu}, x^{\lambda}} \bigg) + \mathcal{O}(h^{2})
  \end{split}
\end{equation}
e in particolare, sfruttando l'ipotesi di campi costanti,
\begin{subequations}
  \begin{align}
    R_{0000} &\approx 0, \\
    R_{i0j0} &\approx \frac{1}{2} \parder{h_{00}}{x^{i},x^{j}}
  \end{align}
\end{subequations}
allora
\begin{equation}
  R_{00} = \eta^{00}R_{0000} + \eta^{ij}R_{i0j0} = \delta^{ij}R_{i0j0} \approx
  \frac{1}{2}\nabla^{2} h_{00} = -\nabla^{2}\frac{\phi}{c^{2}}
\end{equation}
avendo ricordato la~\eqref{eq:h00}.  Dunque mettendo insieme le due espressioni
di $R_{00}$ ottenute troviamo
\index{equazione!di Poisson!per il campo gravitazionale}l'equazione di Poisson
per il campo gravitazionale
\begin{equation}
  \nabla^{2} \phi = 4\pi G\rho.
\end{equation}

\section{Isometrie delo spazio-tempo; vettori di Killing}
\label{sec:vettori di Killing}

Vedi~\textcite[300-306]{ohanian:gravitazione}.

%%% Local Variables:
%%% mode: latex
%%% TeX-master: "../gravitazione"
%%% fill-column: 80
%%% End:
