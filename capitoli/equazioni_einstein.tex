\cleardoublepage
\chapter{Equazioni del campo gravitazionale}
\label{cha:equazioni-campo-grav}

\section{Tensore di Riemann}
\label{sec:tensore-riemann}

\subsection{Definizione e proprietà algebriche}
\label{sec:definizione-proprietà-riemann}

Si può far vedere che l'unico tensore ottenibile dal tensore metrico e dalle sue
derivate prime e seconde è il \emph{tensore di curvatura di Riemann-Christoffel}
o, semplicemente, \index{tensore!di Riemann}\emph{tensore di Riemann}
\begin{equation}
  \tensor{R}{^{\lambda}_{\mu\nu\kappa}}
  = \parder{\tensor{\Gamma}{^{\lambda}_{\mu\nu}}}{x^{\kappa}}
  - \parder{\tensor{\Gamma}{^{\lambda}_{\mu\kappa}}}{x^{\nu}} +
  \tensor{\Gamma}{^{\eta}_{\mu\nu}} \tensor{\Gamma}{^{\lambda}_{\eta\kappa}} -
  \tensor{\Gamma}{^{\eta}_{\mu\kappa}} \tensor{\Gamma}{^{\lambda}_{\eta\nu}}.
\end{equation}
Il tensore di Riemann è lineare nelle derivate seconde del tensore metrico.

Per analizzare le proprietà algebriche del tensore di Riemann conviene
considerare la sua forma completamente covariante
\begin{equation}
  \label{eq:riemann-covariante}
  \begin{split}
    R_{\lambda\mu\nu\kappa} &= g_{\lambda\sigma}
    \tensor{R}{^{\sigma}_{\mu\nu\kappa}} = g_{\lambda\sigma}
    \bigg( \parder{\tensor{\Gamma}{^{\lambda}_{\mu\nu}}}{x^{\kappa}}
    - \parder{\tensor{\Gamma}{^{\lambda}_{\mu\kappa}}}{x^{\nu}} +
    \tensor{\Gamma}{^{\eta}_{\mu\nu}} \tensor{\Gamma}{^{\lambda}_{\eta\kappa}} -
    \tensor{\Gamma}{^{\eta}_{\mu\kappa}}
    \tensor{\Gamma}{^{\lambda}_{\eta\nu}} \bigg) \\
    &= \frac{1}{2} \bigg( \parder{g_{\lambda\nu}}{x^{\kappa}, x^{\mu}}
    - \parder{g_{\mu\nu}}{x^{\kappa}, x^{\lambda}}
    - \parder{g_{\lambda\kappa}}{x^{\nu}, x^{\mu}}
    + \parder{g_{\mu\kappa}}{x^{\nu}, x^{\lambda}} \bigg) \\
    &+ g_{\eta\sigma} (\tensor{\Gamma}{^{\eta}_{\nu\lambda}}
    \tensor{\Gamma}{^{\sigma}_{\mu\kappa}} -
    \tensor{\Gamma}{^{\eta}_{\kappa\lambda}}
    \tensor{\Gamma}{^{\sigma}_{\mu\nu}}).
  \end{split}
\end{equation}
Questa relazione ci tornerà utile quando considereremo il limite newtoniano
delle equazioni di Einstein.  Da questa espressione si vede che
\begin{enumerate}
\item \label{item:antisimmetria-riemann} il tensore di Riemann è antisimmetrico
  rispetto allo scambio fra primo e secondo indice e allo scambio fra terzo e
  quarto indice
  \begin{equation}
    R_{\lambda\mu\nu\kappa} = -R_{\mu\lambda\nu\kappa} =
    -R_{\lambda\mu\kappa\nu} = R_{\mu\lambda\kappa\nu};
  \end{equation}
\item \label{item:simmetria-riemann} il tensore di Riemann
  $R_{\lambda\mu\nu\kappa}$ è simmetrico rispetto allo scambio fra la prima
  coppia indici $(\lambda,\mu)$ con la seconda coppia $(\nu,\kappa)$
  \begin{equation}
    R_{\lambda\mu\nu\kappa} = R_{\nu\kappa\lambda\mu};
  \end{equation}
\item \label{item:ciclicita-riemann} è valida la seguente relazione di ciclicità
  (si noti che il primo indice è tenuto fisso, si fanno ruotare gli altri tre)
  \begin{equation}
    \label{eq:ciclicita-riemann}
    R_{\lambda\mu\nu\kappa} + R_{\lambda\kappa\mu\nu} + R_{\lambda\nu\kappa\mu}
    = 0.
  \end{equation}
\end{enumerate}

A causa delle proprietà appena elencate, delle $4^{4} = 256$ componenti del
tensore di Riemann solo $20$ sono indipendenti.  Per far vedere ciò consideriamo
il caso generale di uno spazio $n$-dimensionale.  Il tensore di Riemann può
essere immaginato come una matrice quadrata simmetrica (per la proprietà di
simmetria~\ref{item:simmetria-riemann}) di ordine $d$ del tipo
$R_{(\lambda\mu)(\nu\kappa)}$, con indici $(\lambda\mu)$ e $(\nu\kappa)$.  Il
valore dell'ordine $d$ di questa matrice è dato dal numero di valori
indipendenti che assumono i due indici che, per la proprietà di
antisimmetria~\ref{item:antisimmetria-riemann}, sono uguali al numero di
elementi indipendenti di una matrice $n \times n$ antisimmetrica, cioè
$d = n(n-1)/2$.  Dunque gli elementi indipendenti di
$R_{(\lambda\mu)(\nu\kappa)}$ si riducono
\begin{equation}
  \frac{1}{2} d (d + 1) = \frac{1}{2}\bigg( \frac{1}{2}n(n-1) \bigg) \bigg(
  \frac{1}{2}n(n-1) +1 \bigg) = \frac{1}{8}n(n-1)(n^{2} - n +2).
\end{equation}
% TODO: provare a capire per bene il perché dell'affermazione successiva e a
% spiegarlo in maniera semplice.  Per esempio vedi
% http://www.maths.tcd.ie/~houghton/TEACHING/442/notes/22Oct2002.pdf
% http://sergeysk.wordpress.com/2011/01/21/number-of-independent-components-of-the-riemann-curvature-tensor/
% http://ned.ipac.caltech.edu/level5/March01/Carroll3/Carroll3.html (è il
% capitolo 3 del Carrol)
Infine, la relazione di ciclicità~\eqref{eq:ciclicita-riemann} corrisponde a
$\binom{n}{4}$ equazioni differenti, allora abbiamo altrettanti ulteriori
vincoli, così che gli elementi indipendenti del tensore di Riemann
$n$-dimensionale sono
\begin{equation}
  C_{n} = \frac{1}{8}n(n-1)(n^{2} - n +2) - \binom{n}{4} =
  \frac{1}{12}n^{2}(n^{2} - 1).
\end{equation}
Ponendo $n=4$ abbiamo che nello spazio-tempo quadrimensionale gli elementi
indipendenti del tensore di Riemann sono $C_{4} = 20$, come avevamo
preannunciato.

La contrazione fra primo e terzo indice del tensore di Riemann dà il
\index{tensore!di Ricci}\emph{tensore di Ricci}
\begin{equation}
  \label{eq:tens-ricci}
  R_{\mu\kappa} = g^{\lambda\nu} R_{\lambda\mu\nu\kappa} =
  \tensor{R}{^{\nu}_{\mu\nu\kappa}}
  = \parder{\tensor{\Gamma}{^{\nu}_{\mu\nu}}}{x^{\kappa}}
  - \parder{\tensor{\Gamma}{^{\nu}_{\mu\kappa}}}{x^{\nu}} +
  \tensor{\Gamma}{^{\eta}_{\mu\nu}} \tensor{\Gamma}{^{\nu}_{\eta\kappa}} -
  \tensor{\Gamma}{^{\eta}_{\mu\kappa}} \tensor{\Gamma}{^{\nu}_{\eta\nu}}.
\end{equation}
Il tensore di Ricci è simmetrico per la proprietà~\ref{item:simmetria-riemann}
di simmetria del tensore di Riemann, infatti
\begin{equation}
  \label{eq:simmetria-riemann}
  R_{\mu\kappa} = g^{\lambda\nu} R_{\lambda\mu\nu\kappa} = g^{\lambda\nu}
  R_{\nu\kappa\lambda\mu} = g^{\nu\lambda} R_{\nu\kappa\lambda\mu} =
  R_{\kappa\mu}.
\end{equation}
Dunque delle $4^{2} = 16$ componenti del tensore solo $10$ sono indipendenti.
Inoltre il tensore di Ricci è essenzialmente l'unico tensore non nullo che si
può ottenere dalla contrazione del tensore di Riemann.  Contraendo primo e
secondo indice, per la proprietà~\ref{item:antisimmetria-riemann} di
antisimmetria del tensore di Riemann si ha
\begin{equation}
  \tensor{R}{^{\mu}_{\mu\nu\kappa}} = g^{\lambda\mu}
  R_{\lambda\mu\nu\kappa} = -g^{\lambda\mu} R_{\mu\lambda\nu\kappa} =
  -g^{\mu\lambda}R_{\mu\lambda\nu\kappa} =
  -g^{\lambda\mu}R_{\lambda\mu\nu\kappa} = 0.
\end{equation}
Analogamente, contraendo terzo e quarto indice si ottiene
\begin{equation}
  \tensor{R}{_{\lambda\mu}^{\kappa}_{\kappa}} = g^{\nu\kappa}
  R_{\lambda\mu\nu\kappa} = 0.
\end{equation}
Contraendo secondo e quarto indice del tensore di Riemann si ottiene di nuovo il
tensore di Ricci, invece contraendo primo e quarto indice oppure secondo terzo
si ottiene ancora il tensore di Ricci così come da noi definito ma con il segno
opposto.  Infatti, moltiplicando tutti i membri
della~\eqref{eq:simmetria-riemann} per $g^{\lambda\nu}$ troviamo
\begin{equation}
  \tensor{R}{^{\nu}_{\mu\nu\kappa}} = -\tensor{R}{_{\mu}^{\nu}_{\nu\kappa}} =
  -\tensor{R}{^{\nu}_{\mu\kappa\nu}} = \tensor{R}{_{\mu}^{\nu}_{\kappa\nu}}.
\end{equation}

L'ulteriore contrazione del tensore di Riemann dà la
\index{curvatura scalare}\emph{curvatura scalare}
\begin{equation}
  R = g^{\mu\nu}R_{\mu\nu} = \tensor{R}{^{\mu}_{\mu}}.
\end{equation}

\subsection{Tensore di Riemann e curvatura dello spazio}
\label{sec:riemann-curvatura}

Anche in uno spazio piatto è possibile introdurre una metrica non costante su
tutto lo spazio che faccia apparire lo spazio curvo, come se ci fosse un campo
gravitazionale.  Questo è il caso, per esempio, delle coordinate sferiche o
cilindriche nello spazio cartesiano $\R^{3}$.  Siamo allora interessati a
cercare uno strumento che ci permetta di discriminare fra uno spazio-tempo
intrinsecamente curvo, perché permeato da un reale campo gravitazionale, e uno
spazio-tempo piatto, cioè lo spazio di Minkowski, in cui si fa uso di coordinate
curvilinee.  Se siamo nella seconda situazione, in questo spazio di Minkowski è
possibile effettuare una trasformazione delle coordinate
$x^{\alpha} \to \xi^{\alpha}(x)$ da un sistema di coordinate curvilinee
$x^{\alpha}$ con tensore metrico $\gamma_{\mu\nu}$ non costante a un sistema di
coordinate $\xi^{\alpha}(x)$ con tensore metrico uguale al tensore di Minkowski
$\eta_{\alpha\beta}$ e in tutto lo spazio si ha
\begin{equation}
  \eta_{\alpha\beta} =
  g_{\mu\nu} \parder{x^{\mu}}{\xi^{\alpha}} \parder{x^{\nu}}{\xi^{\beta}}.
\end{equation}
Per il \index{principio!di equivalenza}principio di equivalenza, in ogni punto
$X$ dello spazio è possibile individuare un sistema di coordinate localmente
inerziali $\xi_{X}$ che soddisfano la relazione precedente in un intorno
infinitesimo di $X$.  Noi invece vogliamo scoprire se esista un sistema di
coordinate $\xi^{\alpha}$ che soddisfi la relazione in tutto lo spazio.

Lo strumento a cui siamo interessati è il tensore di Riemann, infatti è
possibile dimostrare il seguente
\begin{teorema}
  Condizioni necessarie e sufficienti affinché sia possibile effettuare una
  trasformazione di coordinate $x \to \xi$ tale che in tutto lo spazio sia
  valida la relazione
  \begin{equation}
    \label{eq:tensore-spazio-piatto}
    \eta_{\alpha\beta} =
    g_{\mu\nu} \parder{x^{\mu}}{\xi^{\alpha}} \parder{x^{\nu}}{\xi^{\beta}}
  \end{equation}
  sono
  \begin{enumerate}
  \item il \index{tensore!di Riemann}tensore di Riemann si annulla in tutto lo
    spazio
    \begin{equation}
      \tensor{R}{^{\lambda}_{\mu\nu\kappa}} = 0;
    \end{equation}
  \item in qualche punto $X$ la matrice $g_{\mu\nu}(X)$ ha tre autovalori
    positivi e uno negativo.
  \end{enumerate}
\end{teorema}
Il fatto che sia valida la relazione~\eqref{eq:tensore-spazio-piatto} significa
che la metrica è quella usuale dello spazio di Minkowski
$\dd \tau^{2} = \dd t^{2} - \dd\bm{x}^{2}$, l'annullarsi del tensore di Riemann
invece significa che la curvatura dello spazio è nulla, il teorema precedente
dichiara l'equivalenza fra queste due condizioni.  In questo modo il tensore di
Riemann ci permette di verificare se nello spazio sia presente un reale campo
gravitazionale.

Un altro modo che permette di mostrare che il tensore di Riemann caratterizza la
curvatura dello spazio, e quindi evidenzia la presenza di un campo
gravitazionale, è il seguente.  Dato un vettore covariante $V_{\mu}$ si dimostra
(vedi appendice~\ref{sec:dimostr-derivate-miste-vettore}) che le sue derivate
covarianti seconde miste non commutano
\begin{equation}
  \label{eq:derivate-miste-vettore}
  V_{\mu;\nu;\kappa} - V_{\mu;\kappa;\nu} = -V_{\lambda}
  \tensor{R}{^{\lambda}_{\mu\nu\kappa}}.
\end{equation}
Si trovano inoltre analoghe relazioni per i vettori controvarianti o, più in
generale, per i tensori
\begin{subequations}
  \begin{gather}
    \tensor{V}{^{\mu}_{;\nu;\kappa}} - \tensor{V}{^{\mu}_{;\kappa;\nu}} =
    V^{\sigma} \tensor{R}{^{\lambda}_{\sigma\nu\kappa}}, \\
    \tensor{T}{^{\lambda}_{\mu;\nu;\kappa}} -
    \tensor{T}{^{\lambda}_{\mu;\kappa;\nu}} = \tensor{T}{^{\sigma}_{\mu}}
    \tensor{R}{^{\lambda}_{\sigma\nu\kappa}} - \tensor{T}{^{\lambda}_{\sigma}}
    \tensor{R}{^{\sigma}_{\mu\nu\kappa}}.
  \end{gather}
\end{subequations}
Nello spazio di Minkowski le derivate covarianti coincidono con le derivate
ordinarie e le derivate seconde miste commutano
\begin{equation}
  V_{\mu,\nu,\kappa} = \parder{V_{\mu}}{x^{\nu}, x^{\kappa}}
  = \parder{V_{\mu}}{x^{\kappa}, x^{\nu}} = V_{\mu,\kappa,\nu}
\end{equation}
e questo è possibile solo se il tensore di Riemann è identicamente nullo in
tutto lo spazio.

\subsection{Identità di Bianchi}
\label{sec:identita-bianchi}

Si dimostra (vedi l'appendice~\ref{sec:dimostr-identita-bianchi}) la seguente
relazione
\begin{equation}
  \label{eq:bianchi}
  R_{\lambda\mu\nu\kappa;\eta} + R_{\lambda\mu\kappa\eta;\nu} +
  R_{\lambda\mu\eta\nu;\kappa} = 0
\end{equation}
chiamata \index{identità!di Bianchi}\emph{identità di Bianchi}.  Contraiamo gli
indici $\lambda$ e $\nu$ moltiplicando ambo i membri per $g^{\lambda\nu}$
\begin{equation}
  \begin{split}
    \tensor{R}{^{\nu}_{\mu\nu\kappa;\eta}} +
      \tensor{R}{^{\nu}_{\mu\kappa\eta;\nu}} +
    \tensor{R}{^{\nu}_{\mu\eta\nu;\kappa}} &= R_{\mu\kappa;\eta} +
    \tensor{R}{^{\nu}_{\mu\kappa\eta;\nu}} - R_{\mu\eta;\kappa} \\
    &= R_{\mu\kappa;\eta} - \tensor{R}{^{\nu}_{\mu\eta\kappa;\nu}} -
    R_{\mu\eta;\kappa} = 0.
  \end{split}
\end{equation}
Abbiamo sfruttato il fatto che innalzamento degli indici e derivazione
covariante commutano.  Contraendo ulteriormente gli indici $\mu$ e $\kappa$,
cioè moltiplicando gli ultimi due membri per $g^{\mu\kappa}$, otteniamo
\begin{equation}
  R_{;\eta} - \tensor{R}{^{\nu}_{\eta;\nu}} -
  \tensor{R}{^{\kappa}_{\eta;\kappa}} = R_{;\eta} - 2
  \tensor{R}{^{\mu}_{\eta;\mu}} = 0,
\end{equation}
poiché $\kappa$ e $\nu$ sono entrambi indici muti che abbiamo rinominato $\mu$.
Questa relazione può anche essere scritta come
\begin{equation}
  \bigg( \tensor{R}{^{\mu}_{\eta}} - \frac{1}{2} \tensor{\delta}{^{\mu}_{\eta}}
  R \bigg)_{;\mu} = 0
\end{equation}
e moltiplicando ancora per $g^{\eta\nu}$
\begin{equation}
  \bigg( R^{\mu\nu} - \frac{1}{2} g^{\mu\nu} R\bigg)_{;\mu} = 0.
\end{equation}
La quantità all'interno delle parentesi è il
\index{tensore!di Einstein}\emph{tensore di Einstein}
\begin{equation}
  G^{\mu\nu} = R^{\mu\nu} - \frac{1}{2} g^{\mu\nu}R
\end{equation}
e abbiamo dunque mostrato che ha quadridivergenza nulla
\begin{equation}
  \tensor{G}{^{\mu\nu}_{;\mu}} = \bigg( R^{\mu\nu} - \frac{1}{2} g^{\mu\nu}
  R\bigg)_{;\mu} = 0.
\end{equation}

% TODO: scrivere qualcosa.  Vedi lezioni dei giorni 03-04/05/2012
\subsection{\completare{Equazione di deviazione geodetica}}
\label{sec:deviazione-geodetica}

\section{Equazioni di Einstein}
\label{sec:equazioni-einstein}

\emph{Nota: in questo paragrafo indicheremo esplicitamente la velocità della
  luce nel vuoto $c$.}

Vogliamo ora determinare le equazioni del campo gravitazionale $g_{\mu\nu}$ a
partire dal principio variazionale di Hamilton.  Assumiamo che l'azione $S$ sia
somma di una componente $S_{\textup{g}}$ che descrive solo il campo
gravitazionale e una componente $S_{\textup{m}}$ relativa alla materia che
interagisce con il campo.  L'azione deve essere uno scalare, in modo che risulti
invariante per trasformazioni arbitrarie delle coordinate, e dipenderà dal
tensore metrico, dalle sue derivate prime, dalle coordinate generalizzate $q$ e
dalle velocità generalizzate $\lparder{q}{x^{\lambda}}$,
$S = S(g_{\mu\nu}, \lparder{g_{\mu\nu}}{x^{\lambda}}; q,
\lparder{q}{x^{\lambda}})$.
Hilbert propose questa espressione per l'azione del campo
gravitazionale\footnote{Il fattore davanti all'integrale è stato scelto in
  maniera tale che nel limite newtoniano si ottenga l'equazione di Poisson
  $\nabla^{2} \phi = 4\pi G\rho$.  Si noti che i testi
  \textcites{barone:relativita,landau:campo} utilizzano l'espressione
  $S_{\textup{g}} = -c^{3}/(16\pi G) \int R \sqrt{g} \dd^{4} x$ a causa della
  diversa convenzione dei segni rispetto a quella da noi scelta.  Vedi
  pagina~\pageref{eq:convenzione-segni}.}
\begin{equation}
  S_{\textup{g}} = \frac{c^{3}}{16\pi G} \int R \sqrt{g} \dd^{4} x,
\end{equation}
chiamata \index{azione!di Hilbert-Einstein} \emph{azione di Hilbert-Einstein}.
$G = \SI{6.673 84(80)e-11}{\cubic\metre\per\kilo\metre\per\second\squared}$ è la
\index{costante!di gravitazione universale}
\emph{costante di gravitazione universale}.  L'azione della materia ha invece
espressione
\begin{equation}
  S_{\textup{m}} = \frac{1}{c} \int \Lambda \sqrt{g} \dd^{4} x,
\end{equation}
in cui $\Lambda = \Lambda(g^{\mu\nu}, \partial_{\lambda}g^{\mu\nu})$ è la densità
di lagrangiana associata al sistema che interagisce con il campo
gravitazionale.  L'interazione fra il campo e la materia è inclusa in
$S_{\textup{m}}$.

Per determinare le equazioni del campo gravitazionale dobbiamo richiedere che
l'azione sia stazionaria rispetto alle variazioni, a volume fissato,
$\delta g_{\mu\nu}$ del campo che devono soddisfare inoltre la condizione che si
annullino sul bordo dell'ipervolume di integrazione.  Dunque dobbiamo imporre
$\delta S = \delta(S_{\textup{g}} + S_{\textup{m}}) = 0$.  Calcoliamo il
contributo di $\delta S_{\textup{g}}$
\begin{equation}
  \label{eq:var-HE}
  \begin{split}
    \delta \int R \sqrt{g} \dd^{4} x &= \delta \int g^{\mu\nu} R_{\mu\nu}
    \sqrt{g} \dd^{4} x \\
    &= \int (\delta g^{\mu\nu} R_{\mu\nu} \sqrt{g} + g^{\mu\nu} R_{\mu\nu}
    \delta \sqrt{g} + g^{\mu\nu} \delta R_{\mu\nu} \sqrt{g}) \dd^{4} x.
  \end{split}
\end{equation}
Osserviamo che
\begin{equation}
  \delta\sqrt{g} = \parder{\sqrt{g}}{g} \parder{g}{g^{\mu\nu}} \delta g^{\mu\nu}
  = \frac{1}{2 \sqrt{g}} (-g g_{\mu\nu} ) \delta g^{\mu\nu} = - \frac{1}{2}
  g_{\mu\nu} \sqrt{g} \delta g^{\mu\nu},
\end{equation}
in cui abbiamo ricordato la~\eqref{eq:g_mu_rho-determinante2}.  Sostituendo
nella~\eqref{eq:var-HE} otteniamo
\begin{equation}
  \delta \int R \sqrt{g} \dd^{4} x = \int \bigg( R_{\mu\nu} - \frac{1}{2}
  g_{\mu\nu} R \bigg)\sqrt{g} \delta g^{\mu\nu} \dd^{4} x + \int g^{\mu\nu}
  \delta R_{\mu\nu} \sqrt{g} \dd^{4} x.
\end{equation}
Mostriamo che il secondo integrale è nullo.  Innanzitutto la
\index{tensore!di Ricci!variazione del}variazione del tensore di Ricci può
essere espressa in termini di variazioni della connessione affine mediante
l'\index{identità!di Palatini}% TODO: vedere di dimostrarla
\emph{identità di
  Palatini}\footnote{Si
  noti che la variazione della connessione affine è un tensore poiché è la
  differenza fra due connessioni affini, la quale è un tensore come visto nel
  paragrafo~\ref{sec:connessione-affine}.}
\begin{equation}
  \delta R_{\mu\nu} = (\delta \tensor{\Gamma}{^{\lambda}_{\mu\lambda}})_{;\nu} -
  (\delta \tensor{\Gamma}{^{\lambda}_{\mu\nu}})_{;\lambda}.
\end{equation}
Allora si può riscrivere l'integrale in esame come
\begin{equation}
  \int g^{\mu\nu} \delta R_{\mu\nu} \sqrt{g} \dd^{4} x = \int (g^{\mu\nu} \delta
  \tensor{\Gamma}{^{\lambda}_{\mu\lambda;\nu}} - g^{\mu\nu} \delta
  \tensor{\Gamma}{^{\lambda}_{\mu\nu;\lambda}}) \sqrt{g} \dd^{4} x
\end{equation}
e per il teorema di Gauss in forma covariante~\eqref{eq:gauss-covariante}
ricaviamo
\begin{equation}
  \begin{split}
    \int g^{\mu\nu} \delta R_{\mu\nu} \sqrt{g} \dd^{4} x &= \int (\delta
    \tensor{\Gamma}{^{\lambda\nu}_{\lambda;\nu}} - \delta
    \tensor{\Gamma}{^{\lambda\nu}_{\nu;\lambda}}) \sqrt{g} \dd^{4} x \\
    &= \int \delta \tensor{\Gamma}{^{\lambda\nu}_{\lambda}} \sqrt{g} \dd
    \Sigma_{\nu} - \int \delta \tensor{\Gamma}{^{\lambda\nu}_{\lambda}} \sqrt{g}
    \dd \Sigma_{\lambda} = 0
  \end{split}
\end{equation}
poiché gli integrali sulle ipersuperfici vanno calcolati sul bordo
dell'ipervolume costante su cui si calcola la variazione dell'azione e sul bordo
sono nulle le variazioni di $g_{\mu\nu}$ e quindi anche della connessione
affine, la quale dipende dal tensore metrico.  Dunque abbiamo trovato che la
variazione dell'azione di Hilbert-Einstein è
\begin{equation}
  \delta S_{\textup{g}} = \frac{c^{3}}{16\pi G} \int \bigg( R_{\mu\nu} -
  \frac{1}{2} g_{\mu\nu} R \bigg)\sqrt{g} \delta g^{\mu\nu} \dd^{4} x.
\end{equation}

Per calcolare la variazione $\delta S_{\textup{m}}$ dell'azione associata alla
materia ragioniamo in maniera analoga a quanto fatto nel
paragrafo~\ref{sec:tensore-energia-impulso}
\begin{equation}
  \begin{split}
    \delta S_{\textup{m}} &= \delta \int
    \Lambda(g^{\mu\nu}, \tensor{g}{^{\mu\nu}_{,\lambda}}) \sqrt{g} \dd^{4} x =
    \int \bigg( \parder{(\sqrt{g}\Lambda)}{g^{\mu\nu}} \delta g^{\mu\nu}
    + \parder{(\sqrt{g}\Lambda)}{\tensor{g}{^{\mu\nu}_{,\lambda}}} \delta
    \tensor{g}{^{\mu\nu}_{,\lambda}} \bigg) \dd^{4} x \\
    &= \frac{1}{c} \int \bigg( \parder{(\sqrt{g}\Lambda)}{g^{\mu\nu}} \delta
    g^{\mu\nu} + \parder{}{x^{\lambda}}
    \bigg( \parder{(\sqrt{g}\Lambda)}{\tensor{g}{^{\mu\nu}_{,\lambda}}} \delta
    g^{\mu\nu}\bigg) - \delta g^{\mu\nu} \parder{}{x^{\lambda}}
    \bigg( \parder{\sqrt{g}\Lambda}{\tensor{g}{^{\mu\nu}_{,\lambda}}} \bigg)
    \bigg) \dd^{4} x \\
    &= \frac{1}{c} \int \bigg( \parder{(\sqrt{g}\Lambda)}{g^{\mu\nu}}
    - \parder{}{x^{\lambda}}
    \bigg( \parder{\sqrt{g}\Lambda}{\tensor{g}{^{\mu\nu}_{,\lambda}}} \bigg)
    \bigg) \delta g^{\mu\nu} \dd^{4} x.
  \end{split}
\end{equation}
Nell'ultimo passaggio abbiamo sfruttato il fatto che l'integrale del secondo
termine si trasforma, per il teorema di Gauss, in un integrale sul bordo del
volume di integrazione sul quale le variazioni $\delta g^{\mu\nu}$ sono nulle.
Introduciamo il tensore $T_{\mu\nu}$ definito da
\begin{equation}
  \frac{1}{2} \sqrt{g} T_{\mu\nu} = \parder{(\sqrt{g}\Lambda)}{g^{\mu\nu}}
  - \parder{}{x^{\lambda}}
  \bigg( \parder{\sqrt{g}\Lambda}{\tensor{g}{^{\mu\nu}_{,\lambda}}} \bigg)
\end{equation}
% TODO: spiegare da qualche parte che questo è proprio il tensore
% energia-impulso (e mettere \index{tensore!energia-impulso})
così la variazione dell'azione della materia diventa
\begin{equation}
  \delta S_{\textup{m}} = \frac{1}{2c} \int T_{\mu\nu} \delta g^{\mu\nu}
  \sqrt{g} \dd^{4} x
\end{equation}
e la condizione di stazionarietà dell'azione totale del sistema campo
gravitazionale + particelle interagenti è
\begin{equation}
  \delta S = \delta(S_{\textup{g}} + S_{\textup{m}}) = \int \bigg(
  \frac{c^{3}}{16\pi G} \bigg( R_{\mu\nu} - \frac{1}{2} g_{\mu\nu} R \bigg) +
  \frac{1}{2} T_{\mu\nu} \bigg) \sqrt{g} \delta g^{\mu\nu} \dd^{4} x = 0.
\end{equation}
Data l'arbitrarietà delle variazioni $\delta g^{\mu\nu}$ ricaviamo
\begin{equation}
  \label{eq:einstein}
  G_{\mu\nu} = R_{\mu\nu} - \frac{1}{2} g_{\mu\nu} R = -\frac{8\pi G}{c^{4}}
  T_{\mu\nu}
\end{equation}
che sono le \emph{equazioni del campo gravitazionale} in presenza di sorgenti,
chiamate anche \emph{equazioni di Einstein}.

Moltiplicando gli ultimi due membri per $g^{\mu\nu}$, in modo da contrarre gli
indici $\mu$ e $\nu$, otteniamo
\begin{equation}
  R - \frac{1}{2}4R = -R = -\frac{8\pi G}{c^{4}}\tensor{T}{^{\lambda}_{\lambda}}
\end{equation}
e sostituendo nella~\eqref{eq:einstein} otteniamo un'espressione alternativa per
le equazioni di Einstein
\begin{equation}
  R_{\mu\nu} = -\frac{8\pi G}{c^{4}} \bigg( T_{\mu\nu} -
  \frac{1}{2}g_{\mu\nu}\tensor{T}{^{\lambda}_{\lambda}} \bigg).
\end{equation}
Da qui possiamo osservare che nel vuoto, cioè in assenza di materia, il tensore
energia-impulso è nullo quindi le equazioni di Einstein si riducono a
\begin{equation}
  R_{\mu\nu} = 0.
\end{equation}
Si noti che questa equazione non significa che uno spazio-tempo vuoto è anche
piatto, per avere quest'ultima condizione deve verificarsi la condizione più
forte $\tensor{R}{^{\lambda}_{\mu\nu\kappa}} = 0$.

\subsection{Limite newtoniano delle equazioni di Einstein}
\label{sec:limite-newtoniano-einstein}

Vogliamo determinare il limite newtoniano (campi gravitazionali deboli e
statici, velocità non relativistiche) delle equazioni di Einstein e per fare
questo lavoriamo in maniera analoga a quanto fatto nel
paragrafo~\ref{sec:limite-newtoniano}, quindi poniamo
$g_{\mu\nu} = \eta_{\mu\nu} + h_{\mu\nu}$.  Ci aspettiamo che il tensore
energia-impulso sia in queste condizioni
\completare{(il Landau lo dimostra nel paragrafo 35)}
\begin{subequations}
  \begin{align}
    T_{ij} &= \rho c^{2} v_{i} v_{j}, \\
    T_{00} &= \rho c^{2},
  \end{align}
\end{subequations}
in cui $\rho$ è la densità di massa a riposo (e quindi $\rho c^{2}$ è la densità
di energia a riposo) e $v_{\mu}$ la quadrivelocità.  Per l'ipotesi di velocità
non relativistiche si devono trascurare le componenti spaziali della
quadrivelocità rispetto alla componente temporale, cioè $v_{i} = 0$, e quindi
$T_{ij} = 0$ da cui abbiamo inoltre
\begin{equation}
  0 = G_{ij} \approx R_{ij} - \frac{1}{2}\eta_{ij}R.
\end{equation}
Allora l'unica equazione di Einstein rilevante è quella con $G_{00}$.
Contraendo gli indici dell'equazione precedente otteniamo
\begin{equation}
  \eta^{ij}R_{ij} \approx \frac{1}{2} \eta_{ij}\eta^{ij} R = \frac{3}{2} R.
\end{equation}
Il tensore di Ricci può essere approssimato come
\begin{equation}
  R = g^{\mu\nu}R_{\mu\nu} \approx \eta^{\mu\nu} R_{\mu\nu} = \eta^{00}R_{00} +
  \eta^{ij}R_{ij} = -R_{00} + \frac{3}{2} R
\end{equation}
dunque $R \approx 2 R_{00}$.  Sostituendo nell'equazione di Einstein abbiamo
\begin{equation}
  G_{00} = R_{00} - \frac{1}{2} \eta_{00}R = R_{00} + \frac{1}{2} R = 2 R_{00} =
  -\frac{8\pi G}{c^{4}} T_{00} = -\frac{8\pi G}{c^{4}} \rho c^{2}
\end{equation}
da cui
\begin{equation}
  R_{00} = -\frac{4\pi G}{c^{2}} \rho.
\end{equation}
D'altra parte osserviamo che
\begin{equation}
  R_{00} = g^{\mu\nu}R_{\mu 0\nu 0} \approx \eta^{\mu\nu}R_{\mu 0\nu 0}
\end{equation}
Dalla~\eqref{eq:riemann-covariante} abbiamo
\begin{align}
  R_{\lambda\mu\nu\kappa} &\approx \frac{1}{2}
  \bigg( \parder{h_{\lambda\nu}}{x^{\kappa}, x^{\mu}}
  - \parder{h_{\mu\nu}}{x^{\kappa},
    x^{\lambda}}- \parder{h_{\lambda\kappa}}{x^{\nu}, x^{\mu}}
  + \parder{h_{\mu\kappa}}{x^{\nu}, x^{\lambda}} \bigg) \\
  &+ \eta_{\eta\sigma} (\tensor{\Gamma}{^{\eta}_{\nu\lambda}}
  \tensor{\Gamma}{^{\sigma}_{\mu\kappa}} -
  \tensor{\Gamma}{^{\eta}_{\kappa\lambda}}
  \tensor{\Gamma}{^{\sigma}_{\mu\nu}})
\end{align}
e in particolare, sfruttando l'ipotesi di campi costanti,
\begin{subequations}
  \begin{align}
    R_{0000} &\approx 0, \\
    R_{i0j0} &\approx \frac{1}{2} \parder{h_{00}}{x^{i},x^{j}}
  \end{align}
\end{subequations}
allora
\begin{equation}
  R_{00} = \eta^{00}R_{0000} + \eta^{ij}R_{i0j0} = \delta^{ij}R_{i0j0} \approx
  \frac{1}{2}\nabla^{2} h_{00} = -\nabla^{2}\frac{\phi}{c^{2}}
\end{equation}
avendo ricordato la~\eqref{eq:h00}.  Dunque mettendo insieme le due espressioni
di $R_{00}$ ottenute troviamo l'equazione di Poisson per il campo gravitazionale
\begin{equation}
  \nabla^{2} \phi = 4\pi G\rho.
\end{equation}

\section{Campo gravitazionale statico e a simmetria sferica}
\label{sec:campo-statico-sferico}

Vogliamo determinare la metrica per un campo gravitazionale a simmetria sferica
e statico, cioè tale che $\dd\tau^{2} = -g_{\mu\nu}\dd x^{\mu}\dd x^{\nu}$ non
dipenda dalla coordinata temporale $x^{0}$ e dipenda dalle coordinate spaziali
solo attraverso delle quantità invarianti per rotazioni, cioè
$\bm{x} \cdot \dd\bm{x}$, $\bm{x} \cdot \bm{x}$, $\dd\bm{x} \cdot \dd\bm{x}$.
La forma più generale per una metrica con queste proprietà è
\begin{equation}
  \dd\tau^{2} = F(r) \dd t^{2} - 2E(r) \bm{x} \cdot \dd\bm{x} \dd t -
  D(r)(\bm{x} \cdot \dd\bm{x})^{2} - C(r)\dd\bm{x}^{2},
\end{equation}
in cui $F$, $E$, $D$ e $C$ sono funzioni di $r = \norm{\bm{x}}$.  Data la
simmetria del sistema che vogliamo analizzare conviene passare in coordinate
sferiche
\begin{subequations}
  \begin{align}
    x^{1} &= r \sin\theta \cos\phi, \\
    x^{2} &= r \sin\theta \sin\phi, \\
    x^{3} &= r \cos\theta
  \end{align}
\end{subequations}
così che la metrica diventa
\begin{equation}
  \dd\tau^{2} = F(r) \dd t^{2} - 2E(r) r\dd r\dd t - r^{2}D(r)\dd r^{2} -
  C(r)(\dd r^{2} + r^{2}\dd\theta^{2} + r^{2}\sin^{2}\theta \dd\phi^{2}).
\end{equation}
La libertà che si ha, in relatività generale, nella scelta del sistema di
coordinate ci consente di semplificare questa espressione.  Consideriamo dunque
la trasformazione della coordinate temporale $t \to t' = t + \phi(r)$ con la
condizione
\begin{equation}
  \toder{\phi}{r} = - r\frac{E(r)}{F(r)}
\end{equation}
in modo da annullare il termine $g_{rt}$ del tensore metrico.  Così la metrica
diventa
\begin{equation}
  \dd \tau^{2} = F(r)\dd t'^{2} - G(r)\dd r^{2} -C(r)(\dd r^{2} +
  r^{2}\dd\theta^{2} + r^{2}\sin^{2}\theta \dd\phi^{2})
\end{equation}
con
\begin{equation}
  G(r) = r^{2} \bigg(D(r) + \frac{E^{2}(r)}{F(r)} \bigg).
\end{equation}
Effettuando inoltre la trasformazione $r \to r' = C(r)r^{2}$ otteniamo la
\index{metrica!forma standard della}\emph{forma standard} della metrica
\begin{equation}
  \label{eq:metrica-standard}
  \dd\tau^{2} = B(r')\dd t'^{2} - A(r')\dd r'^{2} - r'^{2}(\dd\theta^{2} +
  \sin^{2}\theta \dd\phi^{2}),
\end{equation}
in cui
\begin{subequations}
  \begin{align}
    B(r) &= F(r), \\
    A(r) &= \bigg(1 + \frac{G(r)}{C(r)} \bigg) \bigg( 1 + \frac{r}{2C(r)}
    \toder{C(r)}{r} \bigg)^{-2}.
  \end{align}
\end{subequations}
Rinominando le variabili $r'$ e $t'$ in $r$ e $t$, il tensore metrico è
rappresentato dalla seguente matrice diagonale
\begin{equation}
  g_{\mu\nu} =
  \begin{pmatrix}
    g_{tt}       & g_{tr}       & g_{t\theta}      & g_{t\phi}      \\
    g_{rt}       & g_{rr}       & g_{r\theta}      & g_{r\phi}      \\
    g_{\theta t} & g_{\theta r} & g_{\theta\theta} & g_{\theta\phi} \\
    g_{\phi t}   & g_{\phi r}   & g_{\phi\theta}   & g_{\phi\phi}
  \end{pmatrix}
  =
  \begin{pmatrix}
    -B(r) & 0    & 0     & 0 \\
    0     & A(r) & 0     & 0 \\
    0     & 0    & r^{2} & 0 \\
    0     & 0    & 0     & r^{2}\sin^{2}\theta
  \end{pmatrix}.
\end{equation}
Poiché $g_{\mu\nu}$ è diagonale, gli elementi del tensore metrico controvariante
sono semplicemente $g^{\mu\nu} = 1/g_{\mu\nu}$
\begin{equation}
  g^{\mu\nu} =
  \begin{pmatrix}
    g^{tt}       & g^{tr}       & g^{t\theta}      & g^{t\phi}      \\
    g^{rt}       & g^{rr}       & g^{r\theta}      & g^{r\phi}      \\
    g^{\theta t} & g^{\theta r} & g^{\theta\theta} & g^{\theta\phi} \\
    g^{\phi t}   & g^{\phi r}   & g^{\phi\theta}   & g^{\phi\phi}
  \end{pmatrix}
  =
  \begin{pmatrix}
    -B^{-1}(r) & 0         & 0      & 0 \\
    0          & A^{-1}(r) & 0      & 0 \\
    0          & 0         & r^{-2} & 0 \\
    0          & 0         & 0      & r^{-2}\sin^{-2}\theta
  \end{pmatrix}.
\end{equation}
Osserviamo che $g = -\det(g_{\mu\nu}) = r^{4}A(r)B(r)\sin^{2}\theta$, quindi
l'elemento di volume invariante è
\begin{equation}
  \sqrt{-g} \dd^{4} x = r^{2}\sqrt{A(r) B(r)} \sin\theta \dd^{4} x.
\end{equation}

Le connessioni affini possono essere calcolate con la
formula~\eqref{eq:connessione-metrica}.  Si trova che le uniche componenti non
nulle sono
\begin{subequations}
  \begin{align}
    \tensor{\Gamma}{^{r}_{rr}} &= \frac{A'(r)}{2A(r)}, &
    \tensor{\Gamma}{^{r}_{\theta\theta}} &= -\frac{r}{A(r)}, &
    \tensor{\Gamma}{^{r}_{\phi\phi}} &= -\frac{r \sin^{2}\theta}{A(r)}, \\
    \tensor{\Gamma}{^{r}_{tt}} &= \frac{B'(r)}{2A(r)}, &
    \tensor{\Gamma}{^{\theta}_{r\theta}} &=
    \tensor{\Gamma}{^{\theta}_{\theta r}} = \frac{1}{r}, &
    \tensor{\Gamma}{^{\theta}_{\phi\phi}} &= -\sin\theta \cos\theta, \\
    \tensor{\Gamma}{^{\phi}_{\phi r}} &= \tensor{\Gamma}{^{\phi}_{r\phi}} =
    \frac{1}{r}, & \tensor{\Gamma}{^{\phi}_{\phi\theta}} &=
    \tensor{\Gamma}{^{\phi}_{\theta\phi}} = \cot\theta, &
    \tensor{\Gamma}{^{t}_{tr}} &= \tensor{\Gamma}{^{t}_{rt}} =
    \frac{B'(r)}{2B(r)}.
  \end{align}
\end{subequations}
L'apice indica la derivazione rispetto a $r$.

Dopo di ciò è possibile calcolare il tensore di Ricci usando la sua
definizione~\eqref{eq:tens-ricci}.  Grazie all'invarianza per rotazioni e
l'indipendenza dal tempo della metrica, anche il tensore di Ricci è diagonale
con componenti non nulle
\begin{subequations}
  \label{eq:ricci-sferico}
  \begin{align}
    R_{tt} &= -\frac{B''(r)}{2A(r)} +
    \frac{B'(r)}{A(r)}\frac{B'(r)}{B(r)}\bigg(\frac{A'(r)}{A(r)} +
    \frac{B'(r)}{B(r)}\bigg) -
    \frac{1}{r}\frac{B'(r)}{A(r)}, \\
    R_{rr} &= \frac{B''(r)}{2B(r)} - \frac{1}{4}
    \frac{B'(r)}{B(r)}\bigg(\frac{A'(r)}{A(r)} + \frac{B'(r)}{B(r)}\bigg) -
    \frac{1}{r}\frac{A'(r)}{A(r)}, \\
    R_{\theta\theta} &= -1 + \frac{r}{2A(r)}\bigg(-\frac{A'(r)}{A(r)} +
    \frac{B'(r)}{B(r)}\bigg) + \frac{1}{A(r)}, \\
    R_{\phi\phi} &= R_{\theta\theta} \sin^{2}\theta.
  \end{align}
\end{subequations}
Osserviamo infine che risulta
\begin{equation}
  \label{eq:bar}
  \frac{R_{tt}}{B(r)} + \frac{R_{rr}}{A(r)} =
  -\frac{1}{rA(r)}\bigg(\frac{A'(r)}{A(r)} + \frac{B'(r)}{B(r)}\bigg).
\end{equation}

\subsection{Metrica di Schwarzschild}
\label{sec:metrica-schwarzschild}

La soluzione di Schwarzschild delle equazioni di Einstein descrive il campo
gravitazionale all'esterno di un corpo sferico non rotante e in assenza di
materia, cioè con tensore energia-impulso nullo.  Applicheremo i risultati che
otterremo al campo gravitazionale generato dal Sole.

Abbiamo visto che in assenza di materia le equazioni di Einstein comportano che
$R_{\mu\nu} = 0$.  Di conseguenza, ponendo uguale a $0$ il secondo membro
della~\eqref{eq:bar} abbiamo
\begin{equation}
  \frac{A'(r)}{A(r)} = -\frac{B'(r)}{B(r)}
\end{equation}
da cui
\begin{equation}
  \ln A(r) = - \ln B(r) + \text{costante} \implies A(r)B(R) = \text{costante}.
\end{equation}
Per determinare il valore della costante notiamo che a distanza infinita dal
campo gravitazionale la metrica deve tendere a quella di Minkowski, che in
coordinate sferiche si esprime come
\begin{equation}
  \dd\tau^{2} = \dd t^{2} -\dd r^{2} - r^{2}\dd\theta^{2} - r^{2}\sin^{2}\theta
  \dd\phi^{2}.
\end{equation}
Confrontando con la metrica nella forma standard~\eqref{eq:metrica-standard}
troviamo che deve risultare $A(r), B(R) \xrightarrow{r \to +\infty} 1$ e dalla
costanza del prodotto $A(r)B(r)$ deduciamo che esso vale $1$ e quindi
$A(r) = 1/B(r)$.  Sostituendo questo risultato nelle
componenti~\eqref{eq:ricci-sferico} del tensore di Ricci abbiamo
\begin{subequations}
  \begin{align}
    R_{\theta\theta} &= -1 + rB'(r) + B(r), \\
    R_{rr} &= \frac{B''(r)}{2B(r)} + \frac{B'(r)}{rB(r)} =
    \frac{R'_{\theta\theta}(r)}{2rB(r)}.
  \end{align}
\end{subequations}
Imponendo che anche la componente $R_{\theta\theta}$ si annulli, da cui deriva
inoltre che anche $R_{rr}$ e $R_{\phi\phi}$ sono nulle, si ricava
\begin{equation}
  \toder{}{r}(rBr(r)) = 1 \implies rB(r) = r + \text{costante} \implies B(r) = 1
  + \frac{\text{costante}}{r}.
\end{equation}
Per determinare il valore di questa nuova costante poniamo la condizione che a
grandi distanze dal corpo che genera il campo, in cui il campo diventa debole,
sia valida l'approssimazione vista nel paragrafo~\ref{sec:limite-newtoniano}
\begin{equation}
  g_{00} = -1 + \frac{r_{\textup{S}}}{r},
\end{equation}
con $r_{\textup{S}}$ raggio di Schwarzschild del corpo.  Osservando allora che
$g_{00} = g_{tt} = -B(r)$ abbiamo che la costante che vogliamo determinare vale
proprio $r_{\textup{S}}$ e in definitiva la
\index{metrica!di Schwarzschild}\emph{metrica di Schwarzschild}, per campi a
simmetria sferica e indipendenti dal tempo, è data da
\begin{equation}
  \dd s^{2} = c^{2} \dd\tau^{2} = \bigg(1 - \frac{r_{\textup{S}}}{r} \bigg)c^{2}
  \dd t^{2} - \bigg(1 - \frac{r_{\textup{S}}}{r}\bigg)^{-1}\dd r^{2} -
  r^{2}\dd\theta^{2} - r^{2}\sin^{2}\theta \dd\phi^{2}.
\end{equation}
Osserviamo che questa metrica ha una singolarità per $r = r_{\textup{S}}$ e
un'altra per $r = 0$.  Nel caso specifico del Sole, e di numerosi altri corpi,
la prima singolarità non costituisce un problema poiché la metrica vale
all'esterno del corpo e il raggio di questa stella è molto più piccolo del suo
raggio di Schwarzschild.  Si dimostra inoltre che la singolarità per
$r = r_{\textup{S}}$ è eliminabile adottando un opportuno sistema di coordinate,
quindi non è, in realtà, una singolarità fisica.

Concludiamo osservando che il \index{teorema!di Birkhoff}
\emph{teorema di Birkhoff} afferma che
\emph{un campo gravitazionale a simmetria sferica in assenza di materia deve
  essere statico},
quindi l'ipotesi di staticità del campo è superflua in quanto conseguenza della
sua particolare simmetria.

\section{Verifiche sperimentali della relatività generale}
\label{sec:verififiche-relativita}

Grazie alla metrica di Schwarzschild siamo in grado di fare delle previsioni
teoriche riguardo alcuni fenomeni osservabili nel sistema solare, le quali sono
state confermate sperimentalmente e hanno permesso il riconoscimento della
validità della teoria della relatività generale.

\subsection{Precessione del perielio}
\label{sec:precessione-perielio}

\subsection{Deflessione della luce}
\label{sec:deflessione-luce}

\subsection{Radar echo delay}
\label{sec:radar-echo-delay}



%%% Local Variables:
%%% mode: latex
%%% TeX-master: "../astrofisica-teorica"
%%% fill-column: 80
%%% End:
