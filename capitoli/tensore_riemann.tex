\cleardoublepage
\chapter{Tensore di Riemann}
\label{cha:tensore-riemann}

\section{Tensore di Riemann}
\label{sec:tensore-riemann}

\subsection{Definizione e proprietà algebriche}
\label{sec:definizione-proprietà-riemann}

Si può far vedere\footnote{Vedi~\textcite[133-135]{weinberg:gravitation}.} che
l'unico tensore ottenibile dal tensore metrico e dalle sue derivate prime e
seconde e che sia lineare nelle derivate seconde della metrica
è il \emph{tensore di curvatura di Riemann-Christoffel} o,
semplicemente, \index{tensore!di Riemann}\emph{tensore di Riemann}
\begin{equation}
  \tensor{R}{^{\lambda}_{\mu\nu\kappa}}
  = \parder{\tensor{\Gamma}{^{\lambda}_{\mu\nu}}}{x^{\kappa}}
  - \parder{\tensor{\Gamma}{^{\lambda}_{\mu\kappa}}}{x^{\nu}} +
  \tensor{\Gamma}{^{\eta}_{\mu\nu}} \tensor{\Gamma}{^{\lambda}_{\eta\kappa}} -
  \tensor{\Gamma}{^{\eta}_{\mu\kappa}} \tensor{\Gamma}{^{\lambda}_{\eta\nu}}.
\end{equation}
Il tensore di Riemann è un tensore nel senso specificato nel
paragrafo~\ref{sec:tensori}.\footnote{Non è immediato riconoscere che
  $\tensor{R}{^{\lambda}_{\mu\nu\kappa}}$ sia effettivamente un tensore dato che
  è definito a partire dalla connessione affine, la quale non ha natura
  tensoriale.  Si può trovare una dimostrazione di questa proprietà
  in~\textcite[131-133]{weinberg:gravitation}.}

Per analizzare le proprietà algebriche del tensore di Riemann conviene
considerare la sua forma completamente
covariante~\parencite[141]{weinberg:gravitation}
\begin{equation}
  \label{eq:riemann-covariante}
  \begin{split}
    R_{\lambda\mu\nu\kappa} = g_{\lambda\sigma}  \tensor{R}{^{\sigma}_{\mu\nu\kappa}}
     & =\frac{1}{2} \bigg( \parder{g_{\lambda\nu}}{x^{\kappa}, x^{\mu}}
    - \parder{g_{\mu\nu}}{x^{\kappa}, x^{\lambda}}
    - \parder{g_{\lambda\kappa}}{x^{\nu}, x^{\mu}}
    + \parder{g_{\mu\kappa}}{x^{\nu}, x^{\lambda}} \bigg) \\
     &\quad{} + g_{\eta\sigma} (\tensor{\Gamma}{^{\eta}_{\nu\lambda}}
    \tensor{\Gamma}{^{\sigma}_{\mu\kappa}} -
    \tensor{\Gamma}{^{\eta}_{\kappa\lambda}}
    \tensor{\Gamma}{^{\sigma}_{\mu\nu}}).
  \end{split}
\end{equation}
Questa relazione ci tornerà utile quando considereremo il limite newtoniano
delle equazioni di Einstein.  Da questa espressione si vede che
\begin{enumerate}
\item \label{item:antisimmetria-riemann} il tensore di Riemann è antisimmetrico
  rispetto allo scambio fra primo e secondo indice e allo scambio fra terzo e
  quarto indice
  \begin{equation}
    R_{\lambda\mu\nu\kappa} = -R_{\mu\lambda\nu\kappa} =
    -R_{\lambda\mu\kappa\nu} = R_{\mu\lambda\kappa\nu};
  \end{equation}
\item \label{item:simmetria-riemann} il tensore di Riemann
  $R_{\lambda\mu\nu\kappa}$ è simmetrico rispetto allo scambio fra la prima
  coppia indici $(\lambda \mu)$ con la seconda coppia $(\nu \kappa)$
  \begin{equation}
    R_{\lambda\mu\nu\kappa} = R_{\nu\kappa\lambda\mu}
  \end{equation}
\item \label{item:ciclicita-riemann} è valida la seguente relazione di ciclicità
  (si noti che il primo indice è tenuto fisso, si fanno ruotare gli altri tre)
  \begin{equation}
    \label{eq:ciclicita-riemann}
    R_{\lambda\mu\nu\kappa} + R_{\lambda\kappa\mu\nu} + R_{\lambda\nu\kappa\mu}
    = 0.
  \end{equation}
\end{enumerate}

A causa delle proprietà appena elencate, delle $4^{4} = 256$ componenti del
tensore di Riemann solo $20$ sono indipendenti.  Per far vedere ciò consideriamo
il caso generale di uno spazio $n$-dimensionale.  Il tensore di Riemann può
essere immaginato come una matrice quadrata simmetrica (per la proprietà di
simmetria~\ref{item:simmetria-riemann}) di ordine $d$ del tipo
$R_{(\lambda\mu)(\nu\kappa)}$, con indici $(\lambda\mu)$ e $(\nu\kappa)$.  Il
valore dell'ordine $d$ di questa matrice è dato dal numero di valori
indipendenti che assumono i due indici che, per la proprietà di
antisimmetria~\ref{item:antisimmetria-riemann}, sono uguali al numero di
elementi indipendenti di una matrice $n \times n$ antisimmetrica, cioè
$d = n(n-1)/2$.  Dunque gli elementi indipendenti di
$R_{(\lambda\mu)(\nu\kappa)}$ si riducono
\begin{equation}
  \frac{1}{2} d (d + 1) = \frac{1}{2}\bigg( \frac{1}{2}n(n-1) \bigg) \bigg(
  \frac{1}{2}n(n-1) +1 \bigg) = \frac{1}{8}n(n-1)(n^{2} - n +2).
\end{equation}
% TODO: provare a capire per bene il perché dell'affermazione successiva e a
% spiegarlo in maniera semplice.  Per esempio vedi
% http://www.maths.tcd.ie/~houghton/TEACHING/442/notes/22Oct2002.pdf
% http://sergeysk.wordpress.com/2011/01/21/number-of-independent-components-of-the-riemann-curvature-tensor/
% http://ned.ipac.caltech.edu/level5/March01/Carroll3/Carroll3.html (è il
% capitolo 3 del Carrol)
Infine, la relazione di ciclicità~\eqref{eq:ciclicita-riemann} corrisponde a
$\binom{n}{4}$ equazioni differenti, allora abbiamo altrettanti ulteriori
vincoli, così che gli elementi indipendenti del tensore di Riemann
$n$-dimensionale sono
\begin{equation}
  C_{n} = \frac{1}{8}n(n-1)(n^{2} - n +2) - \binom{n}{4} =
  \frac{1}{12}n^{2}(n^{2} - 1).
\end{equation}
Ponendo $n=4$ abbiamo che nello spazio-tempo quadrimensionale gli elementi
indipendenti del tensore di Riemann sono $C_{4} = 20$, come avevamo
preannunciato.

La contrazione fra primo e terzo indice del tensore di Riemann dà il
\index{tensore!di Ricci}\emph{tensore di Ricci}
\begin{equation}
  \label{eq:tens-ricci}
  R_{\mu\kappa} = g^{\lambda\nu} R_{\lambda\mu\nu\kappa} =
  \tensor{R}{^{\nu}_{\mu\nu\kappa}}
  = \parder{\tensor{\Gamma}{^{\nu}_{\mu\nu}}}{x^{\kappa}}
  - \parder{\tensor{\Gamma}{^{\nu}_{\mu\kappa}}}{x^{\nu}} +
  \tensor{\Gamma}{^{\eta}_{\mu\nu}} \tensor{\Gamma}{^{\nu}_{\eta\kappa}} -
  \tensor{\Gamma}{^{\eta}_{\mu\kappa}} \tensor{\Gamma}{^{\nu}_{\eta\nu}}.
\end{equation}
Il tensore di Ricci è simmetrico per la proprietà~\ref{item:simmetria-riemann}
di simmetria del tensore di Riemann, infatti
\begin{equation}
  \label{eq:simmetria-riemann}
  R_{\mu\kappa} = g^{\lambda\nu} R_{\lambda\mu\nu\kappa} = g^{\lambda\nu}
  R_{\nu\kappa\lambda\mu} = g^{\nu\lambda} R_{\nu\kappa\lambda\mu} =
  R_{\kappa\mu}.
\end{equation}
Dunque delle $4^{2} = 16$ componenti del tensore $R_{\mu \kappa}$
solo $10$ sono indipendenti.
Inoltre il tensore di Ricci è essenzialmente l'unico tensore non nullo che si
può ottenere dalla contrazione del tensore di Riemann.  Contraendo primo e
secondo indice, per la proprietà~\ref{item:antisimmetria-riemann} di
antisimmetria del tensore di Riemann si ha
\begin{equation}
  \tensor{R}{^{\mu}_{\mu\nu\kappa}} = g^{\lambda\mu}
  R_{\lambda\mu\nu\kappa} = -g^{\lambda\mu} R_{\mu\lambda\nu\kappa} =
  -g^{\mu\lambda}R_{\mu\lambda\nu\kappa} =
  -g^{\lambda\mu}R_{\lambda\mu\nu\kappa} = 0.
\end{equation}
Analogamente, contraendo terzo e quarto indice si ottiene
\begin{equation}
  \tensor{R}{_{\lambda\mu}^{\kappa}_{\kappa}} = g^{\nu\kappa}
  R_{\lambda\mu\nu\kappa} = 0.
\end{equation}
Contraendo secondo e quarto indice del tensore di Riemann si ottiene di nuovo il
tensore di Ricci, invece contraendo primo e quarto indice oppure secondo terzo
si ottiene ancora il tensore di Ricci così come da noi definito ma con il segno
opposto.  Infatti, moltiplicando tutti i membri
della~\eqref{eq:simmetria-riemann} per $g^{\lambda\nu}$ troviamo
\begin{equation}
  \tensor{R}{^{\nu}_{\mu\nu\kappa}} = -\tensor{R}{_{\mu}^{\nu}_{\nu\kappa}} =
  -\tensor{R}{^{\nu}_{\mu\kappa\nu}} = \tensor{R}{_{\mu}^{\nu}_{\kappa\nu}}.
\end{equation}

L'ulteriore contrazione del tensore di Ricci dà la
\index{curvatura scalare}\emph{curvatura scalare}
\begin{equation}
  R = g^{\mu\nu}R_{\mu\nu} = \tensor{R}{^{\mu}_{\mu}}.
\end{equation}

\subsection{Tensore di Riemann e curvatura dello spazio}
\label{sec:riemann-curvatura}

È ben noto che nello spazio-tempo piatto è possibile introdurre una metrica
$g_{\mu \nu}(x)$ non costante.  Questo è il caso, per esempio, delle coordinate
$(t,r,\theta, \phi)$ nello spazio-tempo di Minkoowski.  Siamo allora interessati
a cercare uno strumento che ci permetta di discriminare fra uno spazio-tempo
intrinsecamente curvo, con metrica $g_{\mu \nu} \ne \eta_{\mu \nu}$ perché
permeato da un reale campo gravitazionale, e uno spazio-tempo piatto, cioè lo
spazio di Minkowski, in cui si fa uso di coordinate curvilinee.

Se siamo nella seconda situazione, vedremo che è possibile effettuare una
trasformazione $x^{\alpha} \to \xi^{\alpha}(x)$ per cui in tutto lo spazio-tempo
si abbia la metrica di Minkoowski. Si ha quindi
\begin{equation}
  \label{4.15}
  \eta_{\alpha\beta} =
  g_{\mu\nu} \parder{x^{\mu}}{\xi^{\alpha}} \parder{x^{\nu}}{\xi^{\beta}}.
\end{equation}

Osserviamo che per il \index{principio!di equivalenza!forte}principio di
equivalenza, assicura che in ogni punto dello spazio-tempo è sempre possibile
individuare un sistema di coordinate in cui localmente sia soddisfatta la
relazione~\eqref{4.15}; qui invece cerchiamo una condizione che ci permetta di
soddisfare la relazione precedente in tutto lo spazio-tempo.

Lo strumento che cerchiamo è il tensore di Riemann, poiché infatti è possibile
dimostrare\footnote{Vedi~\textcite[138]{weinberg:gravitation}.} il seguente
\begin{teorema}
  Condizioni necessarie e sufficienti affinché sia possibile effettuare una
  trasformazione di coordinate $x^{\mu} \to \xi^{\alpha}$
  tale che in tutto lo spazio sia
  \begin{equation}
    \label{eq:tensore-spazio-piatto}
    \eta_{\alpha\beta} =
    g_{\mu\nu} \parder{x^{\mu}}{\xi^{\alpha}} \parder{x^{\nu}}{\xi^{\beta}}
  \end{equation}
  sono
  \begin{enumerate}
  \item il \index{tensore!di Riemann}tensore di Riemann si annulla in tutto lo
    spazio
    \begin{equation}
      \tensor{R}{^{\lambda}_{\mu\nu\kappa}} = 0;
    \end{equation}
  \item in qualche punto $X$ la matrice $g_{\mu\nu}(X)$ ha tre autovalori
    positivi e uno negativo.
  \end{enumerate}
\end{teorema}

Il fatto che sia valida la relazione~\eqref{eq:tensore-spazio-piatto} significa
che la metrica è quella usuale dello spazio di Minkowski $\dd \tau^{2} = \dd
t^{2} - \dd\bm{x}^{2}$.  L'annullarsi del tensore di Riemann invece significa
che la curvatura dello spazio è nulla ed il teorema precedente dichiara
l'equivalenza fra queste due condizioni.

In questo modo anche se fossimo in presenza di una metrica $g_{\mu \nu}(x)$, il
valore del tensore di Riemann ci permette di decidere se siamo in presenza di un
reale campo gravitazionale (per cui lo spazio è intrinsecamente curvo) o se tale
campo è assennte e $g_{\mu \nu} \ne \eta_{\mu \nu}$ perché usiamo coordinate
curvilinee nello spazio tempo piatto.

\subsection{Commutazione delle derivate covarianti}
\label{sec:commutazione-derivate-covarianti}

Nel calcolo differenziale ordinario, l'ordine di derivazione non ha importanza.
Vediamo cosa succede nel caso di derivazione covariante.  Calcoliamo quindi
$V_{\mu;\nu;\kappa} - V_{\mu;\kappa;\nu}$.  Per questo ricordiamo che la
derivata covariante $V_{\mu;\nu} = V_{\mu,\nu} -
\tensor{\Gamma}{^{\lambda}_{\mu\nu}} V_{\lambda}$ di un vettore covariante
$V_{\mu}$ è un tensore $T_{\mu\nu}$ di rango $2$ completamente covariante e la
sua ulteriore derivata covariante è
\begin{equation}
  \begin{split}
    V_{\mu;\nu;\kappa} &= T_{\mu\nu;\kappa} = T_{\mu\nu,k} -
    \tensor{\Gamma}{^{\sigma}_{\kappa\nu}} T_{\mu\sigma} -
    \tensor{\Gamma}{^{\sigma}_{\mu\kappa}} T_{\sigma\nu} \\
    &= V_{\mu,\nu,\kappa} -
    \tensor{\Gamma}{^{\lambda}_{\mu\nu,\kappa}}V_{\lambda} -
    \tensor{\Gamma}{^{\lambda}_{\mu\nu}} V_{\lambda,\kappa} -
    \tensor{\Gamma}{^{\sigma}_{\kappa\nu}} (V_{\mu,\sigma} -
    \tensor{\Gamma}{^{\lambda}_{\mu\sigma}} V_{\lambda}) -
    \tensor{\Gamma}{^{\sigma}_{\mu\kappa}} (V_{\sigma,\nu} -
    \tensor{\Gamma}{^{\lambda}_{\sigma\nu}} V_{\lambda}).
  \end{split}
\end{equation}
Dopo aver determinato $V_{\mu;\nu;\kappa}$ è sufficiente invertire $\nu$ e
$\kappa$ per trovare $V_{\mu;\kappa;\nu}$
\begin{equation}
  V_{\mu;\kappa;\nu} = V_{\mu,\kappa,\nu} -
    \tensor{\Gamma}{^{\lambda}_{\mu\kappa,\nu}}V_{\lambda} -
    \tensor{\Gamma}{^{\lambda}_{\mu\kappa}} V_{\lambda,\nu} -
    \tensor{\Gamma}{^{\sigma}_{\nu\kappa}} (V_{\mu,\sigma} -
    \tensor{\Gamma}{^{\lambda}_{\mu\sigma}} V_{\lambda}) -
    \tensor{\Gamma}{^{\sigma}_{\mu\nu}} (V_{\sigma,\kappa} -
    \tensor{\Gamma}{^{\lambda}_{\sigma\kappa}} V_{\lambda}).
\end{equation}
Nel valutare la differenza $V_{\mu;\nu;\kappa} - V_{\mu;\kappa;\nu}$ tutti i
termini simmetrici per lo scambio $\nu \leftrightarrow \kappa$ si annullano,
vale a dire i termini contenenti le derivate prime e seconde di $V_{\mu}$
(osserva che $\lambda$ e $\sigma$ sono entrambi indici muti), quindi rimane
\begin{equation}
  V_{\mu;\nu;\kappa} - V_{\mu;\kappa;\nu} =
  (-\tensor{\Gamma}{^{\lambda}_{\mu\nu,\kappa}} +
  \tensor{\Gamma}{^{\sigma}_{\mu\kappa}} \tensor{\Gamma}{^{\lambda}_{\sigma\nu}}
  + \tensor{\Gamma}{_{\mu\kappa,\nu}} - \tensor{\Gamma}{^{\sigma}_{\mu\nu}}
  \tensor{\Gamma}{^{\lambda}_{\sigma\kappa}}) V_{\lambda} =
  -\tensor{R}{^{\lambda}_{\mu\nu\kappa}} V_{\lambda}.
\end{equation}
Abbiamo allora
\begin{equation}
  V_{\mu;\nu;\kappa} - V_{\mu;\kappa;\nu} = -V_{\lambda}
  \tensor{R}{^{\lambda}_{\mu\nu\kappa}}.
  \label{218}
\end{equation}
Si trovano analoghe relazioni per i vettori controvarianti o, più in generale,
per i tensori
\begin{subequations}
  \begin{gather}
    \tensor{V}{^{\mu}_{;\nu;\kappa}} - \tensor{V}{^{\mu}_{;\kappa;\nu}} =
    V^{\sigma} \tensor{R}{^{\mu}_{\sigma\nu\kappa}}, \\
    \tensor{T}{^{\lambda}_{\mu;\nu;\kappa}} -
    \tensor{T}{^{\lambda}_{\mu;\kappa;\nu}} = \tensor{T}{^{\sigma}_{\mu}}
    \tensor{R}{^{\lambda}_{\sigma\nu\kappa}} - \tensor{T}{^{\lambda}_{\sigma}}
    \tensor{R}{^{\sigma}_{\mu\nu\kappa}}.
  \end{gather}
\end{subequations}

Nello spazio di Minkowski le derivate covarianti coincidono con le derivate
ordinarie e le derivate seconde miste commutano
\begin{equation}
  V_{\mu,\nu,\kappa} = \parder{V_{\mu}}{x^{\nu}, x^{\kappa}}
  = \parder{V_{\mu}}{x^{\kappa}, x^{\nu}} = V_{\mu,\kappa,\nu}
\end{equation}
e questo è possibile solo se il tensore di Riemann è identicamente nullo in
tutto lo spazio.  L'equazione~\eqref{218} può considerarsi una ulteriore
definizione del tensore di Riemann.

\subsection{Il trasporto parallelo su una curva chiusa}
\label{sec:trasporto-parallelo-linea-chiusa}

Per trasportare un vettore parallelamente a se stesso basta mantenere invariato
l'angolo che esso forma con la geodetica locale lungo cui è trasportato.  Il
trasporto su un percorso chiuso non riporta, in generale, il vettore su se
stesso; per rendersene conto basta pensare al caso di un percorso costruito
sulla superficie di una sfera con tre archi che delimitano un ottante.  Uno
spazio in cui sia possibile definire univocamente un vettore per trasporto
parallelo su una curva chiusa si dice piatto, altrimenti si dice curvo.  Dalla
definizione di trasporto parallelo si ha
\begin{equation}
  \cdd B_{\beta} = B_{\beta;\mu} \dd x^{\mu}=0.
  \label{4.1}
\end{equation}
Il vettore $B_{\beta}$ torna su se stesso dopo il trasporto parallelo su una
linea chiusa se
\begin{equation}
  \label{4.24}
  \int\limits_{\textup{curva chiusa}} \delta B_{\beta}=0
\end{equation}
cioè se
\begin{equation}
  \int\limits_{\textup{curva chiusa}} \tensor{\Gamma}{^{\alpha}_{\beta \mu}}
  B_{\alpha} \dd x^{\mu} =0.
\end{equation}
Dalla relazione precedente e per il teorema di Stokes, si ha
\begin{equation}
  \left[\tensor{\Gamma}{^{\alpha}_{\beta \nu}} B_{\alpha} \right]_{,\mu} =
  \left[\tensor{\Gamma}{^{\alpha}_{\beta \mu}} B_{\alpha} \right]_{,\nu}.
\end{equation}
Sviluppando questa uguaglianza si ha
\begin{equation}
  \tensor{\Gamma}{^{\alpha}_{\beta \nu ,\mu}} B_{\alpha} +
  \tensor{\Gamma}{^{\alpha}_{\beta \nu}} B_{\alpha ,\mu} =
  \tensor{\Gamma}{^{\alpha}_{\beta \mu, \nu}} B_{\alpha} +
  \tensor{\Gamma}{^{\alpha}_{\beta \mu}} B_{\alpha , \nu},
\end{equation}
da cui
\begin{equation}
  \label{4.5}
  B_{\alpha} \left[ \tensor{\Gamma}{^{\alpha}_{\beta \nu , \mu}}
    -\tensor{\Gamma}{^{\alpha}_{\beta \mu ,\nu}} \right] +
  \tensor{\Gamma}{^{\sigma}_{\beta \nu}} B_{\sigma,\mu} -
  \tensor{\Gamma}{^{\sigma}_{\beta \mu}} B_{\sigma,\nu} = 0.
\end{equation}
Ora dall'equazione~\eqref{4.1} e dalla definizione di derivata covariante segue
\begin{equation}
  B_{\sigma;\nu} = B_{\sigma,\nu}-\tensor{\Gamma}{^{\alpha}_{\sigma \nu}}
  B_{\alpha} = 0,
\end{equation}
ciò implica che $B_{\sigma,\nu}= \tensor{\Gamma}{^{\alpha}_{\sigma \nu}}
B_{\alpha}$ e che $B_{\sigma,\mu}= \tensor{\Gamma}{^{\alpha}_{\sigma \mu}}
B_{\alpha}$.  Tenendo conto di queste relazioni, l'equazione~\eqref{4.5} diventa
\begin{equation}
  B_{\alpha}
  \left[ \tensor{\Gamma}{^{\alpha}_{\beta \nu ,\mu}} -
    \tensor{\Gamma}{^{\alpha}_{\beta \mu ,\nu}} \right] +
  \tensor{\Gamma}{^{\sigma}_{\beta \nu}} \tensor{\Gamma}{^{\alpha}_{\sigma \mu}}
  B_{\alpha} - \tensor{\Gamma}{^{\sigma}_{\beta \mu}}
  \tensor{\Gamma}{^{\alpha}_{\sigma \nu}} B_{\alpha} =0
\end{equation}
da cui
\begin{equation}
  B_{\alpha} \left[
    \tensor{\Gamma}{^{\alpha}_{\beta \nu, \mu}} -
    \tensor{\Gamma}{^{\alpha}_{\beta \mu, \nu}} +
    \tensor{\Gamma}{^{\sigma}_{\beta \nu}} \tensor{\Gamma}{^{\alpha}_{\sigma
        \mu}} - \tensor{\Gamma}{^{\sigma}_{\beta \mu}}
    \tensor{\Gamma}{^{\alpha}_{\sigma \nu}} \right] =0
 \end{equation}
e allora in termini del tensore di Riemann
\begin{equation}
  \label{4.32}
  B_{\alpha} \tensor{R}{^{\alpha}_{\beta \mu \nu}} = 0.
\end{equation}

Abbiamo quindi provato che imporre la condizione~\eqref{4.24} è equivalente a
richiedere che sia valida la condizione~\eqref{4.32}.  Possiamo quindi asserire
che condizione necessaria e sufficiente affinché uno spazio sia piatto (nel
senso della defizione data sopra di trasporto parallelo) è che il tensore di
Riemann sia nullo.

Si ottiene lo stesso risultato operando con vettori controvarianti invece che
covarianti. In questo caso dalla relazione
\begin{equation}
  \cdd A^{\alpha} = \tensor{A}{^{\alpha}_{;\mu}} \dd x^{\mu} =0
\end{equation}
e procedendo nella stessa maniera si ricava:
\begin{equation}
  A^{\alpha} \tensor{R}{_{\alpha \beta \mu \nu}} = 0.
\end{equation}

\subsection{Il trasporto parallelo su un percorso infinitesimo}
\label{sec:trasporto-parallelo-percorso-infinitesimo}

Calcoliamo la variazione delle componenti di un vettore trasportato
parallelamente lungo un percorso chiuso infinitesimo.\footnote{Per approfondire
  questo argomento vedi anche~\textcite[266]{ohanian:gravitazione}.}
Consideriamo due differenti percorsi:
\begin{equation}
  P(x) \to P_1(x^{\mu} + d_1 x^{\mu}) \to Q(x^{\mu} + d_1 x^{\mu} + d_2 x^{\mu})
\end{equation}
che porta da $P(x)$ a $P_1(x^{\mu} + d_1 x^{\mu})$ e infine a $Q(x^{\mu} + d_1
x^{\mu} + d_2 x^{\mu})$ (lungo il percorso infinitesimo $\delta \gamma_1$, vedi
figura), e alternativamente
\begin{equation}
  P(x)\to  P_2(x^{\mu} + d_2 x^{\mu})\to Q(x^{\mu} + d_2 x^{\mu} + d_1 x^{\mu})
\end{equation}
che porta da $P(x)$ a $P_2(x^{\mu} + d_2 x^{\mu})$ e infine a $Q(x^{\mu} + d_2
x^{\mu} + d_1 x^{\mu})$ lungo $\delta \gamma_{2}$.

Il 4-vettore $A_{\alpha}$ in $P_1$ vale
\begin{equation}
  A_{\alpha}(P_1) = A_{\alpha}(P)+ \delta A_{\alpha}(P) = A_{\alpha}(P) +
  \tensor{\Gamma}{^{\beta}_{\alpha \mu}}(P) A_{\beta}(P) d_1 x^{\mu}
\end{equation}
mentre il vettore $A_{\alpha}$ in $Q$ , dopo il percorso $PP_1Q$ è dato da
\begin{equation}
  A_{\alpha}(Q)_{[PP_1Q]} = A_{\alpha}(P_1)+ \delta A_{\alpha}(P_1) =
  A_{\alpha}(P_1) + \tensor{\Gamma}{^{\beta}_{\alpha \mu}}(P_1) A_{\beta}(P_1)
  d_2 x^{\mu}
\end{equation}
Pertanto il vettore $A_{\alpha}$ in $Q$ dopo il percorso $PP_1Q$ diventa
\begin{equation}
  \begin{split}
    A_{\alpha}(Q)_{[PP_1Q]} &= A_{\alpha}(P) +
    \tensor{\Gamma}{^{\beta}_{\alpha\mu}}(P) A_{\beta}(P) d_1 x^{\mu} \\
    &\quad{} + \tensor{\Gamma}{^{\beta}_{\alpha\nu}}(P_1) [ A_{\beta}(P) +
    \tensor{\Gamma}{^{\rho}_{\beta \mu}}(P) A_{\rho}(P) d_1 x^{\mu}] d_2
    x^{\nu}.
    \label{4.10}
  \end{split}
\end{equation}
Sviluppando la connessione affine nell'intorno di $P$ segue
\begin{equation}
  \tensor{\Gamma}{^{\beta}_{\alpha \nu}}(P_1) \simeq
  \tensor{\Gamma}{^{\beta}_{\alpha \nu}}(P) + \tensor{\Gamma}{^{\beta}_{\alpha
      \nu, \mu}}(P)  d_1 x^{\mu}
\end{equation}
e trascurando i termini del terzo ordine, l'equazione~\eqref{4.10} diventa
\begin{equation}
\begin{split}
  A_{\alpha}(Q)_{[PP_1Q]} &= A_{\alpha} + \Gamma^{\beta}_{\alpha \mu} A_{\beta}
  d_1 x^{\mu} + \Gamma^{\beta}_{\alpha \nu} A_{\beta} d_2 x^{\nu} \\
  &\quad{} + \Gamma^{\beta}_{\alpha \nu , \mu} A_{\beta} d_1 x^{\mu} d_2 x^{\nu}
  + \Gamma^{\beta}_{\alpha \nu} \Gamma^{\rho}_{\beta \mu} A_{\rho} d_1 x^{\mu}
  d_2 x^{\nu}.
\end{split}
\end{equation}
Raggruppando i termini e tenendo conto degli indici muti si ha
\begin{equation}
  A_{\alpha}(Q)_{[PP_1Q]} = A_{\alpha} + \tensor{\Gamma}{^{\beta}_{\alpha \mu}}
  A_{\beta} ( d_1 x^{\mu}+d_2 x^{\mu} ) + [ \tensor{\Gamma}{^{\beta}_{\alpha
      \nu, \mu}} + \tensor{\Gamma}{^{\sigma}_{\alpha \nu}}
  \tensor{\Gamma}{^{\beta}_{\sigma \mu}}] A_{\beta} d_1 x^{\mu} d_2 x^{\nu}.
  \label{4.12}
\end{equation}
Similmente il vettore $A_{\alpha}$ in $Q$ dopo il percorso $PP_2Q$ è dato da
\begin{equation}
  A_{\alpha}(Q)_{[PP_2Q]} = A_{\alpha} + \tensor{\Gamma}{^{\beta}_{\alpha \mu}}
  A_{\beta} (d_1 x^{\mu}+d_2 x^{\mu}) + [ \tensor{\Gamma}{^{\beta}_{\alpha \mu,
      \nu}} + \tensor{\Gamma}{^{\sigma}_{\alpha \mu}}
  \tensor{\Gamma}{^{\beta}_{\sigma \nu}}] A_{\beta} d_1 x^{\mu} d_2 x^{\nu}.
  \label{4.13}
\end{equation}
Pertanto la differenza $\delta A_{\alpha}$ per trasporto parallelo del vettore
$A_{\alpha}$ lungo il percorso chiuso $P P_1 Q P_2 P$ è dato dalla differenza
tra l'equazione~\eqref{4.12} e l'equazione~\eqref{4.13}, cioè
\begin{equation}
  \delta A_{\alpha} = - \tensor{R}{^{\beta}_{\alpha \mu \nu}} A_{\beta} d_1
  x^{\mu} d_2 x^{\nu}.
\end{equation}
La relazione precedente dimostra che $\tensor{R}{^{\beta}_{\alpha \mu \nu}}$ è
un tensore poiché moltiplicandolo per tre vettori si ottiene ancora un vettore.

\subsection{Identità di Bianchi}
\label{sec:identita-bianchi}

Si dimostra (vedi l'appendice~\ref{sec:dimostr-identita-bianchi}) la seguente
relazione
\begin{equation}
  \label{eq:bianchi}
  R_{\lambda\mu\nu\kappa;\eta} +
  R_{\lambda\mu\eta\nu;\kappa} + R_{\lambda\mu\kappa\eta;\nu} = 0
\end{equation}
chiamata \index{identità!di Bianchi}\emph{identità di Bianchi}.  Contraiamo gli
indici $\lambda$ e $\nu$ (moltiplicando ambo i membri per $g^{\lambda\nu}$)
\begin{equation}
  \begin{split}
    \tensor{R}{^{\nu}_{\mu\nu\kappa;\eta}} +
      \tensor{R}{^{\nu}_{\mu\kappa\eta;\nu}} +
    \tensor{R}{^{\nu}_{\mu\eta\nu;\kappa}} &= R_{\mu\kappa;\eta} +
    \tensor{R}{^{\nu}_{\mu\kappa\eta;\nu}} - R_{\mu\eta;\kappa} \\
    &= R_{\mu\kappa;\eta} - \tensor{R}{^{\nu}_{\mu\eta\kappa;\nu}} -
    R_{\mu\eta;\kappa} = 0.
  \end{split}
\end{equation}
Contraendo ulteriormente gli indici $\mu$ e $\kappa$
(moltiplicando per $g^{\mu\kappa}$) otteniamo
\begin{equation}
  R_{;\eta} - \tensor{R}{^{\nu}_{\eta;\nu}} -
  \tensor{R}{^{\kappa}_{\eta;\kappa}} = R_{;\eta} - 2
  \tensor{R}{^{\mu}_{\eta;\mu}} = 0,
\end{equation}
poiché $\kappa$ e $\nu$ sono entrambi indici muti che abbiamo rinominato $\mu$.
Questa relazione può anche essere scritta come
\begin{equation}
  \bigg( \tensor{R}{^{\mu}_{\eta}} - \frac{1}{2} \tensor{\delta}{^{\mu}_{\eta}}
  R \bigg)_{;\mu} = 0
\end{equation}
e moltiplicando ancora per $g^{\eta\nu}$
\begin{equation}
  \bigg( R^{\mu\nu} - \frac{1}{2} g^{\mu\nu} R\bigg)_{;\mu} = 0.
\end{equation}
La quantità all'interno delle parentesi è il
\index{tensore!di Einstein}\emph{tensore di Einstein}
\begin{equation}
  G^{\mu\nu} = R^{\mu\nu} - \frac{1}{2} g^{\mu\nu}R
\end{equation}
e abbiamo dunque mostrato che $G^{\mu \nu}$ ha quadridivergenza covariante nulla
\begin{equation}
  \label{eq:divergenza-tens-einstein}
  \tensor{G}{^{\mu\nu}_{;\mu}} = \bigg( R^{\mu\nu} - \frac{1}{2} g^{\mu\nu}
  R\bigg)_{;\mu} = 0.
\end{equation}

\subsection{Equazione della deviazione geodetica}
\label{sec:deviazione-geodetica}

Vogliamo dimostrare che il tensore di Riemann descrive le forze mareali,
equivalentemente che esso descrive l'accelerazione relativa di due particelle in
caduta libera.

Consideriamo due geodetiche a distanza infinitesima: la prima $x^{\mu}(\tau)$
sia associata a una particella di prova $A$, la seconda $x^{\mu}(\tau)
+s^{\mu}(\tau)$ a una particella $B$.  Quindi $s^{\mu}$ è detto \emph{vettore di
  connessione} tra le due geodetiche (vedi figura).

Per le due geodetiche avremo
\begin{subequations}
  \begin{gather}
    \toder[2]{x^{\mu}}{\tau} + \tensor{\Gamma}{^{\mu}_{\alpha \beta}}(x)
    \toder{x^{\alpha}}{\tau} \toder{x^{\beta}}{\tau} = 0,
    \label{116} \\
    \toder[2]{(x^{\mu}+ s^{\mu})}{\tau} + \tensor{\Gamma}{^{\mu}_{\alpha
        \beta}}(x+s) \left(\toder{x^{\alpha}}{\tau} + \toder{s^{\alpha}}{\tau}
    \right) \left( \toder{x^{\beta}}{\tau} + \toder{s^{\beta}}{\tau} \right) =
    0.
    \label{117}
  \end{gather}
\end{subequations}
Nell'equazione precedente usiamo l'approssimazione
\begin{equation}
  \tensor{\Gamma}{^{\mu}_{\alpha \beta}}(x+s) \simeq
  \tensor{\Gamma}{^{\mu}_{\alpha \beta}}(x) + \tensor{\Gamma}{^{\mu}_{\alpha
      \beta ,\sigma}}(x) s^{\sigma}
\end{equation}
Considerando solo i termini al primo ordine in $s^{\mu}$, valutiamo la
differenza tra le equazioni~\eqref{116} e \eqref{117}.  Otteniamo
\begin{equation}
  \toder[2]{s^{\mu}}{\tau} =
  -\tensor{\Gamma}{^{\mu}_{\alpha \beta, \sigma}} s^{\sigma}
  \toder{x^{\alpha}}{\tau} \toder{x^{\beta}}{\tau}
  -2\tensor{\Gamma}{^{\mu}_{\alpha \beta}} \toder{s^{\alpha}}{\tau}
  \toder{x^{\beta}}{\tau}.
  \label{119}
\end{equation}

Valutiamo ora la derivata seconda del vettore connessione $s^{\mu}$.  Abbiamo
visto che la derivata prima di un vettore $s^{\mu}$ lungo una geodetica è
\begin{equation}
  \covder{s^{\mu}}{\tau} = \tensor{s}{^{\mu}_{; \beta}}\toder{x^{\beta}}{\tau} =
  \toder{s^{\mu}}{\tau} + \tensor{\Gamma}{^{\mu}_{\alpha \beta}} s^{\alpha}
  \toder{x^{\beta}}{\tau}.
\end{equation}
Pertanto
\begin{equation}
  \begin{split}
    \covder[2]{s^{\mu}}{\tau} = \covder{}{\tau} \left( \covder{s^{\mu}}{\tau}
    \right) &= \toder{}{\tau} \left( \covder{s^{\mu}}{\tau} \right) +
    \tensor{\Gamma}{^{\mu}_{\alpha \beta}} \toder{s^{\alpha}}{\tau}
    \toder{x^{\beta}}{\tau} + \toder{}{\tau} \left( \toder{s^{\mu}}{\tau} +
      \tensor{\Gamma}{^{\mu}_{\alpha \beta}} s^{\alpha}
      \toder{x^{\beta}}{\tau}\right) \\
    &\quad{} + \tensor{\Gamma}{^{\mu}_{\alpha \beta}} \left(
      \toder{s^{\alpha}}{\tau} + \tensor{\Gamma}{^{\alpha}_{k \lambda}} s^{k}
      \toder{x^{\lambda}}{\tau} \right) \toder{x^{\beta}}{\tau}.
  \end{split}
\end{equation}
Sviluppando si ha
\begin{equation}
  \begin{split}
    \covder[2]{s^{\mu}}{\tau} = & \toder[2]{s^{\mu}}{\tau} +
    \tensor{\Gamma}{^{\mu}_{\alpha \beta , \nu}}
    \toder{x^{\nu}}{\tau} s^{\alpha} \toder{x^{\beta}}{\tau} + 2
    \tensor{\Gamma}{^{\mu}_{\alpha \beta}} \toder{s^{\alpha}}{\tau}
    \toder{x^{\beta}}{\tau} \\
    & - \tensor{\Gamma}{^{\mu}_{\alpha \beta}} s^{\alpha}
    \tensor{\Gamma}{^{\beta}_{k \lambda}} \toder{x^{k}}{\tau}
    \toder{x^{\lambda}}{\tau} + \tensor{\Gamma}{^{\mu}_{\alpha \beta}}
    \tensor{\Gamma}{^{\alpha}_{k \lambda}} s^{k} \toder{x^{\lambda}}{\tau}
    \toder{x^{\beta}}{\tau}.
  \end{split}
  \label{115}
\end{equation}
Il penultimo termine si ottiene calcolando $\ltoder[2]{x^{\beta}}{\tau}= -
\tensor{\Gamma}{^{\beta}_{k \lambda}}
(\ltoder{x^{k}}{\tau})(\ltoder{x^{\lambda}}{\tau})$.

La derivata seconda $\lcovder[2]{s^{\mu}}{\tau}$ può essere ora calcolata
sostituendo l'espressione~\eqref{119} nell'equazione precedente.  Si ha
\begin{equation}
  \begin{split}
    \covder[2]{s^{\mu}}{\tau} &= -\tensor{\Gamma}{^{\mu}_{\alpha \beta,\sigma}}
    s^{\sigma} \toder{x^{\alpha}}{\tau} \toder{x^{\beta }}{\tau} -
    2\tensor{\Gamma}{^{\mu}_{\alpha \beta}} \toder{s^{\alpha}}{\tau}
    \toder{x^{\beta}}{\tau} + \tensor{\Gamma}{^{\mu}_{\alpha \beta , \nu}}
    \toder{x^{\nu}}{\tau} s^{\alpha} \toder{x^{\beta}}{\tau} \\
    &\quad{} + 2\tensor{\Gamma}{^{\mu}_{\alpha \beta}} \toder{s^{\alpha}}{\tau}
    \toder{dx^{\beta}}{d \tau} - \tensor{\Gamma}{^{\mu}_{\alpha \beta}}
    s^{\alpha} \tensor{\Gamma}{^{\beta}_{k \lambda}} \toder{x^{k}}{\tau}
    \toder{x^{\lambda}}{\tau} + \tensor{\Gamma}{^{\mu}_{\alpha \beta}}
    \tensor{\Gamma}{^{\alpha}_{k \lambda}} s^{k} \toder{x^{\lambda}}{\tau}
    \toder{x^{\beta}}{\tau}.
  \end{split}
\end{equation}
Nella precedente secondo e quarto termine si semplificano; inoltre operando
sugli indici muti si possono mettere in evidenza tra i restanti quattro termini
il prodotto $s^{\sigma} (\ltoder{x^{\alpha}}{\tau})
(\ltoder{x^{\beta}}{\tau})$. Allora abbiamo
\begin{equation}
  \covder[2]{s^{\mu}}{\tau} = \tensor{R}{^{\mu}_{\alpha \sigma \beta}}
  s^{\sigma} \toder{x^{\alpha}}{\tau} \toder{x^{\beta}}{\tau}.
\end{equation}
L'equazione precedente è nota come \emph{deviazione geodetica}.

Per campi deboli e statici e per velocità piccole ($V^{\alpha} \simeq
(1,0,0,0)$), trascurando il termine lineare e quadratico in $v$, l'equazione
precedente diventa
\begin{equation}
  \toder[2]{s^{k}}{t} \simeq -\tensor{R}{^k_{0l0}} s^l.
\end{equation}
Inoltre
\begin{subequations}
  \begin{align}
    \tensor{\Gamma}{^{\alpha}_{\beta\mu}} &\simeq\frac{1}{2}\eta^{\alpha \sigma}
                                            (h_{\sigma \beta, \mu}+h_{\mu
                                            \sigma, \beta} - h_{\beta \mu
                                            \sigma}), \\
    \tensor{R}{^{\alpha}_{\beta\mu\nu}} &\simeq\frac{1}{2}\eta^{\alpha \sigma}
                                          (h_{\mu \sigma , \beta , \nu} -
                                          h_{\beta \mu , \sigma , \nu} -
                                          h_{\nu \sigma, \beta , \mu} +
                                          h_{\beta \nu , \sigma , \mu}),
  \end{align}
\end{subequations}
con $h_{00} \simeq -2 \phi_{new}$ e $h_{ij} = \delta_{ij}$.  Quindi i soli
elementi non nulli del tensore di Riemann sono
\begin{equation}
  \tensor{R}{^{k}_{0l0}} \simeq - \frac{1}{2} \eta^{k m} h_{00,m,l}
  \simeq - \frac{1}{2} \delta^{km}
  \parder{}{x^m,x^l} h_{00} \simeq \delta^{ml}
  \parder{}{x^m,x^l}\phi_{\textup{new}}.
\end{equation}

Esiste un'analogia tra tensore di Riemann $R$ e forze mareali, nel senso che:
\begin{itemize}
\item in geometria non-euclidea il tensore di curvatura controlla le dimensioni
  della superficie gaussiana approssimabile con un piano tangente e geometria
  locale euclidea;
\item le forze mareali definiscono le dimensioni della regione entro cui un
  campo gravitazionale può essere eliminato con l'introduzione di un sistema di
  riferimento in caduta libera.
\end{itemize}
Per mostrare questa analogia, consideriamo un satellite in orbita attorno alla
terra e al suo interno due masse di prova $A$ e $B$, disposte nella direzione
radiale, a distanza $ \pm l/2$ dal centro di massa $O$ del satellite.

L'accelerazione centripeta nel punto $O$ (a distanza $r$ dal centro della Terra)
è $a(O) = GM/r^2$; mentre le accelerazioni in $A$ e $B$ sono: $a(A) =
GM/(r-l/2)^2$ e $a(B) = GM/(r+l/2)^2$.

Quindi l'accelerazione residua di $B$ rispetto ad $A$ è:
\begin{equation}
  a_{\textup{residua}} = \frac{GM}{(r+l/2)^2} - \frac{GM}{(r-l/2)^2} \simeq
  \frac{-2lGM}{r^3} \equiv - \underbrace{\frac{2GM}{c^2
      r^3}}_{\si[per-mode=reciprocal]{\per \centi\metre \squared}}
  lc^2.
\end{equation}
Il termine evidenziato ha dimensioni cm$^{-2}$ e può essere identificato con il
tensore di Riemann $R$.  È evidente che possiamo trascurare gli effetti mareali
($a_{\textup{residua}} \ll 1 $) se $ R l c^2 \ll 1 $.

In prossimità della Terra il campo gravitazionale è debole e stazionario.
Allora $R_{0000}=0$ ed $R_{i0j0} \simeq \frac{1}{2} \frac{\partial ^2
  h_{00}}{\partial x^i \partial x^j}$.  Utilizzando l'equazione della deviazione
geodetica possiamo calcolare l'accelerazione della particella $B$ come vista
dalla particella $A$ (accelerazione residua).  Assumendo inoltre che le velocità
siano piccole ($\dd\tau \simeq \dd t$) abbiamo
\begin{equation}
  \toder[2]{}{t} \delta r \simeq \tensor{R}{^r_{0r0}} \delta r \simeq
  \frac{1}{2} \toder[2]{}{r} (-2 \phi_{\textup{new}}) \delta r \simeq
  \frac{-2GM}{r^3} \delta r.
\end{equation}
Questo risultato conferma l'identificazione del tensore di Riemann con le forze
mareali.

%%% Local Variables:
%%% mode: latex
%%% TeX-master: "../gravitazione"
%%% fill-column: 80
%%% End:
